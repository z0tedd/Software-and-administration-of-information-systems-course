\documentclass[a4paper,14pt]{extreport} % добавить leqno в [] для нумерации слева
%\usepackage[14pt]{extsizes}
\usepackage[left=3cm,right=1.5cm,
top=2cm,bottom=2cm,bindingoffset=0cm]{geometry}
\linespread{1.45} %полуторный интервал
\usepackage{titlesec}
%%% Работа с русским языком
\titleformat{\section}{\normalfont\bfseries}{\thechapter}{14pt}{\bfseries}
\usepackage{cmap}					% поиск в PDF
\usepackage{mathtext} 				% русские буквы в фомулах
\usepackage[T2A]{fontenc}			% кодировка
\usepackage[utf8]{inputenc}			% кодировка исходного текста
\usepackage[english,russian]{babel}	% локализация и переносы
\usepackage{ fancyhdr} % улучшенная нумерация страниц
%%% Дополнительная работа с математикой
\usepackage{amsmath,amsfonts,amssymb,amsthm,mathtools} % AMS
\usepackage{icomma} % "Умная" запятая: $0,2$ --- число, $0, 2$ --- перечисление
%\renewcommand{\sfdefault}{cmss}
\usefont{T2A}{cmss}{m}{n}
%% Номера формул
\mathtoolsset{showonlyrefs=true} % Показывать номера только у тех формул, на которые есть \eqref{} в тексте.

%% Шрифты
\usepackage{euscript}	 % Шрифт Евклид
\usepackage{mathrsfs} % Красивый матшрифт
%% Свои команды
\DeclareMathOperator{\sgn}{\mathop{sgn}}

%% Перенос знаков в формулах (по Львовскому)
\newcommand*{\hm}[1]{#1\nobreak\discretionary{}
	{\hbox{$\mathsurround=0pt #1$}}{}}

%\renewcommand{\sfdefault}{cmss}
\usepackage{tempora}
%%% Заголовок
\title{100 Задач по теме "Комбинаторика}
\author{Антон Файтельсон}
\date{}

\begin{document}
			\begin{flushleft}
				\Large 100 Задач по теме "Комбинаторика"		\\
				Выполнил студент 113 группы Файтельсон Антон
				
			\end{flushleft}
			\begin{center}
				\begin{enumerate}
					\item {\large задача  Пароль}\\
						Источник: ICPC 2022-2023 NERC (NEERC), квалификационный этап Чемпионата Юга и Поволжья России \\
						\vspace{15pt}
						Монокарп забыл пароль от своего телефона. Пароль состоит ровно из 6 цифр от 0 до 9 (обратите внимание, что пароль может начинаться с цифры 0).\\
						\vspace{15pt}
						Монокарп помнит, что в его пароле были ровно две различные цифры, причем каждая из этих цифр встречалась в пароле ровно по три раза. Также Монокарп помнит количество цифр (n), которых точно не было в его пароле . \\
						\vspace{15pt}
						Посчитайте количество последовательностей из 6 цифр, которые могли бы быть паролем Монокарпа (то есть которые подходят под все описанные условия).\\
						\vspace{15pt}
						{\large Решение:}\\
						Так как Монокарп помнит количество цифр, которых точно не было в пароле, тогда количество цифр, которые возможно были в пароле равно 
						\begin{equation}
							(10 - n )
						\end{equation}
						
						Возьмем простейший случай, когда всего два возможных претендента на числа в пароле - 1 и 0.\\
						Найдем, сколькими способами мы можем выбрать 3 позиции из 6 возможных:\\
						\begin{equation}
							С_6^3 = \frac{6!}{3!\times3!} = 20
						\end{equation}
						\newpage
						\vspace{15pt} 
						Вернемся к случаю, когда у нас \(10 - n\) возможных чисел. Найдем сколько можно составить пар из этих чисел.
						\begin{equation}
							С_{10-n}^2 = \frac{(10-n)!}{(8-n)!\times2!} = \frac{(10-n)!\times(9-n)!}{2}
						\end{equation}
						
						Тогда решением задачи будет формула:
								\begin{equation}
									С_6^3 \times С_{10-n}^2 = 20 \times \frac{(10-n)!\times(9-n)!}{2} = 10 \times {(10-n)!\times(9-n)!}
								\end{equation}
								
						Ответ: $10 \times {(10-n)!\times(9-n)!}$
						
						 \item {\large задача  }\\
						Источник: C.Якунин сайт kompege.ru\\
						\vspace{15pt}
						Полина составляет 21-буквенные слова из букв слова РЕКОГНОСЦИРОВКА. Каждая гласная в них используется столько раз, сколько в слове \\РЕКОГНОСЦИРОВКА. Каждая согласная может использоваться сколько угодно раз или не использоваться совсем. Сколько слов может составить Полина, если известно, что сумма порядковых номеров гласных букв, в каждом из них, равна 21? Буквы нумеруются слева направо, начиная с единицы.\\
						\vspace{15pt}
						Решение:\\
						Подсчёт конфигураций:
						21 нам даёт единственный набор, состоящий из 6 неповторяющихся гласных: 1 + 2 + 3 + 4 + 5 + 6. Гласные, которые могут стоять на этих местах: 3 буквы О, одна буква А, одна буква И и одна буква Е.  Так как неповторяющихся букв у нас всего 3, значит нам нужно выбрать только для них позиции, значит всего:
						\begin{equation}
						 \frac{6!}{3!} = 4 \times5 \times6 = 120 
						\end{equation}
						Размещение согласных:				
						Любая согласная может занимать одну из следующих 15 позиций. Имеем 15 перемноженных семёрок (размещения с повторениями) или $7^{15}$ = 4747561509943.
						
						\newpage						
						
						Итоговый подсчёт слов:
						Итак, в каждой из 120 конфигураций есть 4747561509943 вариантов. Значит, всего слов: 
						\begin{equation}
							120 \times 4747561509943 = 569707381193160 
						\end{equation}
						
						Ответ: 569707381193160
						
						
						
						
						
						
						
						
						
						
						
						
						
						
						
						\item {\large задача  }\\
						Источник: Дискретная математика. Учебник и задачник для Вузов (Баврин И.И. )\\
						\vspace{15pt}
						Нужно присудить первую, вторую и третью премии на конкурсе, в котором принимают  участие 20 человек . Сколькими способами можно распредилить эти премии?
						\vspace{15pt}
						Решение:\\
						Ответом на данную задачу будут являться количество размещений по 3 человека из 20:
						\begin{equation}
							A_{20}^3 = (20)\times(20-1)\times(20-2) = 20\times 19 \times 18 = 6840
						\end{equation}
						
						Ответ: 6840
						
						
						 
						 \item {\large задача }\\
						 Источник: Дагестанский государственный университет народного хозяйства учебное пособие по дисциплине "математика"  \\
						\vspace{15pt}
						 В театре 10 актеров и 8 актрис. Сколькими способами можно
						 распределить между ними роли в пьесе, в которой 5 мужских и 3
						 женские роли?\\
						 \vspace{15pt}
						 \newpage
						 Решение:\\
						 Так как нам важен порядок при распределении ролей, то нам нужны размещения, тогда 
						 Ответом на данную задачу будут являться произведение размещения из 10 элементов по 5 и размещения из 8 элементов по 3:
						 \begin{equation}
						 	A_{10}^5\times A_{8}^3 = \frac{10!}{5!}\times\frac{8!}{5!}= 10160640
						 \end{equation}
						 
						 Ответ: 10160640
						 
						 \item {\large задача  }\\
						 Источник: Дискретная математика. Учебник и задачник для Вузов (Баврин И.И. )\\
						 \vspace{15pt}
						 Восемь лабораторных животных нужно проранжировать в соответствии с их способностями выполнять определенные задания. Каково число возможных ранжировок, если допустить, что одинаковых способностей нет?\\
						 \vspace{15pt}
						 Решение:\\
						 Так как одинаковых способностей нет, то тогда в комбинациях не будет повторений. 
						 Ответом на данную задачу будут являться количество перестановок из 8 лабораторных животных:
						 \begin{equation}
						 	P_{8} = 8! = 40320
						 \end{equation}
						 
						 Ответ: 40320
						 
						 \item {\large задача  }\\
						 Источник: Дискретная математика. Учебник и задачник для Вузов (Баврин И.И. )\\
						 \vspace{15pt}
						 Комитет состоит из 12 человек. Минимальный кворум на заседаниях этого комитета должен насчитывать восемь членов. Сколькими способами может достигаться минимальный кворум?\\
						 \vspace{15pt}
						 Решение:\\
						 Ответом на данную задачу будут являться количество сочетаний по 8 человека из 12:
						 \begin{equation}
						 	C_{12}^8 =\frac{12!}{8!\times4!} = 495
						 \end{equation}
						 
						 Ответ: 495
						 
						 \item {\large задача  }\\
						 Источник: Дискретная математика. Учебник и задачник для Вузов (Баврин И.И. )\\
						 \vspace{15pt}
						 В лабораторной клетке содержатся 8 белых и 6 коричневых мышей. Найдите число способов выбора пяти мышей из клетки, если они могут быть любого цвета.\\
						 \vspace{15pt}
						 Решение:\\
						 Найдем кол-во мышей в общей сложности $8+6 = 14$, тогда
						 ответом на данную задачу будут являться количество сочетаний по 5 мышей из 14 возможных:
						 \begin{equation}
						 	C_{14}^5 = \frac{14!}{9!\times5!} = 2002
						 \end{equation}
						 
						 Ответ: 2002
				\end{enumerate}
			\end{center}
			%«Дискретная математика»\\
			
			%\vfill



	
	




\end{document}