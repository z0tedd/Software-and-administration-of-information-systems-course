\documentclass[a4paper,14pt]{extreport} % добавить leqno в [] для нумерации слева
%\usepackage[14pt]{extsizes}
\usepackage[left=3cm,right=1.5cm,
top=2cm,bottom=2cm,bindingoffset=0cm]{geometry}
\linespread{1.45} %полуторный интервал
\usepackage{titlesec}
%%% Работа с русским языком
\titleformat{\section}{\normalfont\bfseries}{\thechapter}{14pt}{\bfseries}
\usepackage{cmap}					% поиск в PDF
\usepackage{mathtext} 				% русские буквы в фомулах
\usepackage[T2A]{fontenc}			% кодировка
\usepackage[utf8]{inputenc}			% кодировка исходного текста
\usepackage[english,russian]{babel}	% локализация и переносы
\usepackage{ fancyhdr} % улучшенная нумерация страниц
%%% Дополнительная работа с математикой
\usepackage{amsmath,amsfonts,amssymb,amsthm,mathtools} % AMS
\usepackage{icomma} % "Умная" запятая: $0,2$ --- число, $0, 2$ --- перечисление
%\renewcommand{\sfdefault}{cmss}
\usefont{T2A}{cmss}{m}{n}
%% Номера формул
\mathtoolsset{showonlyrefs=true} % Показывать номера только у тех формул, на которые есть \eqref{} в тексте.

%% Шрифты
\usepackage{euscript}	 % Шрифт Евклид
\usepackage{mathrsfs} % Красивый матшрифт
%% Свои команды
\DeclareMathOperator{\sgn}{\mathop{sgn}}

%% Перенос знаков в формулах (по Львовскому)
\newcommand*{\hm}[1]{#1\nobreak\discretionary{}
	{\hbox{$\mathsurround=0pt #1$}}{}}

%\renewcommand{\sfdefault}{cmss}
\usepackage{tempora}
%%% Заголовок
\title{100 Задач по теме "Комбинаторика}
\author{Антон Файтельсон}
\date{}

\begin{document}
			\begin{flushleft}
				\Large 100 Задач по теме "Комбинаторика"		\\
				Выполнил студент 113 группы Файтельсон Антон
				
			\end{flushleft}
			\begin{center}
				\begin{enumerate}
					\item {\large Задача 1 Пароль}\\
						Источник: ICPC 2022-2023 NERC (NEERC), квалификационный этап Чемпионата Юга и Поволжья России \\
						\vspace{15pt}
						Монокарп забыл пароль от своего телефона. Пароль состоит ровно из 6 цифр от 0 до 9 (обратите внимание, что пароль может начинаться с цифры 0).\\
						\vspace{15pt}
						Монокарп помнит, что в его пароле были ровно две различные цифры, причем каждая из этих цифр встречалась в пароле ровно по три раза. Также Монокарп помнит количество цифр (n), которых точно не было в его пароле . \\
						\vspace{15pt}
						Посчитайте количество последовательностей из 6 цифр, которые могли бы быть паролем Монокарпа (то есть которые подходят под все описанные условия).\\
						\vspace{15pt}
						{\large Решение:}\\
						Так как Монокарп помнит количество цифр, которых точно не было в пароле, тогда количество цифр, которые возможно были в пароле равно 
						\begin{equation}
							(10 - n )
						\end{equation}
						
						Возьмем простейший случай, когда всего два возможных претендента на числа в пароле - 1 и 0.\\
						Найдем, сколькими способами мы можем выбрать 3 позиции из 6 возможных:\\
						\begin{equation}
							С_6^3 = \frac{6!}{3!\times3!} = 20
						\end{equation}
						\newpage
						\vspace{15pt} 
						Вернемся к случаю, когда у нас \(10 - n\) возможных чисел. Найдем сколько можно составить пар из этих чисел.
						\begin{equation}
							С_{10-n}^2 = \frac{(10-n)!}{(8-n)!\times2!} = \frac{(10-n)!\times(9-n)!}{2}
						\end{equation}
						
						Тогда решением задачи будет формула:
								\begin{equation}
									С_6^3 \times С_{10-n}^2 = 20 \times \frac{(10-n)!\times(9-n)!}{2} = 10 \times {(10-n)!\times(9-n)!}
								\end{equation}
								
						Ответ: $10 \times {(10-n)!\times(9-n)!}$
						
						
						\item {\large Задача 2 }\\
						Источник: Дискретная математика. Учебник и задачник для Вузов (Баврин И.И. )\\
						Нужно присудить первую, вторую и третью премии на конкурсе, в котором принимают  участие 20 человек . Сколькими способами можно распредилить эти премии?
						\vspace{15pt}
						Решение:\\
						Ответом на данную задачу будут являться количество размещений по 3 человека из 20:
						\begin{equation}
							A_{20}^3 = (20)\times(20-1)\times(20-2) = 20\times 19 \times 18 = 6840
						\end{equation}
						
						Ответ: 6840
						
						\item {\large Задача 2 }\\
						Источник: Дискретная математика. Учебник и задачник для Вузов (Баврин И.И. )\\
						Нужно присудить первую, вторую и третью премии на конкурсе, в котором принимают  участие 20 человек . Сколькими способами можно распредилить эти премии?
						\vspace{15pt}
						Решение:\\
						Ответом на данную задачу будут являться количество размещений по 3 человека из 20:
						\begin{equation}
							A_{20}^3 = (20)\times(20-1)\times(20-2) = 20\times 19 \times 18 = 6840
						\end{equation}
						
						Ответ: 6840
						 
				\end{enumerate}
			\end{center}
			%«Дискретная математика»\\
			
			%\vfill



	
	

\setcounter{page}{2} 
\newpage
fff
\end{document}