\documentclass[a4paper,14pt]{extreport} % добавить leqno в [] для нумерации слева
%\usepackage[14pt]{extsizes}
\usepackage[left=3cm,right=1.5cm,
top=2cm,bottom=2cm,bindingoffset=0cm]{geometry}
\linespread{1.45} %полуторный интервал
\usepackage{titlesec}
%%% Работа с русским языком
\titleformat{\section}{\normalfont\bfseries}{\thechapter}{14pt}{\bfseries}
\usepackage{cmap}					% поиск в PDF
\usepackage{mathtext} 				% русские буквы в фомулах
\usepackage[T2A]{fontenc}			% кодировка
\usepackage[utf8]{inputenc}			% кодировка исходного текста
\usepackage[english,russian]{babel}	% локализация и переносы
\usepackage{ fancyhdr} % улучшенная нумерация страниц
%%% Дополнительная работа с математикой
\usepackage{amsmath,amsfonts,amssymb,amsthm,mathtools} % AMS
\usepackage{icomma} % "Умная" запятая: $0,2$ --- число, $0, 2$ --- перечисление
%\renewcommand{\sfdefault}{cmss}
\usefont{T2A}{cmss}{m}{n}
%% Номера формул
\mathtoolsset{showonlyrefs=true} % Показывать номера только у тех формул, на которые есть \eqref{} в тексте.

%% Шрифты
\usepackage{euscript}	 % Шрифт Евклид
\usepackage{mathrsfs} % Красивый матшрифт
%% Свои команды
\DeclareMathOperator{\sgn}{\mathop{sgn}}

%% Перенос знаков в формулах (по Львовскому)
\newcommand*{\hm}[1]{#1\nobreak\discretionary{}
	{\hbox{$\mathsurround=0pt #1$}}{}}

%\renewcommand{\sfdefault}{cmss}
\usepackage{tempora}
%%% Заголовок
\title{100 Задач по теме "Комбинаторика}
\author{Антон Файтельсон}
\date{}

\begin{document}
			\begin{flushleft}
				\Large 100 Задач по теме "Комбинаторика"		\\
				Выполнил студент 113 группы Файтельсон Антон
				
			\end{flushleft}
			\begin{center}
				\begin{enumerate}
					\item {\large задача  Пароль}\\
						Источник: ICPC 2022-2023 NERC (NEERC), квалификационный этап Чемпионата Юга и Поволжья России \\
						\vspace{15pt}
						Монокарп забыл пароль от своего телефона. Пароль состоит ровно из 6 цифр от 0 до 9 (обратите внимание, что пароль может начинаться с цифры 0).\\
						\vspace{15pt}
						Монокарп помнит, что в его пароле были ровно две различные цифры, причем каждая из этих цифр встречалась в пароле ровно по три раза. Также Монокарп помнит количество цифр (n), которых точно не было в его пароле . \\
						\vspace{15pt}
						Посчитайте количество последовательностей из 6 цифр, которые могли бы быть паролем Монокарпа (то есть которые подходят под все описанные условия).\\
						\vspace{15pt}
						{\large Решение:}\\
						Так как Монокарп помнит количество цифр, которых точно не было в пароле, тогда количество цифр, которые возможно были в пароле равно 
						\begin{equation}
							(10 - n )
						\end{equation}
						
						Возьмем простейший случай, когда всего два возможных претендента на числа в пароле - 1 и 0.\\
						Найдем, сколькими способами мы можем выбрать 3 позиции из 6 возможных:\\
						\begin{equation}
							С_6^3 = \frac{6!}{3!\times3!} = 20
						\end{equation}
						\newpage
						\vspace{15pt} 
						Вернемся к случаю, когда у нас \(10 - n\) возможных чисел. Найдем сколько можно составить пар из этих чисел.
						\begin{equation}
							С_{10-n}^2 = \frac{(10-n)!}{(8-n)!\times2!} = \frac{(10-n)!\times(9-n)!}{2}
						\end{equation}
						
						Тогда решением задачи будет формула:
								\begin{equation}
									С_6^3 \times С_{10-n}^2 = 20 \times \frac{(10-n)!\times(9-n)!}{2} = 10 \times {(10-n)!\times(9-n)!}
								\end{equation}
								
						Ответ: $10 \times {(10-n)!\times(9-n)!}$
						
						
						
						
						 \item {\large задача  }\\
						Источник: C.Якунин сайт kompege.ru\\
						\vspace{15pt}
						Полина составляет 21-буквенные слова из букв слова РЕКОГНОСЦИРОВКА. Каждая гласная в них используется столько раз, сколько в слове \\РЕКОГНОСЦИРОВКА. Каждая согласная может использоваться сколько угодно раз или не использоваться совсем. Сколько слов может составить Полина, если известно, что сумма порядковых номеров гласных букв, в каждом из них, равна 21? Буквы нумеруются слева направо, начиная с единицы.\\
						\vspace{15pt}
						Решение:\\
						Подсчёт конфигураций:
						21 нам даёт единственный набор, состоящий из 6 неповторяющихся гласных: 1 + 2 + 3 + 4 + 5 + 6. Гласные, которые могут стоять на этих местах: 3 буквы О, одна буква А, одна буква И и одна буква Е.  Так как неповторяющихся букв у нас всего 3, значит нам нужно выбрать только для них позиции, значит всего:
						\begin{equation}
						 \frac{6!}{3!} = 4 \times5 \times6 = 120 
						\end{equation}
						Размещение согласных:				
						Любая согласная может занимать одну из следующих 15 позиций. Имеем 15 перемноженных семёрок (размещения с повторениями) или $7^{15}$ = 4747561509943.
						
						\newpage						
						
						Итоговый подсчёт слов:
						Итак, в каждой из 120 конфигураций есть 4747561509943 вариантов. Значит, всего слов: 
						\begin{equation}
							120 \times 4747561509943 = 569707381193160 
						\end{equation}
						
						Ответ: 569707381193160
						
						\item {\large задача  }\\
						Источник:  Дискретная математика. Учебник и задачник для Вузов (Баврин И.И. )\\
						\vspace{15pt}
						Пусть имеется n сортов мономеров (например, азотистых оснований). Из этих мономеров образуется полимер, который можно представить как цепочку из k мономеров. При этом k, как правило, больше n, и мономеры в цепочке могут повторяться.
						Какое количество различных полимеров длины k можно образовать из данных n сортов мономеров?
						\\
						\vspace{15pt}
						Решение:\\
						Будем считать набор мономеров алфавитом из n элементов. Тогда каждый полимер, состоящий из k мономеров, есть слово длины n. Число таких слов, как известно, равно $n^k$, а число различных полимеров будет в два раза меньше, так как, например, молекулы $a_1a_2a_3$ и $a_3a_2a_1$ не различаются (одна из них превращается в другую, если ее повернуть на $180^{\circ}$).
						
						В частности, если алфавит состоит из четырех азотистых оснований А, Ц, Г и Т (т. е. n = 4), а полимером является ген (средняя длина гена равна 1000 единиц, т. е. k = 1000), то число всевозможных генов, которые можно получить из четырех оснований, равно
						\begin{equation}
							\frac{1}{2}n^k=\frac{1}{2}4^{1000} = \frac{2^{2000}}{2} = 2^{1999}
						\end{equation}
						Это громадное число. По некоторым подсчетам, оно превосходит общее число атомов в Солнечной системе\\
						Ответ: $\frac{1}{2}n^k$
						
						 \item {\large задача  }\\
						Источник:  Дискретная математика. Учебник и задачник для Вузов (Баврин И.И. )\\
						\vspace{15pt}
						Хорошо известно, что хромосому схематично можно представить как цепочку из генов. При этом свойства хромосомы зависят не только от состава генов, но и от их расположения в цепочке. Существуют методы, позволяющие изменить порядок генов в хромосоме. Возникает вопрос: какое количество хромосом можно получить из данной, изменяя в ней порядок следования генов?\\
						\vspace{15pt}
						Решение:\\
						Пусть исходная хромосома состоит из n генов. Обозначим их $a_1, a_2,\ldots, a_n$, и пусть  $А = \{a_1, a_2,\ldots, a_n\}$. Тогда понятно, что каждая хромосома, имеющая данный набор генов, есть перестановка множества А. Число таких перестановок, как известно, равно $n!$.
						
						Ответ: $n!$
						
						
%						 \item {\large задача  }\\
%						Источник:  Дискретная математика. Учебник и задачник для Вузов (Баврин И.И. )\\
%						\vspace{15pt}
%						\\
%						\vspace{15pt}
%						Решение:\\
%						
%						\begin{equation}
%							 
%						\end{equation}
%						Ответ: 
					\item {\large задача  }\\
					Источник: Джеймс Андерсон: Дискретная математика и комбинаторика\\
					\vspace{15pt}
					"Фулл хаус" содержит три карты одного ранга и две карты другого
					ранга. Например, расклад, содержащий три короля и две шестерки представляет
					собой "фулл хаус". Сколько существует 5-карточных раскладов с "фулл хаус"? \\
					\vspace{15pt}
					Решение:\\
					Предположим, что "фулл хаус" составили три короля и две шестерки. Три короля
					выбираются из четырех, поэтому существуют $C_4^3 = 4$  способа выбрать трех
					королей. Две шестерки выбираются из четырех, поэтому существуют $C_4^2 = 6$
					способов выбрать две шестерки. Поэтому, согласно комбинаторному принципу
					умножения существуют $4 \times 6 = 24$ способа выбрать трех королей и две шестерки
					или три карты одного ранга и две карты другого ранга. Существуют 13 способов
					выбрать три карты одного ранга и 12 способов выбрать две карты одного ранга.
					Поэтому существуют $13 \times 12 = 156$ различных способов сочетания рангов. Следовательно, существуют $156 \times 24 = 3744$ возможных 5-карточных раскладов с "фулл хаус".
					
					Ответ: 3744
					
					
						 \item {\large задача }\\
						Источник: Дагестанский государственный университет народного хозяйства учебное пособие по дисциплине "математика"  \\
						\vspace{15pt}
						Cколько существует вариантов опроса 11 учащихся на
						одном занятии, если ни одни из них не будет подвергнут опросу
						дважды и на занятии может быть опрошено любое количество
						учащихся, причем порядок, в котором опрашиваются учащиеся,
						безразличен?\\
						\vspace{15pt}
						
						Решение:\\
						Имеется генеральная совокупность объема 11 учащихся.
						Преподаватель может не опросить ни одного из 11 учащихся, что
						является одним из вариантов. Этому случаю соответствует $C_{11}^0$.
						Преподаватель может опросить только одного из учащихся, таких
						вариантов $C_{11}^1$
						Если преподаватель опросит двух учащихся, то число
						вариантов опроса $C_{11}^2$ и т. д.
						Наконец, могут быть опрошены все учащиеся. Число
						вариантов в этом случае $C_{11}^{11}$
						
						Число всех возможных вариантов опроса можно найти по правилу
						сложения:
						\begin{equation}
							C_{11}^0 + C_{11}^1 + C_{11}^2 + \ldots  + C_{11}^11 = 2^{11} = 2048
						\end{equation}
						
						Ответ: 2048
						
						\item {\large задача  }\\
						Источник: Джеймс Андерсон: Дискретная математика и комбинаторика\\
						\vspace{15pt}
						Десять команд участвуют в розыгрыше первенства по
						футболу, лучшие из которых занимают 1-е, 2-е и 3-е место. Две
						команды, занявшие последние места, не будут участвовать в
						следующем таком же первенстве. Сколько разных вариантов
						результата первенства может быть, если учитывать только положение
						первых трех и последних двух команд.\\
						
					
						\vspace{15pt}
						Решение:\\
						 Имеется генеральная совокупность объема 10
						команд. Из нее будем выбирать 5 команд в 2 этапа:
						1) сначала на первые 3 места из 10 с учетом состава и порядка
						команд;
						2) затем на последние 2 места из оставшихся 7 с учетом только
						состава (порядок выбывших команд не важен). Первые 3 места могут
						быть распределены $A_{10}^3 = \frac{10!}{7!}= 720$ способами.						
						Число способов исключить 2 команды из оставшихся 7 равно $С_{7}^2 = \frac{7!}{2!\times5!}= 21$.
						Согласно правилу умножения получаем, что число разных результатов неравенства равно:
						\begin{equation}
							С_{7}^2 \times A_{10}^3 = 15120
						\end{equation}
						
						Ответ: 15120
						
						
						
						\item {\large задача  }\\
						Источник: Задачи по комбинаторике, Шварц Д. А.\\
						\vspace{15pt}
						У бедного студента осталось гречки на две порции, риса на три порции и макарон на две порции. Сколько у студента способов съесть это на завтраки в течение недели (по одной порции в день)?\\
						\vspace{15pt}
						Решение:\\
						Если студент разложит имеющиеся 7 порций еды по разным тарелкам, то количество вариантов выбора будет составлять $P_7 = 7!$, но поскольку разные порции, например, риса не отличимы, то общее количество необходимо разделить на количество перестановок из 2 элементов $P_2 = 2!$(число способов упорядочить 2 тарелки с гречкой), затем на $P_3 = 3!$ (число способов упорядочить 3 тарелки с рисом), затем на $P_2 = =2!$ (макароны). Итого получим кол-во способов завтракать в течение недели для нашего бедного студента.
						\begin{equation}
							 \frac{7!}{2!\times3!\times2!} = 210
						\end{equation}
						
						Ответ: 210
						
						
						
						 \item {\large задача }\\
						Источник: Дагестанский государственный университет народного хозяйства учебное пособие по дисциплине "математика"  \\
						\vspace{15pt}
						При игре в домино 4 игрока делят поровну 28 костей.
						Сколькими способами они могут это сделать?\\
						\vspace{15pt}
						
						Решение:\\
							Так как у нас всего 28 костей, то каждому игроку необходимо дать $28 \div 4 $ = 7 костей.
							Представим, что кости данные игроку 1 получают некоторое св-во и становяться одинаковыми. Аналогично с другими игроками.
							Тогда кол-во способов поделить кости будет равно
						\begin{equation}
							\overline{P_{7,7,7,7}}=\frac{28!}{7!\times7!\times7!\times7!} = 472518347558400
						\end{equation}
						Данное упрощение имеет смысл так, как нас не волнует порядок костей, которые мы дали игрокам, тогда для упрощения вычисления их можно считать равными
						Ответ: 472518347558400
						
						
						
						
						
						
						
						
						
						
						
						
						
						\item {\large задача  }\\
						Источник: Джеймс Андерсон: Дискретная математика и комбинаторика\\
						\vspace{15pt}
						Сколькими способами можно расположить для фотографирования
						пять мальчиков и пять девочек, если ни две девочки, ни два мальчика не должны
						стоять рядом? \\
						\vspace{15pt}
						Решение:\\
						В данной ситуации первым в ряду может быть либо мальчик,
						либо девочка. Если первой стоит девочка, то ряд имеет вид ДМДМДМДМДМ.
						Имеются 5! способов расставить девочек на позициях Д и 5! способов расставить
						мальчиков на позициях М. Поэтому, существуют $5! \times 5!$ способов расположить
						детей в ряд, если первой стоит девочка. Аналогично, существуют $5! х 5!$ способов
						расположить детей в ряд, если первым стоит мальчик. Таким образом, имеются
						$2 х 5! х 5! = 28800$ способов расположить детей в ряд для фотографирования.
					
						Ответ: 28800
						
						 \item {\large задача  }\\
						Источник: Джеймс Андерсон: Дискретная математика и комбинаторика\\
						\vspace{15pt}
						Сколькими способами можно рассадить 10 человек за круглым
						столом, если имеет значение только порядок соседей.\\
						\vspace{15pt}
						Решение:\\
						Предложим, что место за столом уникально. Посадим одного человека на это место
						 и будем считать его место опорным элементом,
						тогда следующего человека(соседа опорного элемента) можно выбрать $10-1=9$ способами, 
						находящегося рядом с соседом человека можно выбрать $9-1=8$ способами, тогда общее кол-во способов рассадки равно:
						\begin{equation}
							P_{9} = P! = 362880
						\end{equation}
						
						Ответ: 362880
						
						
						
						
						
						
						
						
						
						 \item {\large задача  }\\
						Источник: Джеймс Андерсон: Дискретная математика и комбинаторика\\
						\vspace{15pt}
						Сколько различных четырехзначных чисел можно образовать из
						цифр 1, 2, 3, \ldots, 9, если все цифры в каждом четырехзначном числе различны?\\
						\vspace{15pt}
						Решение:\\
						Для формирования каждого четырехзначного числа выбираем четыре цифры из
						девяти, поэтому всего таких чисел существует:
						\begin{equation}
							A_{9}^4 = \frac{9!}{(9-4)!} = 3024 
						\end{equation}
						
						Ответ: 3024
							
						\item {\large задача  }\\
						Источник: Дискретная математика. Учебник и задачник для Вузов (Баврин И.И. )\\
						\vspace{15pt}
						Нужно присудить первую, вторую и третью премии на конкурсе, в котором принимают  участие 20 человек . Сколькими способами можно распредилить эти премии?\\
						\vspace{15pt}
						Решение:\\
						Ответом на данную задачу будут являться количество размещений по 3 человека из 20:
						\begin{equation}
							A_{20}^3 = (20)\times(20-1)\times(20-2) = 20\times 19 \times 18 = 6840
						\end{equation}
						
						Ответ: 6840
						
						
						 
						 \item {\large задача }\\
						 Источник: Дагестанский государственный университет народного хозяйства учебное пособие по дисциплине "математика"  \\
						\vspace{15pt}
						 В театре 10 актеров и 8 актрис. Сколькими способами можно
						 распределить между ними роли в пьесе, в которой 5 мужских и 3
						 женские роли?\\
						 \vspace{15pt}
					
						 Решение:\\
						 Так как нам важен порядок при распределении ролей, то нам нужны размещения, тогда 
						 Ответом на данную задачу будут являться произведение размещения из 10 элементов по 5 и размещения из 8 элементов по 3:
						 \begin{equation}
						 	A_{10}^5\times A_{8}^3 = \frac{10!}{5!}\times\frac{8!}{5!}= 10160640
						 \end{equation}
						 
						 Ответ: 10160640
						 
						 \item {\large задача  }\\
						 Источник: Дискретная математика. Учебник и задачник для Вузов (Баврин И.И. )\\
						 \vspace{15pt}
						 Восемь лабораторных животных нужно проранжировать в соответствии с их способностями выполнять определенные задания. Каково число возможных ранжировок, если допустить, что одинаковых способностей нет?\\
						 \vspace{15pt}
						 Решение:\\
						 Так как одинаковых способностей нет, то тогда в комбинациях не будет повторений. 
						 Ответом на данную задачу будут являться количество перестановок из 8 лабораторных животных:
						 \begin{equation}
						 	P_{8} = 8! = 40320
						 \end{equation}
						 
						 Ответ: 40320
						 
						 \item {\large задача  }\\
						 Источник: Дискретная математика. Учебник и задачник для Вузов (Баврин И.И. )\\
						 \vspace{15pt}
						 Комитет состоит из 12 человек. Минимальный кворум на заседаниях этого комитета должен насчитывать восемь членов. Сколькими способами может достигаться минимальный кворум?\\
						 \vspace{15pt}
						 Решение:\\
						 Ответом на данную задачу будут являться количество сочетаний по 8 человека из 12:
						 \begin{equation}
						 	C_{12}^8 =\frac{12!}{8!\times4!} = 495
						 \end{equation}
						 
						 Ответ: 495
						 
						 \item {\large задача  }\\
						 Источник: Дискретная математика. Учебник и задачник для Вузов (Баврин И.И. )\\
						 \vspace{15pt}
						 В лабораторной клетке содержатся 8 белых и 6 коричневых мышей. Найдите число способов выбора пяти мышей из клетки, если они могут быть любого цвета.\\
						 \vspace{15pt}
						 Решение:\\
						 Найдем кол-во мышей в общей сложности $8+6 = 14$, тогда
						 ответом на данную задачу будут являться количество сочетаний по 5 мышей из 14 возможных:
						 \begin{equation}
						 	C_{14}^5 = \frac{14!}{9!\times5!} = 2002
						 \end{equation}
						 
						 Ответ: 2002
						 
						  \item {\large задача  }\\
						 Источник: Джеймс Андерсон: Дискретная математика и комбинаторика\\
						 \vspace{15pt}
						 Сколько существует вариантов выбора 5 карт из стандартной ко-
						 колоды, содержащей 52 карты?\\
						 \vspace{15pt}
						 Решение:\\
						 Поскольку порядок карт не имеет значения, речь
						 идет о выборе 5 объектов из 52, поэтому существует всего комбинация:
						 \begin{equation}
						 	C_{52}^5 = \frac{52!}{5!\times47!} = 2598960
						 \end{equation}
						 
						 Ответ: 2598960
						 
						 \item {\large задача  }\\
						 Источник: Джеймс Андерсон: Дискретная математика и комбинаторика\\
						 \vspace{15pt}
						 Сколькими способами можно вытянуть 5 карт трефовой масти из
						 стандартной колоды, содержащей 52 карты?\\
						 \vspace{15pt}
						 Решение:\\
						 В колоде имеется 13 треф, из которых выбираются 5, поэтому существует всего:
						 \begin{equation}
						 	C_{13}^5 = \frac{13!}{8!\times5!} = 1287
						 \end{equation}
						 
						 Ответ: 1287
						 
						  \item {\large задача  }\\
						 Источник: Джеймс Андерсон: Дискретная математика и комбинаторика\\
						 \vspace{15pt}
						 Сколькими способами можно выбрать комитет, включающий 6
						 мужчин и 8 женщин, из группы, состоящей из 12 мужчин и 20 женщин?
						 
						 19!\\
						 \vspace{15pt}
						 Решение:\\
						 Найдем кол-во способов выбора мужчин и кол-во способов выбора женщин
						 
						 \begin{align}
						 	C_{12}^6 &= \frac{12!}{6!\times6!}\\
						 	C_{20}^8 &= \frac{20!}{8!\times12!}  
						 \end{align}
						 
						
						 
					
						 Поэтому, согласно комбинаторному принципу умножения, найдем кол-во способов выбрать комитет:
						 \begin{equation}
						 	C_{20}^8 \times C_{12}^6 = 116396280
						 \end{equation}
						 Ответ: 116396280
				\end{enumerate}
			\end{center}
			%«Дискретная математика»\\
			
			%\vfill



	
	




\end{document}