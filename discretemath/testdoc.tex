% Options for packages loaded elsewhere
\PassOptionsToPackage{unicode}{hyperref}
\PassOptionsToPackage{hyphens}{url}
%
\documentclass[
]{article}
\usepackage{amsmath,amssymb,amsthm}
\usepackage{iftex}

\usepackage{titlesec}
\usepackage{microtype}
\usepackage{cmap}\usepackage{mathtext}
\usepackage[T2A]{fontenc}
\usepackage[utf8]{inputenc}\usepackage[english,russian]{babel}\usepackage{ fancyhdr}\usepackage{amsmath,amsfonts,amssymb,amsthm,mathtools} \usepackage{icomma}\usepackage{euscript}\usepackage{mathrsfs}\usepackage{tempora}
\ifPDFTeX
  \usepackage[T1]{fontenc}
  \usepackage[utf8]{inputenc}
  \usepackage{textcomp} % provide euro and other symbols
\else % if luatex or xetex
  \usepackage{unicode-math} % this also loads fontspec
  \defaultfontfeatures{Scale=MatchLowercase}
  \defaultfontfeatures[\rmfamily]{Ligatures=TeX,Scale=1}
\fi
\usepackage{lmodern}
\ifPDFTeX\else
  % xetex/luatex font selection
\fi
% Use upquote if available, for straight quotes in verbatim environments
\IfFileExists{upquote.sty}{\usepackage{upquote}}{}
\IfFileExists{microtype.sty}{% use microtype if available
  \usepackage[]{microtype}
  \UseMicrotypeSet[protrusion]{basicmath} % disable protrusion for tt fonts
}{}
\makeatletter
\@ifundefined{KOMAClassName}{% if non-KOMA class
  \IfFileExists{parskip.sty}{%
    \usepackage{parskip}
  }{% else
    \setlength{\parindent}{0pt}
    \setlength{\parskip}{6pt plus 2pt minus 1pt}}
}{% if KOMA class
  \KOMAoptions{parskip=half}}
\makeatother
\usepackage{xcolor}
\setlength{\emergencystretch}{3em} % prevent overfull lines
\providecommand{\tightlist}{%
  \setlength{\itemsep}{0pt}\setlength{\parskip}{0pt}}
\setcounter{secnumdepth}{-\maxdimen} % remove section numbering
\ifLuaTeX
  \usepackage{selnolig}  % disable illegal ligatures
\fi
\IfFileExists{bookmark.sty}{\usepackage{bookmark}}{\usepackage{hyperref}}
\IfFileExists{xurl.sty}{\usepackage{xurl}}{} % add URL line breaks if available
\urlstyle{same}
\hypersetup{
  hidelinks,
  pdfcreator={LaTeX via pandoc}}

\author{}
\date{}
\newtheorem{theorem}{Theorem}
\begin{document}

Вот ответы на вопросы для подготовки к экзамену по дискретной
математике:

\begin{enumerate}
\def\labelenumi{\arabic{enumi}.}
\tightlist
\item
  \textbf{Сочетания без повторений и с повторениями}:

  \begin{itemize}
  \tightlist
  \item
    \textbf{Сочетания без повторений}:\\
    \emph{Из \(n\) элементов выбирается \(k\) элементов, порядок не важен и элементы не
    повторяются. } \\Если X -- n-элементное множество, то любое его
    k-элементное подмножество, где n, \(k\in N\) и k\textless=n,
    называется сочетанием из n элементов (множества Х) по k.\\
    \emph{Сочетания неупорядоченные! }\\Число всех сочетаний без
    повторений определяется формулой:
    \(C(n, k) =\)\(\frac{n!}{k!(n-k)!}\).
  \item
    \textbf{Сочетания с повторениями}:\\ 
    Сочетанием с повторениями из
    n-элементов данного множества Х по k элементов, где \(k\in N\)
    называется любое разложение вида:
    \((a_1,a_1,\cdots,a_1;a_2,a_2\cdots,a_2;a_n,a_n,\cdots,a_n)\).\\ Из
    \(n\) элементов выбирается \(k\) элементов, порядок не важен, но
    элементы могут повторяться.\\ Число всех сочетаний с повторениями
    определяется формулой: \(C(n+k-1, k) = \frac{(n+k-1)!}{k!(n-1)!}\).
  \item
    \textbf{Примеры}:

    \begin{itemize}
    \tightlist
    \item
      Без повторений: Сколькими способами можно выбрать 2 студента из 5?
      Ответ:\(C(5, 2) = \frac{5!}{2!(5-2)!} = 10\).
    \item
      С повторениями: Сколькими способами можно выбрать 3 конфеты из 4
      видов? Ответ:\(C(4+3-1, 3) = \frac{6!}{3!3!} = 20\).
    \end{itemize}
  \end{itemize}
\item
  \textbf{Размещения без повторений и с повторениями}:

  \begin{itemize}
  \tightlist
  \item
    \textbf{Размещения без повторений}:\\
    Размещением без повторений из n элементов по k называется
упорядоченное k-элементное подмножество n-элементного множества\\
    Из \(n\) элементов
    выбирается \(k\) элементов, порядок важен и элементы не повторяются.
    Число всех размещений без повторений определяется
    формулой: \\
    \(A(n, k) = \frac{n!}{(n-k)!}\).
  \item
    \textbf{Размещения с повторениями}:\\
    Пусть X – данное n-элементное множество, n$\in$N. Любой кортеж
длины k, где k$\in$N, элементы которого принадлежат множеству X,
называется размещениями с повторениями из n элементов данного
множества.\\
    Из \(n\) элементов
    выбирается \(k\) элементов, порядок важен и элементы могут
    повторяться.\\ Число всех размещений с повторениями определяется
    формулой:\(n^k\).
  \item
    \textbf{Примеры}:

    \begin{itemize}
    \tightlist
    \item
      Без повторений: Сколькими способами можно расставить 3 книги из 5?
      Ответ:\(A(5, 3) = \frac{5!}{(5-3)!} = 60\).
    \item
      С повторениями: Сколькими способами можно создать код из 2
      символов, если доступно 3 различных символа? Ответ:\(3^2 = 9\).
    \end{itemize}
  \end{itemize}
\item
  \textbf{Перестановки без повторений и с повторениями}:

  \begin{itemize}
  \tightlist
  \item
    \textbf{Перестановки без повторений}:\\ 
Расположение различных элементов в определенном порядке
называется перестановкой без повторений из элементов.\\
    Все \(n\) элементов
    располагаются в определенном порядке. Число всех перестановок без
    повторений определяется формулой: \(P(n) = n!\).
  \item
    \textbf{Перестановки с повторениями}:\\ 
    Перестановкой с повторениями из n элементов данного множества с
кортежем кратности $ (r_1,\dots,r_n )$ называется размещение из n 
элементов данного множества по k, где элемент $ a_1 $ повторяется $r_1$ 
раз,элемент $ a_2$ повторяется $r_2$ раз, элемент $ a_3 $ повторяется $r_3$ 
раз. \\ Если среди \(n\) элементов есть
    повторяющиеся элементы, число всех перестановок определяется
    формулой:\\ \(P(n; n_1, n_2, \ldots, n_k) = \frac{n!}{n_1! n_2! \cdots n_k!}\),
    где \(n_1 + n_2 + \cdots + n_k = n\).
  \item
    \textbf{Примеры}:

    \begin{itemize}
    \tightlist
    \item
      Без повторений: Сколькими способами можно расставить 4 человека в
      ряд? Ответ:\(P(4) = 4! = 24\).
    \item
      С повторениями: Сколькими способами можно расставить слово
      ``AAB''? Ответ:\(P(3; 2, 1) = \frac{3!}{2!1!} = 3\).
    \end{itemize}
  \end{itemize}
\item
  \textbf{Основные комбинаторные правила}:

  \begin{itemize}
  \tightlist
  \item
    \textbf{Правило суммы}: Если событие \(A\) может
    произойти \(m\) способами, а событие \(B\)---\(n\) способами, и эти
    события не могут произойти одновременно, то общее число способов
    для \(A\) или \(B\) равно \(m + n\).
  \item
    \textbf{Правило произведения}: Если событие \(A\) может
    произойти \(m\) способами, и после него событие \(B\) может
    произойти \(n\) способами, то общее число способов
    для \(A\) и \(B\) равно \(m \times n\).
  \end{itemize}
\item
  \textbf{Метод включения-исключения}:

  \begin{itemize}
  \tightlist
  \item
    Метод используется для вычисления числа элементов в объединении
    нескольких множеств, корректируя избыточные подсчеты пересечений.
    Формула для трех множеств:
    \begin{equation}\|A \cup B \cup C| = |A| + |B| + |C| - |A \cap B| - |A \cap C| - |B \cap C| + |A \cap B \cap C|\end{equation}.
  \end{itemize}
\item
  \textbf{Определение биномиальных коэффициентов}:

  \begin{itemize}
  \tightlist
  \item
    Биномиальный коэффициент \(C(n, k)\) или \(\binom{n}{k}\) равен числу
    способов выбрать \(k\) элементов из \(n\) и определяется
    формулой: \(\binom{n}{k} = \frac{n!}{k!(n-k)!}\).
    Бином Ньютона:
    \begin{math}
      \begin{aligned}
        &(a+b)^{n}=\sum _{k=0}^{n}{\binom {n}{k}}a^{n-k}b^{k}=\\
        &={\binom {n}{0}}a^{n}+{\binom {n}{1}}a^{n-1}b+\dots +{\binom {n}{k}}a^{n-k}b^{k}+\dots +{\binom {n}{n}}b^{n}
      \end{aligned}
    \end{math}
    

  \end{itemize}
\item
  \textbf{Треугольник Паскаля и его свойства}:

  \begin{itemize}
  \tightlist
  \item
    Треугольник Паскаля --- это таблица биномиальных коэффициентов.
    Каждое число в треугольнике равно сумме двух чисел, расположенных
    над ним. Свойства включают симметрию и свойство начальных строк.
    - Числа треугольника симметричны (равны) относительно вертикальной оси.\\
    - В строке с номером $n$ (нумерация начинается с 0):\\
    - первое и последнее числа равны 1;\\
    - второе и предпоследнее числа равны $n$;\\
    - третье число равно треугольному числу $T_{n-1}=\frac{n(n-1)}{2}$, что также равно сумме номеров предшествующих строк;\\
    - четвёртое число является тетраэдрическим;\\
    - $m$-е число (при нумерации с 0) равно биномиальному коэффициенту $C_n^m=\binom{n}{m}=\frac{n!}{m!(n-m)!}$.\\
    - Сумма чисел восходящей диагонали, начинающейся с первого элемента ( $n$ - 1 )-й строки, есть $n$-е число Фибоначчи:\\
    $$
    \binom{n-1}{0}+\binom{n-2}{1}+\binom{n-3}{2}+\ldots=F_n .
    $$
    - Если вычесть из центрального числа в строке с чётным номером соседнее число из той же строки, то получится число Каталана.\\
    - Сумма чисел $n$-й строки треугольника Паскаля равна $2^n$.\\
    - Все числа в $n$-й строке, кроме единиц, делятся на число $n$ тогда и только тогда, когда $n$ является простым числом ${ }^{[4]}$ (следствие теоремы Люка).\\
    - Если в строке с нечётным номером сложить все числа с порядковыми номерами вида $3 n, 3 n+1,3 n+2$, то первые две суммы будут равны, а третья на 1 меньше.\\
    - Каждое число в треугольнике равно количеству способов добраться до него из вершины, перемещаясь либо вправо-вниз, либо влево-вниз.\\
  \end{itemize}
\item
  \textbf{Основные тождества с биномиальными коэффициентами}:

  \begin{itemize}
  \tightlist
  \item
    Основные тождества включают:

    \begin{itemize}
    \tightlist
    \item
      Симметрия:\(\binom{n}{k} = \binom{n}{n-k}\).
    \item
      Рекуррентное
      соотношение:\(\binom{n}{k} = \binom{n-1}{k-1} + \binom{n-1}{k}\).
    \item 
      \begin{equation*}
      C_{n-1}^{k-1}+C_{n-1}^{k}=C_{n}^{k} 
      \end{equation*}
    \item 
      \begin{equation*}
      C_{n-1}^{k-1}+C_{n-1}^{k} =C_{n}^{k}
      \end{equation*}
    \item
      \begin{equation*}
      C_{n}^{k}=\frac{n}{k} C_{n-1}^{k-1} \tag{1.3}
      \end{equation*}
    \item
      \begin{equation*}
        C_{m+m}^{k}=\sum_{i=0}^{k} C_{n}^{i} \cdot C_{m}^{k-i}
      \end{equation*}
    \item 
      \begin{equation*}
      C_{n}^{k}=C_{n}^{n-k} 
      \end{equation*}
    \end{itemize}
  \end{itemize}
\item
  \textbf{Бином Ньютона. Биномиальные формулы}:

  \begin{itemize}
  \tightlist
  \item
    Бином Ньютона описывает разложение\((a + b)^n\)в сумму:
    \((a + b)^n = \sum_{k=0}^{n} \binom{n}{k} a^{n-k} b^k\).
  \end{itemize}
\item
  \textbf{Свойства бинома Ньютона}:\\
  \begin{tabular}{c}
  $C_n^k=C_{n-1}^{k-1}+C_{n-1}^k$ \\
  \hline$C_n^0+C_n^1+C_n^2+\ldots+C_n^n=2^n$ \\
  \hline$C_n^0-C_n^1+C_n^2-\ldots+(-1)^n C_n^n=0$ \\
  \hline$\left(C_n^0\right)^2+\left(C_n^1\right)^2+\left(C_n^2\right)^2+\ldots+\left(C_n^n\right)^2=C_{2 n}^n$ \\
  \hline
  \end{tabular}
  \begin{itemize}
  \tightlist
  \item
    Основное свойство бинома Ньютона заключается в его применении для
    вычисления коэффициентов в разложении многочленов. Также он
    используется для доказательства тождеств с биномиальными
    коэффициентами.
  \end{itemize}
\item
  \textbf{Булевы функции}:

  \begin{itemize}
  \tightlist
  \item
    Булевы функции --- это функции, которые принимают значения 0 и 1.
    Примеры включают логические операции И (конъюнкция), ИЛИ
    (дизъюнкция), НЕ (отрицание).
  \end{itemize}
\item
  \textbf{ДНФ, КНФ. Определения и примеры}:

  \begin{itemize}
  \tightlist
  \item
    \textbf{Дизъюнктивная нормальная форма (ДНФ)}: Это дизъюнкция
    конъюнкций. Пример:\((A \wedge B) \vee (C \wedge \neg D)\).
  \item
    \textbf{Конъюнктивная нормальная форма (КНФ)}: Это конъюнкция
    дизъюнкций. Пример:\((A \vee B) \wedge (C \vee \neg D)\).
  \end{itemize}
\item
  \textbf{СДНФ, СКНФ. Определения и примеры}:

  \begin{itemize}
  \tightlist
  \item
    \textbf{СДНФ (совершенная дизъюнктивная нормальная форма)}: Это ДНФ,
    в которой каждое слагаемое содержит все переменные.
  \item
    \textbf{СКНФ (совершенная конъюнктивная нормальная форма)}: Это КНФ,
    в которой каждый множитель содержит все переменные.
  \end{itemize}
\item
  \textbf{Многочлены Жегалкина}:

  \begin{itemize}
  \tightlist
  \item
    Многочлен Жегалкина --- это полином, представляющий булеву функцию с
    использованием операции сложения по модулю 2 и умножения.
    Пример:\(f(x, y) = x \oplus y \oplus xy\).
  \end{itemize}
\item
  \textbf{Основные понятия теории графов}:

  \begin{itemize}
  \tightlist
  \item
    Основные понятия включают вершины, ребра, смежность, инцидентность,
    степени вершин.\\
1. Граф: абстрактная структура, состоящая из вершин (узлов) и ребер (связей) между ними.

2. Вершина: элемент графа, обозначающий отдельный узел или объект.

3. Ребро: связь между двумя вершинами графа.

4. Ориентированный граф: граф, в котором каждое ребро имеет направление, указывающее на порядок вершин.

5. Неориентированный граф: граф, в котором ребра не имеют направления.

6. Смежные вершины: вершины, соединенные ребром.

7. Степень вершины: количество ребер, связанных с данной вершиной.

8. Путь: последовательность вершин и ребер, соединяющих эти вершины.
Для орграфов цепь называется путем, а цикл — контуром.

9. Цикл: путь, который начинается и заканчивается в одной и той же вершине.

10. Дерево: связный граф без циклов.

11. Гамильтонов путь: путь, проходящий через каждую вершину графа ровно один раз.

12. Эйлеров путь: путь, проходящий через каждое ребро графа ровно один раз.

13. Матрица смежности: матрица, используемая для представления графа, в которой элементы указывают наличие или отсутствие ребер между вершинами.

14. Матрица инцидентности: матрица, используемая для представления графа, в которой элементы указывают наличие или отсутствие ребер между вершинами и их направление.

15. Подграф: граф, полученный из исходного графа путем удаления некоторых вершин и/или ребер.

Часто рассматриваются следующие родственные графам объекты:
1. Если элементами множества Е являются упорядоченные пары (т. е. Е с Vx V),
то граф называется ориентированным (или орграфом). В этом случае элементы множества V называются узлами, а элементы множества Е — дугами.
2. Если элементом множества Е может быть пара одинаковых (не различных)
элементов V, то такой элемент множества Е называется петлей, а граф называется графом с петлями (или псевдографом).
3. Если Е является не множеством, а мультимножеством, содержащим некоторые элементы по несколько раз, то эти элементы называются кратными
рёбрами, а граф называется мультиграфом.
4. Если элементами множества Е являются не обязательно двухэлементные, а любые (непустые) подмножества множества V, то такие элементы множества Е
называются гипердугами, а граф называется гиперграфом.
5. Если задана функция F : V —» М и/или F : Е —> М, то множество М называется множеством пометок, а граф называется помеченным (или нагруженным).
В качестве множества пометок обычно используются буквы или целые числа.
Если функция F инъективна, то есть разные вершины (рёбра) имеют разные
пометки, то граф называют нумерованным.
  \end{itemize}
\item
  \textbf{Элементы графов}:

  \begin{itemize}
  \tightlist
  \item
    Вершины (узлы), ребра (дуги), петли, кратные ребра, смежные вершины,
    инцидентные вершины и ребра.\\
  \end{itemize}

1. Вершина (узел): Отдельный элемент графа, обозначающий конкретный объект или узел. Вершины могут быть связаны друг с другом ребрами.

2. Ребро (связь): Связь между двумя вершинами графа, указывающая на отношение между этими вершинами. Ребро может быть ориентированным (с указанием направления) или неориентированным.\\
Эти два элемента образуют основу для построения различных структур и алгоритмов в теории графов. В зависимости от конкретной задачи, в графах могут также использоваться дополнительные элементы, такие как веса ребер, метки вершин и другие атрибуты.
После рассмотрения определений, относящихся к графам как к цельным объек-
там, естественно дать определения различным элементам графов.\\ 
7.2.1. Подграфы
Граф G'(V',E') называется подграфом (или частью) графа G(V,E) (обозначает-
ся G' с G), если V' с V к Е' с Е. Если V' = V, то G' называется остовным
подграфом G. Если V' С V к Е' С Е \& (V' ф V V Е' ф Е), то граф G' называет-
ся собственным подграфом графа G. Подграф G'(V',Ef) называется правильным
подграфом графа G(V, Е), если G' содержит все возможные рёбра G:
\/u,veV' ((u,v) е Е (u,v) е Е').
Правильный подграф G'(V', Е') графа G(V, Е) определяется подмножеством вер-
шин V .
ЗАМЕЧАНИЕ
Иногда подграфами называют только правильные подграфы, а неправильные подграфы
называют изграфами.
7.2.2. Валентность
Количество рёбер, инцидентных вершине v, называется степенью (или валент-
ностью) вершины v и обозначается d(v):
Таким образом, степень d{v) вершины v совпадает с количеством смежных с ней
вершин. Количество вершин, не смежных с v, обозначают d(v). Ясно, что
VueV (d{v)+ d{v) = р - 1).
Обозначим минимальную степень вершины графа G через 8(G), а максималь-
ную — через A(G):
8(G(V,E)) =f min d(v), A(G(V,E)) = ma xd(v).
v£V vEV
Ясно, что и Д(G) являются инвариантами. Если степени всех вершин равны
к, то граф называется регулярным степени к:
8(G) = Д(<7) = к, Vu е V (d(v) = к).
Степень регулярности обозначается r(G). Для нерегулярных графов r(G) не опре-
делено.
Примеры
На рис. 7.7 приведена диаграмма регулярного графа степени 3. На рис. 7.6 при-
ведены диаграммы двух регулярных, но неизоморфных графов степени 3.

7.2. Элементы графов 247
Если степень вершины равна нулю (то есть d(v) = 0), то вершина называет-
ся изолированной. Если степень вершины равна единице (то есть d(v) = 1), то
вершина называется концевой, или висячей. Для орграфа число дуг, исходящих
из узла v, называется полустепенью исхода, а число входящих — полустепенью
захода. Обозначаются эти числа, соответственно, d~(v) и d+(v).
Т Е О Р Е М А (Лемма о рукопожатиях) Сумма степеней вершин графа (мультиграфа)
равна удвоенному количеству рёбер:
£ =vev
ДОКАЗАТЕЛЬСТВО При подсчёте суммы степеней вершин каждое ребро учитыва-
ется два раза: для одного конца ребра и для другого. •
С Л Е Д С Т В И Е 1 Число вершин нечётной степени чётно.
ДОКАЗАТЕЛЬСТВО П О теореме сумма степеней всех вершин — чётное число. Сумма
степеней вершин чётной степени чётна, значит, сумма степеней вершин нечётной
степени также чётна, следовательно, их чётное число. •
С Л Е Д С Т В И Е 2 Сумма полустепеней узлов орграфа равна удвоенному количеству
дуг:
ДОКАЗАТЕЛЬСТВО Сумма полустепеней узлов орграфа равна сумме степеней вер-
шин графа (мультиграфа), полученного из орграфа забыванием ориентации
ДУГ. 

\item
  \textbf{Маршруты, цепи, циклы}:

  \begin{itemize}
  \tightlist
  \item
    Маршрут: последовательность смежных ребер.
  \item
    Цепь: маршрут без повторений ребер.
  \item
    Цикл: цепь, начинающаяся и заканчивающаяся в одной и той же вершине.
  \end{itemize}
Маршрутом в графе называется чередующаяся последовательность вершин и
рёбер, начинающаяся и кончающаяся вершиной, vo,e1,v1, е2, v2,..., ejt, Vk, в которой любые два соседних элемента инцидентны, причём однородные элементы
(вершины, рёбра) через один смежны или совпадают.\\ 
Если vq — Vk, то маршрут замкнут, иначе — открыт. Если все рёбра различны,
то маршрут называется цепью. Если все вершины (а значит, и рёбра) различны,
то маршрут называется простой цепыо. В цепи v0,ei,... ,ek,Vk вершины vq И VK
называются концами цепи. Говорят, что цепь с концами uwv соединяет вершины
и и v. Цепь, соединяющая вершины и и г>, обозначается (и, v). Если нужно указать
граф G, которому принадлежит цепь, то добавляют индекс: (u,v)G. Нетрудно
показать, что если есть какая-либо цепь, соединяющая вершины и и v, то есть
и простая цепь, соединяющая эти вершины. Замкнутая цепь называется циклом-,
замкнутая простая цепь называется простым циклом. Число циклов в графе G
обозначается z(G). Граф без циклов называется ациклическим.\\ 

Говорят, что две вершины в графе связаны, если существует соединяющая их
(простая) цепь. Граф, в котором все вершины связаны, называется связным.
Нетрудно показать, что отношение связанности вершин является эквивалентно-
стью. Классы эквивалентности по отношению связанности называются компонен-
тами связности графа. Число компонентов связности графа G обозначается k(G).
Граф G является связным тогда и только тогда, когда k(G) = 1. Если k(G) > 1,
то G — несвязный граф. Граф, состоящий только из изолированных вершин (в
котором k(G) = p(G) и r(G) = 0), называется вполне несвязным.
7.2.5. Расстояние между вершинами, ярусы
и диаметр графа
Длиной маршрута называется количество рёбер в нём (с учётом повторений).
Если маршрут М = vq, e\,v\,e2, v2, • • •, vk, то длина М равна к (обозначается
М — к). Расстоянием между вершинами и wv (обозначается d(u, v)) называется
длина кратчайшей цепи (u,v), а сама кратчайшая цепь называется геодезической,
d(u, г>) min l(n, гЛ1.{<«,«»
Если для любых двух вершин графа существует единственная геодезическая
цепь, то граф называется геодезическим.
З А М Е Ч А Н И Е
Если -<3 ((u,v)), то по определению d(u,v) =f+oo.
Множество вершин, находящихся на заданном расстоянии п от вершины v (обо-
значение D(v,n)), называется ярусом:
D(v, п) {и € V | d(v, и) —п} .
Ясно, что множество вершин V всякого связного графа однозначно разбивает-
ся на ярусы относительно дайной вершины. Диаметром графа G называется
длиннейшая геодезическая. Длина диаметра обозначается D(G):
D(G) max d(u,v).
u,v(EV
7.2.6. Эксцентриситет и центр
Эксцентриситетом e(v) вершины v в связном графе G(V, Е) называется макси-
мальное расстояние от вершины v до других вершин графа G:
е(у) =f maxd(v, и).
uGV
Заметим, что наиболее эксцентричные вершины — это концы диаметра. Радиусом
R(G) графа G называется наименьший из эксцентриситетов вершин:
\item
  \textbf{Виды графов}:

  \begin{itemize}
  \tightlist
  \item
    Ориентированные, неориентированные, взвешенные, невзвешенные,
    планарные, полные графы.
  \end{itemize}
Граф, состоящий из одной вершины, называется тривиальным. Граф, состоящий
из простого цикла с к вершинами, обозначается $С_k$.

Граф, в котором любые две вершины смежны, называется полным.
Полный подграф (некоторого графа) называется кликой (этого графа).

Граф G{V,E) называется двудольным (или биграфом, или чётным графом), если
множество V может быть разбито па два непересекающихся множества V1 и V2
(V1 $\cup$ V2 = V,V1$\cap V2$ = 0), причём всякое ребро из Е инцидентно вершине из V1
и вершине из V2 (то есть соединяет вершину из V1 с вершиной из V2).
V1 и V2 - доли двудольного графа

Если двудольный граф содержит
все рёбра, соединяющие множества V1 и V2, то он называется полным двудольным
графом. Если |V1| = m и |V2| = n, то полный двудольный граф обозначается $K_{m,n}$ 

\begin{theorem}[о двудольном графе]
  Граф является двудольным тогда и только тогда, когда все его простые
циклы имеют чётную длину.
\end{theorem}

Если в графе ориентировать все рёбра, то получится орграф, который называется
направленным, или антисимметричным. Направленный орграф, полученный из
полпого графа, называется турниром.

Если в орграфе полустепень захода некоторого узла равна нулю (то есть d+(v) =
= 0), то такой узел называется источником, если же нулю равна полустепепь
исхода (то есть d~(v) = 0), то узел называется стоком. Направленный слабосвязный (см. 8.5.1) орграф с одним источником и одним стоком называется
сетью.

\item
  \textbf{Операции над графами}:

  \begin{itemize}
  \tightlist
  \item
    Объединение, пересечение, дополнение, декартово произведение графов.
  \end{itemize}
\begin{enumerate}
    \item Дополнением графа $G_1(V_1, E_1)$ (обозначение -- $\overline{G_1(V_1, E_1)}$) называется граф $G_2(V_2, E_2)$, где $V_2 = V_1$ и $E_2 = \overline{E_1} = \{ e \in V_1 \times V_1 \mid e \notin E_1 \} = V \times V \setminus E$.
    
    Пример: $\overline{K_1} = K_1$.
    
    \item Объединением (дизъюнктным) графов $G_1(V_1, E_1)$ и $G_2(V_2, E_2)$ (обозначение -- $G_1(V_1, E_1) \cup G_2(V_2, E_2)$), при условии $V_1 \cap V_2 = \emptyset$, называется граф $G(V, E)$, где $V = V_1 \cup V_2$ и $E = E_1 \cup E_2$.
    
    Пример: $K_{3,3} = C_3 \cup C_3$.
    
    \item Соединением графов $G_1(V_1, E_1)$ и $G_2(V_2, E_2)$ (обозначение -- $G_1(V_1, E_1) + G_2(V_2, E_2)$), при условии $V_1 \cap V_2 = \emptyset$, называется граф $G(V, E)$, где $V = V_1 \cup V_2$ и $E = E_1 \cup E_2 \cup \{ e = (v_1, v_2) \mid v_1 \in V_1 \& v_2 \in V_2 \}$.
    
    Пример: $K_{3,3} = C_3 + C_3$.
    
    \item Удаление вершины $v$ из графа $G_1(V_1, E_1)$ (обозначение -- $G_1(V_1, E_1) - v$ при условии $v \in V_1$) даёт граф $G_2(V_2, E_2)$, где $V_2 = V_1 - v$ и $E_2 = E_1 \setminus \{ e = (v_1, v_2) \mid v_1 = v \vee v_2 = v \}$.
    
    Пример: $C_3 - v = K_2$.
    
    \item Удаление ребра $e$ из графа $G_1(V_1, E_1)$ (обозначение -- $G_1(V_1, E_1) - e$ при условии $e \in E_1$) даёт граф $G_2(V_2, E_2)$, где $V_2 = V_1$ и $E_2 = E_1 - e$.
    
    Пример: $K_2 = \overline{K_2}$.
    
    \item Добавление вершины $v$ в граф $G_1(V_1, E_1)$ (обозначение -- $G_1(V_1, E_1) + v$ при условии $v \notin V_1$) даёт граф $G_2(V_2, E_2)$, где $V_2 = V_1 + v$ и $E_2 = E_1$.
    
    Пример: $K_2 + v = K_2 \cup K_1$.
    
    \item Добавление ребра $e$ в граф $G_1(V_1, E_1)$ (обозначение -- $G_1(V_1, E_1) + e$ при условии $e \notin E_1$) даёт граф $G_2(V_2, E_2)$, где $V_2 = V_1$ и $E_2 = E_1 + e$.
    
    Пример: $C_3 + e = K_1 + e$.
    
    \item Вставление (правильного) подграфа $A$ графа $G_1(V_1, E_1)$ (обозначение -- $G_1(V_1, E_1) / A$ при условии $A \subset V_1, v \notin V_1$) даёт граф $G_2(V_2, E_2)$, где $V_2 = (V_1 \setminus A) + v$,
    
    $E_2 = E_1 \setminus \{ e = (u, w) \mid u \in A \wedge w \in A \} \cup \{ e = (u, v) \mid e \in \Gamma(A) \setminus A \}$.
    
    Пример: $K_4 / C_3 = K_2$.
    
    \item Размножение вершин $v$ графа $G_1(V_1, E_1)$ (обозначение -- $G_1(V_1, E_1) \uparrow v$ при условии $v \in V_1, v' \notin V_1$) даёт граф $G_2(V_2, E_2)$, где
    
    $V_2 = V_1 + v'$ и $E_2 = E_1 \cup \{ (v', w) \} \cup \{ e = (u, v') \mid e \in \Gamma^+(v) \}$.
    
    Пример: $K_2 \uparrow v = C_3$.
\end{enumerate}

Некоторые из примеров, приведённых в определениях операций, нетрудно обобщить. В частности, легко показать, что имеют место следующие соотношения:

\[
K_{m,n} = \overline{K_m} + \overline{K_n}, \quad K_{p-1} = K_p - v, \quad G_1 + v = G \cup K_1, \quad K_{p-1} = K_p / K_2,
\]
\[
K_p / K_1 = K_2, \quad K_p = K_p \uparrow 1.
\]

Введённые операции обладают рядом простых свойств, которые легко вывести из определений. В частности:

\[
G_1 \cup G_2 = G_2 \cup G_1, \quad G_1 + G_2 = G_2 + G_1,
\]
\[
G_1 \cup G_2 \Leftrightarrow \overline{G_1} \cup \overline{G_2}, \quad G_1 + G_2 \Leftrightarrow \overline{G_1} + \overline{G_2}.
\]
\item
  \textbf{Изоморфизм графов}:

  \begin{itemize}
  \tightlist
  \item
    Графы изоморфны, если 
существует биекция между множествами их вершин, сохраняющая смежность.
Говорят, что два графа, $G_1(V_1, E_1)$ и $G_2(V_2, E_2)$, изоморфны (обозначается $G_1 \sim G_2$, или $G_1 = G_2$), если существует биекция $h: V_1 \rightarrow V_2$, сохраняющая смежность (см. 2.1.6):

\[
e_1 = (u, v) \in E_1 \iff e_2 = (h(u), h(v)) \in E_2.
\]
Изоморфизм графов есть отношение эквивалентности.
Графы рассматриваются с точностью до изоморфизма, то есть рассматриваются
классы эквивалентности по отношению изоморфизма
Числовая характеристика, одинаковая для всех изоморфных графов, называется инвариантом графа. Так, p(G) и q(G) — инварианты графа G. Неизвестно никакого простого набора инвариантов, определяющих граф с точностью до
изоморфизма.
  \end{itemize}



\item
  \textbf{Матрицы смежности и инцидентности}:

  \begin{itemize}
  \tightlist
  \item
    \textbf{Матрица смежности}: квадратная матрица, где
    элемент\(a_{ij} = 1\), если вершины\(i\)и\(j\)смежны.
  \item
    \textbf{Матрица инцидентности}: матрица, где строки соответствуют
    вершинам, столбцы --- ребрам, и элемент\(a_{ij} = 1\), если
    вершина\(i\)инцидентна ребру\(j\).
  \end{itemize}
\item
  \textbf{Связность в графах}:
Связный граф — граф, содержащий ровно одну компоненту связности. Это означает, что между любой парой вершин этого графа существует как минимум один путь. 
Компонента связности графа  G  (или просто компонента графа  G) — максимальный (по включению) связный подграф графа G
  \begin{itemize}
  \tightlist
  \item
    Граф связен, если существует путь между любыми двумя его вершинами.
    В ориентированных графах различают сильную и слабую связность.
  \end{itemize}
\item
  \textbf{Матрицы достижимости и контрдостижимости}:

  \begin{itemize}
  \tightlist
  \item
    \textbf{Матрица достижимости}: матрица, где элемент\(a_{ij} = 1\),
    если существует путь из вершины\(i\)в вершину\(j\).
  \item
    \textbf{Матрица контрдостижимости}: обратная матрица достижимости.
  \end{itemize}
\item
  \textbf{Выявление маршрутов с заданным количеством ребер}:

  \begin{itemize}
  \tightlist
  \item
    Использование степеней матрицы смежности для нахождения количества
    маршрутов с точно заданным количеством ребер.
  \end{itemize}
\item
  \textbf{Расстояния в графах}:

  \begin{itemize}
  \tightlist
  \item
    Расстояние между двумя вершинами --- это длина кратчайшего пути
    между ними.
  \end{itemize}
7.2.5. Расстояние между вершинами, ярусы
и диаметр графа
Длиной маршрута называется количество рёбер в нём (с учётом повторений).
Если маршрут М = v0, e1,v1,e2, v2, • • •, ek,vk, то длина М равна к (обозначается
|М| = к). Расстоянием между вершинами u и v (обозначается d(u, v)) называется
длина кратчайшей цепи (u,v), а сама кратчайшая цепь называется геодезической,
d(u, v)  = min(|(u,v)|)
Если для любых двух вершин графа существует единственная геодезическая
цепь, то граф называется геодезическим.
З А М Е Ч А Н И Е
Если -<3 ((u,v)), то по определению d(u,v) =f+oo.
Множество вершин, находящихся на заданном расстоянии п от вершины v (обо-
значение D(v,n)), называется ярусом:
D(v, п) {и € V | d(v, и) —п} .
Ясно, что множество вершин V всякого связного графа однозначно разбивает-
ся на ярусы относительно дайной вершины. Диаметром графа G называется
длиннейшая геодезическая. Длина диаметра обозначается D(G):
D(G) max d(u,v).
u,v(EV
\item
  \textbf{Алгоритм Дейкстры}:

  \begin{itemize}
  \tightlist
  \item
    Алгоритм для нахождения кратчайших путей от одной вершины до всех
    остальных в графе с неотрицательными весами ребер.
  \end{itemize}
\item
  \textbf{Эйлеровы графы. Алгоритм Флери}:

  \begin{itemize}
  \tightlist
  \item
    Эйлеров граф содержит цикл, проходящий через все ребра. Алгоритм
    Флери используется для нахождения эйлерова пути или цикла.

  \item 
Алгоритм был предложен Флёри в 1883 году.
Пусть задан граф G = ( V , E ). Начинаем с некоторой вершины p $\in$ V  и каждый раз вычеркиваем пройденное ребро. Не проходим по ребру, если удаление этого ребра приводит к разбиению графа на две связные компоненты (не считая изолированных вершин), т.е. необходимо проверять, является ли ребро мостом или нет. 
  \end{itemize}
\item
  \textbf{Гамильтоновы графы}:

  \begin{itemize}
  \tightlist
  \item
    Граф содержит гамильтонов цикл, проходящий через все вершины ровно
    один раз.
  \item 
Необходимое условие существования гамильтонова цикла в неориентированном графе: если неориентированный граф G содержит гамильтонов цикл, тогда в нём не существует ни одной вершины x ( i )  с локальной степенью p ( x ( i ) ) < 2 . Доказательство следует из определения.

Условие Поша: Пусть граф G имеет p > 2  вершин. Если для всякого n , 0 < n < ( p - 1 ) / 2 , число вершин со степенями меньшими или равными n меньше, чем n, и для нечетного p  число вершин со степенью ( p - 1 ) / 2  не превосходит ( p - 1 ) / 2 , то G — гамильтонов граф. Это достаточное условие не является необходимым[6].

Как следствие теоремы Поша, получаем более простые и менее сильные достаточные условия, найденные Дираком и Оре.

В 1952 году было сформулировано условие Дирака существования гамильтонова цикла: пусть p  — число вершин в данном графе и p > 3 ; если степень каждой вершины не меньше, чем p/2 , то данный граф — гамильтонов[7].

Условие Оре: пусть p  — количество вершин в данном графе и p > 2 . Если для любой пары несмежных вершин ( x , y )  выполнено неравенство deg x + deg y >= p , то данный граф — гамильтонов (другими словами: сумма степеней любых двух несмежных вершин не меньше общего числа вершин в графе)[7].

Теорема Бонди[англ.] — Хватала обобщает утверждения Дирака и Оре. Граф является гамильтоновым тогда и только тогда, когда его замыкание — гамильтонов граф. Для графа G с n вершинами замыкание строится добавлением в G ребра (u,v) для каждой пары несмежных вершин u и v, сумма степеней которых не меньше n[8]. 
  \end{itemize}
\item
  \textbf{Обходы графа по ширине и глубине}:

  \begin{itemize}
  \tightlist
  \item
    Обход в ширину (BFS): посещение вершин уровня за уровнем.
  \item
    Обход в глубину (DFS): углубление до конца ветки, затем возврат.
  \end{itemize}
\item
  \textbf{Деревья. Основные определения и свойства}:

  \begin{itemize}
  \tightlist
  \item
    Дерево --- связный ациклический граф. Основные свойства: любое
    дерево с\(n\)вершинами имеет\(n-1\)ребро.
  \item 
Граф без циклов называется ациклическим, или лесом. Связный ациклический
граф называется (свободным) деревом. Таким образом, компонентами связности
леса являются деревья.
  \end{itemize}
\item
  \textbf{Ориентированные, упорядоченные и бинарные деревья}:

  \begin{itemize}
  \tightlist
  \item
    Ориентированное дерево: дерево, где каждое ребро имеет направление.
  \item
    Упорядоченное дерево: дерево с фиксированным порядком детей у каждой
    вершины.
  \item
    Бинарное дерево: дерево, в котором каждая вершина имеет не более
    двух детей.
  \item св-ва
    Древочисленность - количество ребер равно кол-ву вершин минус 1 

1. G — дерево, то есть связный граф без циклов,

2. Любые две вершины соединены в G единственной простой цепью

3. G — связный граф, и любое ребро есть мост,

4. G — связный и древочисленный,

5. G — ациклический и древочисленный,

6. G — ациклический и субциклический,

7. G — связный, субциклический и неполный,

8. G — древочисленный и субциклический (за двумя исключениями),
  \end{itemize}
\item
  \textbf{Алгоритм выделения остовного дерева}:
  
  \begin{itemize}
  \tightlist
  \item
    Алгоритмы Крускала и Прима для нахождения минимального остовного
    дерева.
  \item Алгоритм с пар:
    Шаг 1: Выбираем в Г произвольную вершину в1. В в1 образует подграф Г1 графа Г, который является деревом. Полагаем i = 1 

    Шаг 2: Если i = p(кол-во вершин), то задача решена и Гi - искомое оставное дерево в противном случае переходим к шагу 3. 

    Шаг 3: Пусть уже постр. дерево Гi являющееся подграфом графа Г и содержю в нем вершины: в1,в2,$\dots$, вi , где 1<=i<=p-1. 

    Строим граф Гi+1, добавляя к графу Гi новую вершину vi+1 смежную с нек. вершиной vj графа Гi и новое ребро(vi+1,vj) 
    Указанная вершина vi+1 обязательно найдется. Прсваеваем по опр. i=i+1 и переходим к шагу 2
  \end{itemize}
\item
  \textbf{Минимальные остовные деревья нагруженных графов}:

  \begin{itemize}
  \tightlist
  \item
    Использование алгоритмов Крускала и Прима для нахождения
    минимального остовного дерева в графе с весами ребер.
  \item Алгоритм с пар(Алгоритм Прима):
    Шаг 1: Выберем в графе Г ребро минимальной длины вместе с инцидентной ему вершиной оно образует подграф Г2 графа Г, полагая i=2

    Шаг 2: Если i=p, то задача решена, то Гi - минимальное оставное дерево, иначе переходим к шагу 3. 

    Шаг 3: Строим Gi+1, добавляя к графу Гi новое ребро минимально среди всех ребер графа Г, каждое из которых инцидентны какой-нибудь вершине
    графа Гi, и одновременно какой-нибудь инцидентна какой-нибудь вершине графа Г не содерж. в Гiю 
    Вместе с этим ребром включена в Gi+1 и инцидентная ему вершину, не содержащуюся в Gi. Переходим к шагу 2. 
  \end{itemize}
\item
  \textbf{Планарность графов}:

  \begin{itemize}
  \tightlist
  \item
    Граф планарен, если его можно нарисовать на плоскости без
    пересечения ребер. Теорема Куратовского характеризует планарные
    графы.
  \item 
    \begin{theorem}[Теорема Куратовского]
      Граф планарен тогда и только тогда, когда он не содержит подразделений полного графа с пятью вершинами (K5) и полного двудольного графа с тремя вершинами в каждой доле 
    \end{theorem}
  \item
Конечный связный планарный граф удовлетворяет формуле Эйлера:\\ 
    V - E + F = 2,
где V - количество вершин, E - количество рёбер, F - количество граней (областей, ограниченных рёбрами).

  \end{itemize}
\item
  \textbf{Раскраски графов}:

Раскраской графа G называется такое приписывание цветов (натуральных чи-
сел) его вершинам, что никакие две смежные вершины пе получают одинаковый
цвет. Если задана допустимая раскраска графа, использующая т цветов, то го-
ворят, что граф m-раскрашиваемый. Наименьшее возможное количество цветов
в раскраске называется хроматическим числом и обозначается x(G).
  \begin{itemize}
  \tightlist
  \item
    Процесс раскраски вершин графа в минимальное количество цветов так,
    чтобы смежные вершины были окрашены в разные цвета.
  \end{itemize}
\item
  \textbf{Теорема о пяти красках, гипотеза четырех красок}:
  \begin{itemize}
  \tightlist
  \item
    Теорема о пяти красках утверждает, что любой планарный граф можно
    раскрасить не более чем в пять цветов.
  \item
    Гипотеза четырех красок утверждает, что достаточно четырех цветов
    для раскраски любого планарного графа.
  \end{itemize}
}}}}}\end{enumerate}


Эти ответы охватывают основные концепции и примеры для каждого вопроса.

\end{document}
