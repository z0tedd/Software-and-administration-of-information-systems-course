\documentclass[a4paper,14pt]{extreport} % добавить leqno в [] для нумерации слева

%\usepackage[14pt]{extsizes}
\linespread{1.45} %полуторный интервал
\usepackage[left=3cm,right=1.5cm,
top=2cm,bottom=2cm,bindingoffset=0cm]{geometry}

\usepackage{titlesec}
%%% Работа с русским языком
\titleformat{\section}{\normalfont\bfseries}{\thechapter}{14pt}{\bfseries}
\usepackage{cmap}					% поиск в PDF
\usepackage{mathtext} 				% русские буквы в фомулах
\usepackage[T2A]{fontenc}			% кодировка
\usepackage[utf8]{inputenc}			% кодировка исходного текста
\usepackage[english,russian]{babel}	% локализация и переносы
\usepackage{ fancyhdr} % улучшенная нумерация страниц
%%% Дополнительная работа с математикой
\usepackage{amsmath,amsfonts,amssymb,amsthm,mathtools} % AMS
\usepackage{icomma} % "Умная" запятая: $0,2$ --- число, $0, 2$ --- перечисление
%\renewcommand{\sfdefault}{cmss}
\usefont{T2A}{cmss}{m}{n}
%% Номера формул
\mathtoolsset{showonlyrefs=true} % Показывать номера только у тех формул, на которые есть \eqref{} в тексте.

%% Шрифты
\usepackage{euscript}	 % Шрифт Евклид
\usepackage{mathrsfs} % Красивый матшрифт
\usepackage{enumitem}  % Классные списки
%% Свои команды
\DeclareMathOperator{\sgn}{\mathop{sgn}}

%% Перенос знаков в формулах (по Львовскому)
\newcommand*{\hm}[1]{#1\nobreak\discretionary{}
	{\hbox{$\mathsurround=0pt #1$}}{}}
\usepackage{tempora}
%\renewcommand{\sfdefault}{cmss}

%%% Заголовок
\title{100 Задач по теме "Комбинаторика}
\author{Антон Файтельсон}
\date{}
\newtheorem{definition}{Определение}
\newtheorem*{Help}{Краткая сводка}

\begin{document}
\begin{flushleft}
	\Large  Конспект к билетам по мат. анализу	\\
	Выполнил великий и могучий Файтельсон Антон
\end{flushleft}

\begin{center}
	\begin{enumerate}
		\item Аксиоматическое определение множества действительных чисел. Свойство полноты.\\
		\begin{definition}\label{Аксиоматика}
			Множество $\mathbb {R} $ называется множеством действительных (вещественных) чисел, а его элементы — действительными (вещественными) числами, если выполнен следующий комплекс условий, называемый
			аксиоматикой вещественных чисел:
		\end{definition}
		%Краткая сводка
		\begin{Help}
			Закон (правило) f, посредством которого каждому a $\in$ A сопоставляется единственный b $\in$ B, называют отображением. Обычно это записывают так: b = f(a) или f: A $\rightarrow$ B (отображение из A в B). 
		\end{Help}
		%Начало Аксиом
		\begin{enumerate}[label=(\Roman*)]
			\item Аксиомы сложения\\
			
			Определено отображение (Операция сложения)
			\begin{equation}
				+: \mathbb {R} \times \mathbb {R} \rightarrow \mathbb {R},
			\end{equation}
			сопоставляющее каждой упорядоченной паре (x, y) элементов x, y из $\mathbb {R}$
			некоторый элемент x  y $\in \mathbb {R}$ , называемый суммой x и y. При этом выполнены следующие условия:
	
			\begin{itemize}
				\item Нейтральный элемент(называемый в случае сложения нулем)
				\begin{center}
					$\forall x \in \mathbb{R}: \exists\space0: x+0=0+x=x$
				\end{center}
				
				\item Противоположный элемент
				\begin{center}
					$\forall x \in \mathbb{R}: \exists (-x) \in \mathbb{R}: x+(-x)=(-x)+x=0$
				\end{center}
				
				\item Ассоциативность
				\begin{center}
					$\forall x,y,z \in \mathbb{R}: x+(y+z)=(x+y)+z$
				\end{center}
				\newpage
				\item Коммунитативность
				\begin{center}
					$\forall x,y \in \mathbb{R}: x+y=y+x$
				\end{center}
			\end{itemize}
		
		
		%Нужно привести сравнение ассоциативности и коммунитативности с бананоми и строками,чтобы можно ьыло понять
			\item Аксиомы умножения\\
			
			Определено отображение (Операция умножения)
			\begin{equation}
				\bullet: \mathbb {R} \times \mathbb {R} \rightarrow \mathbb {R},
			\end{equation}
			сопоставляющее каждой упорядоченной паре (x, y) элементов x, y из $\mathbb {R}$
			некоторый элемент x $\bullet$ y $\in \mathbb {R}$ , называемый произведением x и y. При этом выполнены следующие условия:
			
			\begin{itemize}
				\item Нейтральный элемент(называемый в случае умножения единицей)
				\begin{center}
					$\forall x \in \mathbb{R}: \exists1\in \mathbb{R}\textbackslash0: x\cdotp1=1\cdotp x=x$
				\end{center}
				
				\item Обратный элемент
				\begin{center}
					$\forall x \in \mathbb{R}\textbackslash0: \exists x^{-1} \in \mathbb{R}: x\cdotp x^{-1} =x^{-1}\cdotp x = 1 $
				\end{center}
				
				\item Ассоциативность
				\begin{center}
					$\forall x,y,z \in \mathbb{R}: x\cdotp(y\cdotp z)=(x\cdotp y)\cdotp z$
				\end{center}
				
				\item Коммунитативность
				\begin{center}
					$\forall x,y \in \mathbb{R}: x\cdotp y=y \cdotp x$
				\end{center}
			\end{itemize}
			
			\item[(I, II)] Связь сложения и умножения (Дистрибутивность умножения к сложению)
				\begin{center}
					$\forall x,y,z \in \mathbb{R}: (x+y)z=xz+yz$
				\end{center}
			\item Аксиомы порядка\\
			Между элементами $\mathbb{R}$ имеется отношение $\leq$, т.е. для элементов x, y из $\mathbb{R}$ 
			установлено, выполняется ли x $\leq$ y или нет. При этом должны удолетворяться следующие условия:
			\begin{itemize}
				\item $\forall x \in \mathbb{R} (x\leq x)$
				\item $(x\leq y) \land (y\leq x) \Rightarrow (x = y)$
				\item $(x\leq y) \land (y\leq z) \Rightarrow (x \leq z)$
				\item $\forall x \in \mathbb{R} \hspace{5px} \forall y \in \mathbb{R} \hspace{5px} (x\leq y) \lor (y \leq x)$\\
				Отношение $\leq$ в $\mathbb{R}$ называется отношением неравенства.
			\end{itemize}
			
			\item[(I, III)] Связь сложения и порядка в $\mathbb{R}$ 
			\begin{center}
				$\forall x,y,z \in \mathbb{R}: (x\leq y) \Rightarrow (x+z \leq y+z)$
			\end{center}
			
			\item[(II, III)] Связь умножения и порядка в $\mathbb{R}$ 
			\begin{center}
				$\forall x,y \in \mathbb{R}: (0 \leq x) \land (0 \leq y) \Rightarrow (0 \leq x\cdotp y)$
			\end{center}
			
			\item Аксиома полноты(непрерывности)\\
			Если X и Y — непустые подмножества $\mathbb{R}$, обладающие тем свойством, что для любых элементов x $\in$ X и y $\in$ Y выполнено x $\leq$ y, то $\exists c \in \mathbb{R}$, что 
			$x\leq c\leq y$ для любых элементов x $\in$ X и y $\in$ Y.\\
			Определение через кванторы(мне было весело это писать):
			\begin{center}
				$\forall x \in X, y \in Y : x\leq y \Rightarrow \exists c \in \mathbb{R}: x\leq c\leq y $
			\end{center}
			%, что для любых элементов x ∈ X
			%и y ∈ Y выполнено x ∂ y, то существует такое c ∈ R, что x ∂ c ∂ y для любых
			%элементов x ∈ X и y ∈ Y .
		\end{enumerate}
		
		
		\item Следствия из аксиом множества действительных чисел.\\
		\textbf{Замечание.} Следствий много, и поэтому часть из них не будет представлено, я хз какие будут на экзамене.
		\begin{enumerate}
			\item Cледствия аксиом сложения
			\begin{itemize}
				\item В множестве действительных чисел имеется только один нуль.
				\begin{proof}
					Если $0_1$ и $0_2$ — нули в $\mathbb{R}$, то по определению нуля
					\begin{equation}
						0_1 = 0_1 + 0_2 = 0_2 + 0_1 = 0_2
					\end{equation}
				\end{proof}
				
				\item В множестве действительных чисел у каждого элемента имеется единственный противоположный элемент.
				\begin{proof}
					Если $x_1$ и $x_2$ — элементы, противоположные x $\in \mathbb{R}$, то
					\begin{equation}
						x_1 = x_1 + 0 = x_1 + (x+x_2) = (x_1+x) + x_2 = 0 + x_2 = x_2
					\end{equation}
				\end{proof}
				
				\item Уравнение $a + x = b$ в $\mathbb{R}$ имеет единственное решение:
				\begin{center}
					$x = b + (-a)$
				\end{center}
				\begin{proof}
					Это вытекает из существования и единственности у каждого элемента
					a $\in \mathbb{R}$ противоположного ему элемента:
					\begin{multline*}
						(a+x=b) \Leftrightarrow ((x+a)+(-a) = b + (-a)) \Leftrightarrow \\
						\Leftrightarrow (x+ (a + (-a)) = b+ (-a)) \Leftrightarrow \\
						\Leftrightarrow (x+0 = b+(-a)) \Leftrightarrow (x = b + (-a))
					\end{multline*}
				\end{proof}
			\end{itemize}
			
			\item Следствия аксиом умножения
			
			\begin{itemize}
				\item В множестве действительных чисел имеется только одна единица.
				\item Для каждого числа x $\neq$ 0 имеется только один обратный элемент $x^{-1}$.
				\item Уравнение a$\cdotp$x = b при a $\in \mathbb{R}\textbackslash0$ имеет притом единственное решение x = b $\cdotp a^{-1}$\\
				\textbf{Замечание.} Доказательства? Нахуй они нужны? Скопируй с верхних следствий епта, если так нужны
			\end{itemize}
			\item Следствия аксиомы связи сложения и умножения
			\begin{itemize}
				\item $\forall x \in \mathbb{R}: x\cdotp 0 = 0 \cdotp x = 0$
				\begin{proof}
					$
						(x\cdotp 0 = x\cdotp (0 + 0) = x\cdotp 0 + x\cdotp 0) \Rightarrow $\\
					$	\Rightarrow (x\cdotp 0 = x\cdotp 0 + 0 = x\cdotp 0 + x\cdotp 0 + (-x)\cdotp 0= 
					x\cdotp 0 + (-x)\cdotp 0 = 0)$\\
				\end{proof}
				\item $x\cdotp y = 0 \Rightarrow (x=0)\lor (y=0)$
				\begin{proof}
					Если, например, y $\neq$ 0, то из единственности решения уравнения x $\cdotp$ y = 0
					относительно х находим х = 0 $\cdotp$ $y^{-1}$ = 0
				\end{proof}
				\item $\forall x \in \mathbb{R}: -x = (-1)\cdotp x$
				\begin{proof}
					$ x + (-1) \cdotp x = (1 + (-1)) \cdotp x = 0 \cdotp x = x \cdotp 0 = 0 = x + (-x)$, и утверждение следует из единственности противоположного элемента.
				\end{proof}
				\item $\forall x \in \mathbb{R}: (-1)(-x) = x$
				\begin{proof}
					Следует из предыдущего док-ва и единственности противоположного элемента.
				\end{proof}
				\item $\forall x \in \mathbb{R}: (-x)(-x) = x\cdotp x$
				\begin{proof}
					$(-x)(-x) = ((-1) \cdotp x)(-x) = $\\
					$= (x \cdotp (-1))(-x) 	 = x((-1)(-x)) = x \cdotp x$
				\end{proof}
			\end{itemize}
			\item  Следствия аксиом порядка.
			
			\begin{Help}
				
			\end{Help}}
		\end{enumerate}
	\end{enumerate}
\end{center}

\end{document}