\documentclass[a4paper,14pt]{extreport} % добавить leqno в [] для нумерации слева
%\usepackage[14pt]{extsizes}
\usepackage[left=3cm,right=1.5cm,
top=2cm,bottom=2cm,bindingoffset=0cm]{geometry}
\linespread{1.45} %полуторный интервал
\usepackage{titlesec}
%%% Работа с русским языком
\titleformat{\section}{\normalfont\bfseries}{\thechapter}{14pt}{\bfseries}
\usepackage{cmap}					% поиск в PDF
\usepackage{mathtext} 				% русские буквы в фомулах
\usepackage[T2A]{fontenc}			% кодировка
\usepackage[utf8]{inputenc}			% кодировка исходного текста
\usepackage[english,russian]{babel}	% локализация и переносы
\usepackage{ fancyhdr} % улучшенная нумерация страниц
%%% Дополнительная работа с математикой
\usepackage{amsmath,amsfonts,amssymb,amsthm,mathtools} % AMS
\usepackage{icomma} % "Умная" запятая: $0,2$ --- число, $0, 2$ --- перечисление
%\renewcommand{\sfdefault}{cmss}
\usefont{T2A}{cmss}{m}{n}
%% Номера формул
\mathtoolsset{showonlyrefs=true} % Показывать номера только у тех формул, на которые есть \eqref{} в тексте.

%% Шрифты
\usepackage{euscript}	 % Шрифт Евклид
\usepackage{mathrsfs} % Красивый матшрифт
%% Свои команды
\DeclareMathOperator{\sgn}{\mathop{sgn}}

%% Перенос знаков в формулах (по Львовскому)
\newcommand*{\hm}[1]{#1\nobreak\discretionary{}
	{\hbox{$\mathsurround=0pt #1$}}{}}

%\renewcommand{\sfdefault}{cmss}
\usepackage{tempora}
%%% Заголовок
\title{50 Задач по теме "Комбинаторика}
\author{Антон Файтельсон}
\date{}

\begin{document}
			\begin{flushleft}
				\Large 50 Задач по теме "Комбинаторика"		\\
				Выполнил студент 113 группы Файтельсон Антон
				
			\end{flushleft}
			\begin{center}
				\begin{enumerate}
					\item {\large задача  Пароль}\\
						Источник: ICPC 2022-2023 NERC (NEERC), квалификационный этап Чемпионата Юга и Поволжья России \\
						\vspace{15pt}
						Монокарп забыл пароль от своего телефона. Пароль состоит ровно из 6 цифр от 0 до 9 (обратите внимание, что пароль может начинаться с цифры 0).\\
						\vspace{15pt}
						Монокарп помнит, что в его пароле были ровно две различные цифры, причем каждая из этих цифр встречалась в пароле ровно по три раза. Также Монокарп помнит количество цифр (n), которых точно не было в его пароле . \\
						\vspace{15pt}
						Посчитайте количество последовательностей из 6 цифр, которые могли бы быть паролем Монокарпа (то есть которые подходят под все описанные условия).\\
						\vspace{15pt}
						{\large Решение:}\\
						Так как Монокарп помнит количество цифр, которых точно не было в пароле, тогда количество цифр, которые возможно были в пароле равно 
						\begin{equation}
							(10 - n )
						\end{equation}
						
						Возьмем простейший случай, когда всего два возможных претендента на числа в пароле - 1 и 0.\\
						Найдем, сколькими способами мы можем выбрать 3 позиции из 6 возможных:\\
						\begin{equation}
							С_6^3 = \frac{6!}{3!\times3!} = 20
						\end{equation}
						\newpage
						\vspace{15pt} 
						Вернемся к случаю, когда у нас \(10 - n\) возможных чисел. Найдем сколько можно составить пар из этих чисел.
						\begin{equation}
							С_{10-n}^2 = \frac{(10-n)!}{(8-n)!\times2!} = \frac{(10-n)!\times(9-n)!}{2}
						\end{equation}
						
						Тогда решением задачи будет формула:
								\begin{equation}
									С_6^3 \times С_{10-n}^2 = 20 \times \frac{(10-n)!\times(9-n)!}{2} = 10 \times {(10-n)!\times(9-n)!}
								\end{equation}
								
						Ответ: $10 \times {(10-n)!\times(9-n)!}$
						
						
						
						
						 \item {\large задача  }\\
						Источник: C.Якунин сайт kompege.ru\\
						\vspace{15pt}
						Полина составляет 21-буквенные слова из букв слова РЕКОГНОСЦИРОВКА. Каждая гласная в них используется столько раз, сколько в слове \\РЕКОГНОСЦИРОВКА. Каждая согласная может использоваться сколько угодно раз или не использоваться совсем. Сколько слов может составить Полина, если известно, что сумма порядковых номеров гласных букв, в каждом из них, равна 21? Буквы нумеруются слева направо, начиная с единицы.\\
						\vspace{15pt}
						Решение:\\
						Подсчёт конфигураций:
						21 нам даёт единственный набор, состоящий из 6 неповторяющихся гласных: 1 + 2 + 3 + 4 + 5 + 6. Гласные, которые могут стоять на этих местах: 3 буквы О, одна буква А, одна буква И и одна буква Е.  Так как неповторяющихся букв у нас всего 3, значит нам нужно выбрать только для них позиции, значит всего:
						\begin{equation}
						 \frac{6!}{3!} = 4 \times5 \times6 = 120 
						\end{equation}
						Размещение согласных:				
						Любая согласная может занимать одну из следующих 15 позиций. Имеем 15 перемноженных семёрок (размещения с повторениями) или $7^{15}$ = 4747561509943.
						
						\newpage						
						
						Итоговый подсчёт слов:
						Итак, в каждой из 120 конфигураций есть 4747561509943 вариантов. Значит, всего слов: 
						\begin{equation}
							120 \times 4747561509943 = 569707381193160 
						\end{equation}
						
						Ответ: 569707381193160
						
						\item {\large задача  }\\
						Источник:  Дискретная математика. Учебник и задачник для Вузов (Баврин И.И. )\\
						\vspace{15pt}
						Пусть имеется n сортов мономеров (например, азотистых оснований). Из этих мономеров образуется полимер, который можно представить как цепочку из k мономеров. При этом k, как правило, больше n, и мономеры в цепочке могут повторяться.
						Какое количество различных полимеров длины k можно образовать из данных n сортов мономеров?
						\\
						\vspace{15pt}
						Решение:\\
						Будем считать набор мономеров алфавитом из n элементов. Тогда каждый полимер, состоящий из k мономеров, есть слово длины n. Число таких слов, как известно, равно $n^k$, а число различных полимеров будет в два раза меньше, так как, например, молекулы $a_1a_2a_3$ и $a_3a_2a_1$ не различаются (одна из них превращается в другую, если ее повернуть на $180^{\circ}$).
						
						В частности, если алфавит состоит из четырех азотистых оснований А, Ц, Г и Т (т. е. n = 4), а полимером является ген (средняя длина гена равна 1000 единиц, т. е. k = 1000), то число всевозможных генов, которые можно получить из четырех оснований, равно
						\begin{equation}
							\frac{1}{2}n^k=\frac{1}{2}4^{1000} = \frac{2^{2000}}{2} = 2^{1999}
						\end{equation}
						Это громадное число. По некоторым подсчетам, оно превосходит общее число атомов в Солнечной системе\\
						Ответ: $\frac{1}{2}n^k$
						
						 \item {\large задача  }\\
						Источник:  Дискретная математика. Учебник и задачник для Вузов (Баврин И.И. )\\
						\vspace{15pt}
						Хорошо известно, что хромосому схематично можно представить как цепочку из генов. При этом свойства хромосомы зависят не только от состава генов, но и от их расположения в цепочке. Существуют методы, позволяющие изменить порядок генов в хромосоме. Возникает вопрос: какое количество хромосом можно получить из данной, изменяя в ней порядок следования генов?\\
						\vspace{15pt}
						Решение:\\
						Пусть исходная хромосома состоит из n генов. Обозначим их $a_1, a_2,\ldots, a_n$, и пусть  $А = \{a_1, a_2,\ldots, a_n\}$. Тогда понятно, что каждая хромосома, имеющая данный набор генов, есть перестановка множества А. Число таких перестановок, как известно, равно $n!$.
						
						Ответ: $n!$
						
						
%						 \item {\large задача  }\\
%						Источник:  Дискретная математика. Учебник и задачник для Вузов (Баврин И.И. )\\
%						\vspace{15pt}
%						\\
%						\vspace{15pt}
%						Решение:\\
%						
%						\begin{equation}
%							 
%						\end{equation}
%						Ответ: 
					\item {\large задача  }\\
					Источник: Джеймс Андерсон: Дискретная математика и комбинаторика\\
					\vspace{15pt}
					"Фулл хаус" содержит три карты одного ранга и две карты другого
					ранга. Например, расклад, содержащий три короля и две шестерки представляет
					собой "фулл хаус". Сколько существует 5-карточных раскладов с "фулл хаус"? \\
					\vspace{15pt}
					Решение:\\
					Предположим, что "фулл хаус" составили три короля и две шестерки. Три короля
					выбираются из четырех, поэтому существуют $C_4^3 = 4$  способа выбрать трех
					королей. Две шестерки выбираются из четырех, поэтому существуют $C_4^2 = 6$
					способов выбрать две шестерки. Поэтому, согласно комбинаторному принципу
					умножения существуют $4 \times 6 = 24$ способа выбрать трех королей и две шестерки
					или три карты одного ранга и две карты другого ранга. Существуют 13 способов
					выбрать три карты одного ранга и 12 способов выбрать две карты одного ранга.
					Поэтому существуют $13 \times 12 = 156$ различных способов сочетания рангов. Следовательно, существуют $156 \times 24 = 3744$ возможных 5-карточных раскладов с "фулл хаус".
					
					Ответ: 3744
					
					
						 \item {\large задача }\\
						Источник: Дагестанский государственный университет народного хозяйства учебное пособие по дисциплине "математика"  \\
						\vspace{15pt}
						Cколько существует вариантов опроса 11 учащихся на
						одном занятии, если ни одни из них не будет подвергнут опросу
						дважды и на занятии может быть опрошено любое количество
						учащихся, причем порядок, в котором опрашиваются учащиеся,
						безразличен?\\
						\vspace{15pt}
						
						Решение:\\
						Имеется генеральная совокупность объема 11 учащихся.
						Преподаватель может не опросить ни одного из 11 учащихся, что
						является одним из вариантов. Этому случаю соответствует $C_{11}^0$.
						Преподаватель может опросить только одного из учащихся, таких
						вариантов $C_{11}^1$
						Если преподаватель опросит двух учащихся, то число
						вариантов опроса $C_{11}^2$ и т. д.
						Наконец, могут быть опрошены все учащиеся. Число
						вариантов в этом случае $C_{11}^{11}$
						
						Число всех возможных вариантов опроса можно найти по правилу
						сложения:
						\begin{equation}
							C_{11}^0 + C_{11}^1 + C_{11}^2 + \ldots  + C_{11}^{11} = 2^{11} = 2048
						\end{equation}
						
						Ответ: 2048
						
						\item {\large задача  }\\
						Источник: Джеймс Андерсон: Дискретная математика и комбинаторика\\
						\vspace{15pt}
						Десять команд участвуют в розыгрыше первенства по
						футболу, лучшие из которых занимают 1-е, 2-е и 3-е место. Две
						команды, занявшие последние места, не будут участвовать в
						следующем таком же первенстве. Сколько разных вариантов
						результата первенства может быть, если учитывать только положение
						первых трех и последних двух команд.\\
						
					
						\vspace{15pt}
						Решение:\\
						 Имеется генеральная совокупность объема 10
						команд. Из нее будем выбирать 5 команд в 2 этапа:
						1) сначала на первые 3 места из 10 с учетом состава и порядка
						команд;
						2) затем на последние 2 места из оставшихся 7 с учетом только
						состава (порядок выбывших команд не важен). Первые 3 места могут
						быть распределены $A_{10}^3 = \frac{10!}{7!}= 720$ способами.						
						Число способов исключить 2 команды из оставшихся 7 равно $С_{7}^2 = \frac{7!}{2!\times5!}= 21$.
						Согласно правилу умножения получаем, что число разных результатов неравенства равно:
						\begin{equation}
							С_{7}^2 \times A_{10}^3 = 15120
						\end{equation}
						
						Ответ: 15120
						
						
						
						\item {\large задача  }\\
						Источник: Задачи по комбинаторике, Шварц Д. А.\\
						\vspace{15pt}
						У бедного студента осталось гречки на две порции, риса на три порции и макарон на две порции. Сколько у студента способов съесть это на завтраки в течение недели (по одной порции в день)?\\
						\vspace{15pt}
						Решение:\\
						Если студент разложит имеющиеся 7 порций еды по разным тарелкам, то количество вариантов выбора будет составлять $P_7 = 7!$, но поскольку разные порции, например, риса не отличимы, то общее количество необходимо разделить на количество перестановок из 2 элементов $P_2 = 2!$(число способов упорядочить 2 тарелки с гречкой), затем на $P_3 = 3!$ (число способов упорядочить 3 тарелки с рисом), затем на $P_2 = =2!$ (макароны). Итого получим кол-во способов завтракать в течение недели для нашего бедного студента.
						\begin{equation}
							 \frac{7!}{2!\times3!\times2!} = 210
						\end{equation}
						
						Ответ: 210
						
						
						
						 \item {\large задача }\\
						Источник: Дагестанский государственный университет народного хозяйства учебное пособие по дисциплине "математика"  \\
						\vspace{15pt}
						При игре в домино 4 игрока делят поровну 28 костей.
						Сколькими способами они могут это сделать?\\
						\vspace{15pt}
						
						Решение:\\
							Так как у нас всего 28 костей, то каждому игроку необходимо дать $28 \div 4 $ = 7 костей.
							Представим, что кости данные игроку 1 получают некоторое св-во и становяться одинаковыми. Аналогично с другими игроками.
							Тогда кол-во способов поделить кости будет равно
						\begin{equation}
							\overline{P_{7,7,7,7}}=\frac{28!}{7!\times7!\times7!\times7!} = 472518347558400
						\end{equation}
						Данное упрощение имеет смысл так, как нас не волнует порядок костей, которые мы дали игрокам, тогда для упрощения вычисления их можно считать равными
						Ответ: 472518347558400
						
						
						
						
						
						
						
						
						
						
						
						
						
						\item {\large задача  }\\
						Источник: Джеймс Андерсон: Дискретная математика и комбинаторика\\
						\vspace{15pt}
						Сколькими способами можно расположить для фотографирования
						пять мальчиков и пять девочек, если ни две девочки, ни два мальчика не должны
						стоять рядом? \\
						\vspace{15pt}
						Решение:\\
						В данной ситуации первым в ряду может быть либо мальчик,
						либо девочка. Если первой стоит девочка, то ряд имеет вид ДМДМДМДМДМ.
						Имеются 5! способов расставить девочек на позициях Д и 5! способов расставить
						мальчиков на позициях М. Поэтому, существуют $5! \times 5!$ способов расположить
						детей в ряд, если первой стоит девочка. Аналогично, существуют $5! х 5!$ способов
						расположить детей в ряд, если первым стоит мальчик. Таким образом, имеются
						$2 х 5! х 5! = 28800$ способов расположить детей в ряд для фотографирования.
					
						Ответ: 28800
						
						 \item {\large задача  }\\
						Источник: Джеймс Андерсон: Дискретная математика и комбинаторика\\
						\vspace{15pt}
						Сколькими способами можно рассадить 10 человек за круглым
						столом, если имеет значение только порядок соседей.\\
						\vspace{15pt}
						Решение:\\
						Предложим, что место за столом уникально. Посадим одного человека на это место
						 и будем считать его место опорным элементом,
						тогда следующего человека(соседа опорного элемента) можно выбрать $10-1=9$ способами, 
						находящегося рядом с соседом человека можно выбрать $9-1=8$ способами, тогда общее кол-во способов рассадки равно:
						\begin{equation}
							P_{9} = P! = 362880
						\end{equation}
						
						Ответ: 362880
						
						
						
						
						
						
						
						
						
						 \item {\large задача  }\\
						Источник: Джеймс Андерсон: Дискретная математика и комбинаторика\\
						\vspace{15pt}
						Сколько различных четырехзначных чисел можно образовать из
						цифр 1, 2, 3, \ldots, 9, если все цифры в каждом четырехзначном числе различны?\\
						\vspace{15pt}
						Решение:\\
						Для формирования каждого четырехзначного числа выбираем четыре цифры из
						девяти, поэтому всего таких чисел существует:
						\begin{equation}
							A_{9}^4 = \frac{9!}{(9-4)!} = 3024 
						\end{equation}
						
						Ответ: 3024
							
						\item {\large задача  }\\
						Источник: Дискретная математика. Учебник и задачник для Вузов (Баврин И.И. )\\
						\vspace{15pt}
						Нужно присудить первую, вторую и третью премии на конкурсе, в котором принимают  участие 20 человек . Сколькими способами можно распредилить эти премии?\\
						\vspace{15pt}
						Решение:\\
						Ответом на данную задачу будут являться количество размещений по 3 человека из 20:
						\begin{equation}
							A_{20}^3 = (20)\times(20-1)\times(20-2) = 20\times 19 \times 18 = 6840
						\end{equation}
						
						Ответ: 6840
						
						
						 
						 \item {\large задача }\\
						 Источник: Дагестанский государственный университет народного хозяйства учебное пособие по дисциплине "математика"  \\
						\vspace{15pt}
						 В театре 10 актеров и 8 актрис. Сколькими способами можно
						 распределить между ними роли в пьесе, в которой 5 мужских и 3
						 женские роли?\\
						 \vspace{15pt}
					
						 Решение:\\
						 Так как нам важен порядок при распределении ролей, то нам нужны размещения, тогда 
						 Ответом на данную задачу будут являться произведение размещения из 10 элементов по 5 и размещения из 8 элементов по 3:
						 \begin{equation}
						 	A_{10}^5\times A_{8}^3 = \frac{10!}{5!}\times\frac{8!}{5!}= 10160640
						 \end{equation}
						 
						 Ответ: 10160640
						 
						 \item {\large задача  }\\
						 Источник: Дискретная математика. Учебник и задачник для Вузов (Баврин И.И. )\\
						 \vspace{15pt}
						 Восемь лабораторных животных нужно проранжировать в соответствии с их способностями выполнять определенные задания. Каково число возможных ранжировок, если допустить, что одинаковых способностей нет?\\
						 \vspace{15pt}
						 Решение:\\
						 Так как одинаковых способностей нет, то тогда в комбинациях не будет повторений. 
						 Ответом на данную задачу будут являться количество перестановок из 8 лабораторных животных:
						 \begin{equation}
						 	P_{8} = 8! = 40320
						 \end{equation}
						 
						 Ответ: 40320
						 
						 \item {\large задача  }\\
						 Источник: Дискретная математика. Учебник и задачник для Вузов (Баврин И.И. )\\
						 \vspace{15pt}
						 Комитет состоит из 12 человек. Минимальный кворум на заседаниях этого комитета должен насчитывать восемь членов. Сколькими способами может достигаться минимальный кворум?\\
						 \vspace{15pt}
						 Решение:\\
						 Ответом на данную задачу будут являться количество сочетаний по 8 человека из 12:
						 \begin{equation}
						 	C_{12}^8 =\frac{12!}{8!\times4!} = 495
						 \end{equation}
						 
						 Ответ: 495
						 
						 \item {\large задача  }\\
						 Источник: Дискретная математика. Учебник и задачник для Вузов (Баврин И.И. )\\
						 \vspace{15pt}
						 В лабораторной клетке содержатся 8 белых и 6 коричневых мышей. Найдите число способов выбора пяти мышей из клетки, если они могут быть любого цвета.\\
						 \vspace{15pt}
						 Решение:\\
						 Найдем кол-во мышей в общей сложности $8+6 = 14$, тогда
						 ответом на данную задачу будут являться количество сочетаний по 5 мышей из 14 возможных:
						 \begin{equation}
						 	C_{14}^5 = \frac{14!}{9!\times5!} = 2002
						 \end{equation}
						 
						 Ответ: 2002
						 
						  \item {\large задача  }\\
						 Источник: Джеймс Андерсон: Дискретная математика и комбинаторика\\
						 \vspace{15pt}
						 Сколько существует вариантов выбора 5 карт из стандартной ко-
						 колоды, содержащей 52 карты?\\
						 \vspace{15pt}
						 Решение:\\
						 Поскольку порядок карт не имеет значения, речь
						 идет о выборе 5 объектов из 52, поэтому существует всего комбинация:
						 \begin{equation}
						 	C_{52}^5 = \frac{52!}{5!\times47!} = 2598960
						 \end{equation}
						 
						 Ответ: 2598960
						 
						 \item {\large задача  }\\
						 Источник: Джеймс Андерсон: Дискретная математика и комбинаторика\\
						 \vspace{15pt}
						 Сколькими способами можно вытянуть 5 карт трефовой масти из
						 стандартной колоды, содержащей 52 карты?\\
						 \vspace{15pt}
						 Решение:\\
						 В колоде имеется 13 треф, из которых выбираются 5, поэтому существует всего:
						 \begin{equation}
						 	C_{13}^5 = \frac{13!}{8!\times5!} = 1287
						 \end{equation}
						 
						 Ответ: 1287
						 
						  \item {\large задача  }\\
						 Источник: Джеймс Андерсон: Дискретная математика и комбинаторика\\
						 \vspace{15pt}
						 Сколькими способами можно выбрать комитет, включающий 6
						 мужчин и 8 женщин, из группы, состоящей из 12 мужчин и 20 женщин?
						 
						 19!\\
						 \vspace{15pt}
						 Решение:\\
						 Найдем кол-во способов выбора мужчин и кол-во способов выбора женщин
						 
						 \begin{align}
						 	C_{12}^6 &= \frac{12!}{6!\times6!}\\
						 	C_{20}^8 &= \frac{20!}{8!\times12!}  
						 \end{align}
						 
						 Поэтому, согласно комбинаторному принципу умножения, найдем кол-во способов выбрать комитет:
						 \begin{equation}
						 	C_{20}^8 \times C_{12}^6 = 116396280
						 \end{equation}
						 Ответ: 116396280
						 
						  \item {\large задача  }\\
						 Источник: Дагестанский государственный университет народного хозяйства учебное пособие по дисциплине "математика"\\
						 \vspace{15pt}
						 В некоторой газете 12 страниц. Необходимо на страницах этой газеты поместить четыре фотографии. Сколькими способами можно это сделать, если ни одна страница газеты не должна содержать более одной фотографии? 
						 \\
						 \vspace{15pt}
						 Решение:\\
						 В данной задаче генеральной совокупностью являются 12 страниц газеты, и выборкой без возвращения 4 выбранные из них страницы для фотографий. В данной задаче важно не только то, какие выбраны страницы, но и в каком порядке (для расположения фотографий). Таким образом, задача сводится к классической задаче об определении числа размещений без повторений из 12 элементов по 4 элемента: 
						 \begin{equation}
						 	А_{12}^4 = \frac{12!}{8!} = 11880
						 \end{equation}
						 
						 Ответ: 1287
						 
						 
						  \item {\large задача  }\\
						 Источник: задача №10 из демоверсии ЕГЭ по информатике 2016 года.\\
						 \vspace{15pt}
						 Игорь составляет таблицу кодовых слов для передачи сообщений, каждому сообщению соответствует своё кодовое слово. В качестве кодовых слов Игорь использует 5-буквенные слова, в которых есть только буквы П, И, Р, причём буква П появляется ровно 1 раз. Каждая из других допустимых букв может встречаться в кодовом слове любое количество раз или не встречаться совсем. Сколько различных кодовых слов может использовать Игорь?
						 \\
						 \vspace{15pt}
						 Решение:\\
						 Буква «П» появляется ровно 1 раз, значит она может находиться на одной из 5 позиций в слове. Буквы «И» и «Р» заполнят остальные 4 позиции. Рассмотрим выборки объема 4 из 2 элементов (k = 4, n = 2). Кодовые слова могут отличаться как порядком следования букв, так и составом, значит, комбинаторная схема – размещения с повторениями. Найдем число таких размещений:
						 \begin{equation}
						 	\overline{А_{2}^4} = 2^4 = 16
						 \end{equation}
						 После применим правило произведения: 
						 \begin{equation}
						 	5 \times 16 = 80
						 \end{equation}
						 Ответ: 80
						 
						 
						  \item {\large задача  }\\
						 Источник: Тренировочная задача №10 для подготовки к ЕГЭ по информатике\\
						 \vspace{15pt}
						Вася составляет 5-буквенные слова из четырехбуквенного алфавита {A, C, R, T}, причём буква А используется в каждом слове ровно 2 раза. Каждая из других допустимых букв может встречаться в слове любое количество раз или не встречаться совсем. Словом, считается любая допустимая последовательность букв, не обязательно осмысленная. Сколько существует таких слов, которые может написать Вася?
						 \\
						 \vspace{15pt}
						 Решение:\\
						 1) пронумеруем позиции в слове, тогда варианты расположений букв «А» можно представить в качестве неупорядоченного выбора двух цифр из пяти. Значит, комбинаторная схема - сочетания без повторений:
						 \begin{equation}
						 	С_{5}^2 = \frac{5!}{2!(5-2)!} = 10
						 \end{equation}
						 2) остальные допустимые символы будут занимать 3 позиции. Эти выборки объемом 3 из 3 элементов будут отличаться как порядком следования, так и набором символов. Очевидно, комбинаторная схема – размещения с повторениями.
						 \begin{equation}
						 	\overline{А_{3}^3} = 3^3 = 27
						 \end{equation}
						 3) применим правило произведения: 
						 \begin{equation}
						 	27 \times 10 = 270
						 \end{equation}
						 Ответ: 270
						 
						  \item {\large задача  }\\
						 Источник: Задача под номером В4 из демонстрационной версии ЕГЭ по информатике 2014 года.\\
						 \vspace{15pt}
						 Для передачи аварийных сигналов договорились использовать специальные цветные сигнальные ракеты, запускаемые последовательно. Одна последовательность ракет – один сигнал; в каком порядке идут цвета – существенно. Какое количество различных сигналов можно передать при помощи запуска ровно пяти таких сигнальных ракет, если в запасе имеются ракеты трёх различных цветов (ракет каждого вида неограниченное количество, цвет ракет в последовательности может повторяться)?
						 \\
						 \vspace{15pt}
						 Решение:\\
						 Ракеты могут быть трех различных цветов, при этом в одной последовательности пять ракет. Значит, рассматривается выборка объема пять из трех элементов (n = 3, k = 5).
						 
						 Определим комбинаторную схему. Два положения в условие задачи:
						 \begin{enumerate}
						 \item«в каком порядке идут цвета – существенно»;
						 \item«цвет ракет в последовательности может повторяться»;
						 \end{enumerate}
						 указывают на то, что – это размещения с повторениями.
						 \begin{equation}
						 	\overline{А_{3}^5} = 3^5 = 243
						 \end{equation}
						 
						 Ответ: 243
						 
						 
						 \item {\large задача  }\\
						 Источник: Задача под номером 10473 c сайта решуегэ.\\
						 \vspace{15pt}
						Шифр кодового замка представляет собой последовательность из пяти символов, каждый из которых является цифрой от 1 до 4. Сколько различных вариантов шифра можно задать, если известно, что цифра 1 встречается ровно два раза, а каждая из других допустимых цифр может встречаться в шифре любое количество раз или не встречаться совсем?
						 \\
						 \vspace{15pt}
						 Решение:\\
						 Количество способов поставить две 1 на пять позиций, можно представить как перестановку с повторениями из 2 единиц и 3 нулей, где 0 - некоторое число:
						 \begin{equation}
						 	P_{2,3} = \frac{5!}{2! \times 3!} = 10
						 \end{equation}
						 После того, как определили позиции двух 1, на оставшиеся позиции можем поставить любое из трёх чисел, это можно сделать
						 \begin{equation}
						 	\overline{А_{3}^3} = 3^3 = 27
						 \end{equation}
						 Итого всего кодов (правило умножения):
						 \begin{equation}
						 	10 \times 27 = 270
						 \end{equation}
						 Ответ: 270
						 
						 
						 
						 \item {\large задача  }\\
						 Источник: Задача под номером 10473 c сайта решуегэ.\\
						 \vspace{15pt}
						 Шифр кодового замка представляет собой последовательность из пяти символов, каждый из которых является цифрой от 1 до 5. Сколько различных вариантов шифра можно задать, если известно, что цифра 1 встречается ровно три раза, а каждая из других допустимых цифр может встречаться в шифре любое количество раз или не встречаться совсем?
						 \\
						 \vspace{15pt}
						 Решение:\\
						 Количество способов поставить три единицы на пять позиций, если дать номера позициям, тогда можно представить количество способов как количество сочетаний из 5 элементов по 3:
						 \begin{equation}
						 	С_{5}^3 = \frac{5!}{3!(5-3)!} = 10
						 \end{equation}
						 После того, как определили позиции трех 1, на оставшиеся две позиции можем поставить любое из четырех чисел, это можно сделать
						 \begin{equation}
						 	\overline{А_{4}^2} = 4^2 = 16
						 \end{equation}
						 Итого всего кодов (правило умножения):
						 \begin{equation}
						 	10 \times 16 = 160
						 \end{equation}
						 Ответ: 160
						 
						 \item {\large задача  }\\
						 Источник: Банк ФИПИ\\
						 \vspace{15pt}
						 Вася составляет 5-буквенные слова, в которых есть только буквы К, Р, А, Н, Т,  причём буква К используется в каждом слове ровно 2 раза. Каждая из других допустимых букв может встречаться в слове любое количество раз или не встречаться совсем. Словом считается любая допустимая последовательность букв, не обязательно осмысленная. Сколько существует таких слов, которые может написать Вася?
						 \\
						 \vspace{15pt}
						 Решение:\\
						 Количество способов поставить две К на пять позиций, можно представить как перестановку с повторениями из 2 единиц и 3 нулей, где 0 - некоторое число:
						 \begin{equation}
						 	P_{2,3} = \frac{5!}{2! \times 3!} = 10
						 \end{equation}
						 После того, как определили позиции двух К, на оставшиеся позиции можем поставить любое из четырех чисел, это можно сделать
						 \begin{equation}
						 	\overline{А_{4}^3} = 4^3 = 64
						 \end{equation}
						 Итого всего кодов (правило умножения):
						 \begin{equation}
						 	10 \times 64 = 640
						 \end{equation}
						 Ответ: 640
						 
						 
						 
						 \item {\large задача  }\\
						 Источник: Яндекс учебник\\
						 \vspace{15pt}
						 Какое количество четырёхзначных чисел, записанных в восьмеричной системе счисления, начинается на цифру 3 и заканчивается на цифру 0, при условии, что все цифры числа различные и никакие две чётные цифры не стоят рядом.
						 \\
						 \vspace{15pt}
						 Решение:\\
						 Так как всего в восьмиричной системе исчисления 8 цифр, тогда в две свободные ячейки мы можем поставить какую-то цифру из 3 четных, а после какую-то цифру из 3 нечетных(четные цифры не стоят рядом, значит на предпоследний разряд можно поставить только нечетное число), или поставить какую-то цифру из 3 нечетный и какую-то цифру из оставшихся 2 нечетных.
						 Тогда по правилу комбинаторного сложения и умножения получаем ответ
						 \begin{equation}
						 	3\times3 + 3\times2 = 15
						 \end{equation}
						 
						 Ответ: 15
						 
						 \item {\large задача  }\\
						 Источник: Банк ФИПИ\\
						 \vspace{15pt}
						 Сколько существует десятичных шестизначных чисел, не содержащих в своей записи цифру 3, в которых все цифры различны и никакие две чётные или две нечётные цифры не стоят рядом?
						 \\
						 \vspace{15pt}
						 Решение:\\
						 На первом месте четное число:  на первом месте - 4 числа (ноль нельзя использовать в старшем разряде), на втором месте: 4 варианта, на третьем месте: 4 (ноль возможно использовать,но одно чётное число уже использовали), на четвёртом месте: 3 (одно нечётное число использовали, и 3 нельзя использовать), на пятом месте: 3 (два числа использовали), на шестом месте: 2 числа( 3 нельзя использовать и два числа использовали). Всего: $(4\times4\times4\times3\times3\times2)$ вариантов\\
						 Случай, если на первом месте нечётное число: $(4\times5\times3\times4\times2\times3)$
						 \begin{equation}
						 	Итого: (4\times4\times4\times3\times3\times2) + (4\times5\times3\times4\times2\times3) = 2592
						 \end{equation}
						 
						 Ответ: 2592
						 
						 
						  \item {\large задача  }\\
						 Источник: Демонстрационный вариант ЕГЭ-2024.\\
						 \vspace{15pt}
						 Сколько существует восьмеричных пятизначных чисел, не содержащих в своей записи цифру 1, в которых все цифры различны и никание две чётные или две нечётные цифры не стоят рядом?
						 \\
						 \vspace{15pt}
						 Решение:\\
						 Так как всего в восьмиричной системе исчисления 8 цифр, то всего четных числе в нашем распоряжении - 4, а нечетных - 3(1 не в счет), тогда можно разделить задачу на два базовых случая: ЧНЧНЧ и НЧНЧН, тогда посчитаем кол-во для данных случаев, аналогично прошлой задаче:
						 \begin{align}
						 	1)& (3\times3\times3\times2\times2) = 108 \\
						 	2)& (3\times4\times2\times3\times1) = 72 
						 \end{align}
						 Итого\space получаем: $72 + 108 = 180$\\
						 Ответ: 180
						 
						 
						  \item {\large задача  }\\
						 Источник: одногруппники\\
						 \vspace{15pt}
						 Сколько можно составить 4-х буквенных слов из букв в словах "брак"?
						 \\
						 \vspace{15pt}
						 Решение:\\
						 \begin{equation}
						 	P_4 = 4! = 24 \space 
						 \end{equation}
						  Ответ: 180
						  
						  \item {\large задача  }\\
						 Источник: одногруппники\\
						 \vspace{15pt}
						 Пусть имеется множество букв \{A,B,C\} записать всевозможные перестановки.
						 \\
						 \vspace{15pt}
						 Ответ:\\
						 (ABC);(ACB);(BCA);(BAC);(CBA);(CAB)
						 \newpage
						 
						  \item {\large задача  }\\
						 Источник: Сайт mathprofi.net(одногруппники)\\
						 \vspace{15pt}
						 Сколько различных автомобильных номеров можно составить из 3 цифр от 1 до 3 и 9 букв русского алфавита, при условии, что номер состоит из 3-х букв и 4-ых цифр?
						 \\
						 \vspace{15pt}
						 Решение:
						 \begin{align}
						 	& (C_9^1)^3=9^3 - \text{способов выбрать 3 буквы}\\
						 	& (C_9^1)^4=9^4 - \text{способов выбрать 4 цифры для номера}\\
						 	& (9^3 \times 9^4 = 9^7) - \text{способов составления}
						 \end{align}
						 Ответ: $9^7$
						 
						  \item {\large задача }\\
						 Источник: Дагестанский государственный университет народного хозяйства учебное пособие по дисциплине "математика"  \\
						 \vspace{15pt}
						 В кондитерском магазине продавались 4 сорта пирожных:
						 наполеоны, эклеры, песочные и слоеные. Сколькими способами можно
						 купить 7 пирожных?\\
						 \vspace{15pt}
						 
						 Решение:\\
						 Очевидно, что порядок, в котором выбираются
						 пирожные, не существен, причем в комбинации могут входить повторяющиеся элементы (например, можно купить 7 эклеров).
						 Следовательно, число способов покупки 7 пирожных определяется
						 числом сочетаний с повторениями из 4 элементов по 7, т.е.
						 \begin{equation}
						 	\overline{С_4^7}=C_{4+7-1}^7=\frac{10!}{7!\times3!}=120
						 \end{equation}
						 Ответ: 120
						 \newpage
						 
						 \item {\large задача }\\
						 Источник: Дагестанский государственный университет народного хозяйства учебное пособие по дисциплине "математика"  \\
						 \vspace{15pt}
						 В технической библиотеке имеются книги по математике,
						 физике, химии и т. д., всего по 16 разделам науки. Поступили
						 очередные 4 заказа на литературу. Сколько существует вариантов
						 такого заказа?
						  \\
						 \vspace{15pt}
						 
						 Решение:\\
						 Так как 4 заказанные книги могут быть и из одного
						 раздела науки, и из разных разделов, при этом порядок выбора
						 разделов не важен, то число вариантов заказа определяется числом
						 сочетаний с повторениями из 16 элементов по 4, т. е.
						 \begin{equation}
						 	\overline{С_{16}^4}=C_{16+4-1}^4=\frac{19!}{4!\times15!}=3876
						 \end{equation}
						 Ответ: 3876
						 
						 
						 
						 \item {\large задача }\\
						 Источник: Дагестанский государственный университет народного хозяйства учебное пособие по дисциплине "математика"  \\
						 \vspace{15pt}
						 Имеются 2 буквы A, 2 буквы B, 2 буквы C. Сколькими
						 способами можно выбрать две из этих шести букв?\\
						 \vspace{15pt}
						 
						 Решение:\\
						 Существует 6 способов выбора 2 букв из 6 с
						 повторениями: (АА), (АВ), (АС), (ВС), (BB), (СС). Порядок следования
						 букв не учитывается.\\
						 Ответ: 6
						 
						 \item {\large задача }\\
						 Источник: Дагестанский государственный университет народного хозяйства учебное пособие по дисциплине "математика"  \\
						 \vspace{15pt}
						 Необходимо выбрать в подарок 4 из 10 имеющихся
						 различных книг. Сколькими способами можно это сделать?\\
						 \vspace{15pt}
						 
						 Решение:\\
						 Генеральной совокупностью является 10 различных
						 книг. Из них нужно выбрать 4, причем порядок выбора книг не играет
						 роли. Нужно найти число сочетаний из 10 элементов по 4:
						 \begin{equation}
						 	С_{10}^4 = \frac{10!}{4!\times6!} = 210
						 \end{equation}
						 Ответ: 210
						 
						 \item {\large задача }\\
						 Источник: Дагестанский государственный университет народного хозяйства учебное пособие по дисциплине "математика"  \\
						 \vspace{15pt}
						 Пусть имеется множество, содержащее 4 буквы \{А, B, С,
						 	D\}. Запишем все возможные сочетания из указанных букв по 3.\\
						 \vspace{15pt}
						 
						 Решение:\\
						 Таких сочетаний 4: ABC, ACD, ABD, BCD. Здесь в число
						 сочетаний не включены, например, АСВ, ВСА, так как они не
						 отличаются по составу от последовательности букв ABC, потому что
						 перестановка элементов нового сочетания не дает.\\
						 Ответ: 4
						 
						 \item {\large задача }\\
						 Источник: Дагестанский государственный университет народного хозяйства учебное пособие по дисциплине "математика"  \\
						 \vspace{15pt}
						 Сколько разных буквосочетаний можно сделать из букв
						 слова «Миссисипи»?\\
						 \vspace{15pt}
						 \newpage
						 Решение:\\
						 Здесь 1 буква «м», 4 буквы «и», 3 буквы «с» и 1 буква
						 «п», всего 9 букв.
						 Следовательно, число перестановок с повторениями равно
						 \begin{equation}
						 	P_{1,4,3,1} = \frac{9!}{1! \times 4! \times 3! \times 1!} = 2520
						 \end{equation}
						 
						 Ответ: 2520
						 
						 
						 
						 \item {\large задача }\\
						 Источник: Дагестанский государственный университет народного хозяйства учебное пособие по дисциплине "математика"  \\
						 \vspace{15pt}
						 Сколькими способами можно расставить девять
						 различных книг на полке, чтобы определенные четыре книги стояли
						 рядом?\\
						 \vspace{15pt}
						 
						 Решение:\\
						 В исходной генеральной совокупности - 9 разных книг.
						 Будем считать выделенные 4 книги за одну. Тогда для остальных 6
						 книг существует $P_6 = 6! = 720$ перестановок. Однако четыре
						 определенные книги можно переставить между собой $P_4 = 4! = 24$
						 способами. По правилу умножения имеем $P_6\times P_4 = 720 \times 24 = 17280$
						 
						 Ответ: 17280
						 
						 
						 \item {\large задача }\\
						 Источник: Дагестанский государственный университет народного хозяйства учебное пособие по дисциплине "математика"  \\
						 \vspace{15pt}
						 У мальчика остались от набора для настольной игры
						 штампы с цифрами 1, 3 и 7. Он решил с помощью этих штампов
						 нанести на все книги пятизначные номера – составить каталог. Сколько
						 различных пятизначных номеров может составить мальчик?\\
						 \vspace{15pt}
						 \newpage
						 Решение:\\
						 Можно считать, что опыт состоит в 5-кратном выборе с
						 возращением одной из 3 цифр \{1, 3, 7\}. Таким образом, число
						 пятизначных номеров определяется числом размещений с
						 повторениями из 3 элементов по 5:
						 \begin{equation}
						 	\overline{А_{3}^5} = 3^5 = 243
						 \end{equation}
						 
						 Ответ: 243
						 
						 \item {\large задача }\\
						 Источник: Дагестанский государственный университет народного хозяйства учебное пособие по дисциплине "математика"  \\
						 \vspace{15pt}
						 В некоторой газете 12 страниц. Необходимо на страницах
						 этой газеты поместить четыре фотографии. Сколькими способами
						 можно это сделать, если ни одна страница газеты не должна содержать
						 более одной фотографии?\\
						 \vspace{15pt}
				
						 Решение:\\
						 В данной задаче генеральной совокупностью являются
						 12 страниц газеты, и выборкой без возвращения 4 выбранные из них
						 страницы для фотографий. В данной задаче важно не только то, какие
						 выбраны страницы, но и в каком порядке (для расположения
						 фотографий). Таким образом, задача сводится к классической задаче об
						 определении числа размещений без повторений из 12 элементов по 4
						 элемента:
						 \begin{equation}
						 	А_{12}^4 = \frac{12!}{(12-4)!} = 11880
						 \end{equation}
						 
						 Ответ: 11880
						 
						 \newpage
						  \item {\large задача  }\\
						 Источник: Джеймс Андерсон: Дискретная математика и комбинаторика\\
						 \vspace{15pt}
						 К несчастью, судья на выставке цветов не разбирается в орхидеях. Если он
						 выбирает победителей случайным образом среди 18 участниц, то сколько
						 имеется способов вручить первый, второй и третий приз?\\
						 \vspace{15pt}
						 Решение:\\
						 Можно выдать первый приз кому-то из 18 участников, второй приз (18-1) участникам и т.д.:
						 \begin{equation}
						 	18 \times 17 \times 16 = 4896
						 \end{equation}
						 
						 Ответ: 4896
						 
						 
						  \item {\large задача  }\\
						 Источник: Джеймс Андерсон: Дискретная математика и комбинаторика\\
						 \vspace{15pt}
						 Сколько существует способов рассадить за круглым столом пятерых мужчин
						 и пятерых женщин, если двое мужчин не должны сидеть рядом?
						 \\
						 \vspace{15pt}
						 Решение:\\
						 Возьмем случайную женщину за опорный элемент и будем садить других женщин через одну позицию от опорной, тогда кол-во способов рассадить их равно: $P_4 = 24$. Оставшиеся места заполняем мужчинами, тогда кол-во способов рассадить мужчин будет равно:
						 $P_5 = 120$, тогда общее кол-во способов равно: $P_5 \times P_4 = 24 \times 120 = 2880$\\
						 Ответ: 2880
						 
						 
						  \item {\large задача  }\\
						 Источник: Джеймс Андерсон: Дискретная математика и комбинаторика\\
						 \vspace{15pt}
						 Сколько существует способов составить комитет из 6 мужчин и 7 женщин,
						 если организация состоит из 15 мужчин и 20 женщин?
						 \\
						 \vspace{15pt}
						 Решение:\\
						 Кол-во способов составить комитет по 6 мужчин из 15: \begin{equation}
						 	C_{15}^6 = \frac{15!}{9!\times6!} = 5005
						 \end{equation}
						 Кол-во способов составить комитет по 7 женщин из 20: \begin{equation}C_{20}^7 = \frac{20!}{7!\times13!} = 77520\end{equation}
						 Всего кол-во способов составить комитет: \begin{equation}C_{20}^7 \times C_{15}^6 = 387987600\end{equation}
						 Ответ: 387987600
						 
						 
						 
						 
						 
						  \item {\large задача  }\\
						 Источник: Джеймс Андерсон: Дискретная математика и комбинаторика\\
						 \vspace{15pt}
						 Пять пар идут в кино. Сколькими способами они могут занять места, если они могут сидеть в любом порядке?
						 \\
						 \vspace{15pt}
						 Решение:\\
						 Всего 10 человек, а значит ответом будет кол-во перестановок из 10 элементов
						 \begin{equation}P_{10} = 10! = 3628800\end{equation}
						 Ответ: 3628800
						 
						  \item {\large задача  }\\
						 Источник: Джеймс Андерсон: Дискретная математика и комбинаторика\\
						 \vspace{15pt}
						 Пять пар идут в кино. Сколькими способами они могут занять места, если все пять пар сидят подряд?
						 \\
						 \vspace{15pt}
						 Решение:\\
						 Пару можно разделить на человека 1 и на человека 2, пусть 1 - человек 1 сидит на первом месте, а 2 - человек 2 сидит на втором месте, тогда всего вариантов пар у нас будет
						 \begin{equation}\overline{A_2^5} = 2^5=32\end{equation}
						 Так как пары можно еще между собой перетасовывать, тогда получаем кол-во всех способов занять места в данном случае
						 \begin{equation}\overline{A_2^5} \times P_5 = 5!\times 32 = 3840\end{equation}
						 Ответ: 3840
						 
						 
						  \item {\large задача  }\\
						 Источник: Джеймс Андерсон: Дискретная математика и комбинаторика\\
						 \vspace{15pt}
						 Сколько трехзначных чисел можно образовать, используя цифры 2, 3, 4, 5, 6, 8 и 9?
						 \\
						 \vspace{15pt}
						 Решение:\\
						 Кол-во способов выбрать первое число - 7, второе 7-1, третье - 7-2, тогда
						 всего:
						 \begin{equation}7\times 6 \times 5 = 210\end{equation}
						 Ответ: 210
						 
						 \item {\large задача  }\\
						 Источник: Джеймс Андерсон: Дискретная математика и комбинаторика\\
						 \vspace{15pt}
						 Сколькими способами девять человек могут расположиться в ряд?
						 \\
						 \vspace{15pt}
						 Решение:\\
						 Кол-во способов есть количество перестановок из 9 элементов:
						 \begin{equation}P_9 = 9!\end{equation}
						 Ответ: 362880
						 
						 \item {\large задача  }\\
						 Источник: Джеймс Андерсон: Дискретная математика и комбинаторика\\
						 \vspace{15pt}
						 В скачках участвуют десять лошадей. Сколько существует вариантов
						 призовой тройки лошадей?
						 \\
						 \vspace{15pt}
						 Решение:\\
						 Кол-во способов есть количество сочетаний из 10 элементов по 3:
						 \begin{equation}C_{10}^3 = \frac{10!}{3! \times 7! }= 120\end{equation}
						 Ответ: 120
						 
				\end{enumerate}
			\end{center}
			%«Дискретная математика»\\
			
			%\vfill



	
	




\end{document}