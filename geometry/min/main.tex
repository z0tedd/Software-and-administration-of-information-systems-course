\documentclass{article}
\usepackage{amsmath}
\usepackage{pgfplots}
\usepackage[left=3cm,right=1.5cm,
top=2cm,bottom=2cm,bindingoffset=0cm]{geometry}
\usepackage{tikz}
\usepackage[T2A]{fontenc}
\usepackage[utf8]{inputenc}
\usepackage[russian]{babel}


% \section*{Задача 5}
%
% \begin{minipage}[t]{0.45\textwidth}
%
% \textbf{Дано:}
% \textbf{Найти: }
% \end{minipage}
% \begin{minipage}[t]{0.45\textwidth}
%   \vspace{-\baselineskip} % Required for vertically aligning minipages
%
% \begin{center}
% \begin{tikzpicture}
%     \draw[thick] (0,0) node[left] {$A$} -- (2,2) node[above] {$B$} -- (4,0) node[right] {$C$} -- cycle;
%     \draw[dashed] (2,2) -- (2,0) node[below] {$H$};
% \end{tikzpicture}
% \end{center}
% \end{minipage}
% \\
% \textbf{Решение:}
% \begin{enumerate}
% \end{enumerate}
%
% \textbf{Oтвет: }2 
\begin{document}
\section*{Задача 1}
\begin{minipage}[t]{0.45\textwidth}
\textbf{Дано:}\\
A(1, 2)\\
B(4, 4)\\
C(2, -2)\\
\textbf{Составить:} ур-ние медианы треугольника $ABC$, проходящую через вершину $A$.
\end{minipage}
\begin{minipage}[t]{0.45\textwidth}
	\vspace{-\baselineskip} % Required for vertically aligning minipages
\begin{center}
\begin{tikzpicture}
    \draw[thick] (0,0) node[left] {$C$} -- (2,2) node[above] {$A$} -- (4,0) node[right] {$B$} -- cycle;
    \draw[dashed] (2,2) -- (2,0) node[below] {$M$};
\end{tikzpicture}
\end{center}
\end{minipage}
\\
\textbf{Решение:}
\begin{enumerate}

\item Пусть $AM$ -- медиана, тогда точка $M$ -- середина отрезка $BC$, значит:
\[
M\left(\frac{4 + 2}{2}, \frac{4 + (-2)}{2}\right) = M(3, 1)
\]

\item $\overline{AM}: \{2, -1\}$, $ M(3,1)$ $\Rightarrow AM: \frac{x-3}{2} = \frac{y-1}{-1}$
\end{enumerate}
\textbf{Oтвет:} $\frac{x-3}{2} = \frac{y-1}{-1}$
%%%%%%%%%%%%%%%%%%%%%%%%%%%%%%%%%%%%%%%%
\section*{Задача 2}
\begin{minipage}[t]{0.45\textwidth}
   
\textbf{Дано:}

$M(8, 11)$\\
$l: 2x + 3y + 3 = 0$\\


\textbf{Найти:} точку симметричную $M$ относительно $l$.\\

\end{minipage}
\begin{minipage}[t]{0.45\textwidth}
	\vspace{-\baselineskip} % Required for vertically aligning minipages

\begin{center}
\begin{tikzpicture}
    \draw[thick] (0,-1) -- (3,1.5) node[right] {$e$};
    \fill (1,1) circle (2pt) node[above] {$M$};
    \fill (2,0) circle (2pt) node[below] {$M'$};
\end{tikzpicture}
\end{center}
\end{minipage}
\\
\textbf{Решение:}
\begin{enumerate}
  \item Из уравнения прямой $l$ получаем вектор нормали:
\[
  2x + 3y + 3 = 0 \Rightarrow \overline{n} \{2, 3\}
\]

\item Вектор нормали будет являться направляющим вектором к\\ прямой $MM'$,где  $M'$ - искомая точка.
  \[
    \text{Ур-ние $MM`$:} 
    \begin{cases}
       2t + 8 = x\\ 
       3t + 11 = x
    \end{cases}
    \Rightarrow 
    \begin{cases}
      t = \frac{x}{2} - 4\\ 
      t= \frac{x}{3} - \frac{11}{3}
    \end{cases}
    \Rightarrow 
    x/2 - y/3 - 1/3 = 0
  \]
  ММ`: 3x - 2y - 2 = 0 
  \item Найдем О - т. пересечения MM` и l: 
    \[
      \begin{cases}
         3x - 2y - 2 =0\\
         2x + 3y + 3 =0
      \end{cases} \Rightarrow 
      \left(\begin{array}{rr|r}
          2 & 3 & -3\\
          3 &-2 & 2

      \end{array}\right)\Rightarrow 
\left(\begin{array}{rr|r}
          -1 & 5 & -5\\
          3 &-2 & 2
\end{array}\right)\Rightarrow 
\left(\begin{array}{rr|r}
          -1 & 5 & -5\\
          0 &13 & -13
      \end{array}\right)\Rightarrow
    \] 
    \[
      \Rightarrow
      \begin{cases}
         y = -1\\
         5y + 5 = x
      \end{cases}\Rightarrow
      O(0;-1)
    \]
  \item OM = OM' И M' $\in$ МM'
\[
    \begin{cases}
         3x - 2y - 2 =0\\
         8^2 + 12^2 = (y+1)^2 + x^2
      \end{cases}
      \Rightarrow
 \begin{cases}
   x = \frac{2}{3}(y+1)\\
   \frac{13}{9}(y+1)^2 = 208
      \end{cases}
      \Rightarrow
      \begin{cases}
   x = \frac{2}{3}(y+1)\\
         y +1 = \pm 12 
      \end{cases}
      \Rightarrow
      M'(-8;13)
\]
\end{enumerate}

\textbf{Oтвет:} $M'(-8;13)$

\section*{Задача 3}

\begin{minipage}[t]{0.45\textwidth}
   
\textbf{Дано:}
A(-2,3)\\ 
B(7,-3)\\ 
C(4,8)\\
\textbf{Составить:}уравнение высоты 
треугольника $ABC$, проходящего через вершину B.\\
\end{minipage}
\begin{minipage}[t]{0.45\textwidth}
	\vspace{-\baselineskip} % Required for vertically aligning minipages

\begin{center}
\begin{tikzpicture}
    \draw[thick] (0,0) node[left] {$A$} -- (2,2) node[above] {$B$} -- (4,0) node[right] {$C$} -- cycle;
    \draw[dashed] (2,2) -- (2,0) node[below] {$H$};
\end{tikzpicture}
\end{center}
\end{minipage}
\\
\textbf{Решение:}
\begin{enumerate}
  \item $\overline{AC}(6,5) \Rightarrow \overline{AC} \perp \overline{BH}$
  \item Ур-ние прямой через точку и вектор нормали:
    \[
      \textbf{Ответ - BH: } (x-7)6 + (y+3)5 = 0 
    \]
\end{enumerate}


\section*{Задача 4}

\begin{minipage}[t]{0.45\textwidth}
   
\textbf{Дано:}
A(1,1)\\ 
l: x-y-2=0\\ 
\textbf{Найти: } S\\
\end{minipage}
\begin{minipage}[t]{0.45\textwidth}
	\vspace{-\baselineskip} % Required for vertically aligning minipages

\begin{center}
\begin{tikzpicture}
    \draw[thick] (0,0) -- (0,2) -- (2,2) -- (2,0) -- cycle;
    \draw[thick] (-0.75,2) -- (0,2) -- (2,2) -- (2.75,2);
    \draw[thick] (0,-0.75) -- (0,0) -- (0,2) -- (0,2.75);
     \node[below left] at (0,0) {A};
    \node[below right] at (2,0) {D};
    \node[above left] at (0,2) {B};
    \node[above right] at (2,2) {C};
    \node[left] at (0,1) {b};
    \node[above] at (1,2) {l};
\end{tikzpicture}
\end{center}
\end{minipage}
\\
\textbf{Решение:}
\begin{enumerate}
  \item По уравнению прямой становится ясно, что 
    точка А не лежит на l.
  \item Из уравнения l найдем вектор нормали к данной прямой,
    он будет являтся направляющим вектором некоторой прямой. На этой прямой
    будет лежать точка А, а также сторона квадрата. \\ 
    $\overline{n}(1,-1)$
  \item Найдем ту самую, некоторую прямую, и обозначим ее b. 
    Для простоты вычисления возьмем вектор нормали к b (1,1)\\ 
  $b: 1(x-1) + 1(y-1) =0 \Leftrightarrow x+y-2 = 0$
\item Найдем еще одну вершину квадрата она будет лежать в
  пересечении этих прямых, назовем ее B.  
  \[
  \begin{cases}
    x-y-2=0\\ 
    x+y-2 = 0\\ 
  \end{cases}
  \Rightarrow \text{B: }
\begin{cases}
  x = 2 
  y = 0 
\end{cases}
  \]
\item Тогда длина стороны квадрата равна $|\overline{AB}| = \sqrt{1+1} = \sqrt{2}$. 
  Тогда площадь равна $\sqrt{2}^2 = 2$ 
  
\end{enumerate}

\textbf{Oтвет: }2 

\section*{Задача 5}

\begin{minipage}[t]{0.45\textwidth}
   
\textbf{Дано:} \\ 
A(-1,-1)\\ 
B(5,-3)\\
C(-3,-3)\\
\textbf{Составить ур-ние: } средней линии $ABC$, параллельной $BC$
\end{minipage}
\begin{minipage}[t]{0.45\textwidth}
	\vspace{-\baselineskip} % Required for vertically aligning minipages

\begin{center}
\begin{tikzpicture}
    \draw[thick] (0,0) -- (2,2) -- (4,0) -- cycle;
    \draw[thick] (0,1) -- (1,1) -- (3,1) -- (4,1);
    \node[below left] at (0,0) {B};
    \node[below right] at (4,0) {C};
    \node[above] at (2,2) {A};
    \node[above left] at (1,1) {M};
    \node[above right] at (3,1) {N};
\end{tikzpicture}
\end{center}
\end{minipage}
\\
\textbf{Решение:}
\begin{enumerate}
  \item Найдем $\overline{MN}$, где M и N - середины AB и AC \\ 
    \[
      M\{\frac{-1+5}{2}, \frac{-1-3}{2}\}, N\{\frac{-1-3}{2} ,\frac{-1-3}{2}\} \Rightarrow \overline{MN}\{-4,0\}
    \]
  \item Найдем ур-ние прямой MN\\
    \[
      MN: 
      \displaystyle\frac{x-2}{-4} = \displaystyle\frac{y+2}{0} \Leftrightarrow 
      (x-2)*0 = (y+2)*(-4) \Rightarrow y=-2 
    \]
\end{enumerate}

\textbf{Oтвет: } y=-2


\section*{Задача 6}

\begin{minipage}[t]{0.45\textwidth}

\textbf{Дано:}\\ 
A(-5,1)\\ 
B(4,-2)\\ 
C(3,0)\\ 
D(-3,-3)\\ 
\textbf{Найти: } пересечение AB и CD
\end{minipage}
\begin{minipage}[t]{0.45\textwidth}
  \vspace{-\baselineskip} % Required for vertically aligning minipages

\begin{center}
\begin{tikzpicture}
  \draw[thick] (0,0) node[above] {A} -- (5,3) node[above]{B};
  \draw[thick] (0,3) node[above] {C}-- (5,0) node[above]{D};
  \node[above] at (intersection of 0,0 -- 5,3 and 0,3 -- 5,0) {M};
\end{tikzpicture}
\end{center}
\end{minipage}
\\
\textbf{Решение:}
\begin{enumerate}
  \item  $\overline{AB}(9,-3) \Rightarrow AB: \displaystyle\frac{x+5}{9} = \displaystyle\frac{y-1}{-3}
    \Leftrightarrow AB: -3x-9y -6 = 0$
  \item  $\overline{CD}(-6,-3) \Rightarrow CD: \displaystyle\frac{x-3}{-6} = \displaystyle\frac{y-0}{-3}
    \Leftrightarrow CD: -3x+ 6y +9 = 0$
  \item Найдем пересечение данных уравнений прямых:
    \[
      \begin{cases}
        -3x - 9y - 6 = 0 \\ 
        -3x + 6y + 9 = 0 
      \end{cases}
      \Leftrightarrow
      \begin{cases}
       x+3y + 2 = 0\\ 
       x-2y -3 = 0
      \end{cases}
      \Rightarrow 5y = -5 \Rightarrow 
      \begin{cases}
        y = -1 \\ 
        x = 1
      \end{cases}
    \]
\end{enumerate}
\textbf{Oтвет: }(1,-1) 

\section*{Задача 7}

\begin{minipage}[t]{0.45\textwidth}

\textbf{Дано:}\\
P(9,0,-2)\\ 
a: 4x - 6y - z + 15 = 0\\ 
\textbf{Найти: } точку симметричную данной отн. а 
\end{minipage}
\begin{minipage}[t]{0.45\textwidth}
  \vspace{-\baselineskip} % Required for vertically aligning minipages

\begin{center}
\begin{tikzpicture}
  \draw[thick] (0,0) node[above] {$\alpha$} -- (2,1) -- (5,1) -- (3,0) -- cycle;
  \draw[dashed] (2.25,0) -- (2.25, 0.75);
  \draw[thick] (2.25, -0.5) node[right] {P'} -- (2.25,0) ; 
  \draw[thick] (2.25, 2) node[right] {P} -- (2.25,0.75) ; 
    \fill (2.25,0.75) circle (0.5pt) node[right] {$M$};
\end{tikzpicture}
\end{center}
\end{minipage}
\\
\textbf{Решение:}
\begin{enumerate}
  \item a $\Rightarrow \overline{n}(4,-6,-1) $ - вектор нормали 
  \item Найдем прямую PM, где PM $\perp$ a, а M - т.пересечения PM и а
  \[
    PM: \frac{x-9}{4} = \frac{y-0}{-6 } = \frac{z+2}{-1}
  \] 
\item Найдем точку M $\in$ a: 
  \[
    \begin{cases}
      \frac{y}{-6} = \frac{x-9}{4}\\ 
      \frac{y}{-6} = \frac{z+2}{-1}\\ 
     4x - 6y - z + 15 = 0\\ 
    \end{cases}
    \Leftrightarrow 
    \begin{cases}
      x = -\frac{2}{3}y + 9\\ 
      z = \frac{1}{6}y - 2 \\ 
      4x - 6y - z + 15 = 0\\ 
    \end{cases}
    \Rightarrow -\frac{8}{3}y + 36 - 6y - \frac{1}{6}y + 2 + 15 = 0| \cdot 6
  \]
  \[
    -16y + 6\cdot 36 -36y - y + 12 + 15\cdot 6 = 0 \Leftrightarrow -53y = -6\cdot 53 \Rightarrow M:
    \begin{cases}
      x = 5\\ 
      y = 6\\ 
      z = -1\\
    \end{cases}
  \]
\item точка M - середина отрезка PP', где P'(x,y) - точка симметричная данной, тогда:
\[
  \begin{cases}
    \frac{x+9}{2} = 5\\ 
    \frac{y+0}{2} = 6\\ 
    \frac{z-2}{2} = -1\\
  \end{cases}
  \Rightarrow P': 
  \begin{cases}
    x = 1\\ 
    y = 12\\ 
    z = 0
  \end{cases}
\]
\end{enumerate}

\textbf{Oтвет: }P'(1,12,0)

\section*{Задача 8}

\begin{minipage}[t]{0.45\textwidth}

\textbf{Дано:}\\ 
OY $\Rightarrow$ M(0,y,0)\\ 
$\alpha:$ 3x-6y -2z+6 = 0\\ 
$\rho(M,\alpha)$ = 6\\
\textbf{Найти: }M
\end{minipage}
\begin{minipage}[t]{0.45\textwidth}
  \vspace{-\baselineskip} % Required for vertically aligning minipages

\begin{center}
\begin{tikzpicture}
  \draw[thick] (0,0) node[above] {$\alpha$} -- (2,1) -- (5,1) -- (3,0) -- cycle;
  \draw[dashed] (2.25,0) -- (2.25, 0.75);
  \draw[thick] (2.25, -0.5) node[right] {$M_1$} -- (2.25,0) ; 

  \draw[thick] (2.25, 2) node[right] {$M_2$} -- (2.25,0.75) ; 
\end{tikzpicture}
\end{center}
\end{minipage}
\\
\textbf{Решение:}
\begin{enumerate}
  \item Расстояние от точки до плоскости находится по формуле:
    \[
      \rho((x_0,y_0,z_0),ax+by+cz+d) = \frac{|ax_0+by_0+cz_0+d|}{\sqrt{a^2+b^2+c^2}}
    \]
  \item Подставим значения под формулу:
    \[
      6 = \frac{|-6y + 6|}{7} \Rightarrow \left[\begin{aligned}
        -&6y+6 = 42\\ 
        &6y -6 = 42
      \end{aligned} \right.
\Rightarrow \left[\begin{aligned}
      &M_1(0,-6,0)\\ 
      &M_2(0,6,0)
      \end{aligned} \right.
    \]
\end{enumerate}

\textbf{Oтвет: } 
$\left[\begin{aligned}
        &M_1(0,-6,0)\\ 
        &M_2(0,6,0)
      \end{aligned} \right.$ 


\section*{Задача 9}

\begin{minipage}[t]{0.45\textwidth}

\textbf{Дано:}\\ 
\begin{math}
  \alpha: 2x-3y-2z + 3 = 0\\ 
  \beta: x+4y - 4 = 0\\ 
\end{math}

\textbf{Найти: } Уравнения плоскостей, делящих пополам двугранные углы, образованные данными плоскостями.
\end{minipage}
\begin{minipage}[t]{0.45\textwidth}
  \vspace{-\baselineskip} % Required for vertically aligning minipages

\begin{center}
\begin{tikzpicture}
\begin{axis}[domain=-1:1,y domain=-1:1]
        \addplot3[surf, opacity=0.35] {0};
        \addplot3[surf, opacity=0.35] {x};
        \addplot3[surf, opacity=0.35] {x*0.5};
        %\addplot3[surf, opacity=0.35] {x*(-1.5)};
        \addplot3+[mark=none,thick]({0},{y},{0});
    \end{axis}
\end{tikzpicture}
\end{center}
\end{minipage}
\\
\textbf{Решение:}
\begin{enumerate}
  \item Расстояние от точки до плоскости вычисляется по формуле:
    \[
      d = \frac{|ax_0+by_0 + cz_0 + d|}{\sqrt{a^2 + b^2 + c^2}}  
    \]
  \item Если плоскость является биссектрисой двугранного угла,
    тогда все ее точки равноудаленны от плоскостей, образующих двугранный угол:
    \[
      \frac{|2x - 3y - 2z + 3|}{\sqrt{4+9+4}} = \frac{|x+4y - 4|}{\sqrt{1+4^2}}
      \Leftrightarrow |2x - 3y - 2z + 3| = |x+4y-4| \Rightarrow 
      \left[
        \begin{aligned}
         &2x-3y-2z + 3 = x+4y-4\\ 
         &2x-3y-2z + 3 = -x-4y+4
        \end{aligned}
      \right. \Leftrightarrow 
    \]\[ \Leftrightarrow
 \left[
        \begin{aligned}
         &\phi_1: x-7y+7 = 0\\ 
         &\phi_2: 3x+y-2z - 1 = 0 
        \end{aligned}
      \right. 
    \]
\end{enumerate}

\textbf{Oтвет: }\begin{math} 
 \left[
        \begin{aligned}
         &\phi_1: x-7y+7 = 0\\ 
         &\phi_2: 3x+y-2z - 1 = 0 
        \end{aligned}
      \right. 
\end{math}
 
\section*{Задача 10}

\begin{minipage}[t]{0.45\textwidth}

\textbf{Дано:}\\ 
A(5,-9,-2)\\ 
B(3,-6,-1)\\ 
C(4,-9,-1)\\ 
$a: \frac{x}{3} = \frac{y+5}{-6} = \frac{z-4}{-1} $\\ 
\textbf{Выяснить: }взаимное расположение прямой и плоскости, 
проходящей через 3 точки. 
\end{minipage}
\begin{minipage}[t]{0.45\textwidth}
  \vspace{-\baselineskip} % Required for vertically aligning minipages

\begin{center}
\begin{tikzpicture}
    \draw[thick] (0,0, 0) -- (2,0,0)  -- (2,0,2) -- (0,0,2) node[left] {$\alpha$}  -- cycle;
    \draw[thick] (0,1,3)  -- (3,1,0) node[below] {$a$};
\end{tikzpicture}
\end{center}
\end{minipage}
\\
\textbf{Решение:}
\begin{enumerate}
  \item Уравнение плоскости через 3 точки находится через матрицу вида:
    \[
      \begin{vmatrix}
        x-x_0 &y-y_0 &z-z_0\\ 
        x_1-x_0 &y_1-y_0 &z_1-z_0\\ 
        x_2-x_0 &y_2-y_0 &z_2-z_0\\ 
      \end{vmatrix}
    \]
  \item Уравнение плоскости $ \alpha $ через 3 данные точки имеет вид:
    \[
      \begin{vmatrix}
       x-5 &y+9 &z+2\\ 
       -2 &3 &1\\ 
       -1 &0 &1 
      \end{vmatrix}
    \Leftrightarrow 3(x-5) + (-2+1)(y+9) + 3(z+2) = 0 \Leftrightarrow\]\[
      \Leftrightarrow 3x+y+3z=0
    \]
  \item Проверим пересекаются ли прямая и плоскость:
    \[
      \begin{cases}
        \frac{x}{3} = \frac{y+5}{-6} = \frac{z-4}{-1}\\
        3x+y+3z=0
      \end{cases}
      \Leftrightarrow
      \begin{cases}
       y = 6z -29\\ 
       x = -3z+12\\ 
       3x + y + 3z = 0 
      \end{cases}
      \Rightarrow
      -9z + 36 +6z - 29 + 3z = 0 \Leftrightarrow
    \]
    7 = 0 - неверно, значит прямая и плоскость параллельны.
\end{enumerate}
\textbf{Oтвет: }Параллельны
\section*{Задача 11}

\begin{minipage}[t]{0.45\textwidth}

\textbf{Дано:}\\ 
a: \begin{math}
  \begin{cases}
    
  4x - 6y - z - 49 = 0 \\ 
  5x - 8y -2z - 65 = 0
  \end{cases}
\end{math}
b: \begin{math}
  \begin{cases}
    x = 3 + 4t\\ 
    y = -6 + 3t\\ 
    z = -1 - 2t
  \end{cases}
\end{math}\\ 
\textbf{Выяснить: }взаимное расположение прямых\\
\end{minipage}
\begin{minipage}[t]{0.45\textwidth}
  \vspace{-\baselineskip} % Required for vertically aligning minipages

\begin{center}
\begin{tikzpicture}
  \draw[thick] (-2,0) node[above] {a} -- (2,0) node[below] {$b$};
\end{tikzpicture}
\end{center}
\end{minipage}
\\
\textbf{Решение:}
\begin{enumerate}
  \item Найдем вектора нормали к обр. плоскостям прямой а,
    произведение из векторов нормали есть направляющий вектор
    к прямой а. 
    \[
      \begin{cases}
          4x - 6y - z - 49 = 0 \\ 
  5x - 8y -2z - 65 = 0

\end{cases} \Rightarrow 
\begin{aligned}
  n_1(4,-6,-1)\\ 
  n_2(5,-8,-2)
\end{aligned} \Rightarrow l_a = [n_1,n_2] = 
\begin{vmatrix}
  i &j &k\\ 
  4 &-6 &-1\\ 
  5 &-8 &-2 
\end{vmatrix} = 4i+3j-2k \Rightarrow l_a(4,3,-2)
    \]
  \item Приведем прямую b к каноническому виду, получим точку на этой прямой и напр. вектор. 
\[
   \begin{cases}
    x = 3 + 4t\\ 
    y = -6 + 3t\\ 
    z = -1 - 2t
  \end{cases}
\Leftrightarrow 
\begin{cases}
  \frac{x-3}{4} = t\\ 
  \frac{y+6}{3} = t\\ 
  \frac{z+1}{-2} = t 
\end{cases}
\Rightarrow l_b(4,3,-2), M(3,-6,1) \in b 
\]
Так как $l_b$ = $l_a$, тогда прямые коллинеарны. 
\item Подставим точку M в уравнение прямой а:
  \[
    \begin{cases}
      12= 36+1 - 49 = 0 \text{ - верно}\\
      15 + 48 + 2 - 65 = 0  \text{ - верно}
    \end{cases}
    \Rightarrow \text{Прямые совпадают}
  \]
  
\end{enumerate}

\textbf{Oтвет: }Прямые совпадают.

\section*{Задача 12}

\begin{minipage}[t]{0.45\textwidth}

\textbf{Дано:}\\ 
a: \begin{math}
  \begin{cases}
    3x+ 8y + 5z + 10 = 0\\ 
    5x - 4y + z + 2 = 0
  \end{cases}
\end{math}\\
b: \begin{math}
  \begin{cases}
    \frac{x-3}{2} = \frac{y+6}{3} = \frac{z+4}{-6}
  \end{cases}
\end{math}\\

\textbf{Найти: }расстояние между прямыми
\end{minipage}
\begin{minipage}[t]{0.45\textwidth}
  \vspace{-\baselineskip} % Required for vertically aligning minipages

\begin{center}
\begin{tikzpicture}
  \draw[thick] (-2,0) node[above] {a} -- (2,0) ;
  \draw[thick] (-2,-1) -- (2,-1) node[below] {$b$};
\end{tikzpicture}
\end{center}
\end{minipage}
\\
\textbf{Решение:}
\begin{enumerate}
  \item Получим направляющий вектор из прямой а ($[\overline{n_1},\overline{n_2}] = l_a$):
    \[
      l_a: 
      \begin{vmatrix}
        i &j &k \\ 
        3 &8 &5\\ 
        9 &-4 &1
      \end{vmatrix}
    = 28i + 42j -84k \Rightarrow l_a(28,42,-84)~ l_a(2,3,-6)\]
  \item Получим точку $M_a(x,y,z) \in a$ путем обнуления координаты x в системе уравнений:
    \[
      \begin{cases}
        8y + 5z + 10 = 0\\ 
        -4y + z + 2 = 0 
      \end{cases}
    \Rightarrow y = 0 \Rightarrow z = -2 \Rightarrow M_a(0,0,-2)\]
  \item Проанализиров уравнение прямой b, получим $l_b(2,3,-6), M_b(3,-6,-4) \in b$. Направляющие
    вектора прямой а и прямой b совпадают, а значит считать расстояние между прямыми нужно через 
    векторное произведение $M_aM_b()$ и $l_b$. Напишем общее уравнение для вычисление расстояния между
    параллельными прямыми:
    \[
      \rho(a,b) = \frac{|[M_aM_b, l_b]|}{|l_b|} 
    \]
  \item Решим по последней формуле, выполнив промежуточные вычисления:
    \[
      \begin{Vmatrix}
        i &j &k\\ 
        3 &-6 &-2\\ 
        2 &3 &-6
      \end{Vmatrix} = |42i + 14j + 21k| = \sqrt{42^2 + 14^2 + 21^2} = 49\]\[
      \rho(a,b) = \frac{49}{\sqrt{2^2+3^2+(-6)^2}} = 7 
    \]
\end{enumerate}

\textbf{Oтвет: }7


\section*{Задача 13}



\begin{minipage}[t]{0.45\textwidth}

\textbf{Дано:}\\ 

a: \begin{math}
  \begin{cases}
    5x-3y+4z-35 = 0\\ 
    7x - 3y + 5z - 49 = 0 
  \end{cases}
\end{math}\\
b: \begin{math}  
\frac{x-3}{9} = \frac{y-3}{2} = \frac{z+2}{-1}
\end{math}\\
\textbf{Найти: }угол между прямыми
\end{minipage}
\begin{minipage}[t]{0.45\textwidth}
  \vspace{-\baselineskip} % Required for vertically aligning minipages

\begin{center}
\begin{tikzpicture}
  \draw[thick] (0,0,0) -- (0,0,2) node[below] {a};
  \draw[thick, rotate around y = 60] (0,1,0) -- (0,1,2) node[below] {b};
\end{tikzpicture}
\end{center}
\end{minipage}
\\
\textbf{Решение:}
\begin{enumerate}
  \item Найдем направляющий вектор к прямой а:
    \[
      l_a = [n_1(5,-3,4),n_2(7,-3,5)] = 
      \begin{vmatrix}
        i &j &k\\ 
        5 &-3 &4\\ 
        7 &-3 &5 
      \end{vmatrix} = -3x + 3y + 6z \Rightarrow l_a(-3,3,6)
    \]
    
    
    \item Формула для подсчета угла между прямыми a и b по направляющим векторам $l_a$ и $l_b$: 
      \[
        \alpha = arccos(\left|\frac{(l_a,l_b)}{|l_a||l_b|} \right|)
      \]
      \[
        \alpha = arccos(\left|\frac{-27}{\sqrt{3^2+3^2+6^2}\sqrt{9^2 + 2^2 + 1}}\right|) \Rightarrow \alpha = arccos(\frac{3\sqrt{129}}{86})
      \]
\end{enumerate}

\textbf{Oтвет: }$ arccos(\frac{3\sqrt{129}}{86})$ 

\section*{Задача 14}

\begin{minipage}[t]{0.45\textwidth}

\textbf{Дано:}\\ 
a: \begin{math}
  \frac{x}{1} = \frac{y+3}{0} = \frac{z+3}{1}
\end{math}\\ 
b: \begin{math}
  \frac{x-27}{21}=\frac{y+4}{6}=\frac{z-5}{14}
\end{math}

\textbf{Найти: }
\end{minipage}
\begin{minipage}[t]{0.45\textwidth}
  \vspace{-\baselineskip} % Required for vertically aligning minipages

\begin{center}
\begin{tikzpicture}
  \draw[thick] (0,0, 0) -- (2,0,0)  -- (2,0,2) node[below]{$l_a$} -- (0,0,2) node[left] {$ M_a $}  -- cycle;
  \draw[thick] (0.25,1, 0)node[left] {$l_b$}  -- (2.25,1,0) -- (2.25,1,2) -- (0.25,1,2) node[left] {$M_b$}  -- cycle;
    \draw[thick] (2.25,1,0) -- (2,0,0);
    \draw[thick] (2.25,1,2) -- (2,0,2);
    \draw[thick] (0.25,1,2) -- (0,0,2);
    \draw[thick] (0.25,1,0) -- (0,0,0);
\end{tikzpicture}
\end{center}
\end{minipage}
\\
\textbf{Решение:}
\begin{enumerate}
  \item Найдем для прямых точку и направляющий вектор:
    \[
      a\Rightarrow \begin{aligned}
        &l_a(1,0,1)\\ 
        &M_a(0,-3,-3)
      \end{aligned}
    \]
    \[
      b\Rightarrow \begin{aligned}
        &l_b(21,6,14)\\ 
        &M_b(27,-4,5)
      \end{aligned}
    \]
  \item Расстояние между скрещивающимися прямыми(направляющие вектора разные) вычисляется по формуле:
    \[
      \rho(a,b) = \frac{(M_aM_b,l_a,l_b)}{|[l_a,l_b]|}
    \]
  \item Промежуточные вычисления:
    \[
      (M_aM_b,l_a,l_b) = \begin{vmatrix}
        27 &-1 &8\\
        1 &0 &1 \\ 
        21 &6 &14
      \end{vmatrix} = 121
    \]
    \[
      [l_a,l_b] = \begin{Vmatrix}
        i &j &k\\ 
        1 &0 &1\\ 
        21 &6 &14
      \end{Vmatrix} =| -6i +7j +6k| = 11  
    \]
    \[
      \rho(a,b) = \frac{121}{11} = 11
    \]
\end{enumerate}

\textbf{Oтвет: }11 
\section*{Задача 15}

\begin{minipage}[t]{0.45\textwidth}

\textbf{Дано:}\\ 
P(21,11,-6)\\ 
a: $ \frac{x}{5} = \frac{y+1}{6} = \frac{z-3}{-1}$\\ 
\textbf{Найти: }точку симметричную P отн. а. 
\end{minipage}
\begin{minipage}[t]{0.45\textwidth}
  \vspace{-\baselineskip} % Required for vertically aligning minipages

\begin{center}
\begin{tikzpicture}
    \draw[thick] (0,-1) -- (3,1.5) node[right] {$a$};
    \fill (1,1) circle (2pt) node[above] {$P$};
    \fill (2,0) circle (2pt) node[below] {$P'$};
\end{tikzpicture}
\end{center}
\end{minipage}
\\
\textbf{Решение:}
\begin{enumerate}
  \item a $ \Rightarrow l(5,6,-1)  $
  \item Найдем уравнение плоскости $\alpha$ , проходящей через P' и перпендикулярной а. 
    \[
      5(x-21) + 6(y-11) -(z+6) = 0 \Leftrightarrow 5x+6y-z-177 =0 
    \]
    \item Найдем M - т. пересечения $ \alpha $ и а:
      \[
        \begin{cases}
5x+6y-z-177 =0 \\ 
\frac{x}{5} = \frac{y+1}{6} \\
\frac{y+1}{6} = \frac{z-3}{-1}
        \end{cases} 
      \Rightarrow M(15,17,0)\] 
\item M - середина PP', где P'(x,y,z) - искомая точка
  \[
    \begin{cases}  
    \frac{x+21}{2} = 15 \\ 
    \frac{y+11}{2} = 17 \\ 
    \frac{z-6}{2} = 0 
    \end{cases} \Rightarrow P'(9,23,6)
  \]
\end{enumerate}
\textbf{Oтвет: } P'(9,23,6) 

\section*{Задача 16}

\begin{minipage}[t]{0.45\textwidth}

\textbf{Дано:}\\ 
P(9,0,3)\\ 
a: 4x +3y + z -13 = 0\\ 
\textbf{Найти: }Проекцию точки Р на а
\end{minipage}
\begin{minipage}[t]{0.45\textwidth}
  \vspace{-\baselineskip} % Required for vertically aligning minipages

\begin{center}
\begin{tikzpicture}
  \draw[thick] (0,0) node[above] {$\alpha$} -- (2,1) -- (5,1) -- (3,0) -- cycle;
  \draw[thick] (2.25, 2) node[right] {$P$} -- (2.25,0.75) node[right]{P'} ; 
\end{tikzpicture}
\end{center}
\end{minipage}
\\
\textbf{Решение:}
\begin{enumerate}
  \item $ a \Rightarrow n(4,3,1) \perp a$ 
  \item Найдем прямую $l \perp a $ и $ P \in l $:
    \[
      l: \frac{x-9}{4} = \frac{y-0}{3} = \frac{z-3}{1}
    \]
  \item Найдем искомую точку P' - пересечение l и a. 
    \[
      \begin{cases}
        \frac{x-9}{4} = \frac{y-0}{3} \\ 
 \frac{y-0}{3} = \frac{z-3}{1}\\
 4x +3y + z -13 = 0\\ 
      \end{cases}
      \Rightarrow P'(5,-3,2)
    \]
\end{enumerate}

\textbf{Oтвет: } P'(5,-3,2)
\section*{Задача 17}

\begin{minipage}[t]{0.45\textwidth}

\textbf{Дано:}\\ 
P(-6,-4,-6)\\
a: $ \frac{x-3}{4}= \frac{y+1}{1} = \frac{z-3}{5} $\\
\textbf{Найти: } Проекцию Р на а
\end{minipage}
\begin{minipage}[t]{0.45\textwidth}
  \vspace{-\baselineskip} % Required for vertically aligning minipages

\begin{center}
\begin{tikzpicture}
  \draw[thick, rotate = 15] (1,0) node[left] {P'}-- (1,1) node[left] {P};
  \draw[thick, rotate = 15] (0,0) -- (5,0);node[left] {a};
  \fill[rotate = 15] (1,0) circle(2 pt) ;
\end{tikzpicture}
\end{center}
\end{minipage}
\\
\textbf{Решение:}
\begin{enumerate}
  \item $ a \Rightarrow l(4,1,5) \Rightarrow \alpha \perp a \land P \in \alpha$
    \[
      \alpha: 4(x+6) + (y+4) + 5(z+6) = 0 \Leftrightarrow 4x+y+5z-58 =0 
    \]
  \item $P' \in \alpha \land P' \in a $: 
    \[
      \begin{cases}
        4x+y+5z_58 =0\\ 
 \frac{x-3}{4}= \frac{y+1}{1}\\ 
 \frac{y+1}{1} = \frac{z-3}{5} 
      \end{cases}
      \Rightarrow P'(-5,-3,-7)
    \]
\end{enumerate}

\textbf{Oтвет: } P'(-5,-3,-7)

\section*{Задача 18}

\begin{minipage}[t]{0.45\textwidth}

\textbf{Дано:}
a: $ \begin{cases}
  4x+3y+z-29 = 0\\ 
  5x+4y+z-36 = 0\\ 
\end{cases} $\\
$ \alpha $: 3x + 2y+z+4=0 \\
\textbf{Выяснить: }взаимное расположение прямой и плоскости 
\end{minipage}
\begin{minipage}[t]{0.45\textwidth}
  \vspace{-\baselineskip} % Required for vertically aligning minipages

\begin{center}
\begin{tikzpicture}
    \draw[thick] (0,0, 0) -- (2,0,0)  -- (2,0,2) -- (0,0,2) node[left] {$\alpha$}  -- cycle;
    \draw[thick] (0,1,3)  -- (3,1,0) node[below] {$a$};
\end{tikzpicture}
\end{center}
\end{minipage}
\\
\textbf{Решение:}
\begin{enumerate}
  \item $a \Rightarrow l_a = [n_1(4,3,1),n_2(5,4,1)] = -i + j + k \Leftrightarrow l_a(-1,1,1) $ 
  \item Проверим есть ли пересечение а и $ \alpha $:
    \[
      \begin{cases}
         4x+3y+z-29 = 0\\ 
  5x+4y+z-36 = 0\\ 
 3x + 2y+z+4=0 
      \end{cases}
      \Rightarrow \begin{pmatrix}
        4&3&1&\vrule&29\\ 
        3&2&1&\vrule&-4\\ 
        5&4&1&\vrule&36 
      \end{pmatrix}
\sim \begin{pmatrix}
        1&1&0&\vrule&33\\ 
        3&2&1&\vrule&-4\\ 
        5&4&1&\vrule&36 
      \end{pmatrix} \sim \]\[
\sim \begin{pmatrix}
        1&1&0&\vrule&33\\ 
        0&-1&1&\vrule&-4-33*3\\ 
        0&-1&1&\vrule&36 - 33*5 
      \end{pmatrix} \Rightarrow \text{ (н. стр. - ср. стр) } 0 = 40 -33*2 - \text{ неверно, тогда }  a || \alpha
    \]
\end{enumerate}

\textbf{Oтвет: } Прямая параллельна плоскости
\end{document}

