\documentclass{article}
\usepackage{amsmath}
\usepackage[left=3cm,right=1.5cm,
top=2cm,bottom=2cm,bindingoffset=0cm]{geometry}
\usepackage{tikz}
\usepackage[T2A]{fontenc}
\usepackage[utf8]{inputenc}
\usepackage[russian]{babel}


% \section*{Задача 5}
%
% \begin{minipage}[t]{0.45\textwidth}
%
% \textbf{Дано:}
% \textbf{Найти: }
% \end{minipage}
% \begin{minipage}[t]{0.45\textwidth}
%   \vspace{-\baselineskip} % Required for vertically aligning minipages
%
% \begin{center}
% \begin{tikzpicture}
%     \draw[thick] (0,0) node[left] {$A$} -- (2,2) node[above] {$B$} -- (4,0) node[right] {$C$} -- cycle;
%     \draw[dashed] (2,2) -- (2,0) node[below] {$H$};
% \end{tikzpicture}
% \end{center}
% \end{minipage}
% \\
% \textbf{Решение:}
% \begin{enumerate}
% \end{enumerate}
%
% \textbf{Oтвет: }2 
\begin{document}
\section*{Задача 1}
\begin{minipage}[t]{0.45\textwidth}
\textbf{Дано:}\\
A(1, 2)\\
B(4, 4)\\
C(2, -2)\\
\textbf{Составить:} ур-ние медианы треугольника $ABC$, проходящую через вершину $A$.
\end{minipage}
\begin{minipage}[t]{0.45\textwidth}
	\vspace{-\baselineskip} % Required for vertically aligning minipages
\begin{center}
\begin{tikzpicture}
    \draw[thick] (0,0) node[left] {$C$} -- (2,2) node[above] {$A$} -- (4,0) node[right] {$B$} -- cycle;
    \draw[dashed] (2,2) -- (2,0) node[below] {$M$};
\end{tikzpicture}
\end{center}
\end{minipage}
\\
\textbf{Решение:}
\begin{enumerate}

\item Пусть $AM$ -- медиана, тогда точка $M$ -- середина отрезка $BC$, значит:
\[
M\left(\frac{4 + 2}{2}, \frac{4 + (-2)}{2}\right) = M(3, 1)
\]

\item $\overline{AM}: \{2, -1\}$, $ M(3,1)$ $\Rightarrow AM: \frac{x-3}{2} = \frac{y-1}{-1}$
\end{enumerate}
\textbf{Oтвет:} $\frac{x-3}{2} = \frac{y-1}{-1}$
%%%%%%%%%%%%%%%%%%%%%%%%%%%%%%%%%%%%%%%%
\section*{Задача 2}
\begin{minipage}[t]{0.45\textwidth}
   
\textbf{Дано:}

$M(8, 11)$\\
$l: 2x + 3y + 3 = 0$\\


\textbf{Найти:} точку симметричную $M$ относительно $l$.\\

\end{minipage}
\begin{minipage}[t]{0.45\textwidth}
	\vspace{-\baselineskip} % Required for vertically aligning minipages

\begin{center}
\begin{tikzpicture}
    \draw[thick] (0,-1) -- (3,1.5) node[right] {$e$};
    \fill (1,1) circle (2pt) node[above] {$M$};
    \fill (2,0) circle (2pt) node[below] {$M'$};
\end{tikzpicture}
\end{center}
\end{minipage}
\\
\textbf{Решение:}
\begin{enumerate}
  \item Из уравнения прямой $l$ получаем вектор нормали:
\[
  2x + 3y + 3 = 0 \Rightarrow \overline{n} \{2, 3\}
\]

\item Вектор нормали будет являться направляющим вектором к\\ прямой $MM'$,где  $M'$ - искомая точка.
  \[
    \text{Ур-ние $MM`$:} 
    \begin{cases}
       2t + 8 = x\\ 
       3t + 11 = x
    \end{cases}
    \Rightarrow 
    \begin{cases}
      t = \frac{x}{2} - 4\\ 
      t= \frac{x}{3} - \frac{11}{3}
    \end{cases}
    \Rightarrow 
    x/2 - y/3 - 1/3 = 0
  \]
  ММ`: 3x - 2y - 2 = 0 
  \item Найдем О - т. пересечения MM` и l: 
    \[
      \begin{cases}
         3x - 2y - 2 =0\\
         2x + 3y + 3 =0
      \end{cases} \Rightarrow 
      \left(\begin{array}{rr|r}
          2 & 3 & -3\\
          3 &-2 & 2

      \end{array}\right)\Rightarrow 
\left(\begin{array}{rr|r}
          -1 & 5 & -5\\
          3 &-2 & 2
\end{array}\right)\Rightarrow 
\left(\begin{array}{rr|r}
          -1 & 5 & -5\\
          0 &13 & -13
      \end{array}\right)\Rightarrow
    \] 
    \[
      \Rightarrow
      \begin{cases}
         y = -1\\
         5y + 5 = x
      \end{cases}\Rightarrow
      O(0;-1)
    \]
  \item OM = OM' И M' $\in$ МM'
\[
    \begin{cases}
         3x - 2y - 2 =0\\
         8^2 + 12^2 = (y+1)^2 + x^2
      \end{cases}
      \Rightarrow
 \begin{cases}
   x = \frac{2}{3}(y+1)\\
   \frac{13}{9}(y+1)^2 = 208
      \end{cases}
      \Rightarrow
      \begin{cases}
   x = \frac{2}{3}(y+1)\\
         y +1 = \pm 12 
      \end{cases}
      \Rightarrow
      M'(-8;13)
\]
\end{enumerate}

\textbf{Oтвет:} $M'(-8;13)$

\section*{Задача 3}

\begin{minipage}[t]{0.45\textwidth}
   
\textbf{Дано:}
A(-2,3)\\ 
B(7,-3)\\ 
C(4,8)\\
\textbf{Составить:}уравнение высоты 
треугольника $ABC$, проходящего через вершину B.\\
\end{minipage}
\begin{minipage}[t]{0.45\textwidth}
	\vspace{-\baselineskip} % Required for vertically aligning minipages

\begin{center}
\begin{tikzpicture}
    \draw[thick] (0,0) node[left] {$A$} -- (2,2) node[above] {$B$} -- (4,0) node[right] {$C$} -- cycle;
    \draw[dashed] (2,2) -- (2,0) node[below] {$H$};
\end{tikzpicture}
\end{center}
\end{minipage}
\\
\textbf{Решение:}
\begin{enumerate}
  \item $\overline{AC}(6,5) \Rightarrow \overline{AC} \perp \overline{BH}$
  \item Ур-ние прямой через точку и вектор нормали:
    \[
      \textbf{Ответ - BH: } (x-7)6 + (y+3)5 = 0 
    \]
\end{enumerate}


\section*{Задача 4}

\begin{minipage}[t]{0.45\textwidth}
   
\textbf{Дано:}
A(1,1)\\ 
l: x-y-2=0\\ 
\textbf{Найти: } S\\
\end{minipage}
\begin{minipage}[t]{0.45\textwidth}
	\vspace{-\baselineskip} % Required for vertically aligning minipages

\begin{center}
\begin{tikzpicture}
    \draw[thick] (0,0) node[left] {$A$} -- (2,2) node[above] {$B$} -- (4,0) node[right] {$C$} -- cycle;
    \draw[dashed] (2,2) -- (2,0) node[below] {$H$};
\end{tikzpicture}
\end{center}
\end{minipage}
\\
\textbf{Решение:}
\begin{enumerate}
  \item По уравнению прямой становится ясно, что 
    точка А не лежит на l.
  \item Из уравнения l найдем вектор нормали к данной прямой,
    он будет являтся направляющим вектором некоторой прямой. На этой прямой
    будет лежать точка А, а также сторона квадрата. \\ 
    $\overline{n}(1,-1)$
  \item Найдем ту самую, некоторую прямую, и обозначим ее b. 
    Для простоты вычисления возьмем вектор нормали к b (1,1)\\ 
  $b: 1(x-1) + 1(y-1) =0 \Leftrightarrow x+y-2 = 0$
\item Найдем еще одну вершину квадрата она будет лежать в
  пересечении этих прямых, назовем ее B.  
  \[
  \begin{cases}
    x-y-2=0\\ 
    x+y-2 = 0\\ 
  \end{cases}
  \Rightarrow \text{B: }
\begin{cases}
  x = 2 
  y = 0 
\end{cases}
  \]
\item Тогда длина стороны квадрата равна $|\overline{AB}| = \sqrt{1+1} = \sqrt{2}$. 
  Тогда площадь равна $\sqrt{2}^2 = 2$ 
  
\end{enumerate}

\textbf{Oтвет: }2 

\section*{Задача 5}

\begin{minipage}[t]{0.45\textwidth}
   
\textbf{Дано:} \\ 
A(-1,-1)\\ 
B(5,-3)
C(-3,-3)\\
\textbf{Найти: }
\end{minipage}
\begin{minipage}[t]{0.45\textwidth}
	\vspace{-\baselineskip} % Required for vertically aligning minipages

\begin{center}
\begin{tikzpicture}
    \draw[thick] (0,0) node[left] {$A$} -- (2,2) node[above] {$B$} -- (4,0) node[right] {$C$} -- cycle;
    \draw[dashed] (2,2) -- (2,0) node[below] {$H$};
\end{tikzpicture}
\end{center}
\end{minipage}
\\
\textbf{Решение:}
\begin{enumerate}
\end{enumerate}

\textbf{Oтвет: }2 



\end{document}

