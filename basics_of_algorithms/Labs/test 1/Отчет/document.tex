\documentclass[a4paper,14pt]{extreport} % добавить leqno в [] для нумерации слева
%\usepackage[14pt]{extsizes}
\usepackage[left=3cm,right=1.5cm,
top=2cm,bottom=2cm,bindingoffset=0cm]{geometry}
\linespread{1.45} %полуторный интервал
\usepackage{titlesec}
%%% Работа с русским языком
\titleformat{\section}{\normalfont\bfseries}{\thechapter}{14pt}{\bfseries}
\usepackage{cmap}					% поиск в PDF
\usepackage{mathtext} 				% русские буквы в фомулах
\usepackage[T2A]{fontenc}			% кодировка
\usepackage[utf8]{inputenc}			% кодировка исходного текста
\usepackage[english,russian]{babel}	% локализация и переносы
\usepackage{ fancyhdr} % улучшенная нумерация страниц
%%% Дополнительная работа с математикой
\usepackage{amsmath,amsfonts,amssymb,amsthm,mathtools} % AMS
\usepackage{icomma} % "Умная" запятая: $0,2$ --- число, $0, 2$ --- перечисление
%\renewcommand{\sfdefault}{cmss}
\usefont{T2A}{cmss}{m}{n}
%% Номера формул
\mathtoolsset{showonlyrefs=true} % Показывать номера только у тех формул, на которые есть \eqref{} в тексте.

%% Шрифты
\usepackage{euscript}	 % Шрифт Евклид
\usepackage{mathrsfs} % Красивый матшрифт
%% Свои команды
\DeclareMathOperator{\sgn}{\mathop{sgn}}

%% Перенос знаков в формулах (по Львовскому)
\newcommand*{\hm}[1]{#1\nobreak\discretionary{}
	{\hbox{$\mathsurround=0pt #1$}}{}}

%\renewcommand{\sfdefault}{cmss}
\usepackage{tempora}
%%% Заголовок
\begin{document}
	\begin{titlepage}
		
		\begin{center}
			\vfill
			%\framepage
			
			Министерство науки и высшего образования РФ\\
			Федеральное государственное бюджетное образовательное учреждение
			высшего образования\\
			«КУРСКИЙ ГОСУДАРСТВЕННЫЙ УНИВЕРСИТЕТ»\\
			\ \\
			\vspace{1em}
	
			\hfill\vbox
			{
				\hbox{\hspace{1cm}Кафедра программного}
				\hbox{обеспечения и администрирования}
				\hbox{информационных систем}
				\hbox{\hspace{1cm}Направление подготовки}
				\hbox{математическое обеспечение и}
				\hbox{администрирование}
				\hbox{информационных систем}
				\hbox{\hspace{1cm}Форма обучения очная}
			}
			%\hfill\vbox
			%{
			%	\hbox{Кафедра иного разума}
			%
			%}
			
			\vfill
			
			{\bf Отчет\\
				по лабораторной работе №2\\}
			«Программирование линейных алгоритмов на языке C++»\\
			дисциплина «Основы алгоритмизации и программирования»\\
			
			\vfill
			
			{
				
				\hbox{Выполнил:}
				\hbox to 16cm {студент группы 113 \hfill Файтельсон А.А.}
				\hbox{}
				\hbox{Проверил:}
				\hbox to 15.23cm {старший преподаватель кафедры ПОиАИС\hfill Ураева Е.Е.}%\hspace{0.75cm}}
			}
			
			\vfill
			
			Курск, 2023
		\end{center}
		
	\end{titlepage}
	
	\section*{Ответы на теоретические вопросы}
	\begin{enumerate}
		\item Напишите наименования и функциональное назначение 5-ти или
		более различных текстовых визуальных компонентов.
		\begin{enumerate}
			\item  \textbf{Текстовая метка QLabel}\\
			Компонент для отображения текста или изображения. Используется для вывода информации пользователю.
			
			\item \textbf{QLineEdit}\\
			Компонент для ввода однострочного текста. Используется для получения данных от пользователя.
			
			\item \textbf{QTextEdit}\\
			Компонент для ввода многострочного текста. Также используется для получения данных от пользователя.
			
			\item \textbf{QSpinBox}\\
			Компонент для ввода	и изменения целого числа. Позволяет пользователю выбирать числа с помощью кнопок или вводить их вручную.
			
			\item \textbf{QComboBox}\\
			Компонент с возможностью выбора одного из предложенных вариантов. Используется для выбора элемента из заданного набора значений.
		\end{enumerate}
		\item Напишите наименования и функциональное назначение 3-х или
		более различных управляющих визуальных компонентов (кнопок).
		\begin{enumerate}
			\item \textbf{QPushButton}\\
			Стандартная кнопка для выполнения определенного действия при нажатии на нее.
			\item \textbf{QRadioButton}\\
			Стандартная кнопка-переключатель для выбора одного варианта из нескольких предложенных.
			\item \textbf{QCheckBox}\\
			Стандартная кнопка-переключатель для выбора одного или нескольких вариантов из предложенных.
		\end{enumerate}
		
		\item Напишите наименования и функциональное назначение 3-х или
		более различных событий (слотов).
		\begin{enumerate}
			\item \textbf{Click}\\
			Данное событие вызывается при нажатии на кнопку. Можно использовать этот слот для выполнения определенного действия.
			\item \textbf{DoubleClick}\\
			Данное событие вызывается при двойном нажатии на кнопку. Можно использовать этот слот для выполнения специфического действия, которое должно произойти при двойном щелчке.
			\item \textbf{Change}\\
			Данное событие вызывается при изменении состояния кнопки или выбора варианта. Можно использовать этот слот для выполнения действий, зависящих от изменения состояния кнопки или выбора варианта.
		\end{enumerate}
		
		\item Опишите, чем различаются сборка проекта и запуск программы.\\
		\textbf{Сборка проекта} относится к процессу компиляции и сборки всех файлов и зависимостей проекта в исполняемый файл или библиотеку.\\ \textbf{Запуск программы} относится к процессу запуска исполняемого файла после его успешной сборки. 
		
		\item Укажите не менее двух способов пошагового запуска программы
		на выполнение. Для индивидуального задания приведите снимок экрана с
		редактором кода в процессе отладки.
	
	\end{enumerate}
	
	
	\setcounter{page}{2} 
	\newpage
	fff
\end{document}