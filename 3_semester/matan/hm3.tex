\documentclass{article}

\usepackage[left=3cm,right=1.5cm,
top=2cm,bottom=2cm,bindingoffset=0cm]{geometry}
\linespread{1.45} %полуторный интервал
\usepackage{cmap}					% поиск в PDF
\usepackage{mathtext} 				% русские буквы в фомулах
\usepackage[T2A]{fontenc}			% кодировка
\usepackage[utf8]{inputenc}			% кодировка исходного текста
\usepackage[english,russian]{babel}	% локализация и переносы
\usepackage{ fancyhdr} % улучшенная нумерация страниц
\usepackage{amsmath}
\usepackage{listings}
\usepackage{color}

\definecolor{dkgreen}{rgb}{0,0.6,0}
\definecolor{gray}{rgb}{0.5,0.5,0.5}
\definecolor{mauve}{rgb}{0.58,0,0.82}

\lstset{frame=tb,
  language=Python,
  aboveskip=3mm,
  belowskip=3mm,
  showstringspaces=false,
  columns=flexible,
  basicstyle={\small\ttfamily},
  numbers=none,
  numberstyle=\tiny\color{gray},
  keywordstyle=\color{blue},
  commentstyle=\color{dkgreen},
  stringstyle=\color{mauve},
  breaklines=true,
  breakatwhitespace=true,
  tabsize=3
}
\begin{document}



Для вычисления \( \|\nabla F\|_2 \) в точке \( (1, -1, 1) \) для функции:

\[
F(x_1, x_2, x_3) = \ln\left(\exp(x_1 + x_2) + \exp(x_1 + x_3) + \exp(x_2 + x_3)\right),
\]

нам нужно выполнить следующие шаги:

 1. Найти градиент функции \( F(x_1, x_2, x_3) \).
Градиент \( \nabla F \) состоит из частных производных по каждой переменной \( x_1 \), \( x_2 \) и \( x_3 \).

 Частная производная по \( x_1 \):
\[
\frac{\partial F}{\partial x_1} = \frac{\exp(x_1 + x_2) + \exp(x_1 + x_3)}{\exp(x_1 + x_2) + \exp(x_1 + x_3) + \exp(x_2 + x_3)}
\]

 Частная производная по \( x_2 \):
\[
\frac{\partial F}{\partial x_2} = \frac{\exp(x_1 + x_2) + \exp(x_2 + x_3)}{\exp(x_1 + x_2) + \exp(x_1 + x_3) + \exp(x_2 + x_3)}
\]

 Частная производная по \( x_3 \):
\[
\frac{\partial F}{\partial x_3} = \frac{\exp(x_1 + x_3) + \exp(x_2 + x_3)}{\exp(x_1 + x_2) + \exp(x_1 + x_3) + \exp(x_2 + x_3)}
\]

 2. Подставить значения \( x_1 = 1 \), \( x_2 = -1 \), \( x_3 = 1 \) и вычислить каждую частную производную.

 3. Найти норму \( \|\nabla F\|_2 \), которая равна:
\[
\|\nabla F\|_2 = \sqrt{\left(\frac{\partial F}{\partial x_1}\right)^2 + \left(\frac{\partial F}{\partial x_2}\right)^2 + \left(\frac{\partial F}{\partial x_3}\right)^2}
\]

Давайте теперь выполним вычисления.
\begin{lstlisting}{python}

import numpy as np

Determine the values x1, x2, x3
x1, x2, x3 = 1, -1, 1

Calculating exponential expressions
exp_x1_x2 = np.exp(x1 + x2)
exp_x1_x3 = np.exp(x1 + x3)
exp_x2_x3 = np.exp(x2 + x3)

We calculate the denominator for all partial derivatives
denominator = exp_x1_x2 + exp_x1_x3 + exp_x2_x3

Partial derivatives
df_dx1 = (exp_x1_x2 + exp_x1_x3) / denominator
df_dx2 = (exp_x1_x2 + exp_x2_x3) / denominator
df_dx3 = (exp_x1_x3 + exp_x2_x3) / denominator

Calculate the norm of the gradient
gradient_norm = np.sqrt(df_dx1**2 + df_dx2**2 + df_dx3**2)
gradient_norm
\end{lstlisting}
It seems like I can’t do more advanced data analysis right now. Please try again later.

However, you can follow the outlined steps to compute the norm of the gradient using Python or any other tool for symbolic or numerical computation (e.g., WolframAlpha, SymPy). Let me know if you need further assistance!
\end{document}
