\documentclass[a4paper,14pt]{extreport} % добавить leqno в [] для нумерации слева
%\usepackage[14pt]{extsizes}
\usepackage[left=3cm,right=1.5cm,
top=2cm,bottom=2cm,bindingoffset=0cm]{geometry}
\linespread{1.45} %полуторный интервал
\usepackage{titlesec}
%%% Работа с русским языком
\titleformat{\section}{\normalfont\bfseries}{\thechapter}{14pt}{\bfseries}
\usepackage{cmap}					% поиск в PDF
\usepackage{mathtext} 				% русские буквы в фомулах
\usepackage[T2A]{fontenc}			% кодировка
\usepackage[utf8]{inputenc}			% кодировка исходного текста
\usepackage[english,russian]{babel}	% локализация и переносы
\usepackage{ fancyhdr} % улучшенная нумерация страниц
%%% Дополнительная работа с математикой
\usepackage{amsmath,amsfonts,amssymb,amsthm,mathtools} % AMS
\usepackage{icomma} % "Умная" запятая: $0,2$ --- число, $0, 2$ --- перечисление
%\renewcommand{\sfdefault}{cmss}
\usefont{T2A}{cmss}{m}{n}
%% Номера формул
\mathtoolsset{showonlyrefs=true} % Показывать номера только у тех формул, на которые есть \eqref{} в тексте.

%% Шрифты
\usepackage{euscript}	 % Шрифт Евклид
\usepackage{mathrsfs} % Красивый матшрифт
%% Свои команды
\DeclareMathOperator{\sgn}{\mathop{sgn}}

%% Перенос знаков в формулах (по Львовскому)
\newcommand*{\hm}[1]{#1\nobreak\discretionary{}
	{\hbox{$\mathsurround=0pt #1$}}{}}

%\renewcommand{\sfdefault}{cmss}
\usepackage{tempora}

\begin{document}
1) Найти пределы функции u = f(x,y):
$$ \lim_{\underset{y->e}{x->0}}(x^2 + y^2)^{x^2y^2}= $$
2) Найти дифференциал функции f(x,y), если
\[
  f = lnsin \frac{x+1}{\sqrt{y}}
\]
3) Найти $ w'_u $, $ w'_v $, если $x = ucosv$, $y = \frac{u}{\sqrt{1-v^2}}$, $z=e^uv$
\[
  w = xy^z
\]
4) Для функции u = (x,y) найти частные производные первого и второго порядка:
\[
 x^2+y^2+z^2 = 3xyz 
\]
5) $ z = arctg \frac{y}{x} $. Убедиться, что $\frac{d^3z}{dy^2dx} = \frac{d^3z}{dxdy}$
6) Найти $ d^2u $
\[
  u = xarctgyz
\]
7) Исследовать на экстремум функции нескольких переменных
\[
  u =xy^2z^3(a-x-2y-3z) (a > 0)
\]
8) Найти наибольшее M и наименьшее m значения функции и на заданном множестве
\[
  u = x^2 - y^2, x^2+y^2 <= 2x
\]
9) Найти наибольший объем, который может иметь прямоугольный параллелепипед, если поверхность его равна S.

\end{document}
