
\documentclass[a4paper,14pt]{extreport} % добавить leqno в [] для нумерации слева
%\usepackage[14pt]{extsizes}
\usepackage[left=3cm,right=1.5cm,
top=2cm,bottom=2cm,bindingoffset=0cm]{geometry}
\linespread{1.45} %полуторный интервал
\usepackage{titlesec}
%%% Работа с русским языком
\titleformat{\section}{\normalfont\bfseries}{\thechapter}{14pt}{\bfseries}
\usepackage{cmap}					% поиск в PDF
\usepackage{mathtext} 				% русские буквы в фомулах
\usepackage[T2A]{fontenc}			% кодировка
\usepackage[utf8]{inputenc}			% кодировка исходного текста
\usepackage[english,russian]{babel}	% локализация и переносы
\usepackage{ fancyhdr} % улучшенная нумерация страниц
%%% Дополнительная работа с математикой
\usepackage{amsmath,amsfonts,amssymb,amsthm,mathtools} % AMS
\usepackage{icomma} % "Умная" запятая: $0,2$ --- число, $0, 2$ --- перечисление
%\renewcommand{\sfdefault}{cmss}
\usefont{T2A}{cmss}{m}{n}
%% Номера формул
\mathtoolsset{showonlyrefs=true} % Показывать номера только у тех формул, на которые есть \eqref{} в тексте.

%% Шрифты
\usepackage{euscript}	 % Шрифт Евклид
\usepackage{mathrsfs} % Красивый матшрифт
%% Свои команды
\DeclareMathOperator{\sgn}{\mathop{sgn}}

%% Перенос знаков в формулах (по Львовскому)
\newcommand*{\hm}[1]{ 1\nobreak\discretionary{}
	{\hbox{$\mathsurround=0pt  1$}}{}}

%\renewcommand{\sfdefault}{cmss}
\usepackage{tempora}

\begin{document}
\chapter{Первый билет}
\section*{Задача}
Известен объем тетраэдра и три его вершины. Найти координаты 4 вершины, 
известно, что она лежит на какой-то оси. \\ 
Решение идет через формулу:
\[
  V = \frac{1}{6}|(\bar{AB}, \bar{AC}, \bar{AD})|
\]
Пример задачи: \\
Даны три вершины пирамиды A(1,2,3), B(4,-5,6), C(7,8,10). Объем пирамиды равен 10 кубических единиц.
Найдите координаты четвертой вершины D(x,y,z), если известно, что она лежит на плоскости xy (то есть z=0).

Решение:
Пользуясь формулой получаем:
\[
  V = \frac{1}{6} \begin{vmatrix}
    x_b - x_a; y_b - y_a; z_b -z_a \\
    x_c - x_a; y_c - y_a; z_c -z_a\\
    x_d - x_a; y_d - y_a; z_d -z_a
  \end{vmatrix} => 10 = \frac{1}{6} \begin{vmatrix}
    3 ; -7 ; 3 ; \\
    6;   6;  7;\\ 
    x-1;y-2;z-3
  \end{vmatrix} =>
\]
Т.к D лежит на оси, то можно обнулить 2 координаты:  -107 -67x = 60, тогда 
точка D имеет координаты ($ -\frac{167}{67} $, 0,0) \\ 

\section*{Задача}
Найти точку $ M_1(x_1,y_1,z_1)$ , симметричную точке $M(x_0,y_0,z_0)$ отн. плоскости $ \alpha: Ax + By + Cz+ D = 0$
1) Определяем вектор нормали $ \vec{n}(A,B,C)$
2) Находим прямую l, проходящую через точку и параллельную вектору нормали. Находим прямую в параметрическом виде
$$
\begin{cases}
  x = x_0 + At \\
  y = y_0 + bt \\
  z = z_0 + Ct \\
\end{cases}
$$
3) Найдем т. К, где $ l \cap \alpha  = K $ 
\[
  \begin{cases}
    
  x = x_0 + At \\
  y = y_0 + At \\
  z = z_0 + Ct \\
\alpha: Ax + By + Cz+ D = 0
  \end{cases} => (Ax_0 + By_0 + Cz_0 - D) +(A^2 + B^2 + C^2) t  =>  
\]
  \[    
  t = \frac{-(Ax_0 + By_0 + Cz_0 - D)}{(A^2 + B^2 + C^2)}
  \]
  \[
    K: \begin{cases}
  x_k = x_0 + A\frac{-(Ax_0 + By_0 + Cz_0 - D)}{(A^2 + B^2 + C^2)} \\
  y_k = y_0 + A\frac{-(Ax_0 + By_0 + Cz_0 - D)}{(A^2 + B^2 + C^2)} \\
  z_k = z_0 + C\frac{-(Ax_0 + By_0 + Cz_0 - D)}{(A^2 + B^2 + C^2)} \\
    \end{cases}
  \]

4) K - Cередина отрезка $ MM_1 $, получаем:
\[
  M_1: \begin{cases}
    x_1 = 2(x_0 + A\frac{-(Ax_0 + By_0 + Cz_0 - D)}{(A^2 + B^2 + C^2)} ) - x_0\\
    y_1 = 2(y_0 + A\frac{-(Ax_0 + By_0 + Cz_0 - D)}{(A^2 + B^2 + C^2)} ) - y_0\\
    z_1 = 2(z_0 + A\frac{-(Ax_0 + By_0 + Cz_0 - D)}{(A^2 + B^2 + C^2)} ) - z_0
  \end{cases}
\]
Это и есть ответ

\section*{Задача}
Найди прямолинейные образующие фигуры F(x,y,z) =0, параллельные плоскости $\alpha:$ Ax + By + Cz + D = 0 \\ 
Решение: \\ 
Если прямая d параллельна плоскости, 
то направляющий вектор l прямой перпендикулярен вектору нормали плоскости.$ l \perp n =>(\vec{l}, \vec{n}) = 0 $ \\
Можно взять два стула: l(0,m,n) и l(1,m,n)

1) Пишем уравнение прямолинейной образующей в параметрическом виде:
Возьмем: l(0,m,n) => m*B + n*C = 0 => $ m = \frac{-n*c}{B} $ => l(0,$\frac{-n*c}{B}$, n)\\ 
Можем взять параллельный l вектор, назовем его также, просто разделим его координаты на n, получим:
$l(0,\frac{-c}{b}, 1)$\\
Тогда ур-ние прямолинейной образующей:
\[
  d:   \begin{cases}
    x = x_0 \\ 
    y = y_0 - \frac{c}{b}t \\ 
    z = z_0 + t
  \end{cases}
\]

2)Подставляем ур-е прямой в уравнение фигуры F; \\
Получаем: $F(x_0, y_0 - \frac{c}{b}t, z_0 + t) = 0$
Группирум относительно t, и так как t - постоянно изм. параметр, то у нас выражение в скобках должны быть равны нулю. 
Из них находим $x_0$, $y_0$, $z_0$ и подставляем их в ур-е образующей, это и будет ответ \\
\textit{Пример:}
\[
\begin{aligned}
  &t^2|   &\frac{100}{4} - \frac{225}{9} = 0\\
  &t^1|   &\frac{20y_0}{4} =20\\
  &t^0|  &\frac{y_0^2}{4} = 10x_0\\
\end{aligned} =>
  \begin{cases}
    y_0=4\\ 
    x_0 = \frac{4}{10}\\
  \end{cases}
\]  
Получим:
\[
  l:\begin{cases}
    x = 2t+0.4\\ 
    y = 10t + 4\\ 
    z = -15t
  \end{cases}
\]
Повторить пункты аналогично со случаем l(1,m,n)

\chapter{Второй билет}

\section*{Задача}
Известны координаты концов отрезка, найти точку которая делит отрезок в каком-то отношении. 
Формула деления на отрезков. 
Пример:
Концы отрезков $ A(x_1, y_1) $ , $ B(x_2, y_2) $. Найти $ C(x_0,y_0) \in AB$, если известно 
$ \frac{AC}{CB} =  \lambda$\\
Решение:\\ 
\[
\frac{AC}{CB} =  \lambda =>
  \frac{x_0 - x_1}{x_2-x_0} = \lambda ; 
  \frac{y_0 - y_1}{y_2-y_0} = \lambda ; 
  \frac{z_0 - z_1}{z_2-z_0} = \lambda ; 
\]
Получаем: 
\[
C: 
\begin{cases}
  x_0 = \frac{x_1+\lambda x_2}{1+\lambda}\\
  y_0 = \frac{y_1+\lambda y_2}{1+\lambda}\\
  z_0 = \frac{z_1+\lambda z_2}{1+\lambda}
\end{cases}
\]
\section*{Задача}
Найти проекцию точки M(\begin{math}
  x_1, y_1,z_1
\end{math}) на плоскость $\alpha$: Ax +By +Cz + D = 0 
Решение:\\
1) Находим вектор нормали n(A,B,C) \\
2) Находим прямую $ d || \bar{n}$, где $ M \in d$. Записываем параметрически.
\begin{equation}d:
\begin{cases}
  x = x_0 + At\\
  y = y_0 + Bt\\
  z = z_0 + Ct\\
\end{cases}
\end{equation}
3) Находим $ d \cap \alpha $: \\ 
\[
  \begin{cases}
    Ax +By +Cz + D = 0 \\
  x = x_0 + At\\
  y = y_0 + Bt\\
  z = z_0 + Ct
\end{cases} => (Ax_0 + By_0 + Cz_0 + D) + t(A^2 + B^2 + C^2) = 0 
\]
\[
  t =-\frac{(Ax_0 + By_0 + Cz_0 + D)}{(A^2 + B^2 + C^2)} => M':
  \begin{cases}
    
  x = x_0 - A\frac{(Ax_0 + By_0 + Cz_0 + D)}{(A^2 + B^2 + C^2)} \\
  y = y_0 - B\frac{(Ax_0 + By_0 + Cz_0 + D)}{(A^2 + B^2 + C^2)}\\
  z = z_0 - C\frac{(Ax_0 + By_0 + Cz_0 + D)}{(A^2 + B^2 + C^2)}
  \end{cases}
\]
M' - ответ. 

\section*{Задача}
Найти ур-ния прямолинейных образующих некоторой фигуры F(x,y,z), проходящих через точку A($x_0,y_0,z_0$).\\

Решение:\\
1) Пусть $\vec{l}(a,b,c)$ - направляющий вектор, прямолинейной образующей $ l $, где $ A \in l $
\[
  l: \begin{cases}
     x = at + x_0\\ 
     y = bt + y_0\\ 
     z = ct + z_0\\
   \end{cases} \text{, Подставляем в уравнение из условия}
\]
$F(at + x_0),(bt + y_0),( ct + z_0) )$
\[
\]
Так как ур-ние должно выполнятся $ \forall t $, то тогда должно выполнятся:
\[
\begin{aligned}
  &t^2|   &G(A,B,C) = 0\\
  &t^1|   &H(A,B,C) = 0\\
  &t^0|  &0 = 0\\
\end{aligned} =>
  \begin{cases}
G(A,B,C) = 0\\
H(A,B,C) = 0\\
  \end{cases}
\]  
Для направ. вектора $\vec{l}(a,b,c)$ возьмем 2 варианта $\vec{l}(0,b,c)$, $\vec{l}(1,b,c)$ \\
$\vec{l}(0,b,c)$ и подставляем в G, H: 
\[ 
  \begin{cases}
     -36b^2 + 225c^2 = 0\\
     - 144b + 360c = 0\\
   \end{cases} => b = \text{Дупля}; C = \text{Дупля} 
\]  
\\
Получили $l_1$
$\vec{l}(1,b,c)$: 
\[ 
  \begin{cases}
G(A,B,C) = 0\\
H(A,B,C) = 0\\
   \end{cases} => b = \text{Дупля}; C = \text{Дупля} 
\]  
Получили $l_2$
2) Подставляем значения напр. векторов в l: \\
\[
  l1:
  \begin{cases} 
     x = x_0 + a_1t\\ 
     y = y_0 + b_1t\\ 
     z = z_0 + c_1t\\ 
 \end{cases} \text{ , } l2:
\begin{cases} 
     x = x_0 + a_2t\\ 
     y = y_0 + b_2t\\ 
     z = z_0 + c_2t\\ 
\end{cases}
\]
Что и является ответом.\\
\paragraph{Пример}

Напишите уравнения прямолинейных образующих однополосного гиперболоида: $100x^2 - 36y^2 + 225z^2 = 900$, проходящих через точку $A(3; 2; 0,8)$.

Решение:\\
1) Пусть $\vec{l}(a,b,c)$ - направляющий вектор, прямолинейной образующей $ l $, где $ A \in l $
\[
  l: \begin{cases}
     x = at + 3\\ 
     y = bt + 2\\ 
     z = ct + 0.8\\
   \end{cases} \text{, Подставляем в уравнение из условия}
\]
\[
  100(at+3)^2 - 36(bt+2)^2 + 225(ct+0.8) - 900 = 0
\]
Так как ур-ние должно выполнятся $ \forall t $, то тогда должно выполнятся:
\[
\begin{aligned}
  &t^2|   &100a^2 -36b^2 + 225c^2 = 0\\
  &t^1|   &600at - 144b + 360c = 0\\
  &t^0|  &900-144+144-900 = 0\\
\end{aligned} =>
  \begin{cases}
    100a^2 -36b^2 + 225c^2 = 0\\
    600at - 144b + 360c = 0\\
  \end{cases}
\]  
Для направ. вектора $\vec{l}(a,b,c)$ возьмем 2 варианта $\vec{l}(0,b,c)$, $\vec{l}(1,b,c)$ \\
$\vec{l}(0,b,c)$: 
\[ 
  \begin{cases}
     -36b^2 + 225c^2 = 0\\
     - 144b + 360c = 0\\
  \end{cases} => b = \frac{360}{144}c = \frac{5}{2}c 
\]  
\[
  -36  \frac{25}{4}c^2 + 225c^2 = 0 => 0c^2 = 0 => c \in \mathbb{R}
\]
При с = 0, b = 0, получается нулевой вектор, поэтому данный вектором будет $ \vec{l1}(0,\frac{5}{2}c, c) \space c \in \mathbb{R}/ {0}$ \\
\\
$\vec{l}(1,b,c)$: 
\[ 
  \begin{cases}
     100 -36b^2 + 225c^2 = 0\\
     600 - 144b + 360c = 0\\
   \end{cases} => b = \frac{29}{12}, \space c = -\frac{7}{10}\text{ (Подсчитано на калькуляторе)}
\]  
$\vec{l2}(1,\frac{29}{12},-\frac{7}{10})$ \\
2) Подставляем значения напр. векторов в l: \\
\[
  l1:
  \begin{cases} 
     x = 3\\ 
     y = \frac{5}{2}ct + 2\\ 
     z = ct + 0.8\\
 \end{cases} \text{ , где } \space c \in \mathbb{R}/ {0} \text{ , } l2:
\begin{cases} 
     x = t+ 3\\ 
     y = \frac{29}{12}t + 2\\ 
     z = \frac{-7}{10}t + 0.8\\
\end{cases}
\]
Что и является ответом.\\

\section*{Задача}
Известны координаты двух точек A($ x_1+a,y_1,z_1 $), B($x_2,y_2,z_2$) одна из координат с параметром a, известно, что расстояние между ними равно c.
Найти параметр a;\\
Решение:
\[
c = \sqrt{(x_1+a-x_0)^2 + (y_1-y_0)^2} ->\]
\[
   c^2 = x_1^2 + a^2 + x_0^2 + 2ax_1 - 2x_0x_1 - 2ax_0 + y_1^2 -2y_0y_1+y_0^2 \\ 
\]
\[ 
   a^2 + 2a(x_1 - x_0) + x_1^2 +x_0^2 -2x_0x_1 + y^2_1 - 2y_0y_1 + y^2_0 - c^2  = 0
\]
\[
  a = -(x_1-x_0) \pm \sqrt{(x_1-x_0)^2 - x_1^2 +x_0^2 -2x_0x_1 + y^2_1 - 2y_0y_1 + y^2_0 - c^2 }
\]
\section*{Задача}
Найти кратчайшее расстояние между прямыми $l_1(a,b,c), l_2(d,e,f) $в пространстве:
1) Прямые скрещиваются(определяются по направляющему вектору)<=>\\Координаты направляющих векторов не пропорциональны!
Берем по точке на каждой прямой, точка M на одной,точка N на другой, получаем вектор MN(x,y,z)\\
Строим параллепипед на MN, l1,l2  $h =  \frac{|(MN, l_1, l_2)|}{|[l_1,l_2]|} $,где 
\[
  |(MN, l_1, l_2)|= \begin{vmatrix} 
    x; y; z\\
    a; b;c \\ 
    d; e ;f ;\\  
  \end{vmatrix} ; \text{ модуль от }[l_1, l_2] = \begin{vmatrix} \bar{i} \bar{j} \bar{k} \\ 
  a ; b;c;\\
  d;e;f;
  \end{vmatrix} 
\] 
Найти координаты вектора, если известны координаты базисных векторов. 
Базис = ${e_1,e_2,e_3} $
$ \bar{x} = с1 e1 + c2 e2 + c3 e3 $, где х(c1; c2; c3)

Пример\\ 
Найдите координатный вектор \(\mathbf{x}_{\mathcal{B}}\) для \(\mathbf{x} = \begin{bmatrix} 3 \\ -5 \\ 4 \end{bmatrix}\) относительно базиса \(\mathcal{B} = \left\{ \begin{bmatrix} 1 \\ 0 \\ 3 \end{bmatrix}, \begin{bmatrix} 2 \\ 1 \\ 8 \end{bmatrix}, \begin{bmatrix} 1 \\ -1 \\ 2 \end{bmatrix} \right\}\).

 Координаты \(c_1, c_2, c_3\) вектора \(\mathbf{x}\) относительно базиса \(\mathcal{B}\) удовлетворяют уравнению:

\[ c_1 \begin{bmatrix} 1 \\ 0 \\ 3 \end{bmatrix} + c_2 \begin{bmatrix} 2 \\ 1 \\ 8 \end{bmatrix} + c_3 \begin{bmatrix} 1 \\ -1 \\ 2 \end{bmatrix} = \begin{bmatrix} 3 \\ -5 \\ 4 \end{bmatrix} \]

или

\[ \begin{bmatrix} 1 & 2 & 1 \\ 0 & 1 & -1 \\ 3 & 8 & 2 \end{bmatrix} \begin{bmatrix} c_1 \\ c_2 \\ c_3 \end{bmatrix} = \begin{bmatrix} 3 \\ -5 \\ 4 \end{bmatrix} \tag{3} \]

Увеличенная матрица из уравнения (3) приводится к виду:

\[ \begin{bmatrix} 1 & 2 & 1 & 3 \\ 0 & 1 & -1 & -5 \\ 3 & 8 & 2 & 4 \end{bmatrix} \sim \begin{bmatrix} 1 & 0 & 0 & -2 \\ 0 & 1 & 0 & 0 \\ 0 & 0 & 1 & 5 \end{bmatrix} \]

Таким образом, \(\mathbf{x}_{\mathcal{B}} = \begin{bmatrix} c_1 \\ c_2 \\ c_3 \end{bmatrix} = \begin{bmatrix} -2 \\ 0 \\ 5 \end{bmatrix}\) и \(\mathbf{x} = -2 \begin{bmatrix} 1 \\ 0 \\ 3 \end{bmatrix} + 0 \begin{bmatrix} 2 \\ 1 \\ 8 \end{bmatrix} + 5 \begin{bmatrix} 1 \\ -1 \\ 2 \end{bmatrix}\).
Тогда координаты x(-2, 0, 5)
\section*{Задача}
Нахождение кр. расстояния в пр-ре 
Составить канонические ур-ния эллипса , гиперболический, 
когда известны фокус, большая или малая ось , эксцентриситент 
Примеры задач\\

Каноническое уравнение эллипса с центром в начале координат имеет вид:
\[
\frac{x^2}{a^2} + \frac{y^2}{b^2} = 1,
\]
где:
\begin{itemize}
    \item \(a\) — большая полуось,
    \item \(b\) — малая полуось,
    \item \(c\) — фокусное расстояние, связанное с \(a\) и \(b\) соотношением \(c^2 = a^2 - b^2\),
    \item Эксцентриситет \(e\) определяется как \(e = \frac{c}{a}\).
\end{itemize}

     2. Каноническое уравнение гиперболы:
Каноническое уравнение гиперболы с центром в начале координат имеет вид:
\[
\frac{x^2}{a^2} - \frac{y^2}{b^2} = 1,
\]
где:
\begin{itemize}
    \item \(a\) — действительная полуось,
    \item \(b\) — мнимая полуось,
    \item \(c\) — фокусное расстояние, связанное с \(a\) и \(b\) соотношением \(c^2 = a^2 + b^2\),
    \item Эксцентриситет \(e\) определяется как \(e = \frac{c}{a}\).
\end{itemize}

---

    Примеры задач

     Задача 1 (Эллипс):
Даны фокусы эллипса \(F_1(-3, 0)\) и \(F_2(3, 0)\), а также большая полуось \(a = 5\). Найти каноническое уравнение эллипса и его эксцентриситет.

 Решение: 
\begin{enumerate}
    \item Фокусное расстояние \(c = 3\) (так как фокусы находятся на расстоянии 3 от центра).
    \item По формуле \(c^2 = a^2 - b^2\) находим \(b\):
    \[
    b^2 = a^2 - c^2 = 25 - 9 = 16 \Rightarrow b = 4.
    \]
    \item Каноническое уравнение эллипса:
    \[
    \frac{x^2}{25} + \frac{y^2}{16} = 1.
    \]
    \item Эксцентриситет:
    \[
    e = \frac{c}{a} = \frac{3}{5} = 0.6.
    \]
\end{enumerate}

Ответ:
\[
\boxed{\frac{x^2}{25} + \frac{y^2}{16} = 1, \quad e = 0.6}.
\]

---

     Задача 2 (Гипербола):
Даны фокусы гиперболы \(F_1(-5, 0)\) и \(F_2(5, 0)\), а также эксцентриситет \(e = \frac{5}{3}\). Найти каноническое уравнение гиперболы.

 Решение: 
\begin{enumerate}
    \item Фокусное расстояние \(c = 5\) (так как фокусы находятся на расстоянии 5 от центра).
    \item По формуле \(e = \frac{c}{a}\) находим \(a\):
    \[
    a = \frac{c}{e} = \frac{5}{\frac{5}{3}} = 3.
    \]
    \item По формуле \(c^2 = a^2 + b^2\) находим \(b\):
    \[
    b^2 = c^2 - a^2 = 25 - 9 = 16 \Rightarrow b = 4.
    \]
    \item Каноническое уравнение гиперболы:
    \[
    \frac{x^2}{9} - \frac{y^2}{16} = 1.
    \]
\end{enumerate}

 Ответ: 
\[
\boxed{\frac{x^2}{9} - \frac{y^2}{16} = 1}.
\]

---

     Задача 3 (Эллипс):
Даны малая полуось эллипса \(b = 4\) и эксцентриситет \(e = 0.8\). Найти каноническое уравнение эллипса.

 Решение: 
\begin{enumerate}
    \item По формуле \(e = \frac{c}{a}\) выразим \(c\):
    \[
    c = e \cdot a = 0.8a.
    \]
    \item По формуле \(c^2 = a^2 - b^2\) подставляем \(c = 0.8a\):
    \[
    (0.8a)^2 = a^2 - 16 \Rightarrow 0.64a^2 = a^2 - 16.
    \]
    \item Решаем уравнение относительно \(a\):
    \[
    0.36a^2 = 16 \Rightarrow a^2 = \frac{16}{0.36} = \frac{1600}{36} = \frac{400}{9} \Rightarrow a = \frac{20}{3}.
    \]
    \item Теперь находим \(c\):
    \[
    c = 0.8 \cdot \frac{20}{3} = \frac{16}{3}.
    \]
    \item Каноническое уравнение эллипса:
    \[
    \frac{x^2}{\left(\frac{400}{9}\right)} + \frac{y^2}{16} = 1 \Rightarrow \frac{9x^2}{400} + \frac{y^2}{16} = 1.
    \]
\end{enumerate}

 Ответ: 
\[
\boxed{\frac{9x^2}{400} + \frac{y^2}{16} = 1}.
\]

---

     Задача 4 (Гипербола):
Даны действительная полуось гиперболы \(a = 6\) и эксцентриситет \(e = 1.5\). Найти каноническое уравнение гиперболы.

 Решение: 
\begin{enumerate}
    \item По формуле \(e = \frac{c}{a}\) находим \(c\):
    \[
    c = e \cdot a = 1.5 \cdot 6 = 9.
    \]
    \item По формуле \(c^2 = a^2 + b^2\) находим \(b\):
    \[
    b^2 = c^2 - a^2 = 81 - 36 = 45 \Rightarrow b = \sqrt{45} = 3\sqrt{5}.
    \]
    \item Каноническое уравнение гиперболы:
    \[
    \frac{x^2}{36} - \frac{y^2}{45} = 1.
    \]
\end{enumerate}

 Ответ: 
\[
\boxed{\frac{x^2}{36} - \frac{y^2}{45} = 1}.
\]
Вот перевод решений задач в LaTeX:

---

 Задача 1 (Эллипс):   
Составьте каноническое уравнение эллипса, если один из его фокусов находится в точке \((3, 0)\), длина большой оси равна \(10\), а эксцентриситет \(\varepsilon = 0.6\).

 Решение:   
1. Центр эллипса — \((0, 0)\), так как фокус лежит на оси \(Ox\).  
2. Длина большой оси: \(2a = 10 \Rightarrow a = 5\).  
3. Эксцентриситет: \(\varepsilon = \frac{c}{a} \Rightarrow c = \varepsilon \cdot a = 0.6 \cdot 5 = 3\).  
4. Фокус в точке \((3, 0)\) подтверждает \(c = 3\).  
5. Находим \(b\): \(b^2 = a^2 - c^2 = 25 - 9 = 16 \Rightarrow b = 4\).  
6. Уравнение эллипса: \(\frac{x^2}{25} + \frac{y^2}{16} = 1\).

 Ответ:   
\[
\boxed{\frac{x^2}{25} + \frac{y^2}{16} = 1}
\]

---

 Задача 2 (Гипербола):   
Составьте каноническое уравнение гиперболы, если один из фокусов находится в точке \((5, 0)\), длина действительной оси равна \(6\), а эксцентриситет \(\varepsilon = \frac{5}{3}\).

 Решение:   
1. Центр гиперболы — \((0, 0)\), фокус на оси \(Ox\): \(\frac{x^2}{a^2} - \frac{y^2}{b^2} = 1\).  
2. Длина действительной оси: \(2a = 6 \Rightarrow a = 3\).  
3. Эксцентриситет: \(\varepsilon = \frac{c}{a} \Rightarrow c = \varepsilon \cdot a = \frac{5}{3} \cdot 3 = 5\).  
4. Фокус в точке \((5, 0)\) подтверждает \(c = 5\).  
5. Находим \(b\): \(c^2 = a^2 + b^2 \Rightarrow 25 = 9 + b^2 \Rightarrow b^2 = 16 \Rightarrow b = 4\).  
6. Уравнение гиперболы: \(\frac{x^2}{9} - \frac{y^2}{16} = 1\).

 Ответ:   
\[
\boxed{\frac{x^2}{9} - \frac{y^2}{16} = 1}
\]

---

 Задача 3 (Эллипс):   
Составьте каноническое уравнение эллипса с фокусом в точке \((0, 4)\), длиной малой оси \(6\) и эксцентриситетом \(\varepsilon = 0.8\).

 Решение:   
1. Центр эллипса — \((0, 0)\), фокус на оси \(Oy\): \(\frac{x^2}{b^2} + \frac{y^2}{a^2} = 1\).  
2. Длина малой оси: \(2b = 6 \Rightarrow b = 3\).  
3. Эксцентриситет: \(\varepsilon = \frac{c}{a} \Rightarrow c = \varepsilon \cdot a = 0.8a\).  
4. Фокус в точке \((0, 4)\) \(\Rightarrow c = 4 \Rightarrow 0.8a = 4 \Rightarrow a = 5\).  
5. Проверка: \(c^2 = a^2 - b^2 \Rightarrow 16 = 25 - 9\) (верно).  
6. Уравнение эллипса: \(\frac{x^2}{9} + \frac{y^2}{25} = 1\).

 Ответ:   
\[
\boxed{\frac{x^2}{9} + \frac{y^2}{25} = 1}
\]

---

 Задача 4 (Гипербола):   
Составьте каноническое уравнение гиперболы с фокусом в точке \((0, -5)\), длиной мнимой оси \(8\) и эксцентриситетом \(\varepsilon = \frac{5}{3}\).

 Решение:   
1. Центр гиперболы — \((0, 0)\), фокус на оси \(Oy\): \(\frac{y^2}{a^2} - \frac{x^2}{b^2} = 1\).  
2. Длина мнимой оси: \(2b = 8 \Rightarrow b = 4\).  
3. Эксцентриситет: \(\varepsilon = \frac{c}{a} \Rightarrow c = \varepsilon \cdot a = \frac{5}{3}a\).  
4. Фокус в точке \((0, -5)\) \(\Rightarrow c = 5 \Rightarrow \frac{5}{3}a = 5 \Rightarrow a = 3\).  
5. Проверка: \(c^2 = a^2 + b^2 \Rightarrow 25 = 9 + 16\) (верно).  
6. Уравнение гиперболы: \(\frac{y^2}{9} - \frac{x^2}{16} = 1\).

 Ответ:   
\[
\boxed{\frac{y^2}{9} - \frac{x^2}{16} = 1}
\]
\section*{Задача}
Найти объем тетраэдра, когда известны 4 его вершины. \\
Решение:\\
1) Находим координаты векторов 3 векторов, выходящих из одной точки, назовем их a1(x1;y1;z1), a2(x2;y2;z2),a3(x3;y3;z3)\\
2) V = $ \frac{1}{6}(a1,a2,a3) $ = $ \begin{vmatrix}
  x1;y1;z1\\
x2;y2;z2\\x3;y3;z3\\
\end{vmatrix} $, что являяется ответом


\chapter{Дальше начинается ИИ..}

\section*{Задача}
Составить ур-ние общего перпендикуляра для данных прямых \\
Решение:\\
1) Определяем напр. вектора и точки \\
2) Составляем ур-ния плоскости через направляющий вектор одной прямой \\
и точку другой прямой, этот направляющий вектор - нормаль для плоскости. \\
Аналогично делаем для другой прямой и другой точки \\
Составляем систему из этих двух плоскостей и все Это ответ \\

Даны две прямые в пространстве. Необходимо найти уравнение общего перпендикуляра для этих прямых. 

 Решение: 

1.  Определим направляющие векторы и точки прямых. 

   Пусть даны две прямые:

   \[
   L_1: \vec{r}_1 = \vec{a}_1 + t\vec{b}_1
   \]
   \[
   L_2: \vec{r}_2 = \vec{a}_2 + s\vec{b}_2
   \]

   где \(\vec{a}_1\) и \(\vec{a}_2\) — точки на прямых \(L_1\) и \(L_2\), \(\vec{b}_1\) и \(\vec{b}_2\) — направляющие векторы прямых.

2.  Составим уравнения плоскостей. 

   - Первая плоскость \(\Pi_1\) проходит через прямую \(L_1\) и параллельна направляющему вектору \(\vec{b}_2\) прямой \(L_2\). Нормаль к плоскости \(\Pi_1\) будет перпендикулярна как \(\vec{b}_1\), так и \(\vec{b}_2\), поэтому её можно найти как векторное произведение:

     \[
     \vec{n}_1 = \vec{b}_1 \times \vec{b}_2
     \]

     Уравнение плоскости \(\Pi_1\):

     \[
     (\vec{r} - \vec{a}_1) \cdot \vec{n}_1 = 0
     \]

   - Вторая плоскость \(\Pi_2\) проходит через прямую \(L_2\) и параллельна направляющему вектору \(\vec{b}_1\) прямой \(L_1\). Нормаль к плоскости \(\Pi_2\) также будет перпендикулярна как \(\vec{b}_1\), так и \(\vec{b}_2\), поэтому её можно найти как векторное произведение:

     \[
     \vec{n}_2 = \vec{b}_1 \times \vec{b}_2
     \]

     Уравнение плоскости \(\Pi_2\):

     \[
     (\vec{r} - \vec{a}_2) \cdot \vec{n}_2 = 0
     \]

3.  Составим систему уравнений плоскостей. 

   Общий перпендикуляр будет линией пересечения плоскостей \(\Pi_1\) и \(\Pi_2\). Уравнение общего перпендикуляра можно записать в виде системы:

   \[
   \begin{cases}
   (\vec{r} - \vec{a}_1) \cdot (\vec{b}_1 \times \vec{b}_2) = 0 \\
   (\vec{r} - \vec{a}_2) \cdot (\vec{b}_1 \times \vec{b}_2) = 0
   \end{cases}
   \]

   Это и есть уравнение общего перпендикуляра.

 Ответ: 

Уравнение общего перпендикуляра для прямых \(L_1\) и \(L_2\) задаётся системой:

\[
\boxed{
\begin{cases}
(\vec{r} - \vec{a}_1) \cdot (\vec{b}_1 \times \vec{b}_2) = 0 \\
(\vec{r} - \vec{a}_2) \cdot (\vec{b}_1 \times \vec{b}_2) = 0
\end{cases}
}
\]
\section*{Задача}
Вычислить объем параллепипеда по координатам. 
1) Находим координаты векторов, выходящих из одной точки, а дальше смешанное произведение 
Чтобы вычислить объём параллелепипеда, построенного на трёх векторах, выходящих из одной вершины, используем смешанное произведение этих векторов. Объём равен модулю смешанного произведения.

 Дано: 
\[
\mathbf{a} = \begin{pmatrix}1 \\ 0 \\ 2\end{pmatrix}, \quad
\mathbf{b} = \begin{pmatrix}3 \\ 4 \\ 5\end{pmatrix}, \quad
\mathbf{c} = \begin{pmatrix}2 \\ 1 \\ 7\end{pmatrix}
\]

 Решение: 

1.  Найдём векторное произведение векторов \(\mathbf{b}\) и \(\mathbf{c}\): 
\[
\mathbf{b} \times \mathbf{c} = \begin{vmatrix}
\mathbf{i} & \mathbf{j} & \mathbf{k} \\
3 & 4 & 5 \\
2 & 1 & 7
\end{vmatrix}
= \mathbf{i}(4 \cdot 7 - 5 \cdot 1) - \mathbf{j}(3 \cdot 7 - 5 \cdot 2) + \mathbf{k}(3 \cdot 1 - 4 \cdot 2)
\]
\[
= \mathbf{i}(28 - 5) - \mathbf{j}(21 - 10) + \mathbf{k}(3 - 8) = 23\mathbf{i} - 11\mathbf{j} - 5\mathbf{k} = \begin{pmatrix}23 \\ -11 \\ -5\end{pmatrix}
\]

2.  Вычислим скалярное произведение вектора \(\mathbf{a}\) и результата векторного произведения: 
\[
\mathbf{a} \cdot (\mathbf{b} \times \mathbf{c}) = \begin{pmatrix}1 \\ 0 \\ 2\end{pmatrix} \cdot \begin{pmatrix}23 \\ -11 \\ -5\end{pmatrix} = 1 \cdot 23 + 0 \cdot (-11) + 2 \cdot (-5) = 23 - 0 - 10 = 13
\]

3.  Найдём объём параллелепипеда: 
\[
V = |13| = 13
\]

 Ответ:  Объём параллелепипеда равен \(\boxed{13}\).
% ___ 
\section*{Задача}
С-вить ур-ние прямой, проходящей через точку, перпенд. вектору, пересек. другую прямую. \\
Решение:\\
Берем направляющий вектор (0,m,n), (1,m,n). Каждый случай р/м отдельно. \\
Сост. направляющий вектор, так как прямая перпенд. вектору, скалярно произведение обратится в ноль, значит будет связь между m,n \\
Переписываем канонические ур-ние прямы как пересечение двех плоскостей\\
Решение будет совместным \\
 Уравнение прямой, проходящей через заданную точку, перпендикулярной вектору и пересекающей другую прямую \\

 Дано: 
- Точка \( M_0(x_0, y_0, z_0) \), через которую проходит искомая прямая.
- Вектор \( \vec{N} = (a, b, c) \), которому прямая перпендикулярна.
- Прямая \( L \), заданная каноническими уравнениями:  
  \( \frac{x - x_1}{l} = \frac{y - y_1}{m} = \frac{z - z_1}{n} \).

 Решение: 
1.  Направляющий вектор искомой прямой   
   Пусть направляющий вектор искомой прямой \( \vec{s} = (p, q, r) \). Так как прямая перпендикулярна вектору \( \vec{N} \), их скалярное произведение равно нулю:  
   \[
   a \cdot p + b \cdot q + c \cdot r = 0. \quad (1)
   \]

2.  Условие пересечения с прямой \( L \)   
   Прямые пересекаются, если они лежат в одной плоскости. Для этого смешанное произведение векторов \( \overrightarrow{M_0M_1} \), \( \vec{s} \) и направляющего вектора \( \vec{s_L} = (l, m, n) \) прямой \( L \) должно быть равно нулю:  
   \[
   \begin{vmatrix}
   x_1 - x_0 & y_1 - y_0 & z_1 - z_0 \\
   p & q & r \\
   l & m & n
   \end{vmatrix} = 0. \quad (2)
   \]

3.  Определение направляющего вектора   
   Рассмотрим два случая для упрощения вычислений:  
   -  Случай 1:  \( p = 1 \).  
     Из уравнения (1):  
     \[
     a + b \cdot q + c \cdot r = 0 \implies q = -\frac{a + c \cdot r}{b} \quad (b \neq 0).
     \]  
     Подставляем \( q \) и \( p = 1 \) в уравнение (2) и находим \( r \).  
   -  Случай 2:  \( p = 0 \).  
     Из уравнения (1):  
     \[
     b \cdot q + c \cdot r = 0 \implies q = -\frac{c}{b} \cdot r \quad (b \neq 0).
     \]  
     Подставляем \( p = 0 \) и \( q \) в уравнение (2) и находим \( r \).

4.  Запись уравнения прямой   
   После нахождения \( p, q, r \), записываем канонические уравнения искомой прямой:  
   \[
   \frac{x - x_0}{p} = \frac{y - y_0}{q} = \frac{z - z_0}{r}.
   \]

 Пример: 
Пусть \( M_0(1, 2, 3) \), \( \vec{N} = (2, -1, 4) \), прямая \( L \):  
\( \frac{x - 0}{1} = \frac{y - 1}{2} = \frac{z - 0}{-1} \).

1.  Случай 1:  \( p = 1 \).  
   Из уравнения (1):  
   \[
   2 \cdot 1 + (-1) \cdot q + 4 \cdot r = 0 \implies -q + 4r = -2 \implies q = 4r + 2.
   \]  
   Подставляем \( \overrightarrow{M_0M_1} = (-1, -1, -3) \), \( \vec{s} = (1, 4r + 2, r) \), \( \vec{s_L} = (1, 2, -1) \) в (2):  
   \[
   \begin{vmatrix}
   -1 & -1 & -3 \\
   1 & 4r + 2 & r \\
   1 & 2 & -1
   \end{vmatrix} = 0.
   \]  
   Раскрывая определитель, находим \( r = 1 \), тогда \( q = 6 \).  
   Направляющий вектор: \( \vec{s} = (1, 6, 1) \).  
   Уравнение прямой:  
   \[
   \frac{x - 1}{1} = \frac{y - 2}{6} = \frac{z - 3}{1}.
   \]

 Ответ:   
\[
\boxed{\frac{x - 1}{1} = \frac{y - 2}{6} = \frac{z - 3}{1}}
\] 
\section*{Задача}
Определить вид линии пересечения некоторой фигуры с плоскостью. \\
Выражаем одну из координат из ур-ний плоскости и подставляем в ур-ние фигуры. \\
Определям гиперболический, элиптический или гиперболический, одним из способов:\\
1) Через определитель (см. в теорию как по определит. тип по определителю)\\
или выделить полные квадраты(приведение к кан. виду)\\

\[
Ax^2+By^2+Cz^2+2Dxy+2Exz+2Fyz+2Gx+2Hy+2Iz+J=0,
\]
и плоскость
\[
z=k.
\]
Подставляем \(z=k\) в уравнение поверхности:
\[
Ax^2+By^2+Ck^2+2Dxy+2Exk+2Fyk+2Gx+2Hy+2Ik+J=0.
\]
Обозначим \(J'=Ck^2+2Ik+J\). Тогда получаем уравнение
\[
Ax^2+By^2+2Dxy+2(Ex+Fy)k+2Gx+2Hy+J'=0,
\]
которое можно записать в виде
\[
Ax^2+2Dxy+By^2+2Lx+2My+N=0,
\]
где \(L\) и \(M\) — новые коэффициенты, а \(N=J'\).

Для классификации кривой вводится дискриминант
\[
\delta = D^2 - AB.
\]
Тогда:
\begin{itemize}
  \item Если \(\delta < 0\), то кривая эллиптическая (или окружность при \(A=B\) и \(D=0\)).
  \item Если \(\delta = 0\), то кривая параболическая (либо вырожденный случай).
  \item Если \(\delta > 0\), то кривая гиперболическая.
\end{itemize}

Альтернативный метод заключается в приведении уравнения к каноническому виду с выделением полных квадратов. После поворота координат (если необходимо) и сдвига можно получить одно из канонических уравнений:
\[
\frac{(x-h)^2}{a^2}+\frac{(y-k)^2}{b^2}=1, \quad \frac{(x-h)^2}{a^2}-\frac{(y-k)^2}{b^2}=1, \quad (x-h)^2=2p(y-k),
\]
которые соответствуют эллипсу (или окружности), гиперболе и параболе соответственно.
%
% __ 
% Найти длину высоты тетраэдра. 
% Решение:
% Находим объем через смешанное произведение, находим площадь основания
% через векторное произведение и берем соотв. отношение. 
% __ 
% Найти точку сим. относительно плоскости, но ур-ние пл-ти нужно составить. 
% В данном случае по трем точкам
% __ 
% Написать кан. ур-ние гиперболоида или параболоида, проход. через точку или перес. плоскость 
% Сначала составляем в общем виде канонический вид. Потом подставили, сравнили 
% 2 параметра выплывут из общего и т.д. 
% __ 
% Даны координаты вершин тетраэдра, найти высоту 


\section*{Задача}
Пусть заданы координаты вершин тетраэдра
\[
A(x_A,\, y_A,\, z_A),\quad B(x_B,\, y_B,\, z_B),\quad C(x_C,\, y_C,\, z_C),\quad D(x_D,\, y_D,\, z_D).
\]
Выберем треугольник \(ABC\) за основание, тогда высота \(h\) определяется как расстояние от точки \(D\) до плоскости, содержащей \(ABC\).

Сначала определим векторы:
\[
\vec{AB} = \begin{pmatrix} x_B - x_A \\[1mm] y_B - y_A \\[1mm] z_B - z_A \end{pmatrix}, \quad
\vec{AC} = \begin{pmatrix} x_C - x_A \\[1mm] y_C - y_A \\[1mm] z_C - z_A \end{pmatrix}.
\]
Нормальный вектор к плоскости \(ABC\) задаётся векторным произведением:
\[
\vec{n} = \vec{AB} \times \vec{AC} =
\begin{pmatrix}
(y_B - y_A)(z_C - z_A) - (z_B - z_A)(y_C - y_A)\\[1mm]
(z_B - z_A)(x_C - x_A) - (x_B - x_A)(z_C - z_A)\\[1mm]
(x_B - x_A)(y_C - y_A) - (y_B - y_A)(x_C - x_A)
\end{pmatrix}.
\]
Пусть также
\[
\vec{AD} = \begin{pmatrix} x_D - x_A \\[1mm] y_D - y_A \\[1mm] z_D - z_A \end{pmatrix}.
\]
Тогда высота \(h\) вычисляется по формуле:
\[
h = \frac{\left| \vec{n} \cdot \vec{AD} \right|}{\|\vec{n}\|},
\]
где
\[
\|\vec{n}\| = \sqrt{n_x^2 + n_y^2 + n_z^2}.
\]

Альтернативно, пользуясь объёмом \(V\) тетраэдра
\[
V = \frac{1}{6} \left| \det\bigl[\vec{AB},\, \vec{AC},\, \vec{AD}\bigr] \right|
\]
и площадью основания \(S_{ABC} = \frac{1}{2}\|\vec{n}\|\), получаем:
\[
h = \frac{3V}{S_{ABC}} = \frac{\left| \det\bigl[\vec{AB},\, \vec{AC},\, \vec{AD}\bigr] \right|}{\|\vec{n}\|}.
\]

Таким образом, окончательная формула для высоты имеет вид:
\[
\boxed{h = \frac{\left| \vec{n} \cdot \vec{AD} \right|}{\|\vec{n}\|}, \quad \text{где} \quad \vec{n} = \vec{AB} \times \vec{AC} \quad \text{и} \quad \vec{AD} = D - A.}
\]

% __ 
\section*{Задача}
Найти координаты точки, симметричной отн. прямой\\
\textbf{Дано:}\\
\begin{itemize}
\item Точка M(x0,y0)M(x0,y0).
\item Прямая l:ax+by+c=0l:ax+by+c=0.
\end{itemize}
\textbf{Решение:}

\begin{enumerate}
\item \textbf{Направляющий вектор прямой l:}
\textbf{Найти:} Точку M'(x',y')M'(x',y'), симметричную точке MM относительно прямой l.
Вектор нормали прямой \( l \) равен \( \overline{n} = (a, b) \). Этот вектор перпендикулярен прямой \( l \).

\item \textbf{Уравнение прямой \( MM' \):}

Поскольку \( M' \) симметрична \( M \) относительно прямой \( l \), прямая \( MM' \) должна быть перпендикулярна прямой \( l \). Следовательно, направляющий вектор прямой \( MM' \) совпадает с вектором нормали прямой \( l \), то есть \( \overline{n} = (a, b) \).

Уравнение прямой \( MM' \) в параметрическом виде:
\[
\begin{cases}
    x = x_0 + a t, \\
    y = y_0 + b t,
\end{cases}
\]
где \( t \) — параметр.

\item \textbf{Точка пересечения \( O \) прямых \( l \) и \( MM' \):}

Точка \( O \) лежит на прямой \( l \), поэтому её координаты удовлетворяют уравнению \( ax + by + c = 0 \). Подставим параметрические уравнения прямой \( MM' \) в уравнение прямой \( l \):
\[
a(x_0 + a t) + b(y_0 + b t) + c = 0.
\]
Раскроем скобки:
\[
a x_0 + a^2 t + b y_0 + b^2 t + c = 0.
\]
Сгруппируем слагаемые:
\[
(a^2 + b^2) t + (a x_0 + b y_0 + c) = 0.
\]
Выразим параметр \( t \):
\[
t = -\frac{a x_0 + b y_0 + c}{a^2 + b^2}.
\]

Подставим \( t \) в параметрические уравнения прямой \( MM' \), чтобы найти координаты точки \( O \):
\[
O\left(x_0 + a \left(-\frac{a x_0 + b y_0 + c}{a^2 + b^2}\right), y_0 + b \left(-\frac{a x_0 + b y_0 + c}{a^2 + b^2}\right)\right).
\]
Упростим:
\[
O\left(x_0 - \frac{a(a x_0 + b y_0 + c)}{a^2 + b^2}, y_0 - \frac{b(a x_0 + b y_0 + c)}{a^2 + b^2}\right).
\]

\item \textbf{Координаты точки \( M' \):}

Точка \( O \) является серединой отрезка \( MM' \). Используем формулу середины отрезка:
\[
O\left(\frac{x_0 + x'}{2}, \frac{y_0 + y'}{2}\right).
\]
Приравняем координаты точки \( O \):
\[
\frac{x_0 + x'}{2} = x_0 - \frac{a(a x_0 + b y_0 + c)}{a^2 + b^2},
\]
\[
\frac{y_0 + y'}{2} = y_0 - \frac{b(a x_0 + b y_0 + c)}{a^2 + b^2}.
\]
Умножим обе части на 2:
\[
x_0 + x' = 2x_0 - \frac{2a(a x_0 + b y_0 + c)}{a^2 + b^2},
\]
\[
y_0 + y' = 2y_0 - \frac{2b(a x_0 + b y_0 + c)}{a^2 + b^2}.
\]
Выразим \( x' \) и \( y' \):
\[
x' = x_0 - \frac{2a(a x_0 + b y_0 + c)}{a^2 + b^2},
\]
\[
y' = y_0 - \frac{2b(a x_0 + b y_0 + c)}{a^2 + b^2}.
\] 
\end{enumerate}

\textbf{Ответ:}

Координаты точки \( M' \), симметричной точке \( M(x_0, y_0) \) относительно прямой \( l: ax + by + c = 0 \), вычисляются по формулам:
\[
x' = x_0 - \frac{2a(a x_0 + b y_0 + c)}{a^2 + b^2},
\]
\[
y' = y_0 - \frac{2b(a x_0 + b y_0 + c)}{a^2 + b^2}.
\]
\section*{Задача}
\textbf{Вопрос был написан ШИЗОМ, ахтунг!!}\\
Рассмотрим задачу нахождения точки, симметричной данной точке относительно прямой, заданной различными способами. Для этого рассмотрим все возможные варианты задания прямой и подробно распишем каждый шаг решения.

    1. Прямая задана параметрически

Пусть прямая задана параметрически:
\[
\begin{cases}
x = x_0 + at, \\
y = y_0 + bt, \\
z = z_0 + ct,
\end{cases}
\]
где \((x_0, y_0, z_0)\) — точка на прямой, \((a, b, c)\) — направляющий вектор прямой, \(t\) — параметр.

     Шаг 1: Заданная точка

Пусть дана точка \(P(x_1, y_1, z_1)\), которую нужно отразить относительно прямой.

     Шаг 2: Уравнение плоскости, перпендикулярной прямой и проходящей через точку \(P\)

Направляющий вектор прямой \((a, b, c)\) будет нормальным вектором искомой плоскости. Уравнение плоскости:
\[
a(x - x_1) + b(y - y_1) + c(z - z_1) = 0.
\]

     Шаг 3: Точка пересечения плоскости и прямой

Подставим параметрические уравнения прямой в уравнение плоскости:
\[
a(x_0 + at - x_1) + b(y_0 + bt - y_1) + c(z_0 + ct - z_1) = 0.
\]
Раскроем скобки и выразим параметр \(t\):
\[
a(x_0 - x_1) + b(y_0 - y_1) + c(z_0 - z_1) + t(a^2 + b^2 + c^2) = 0.
\]
Отсюда:
\[
t = -\frac{a(x_0 - x_1) + b(y_0 - y_1) + c(z_0 - z_1)}{a^2 + b^2 + c^2}.
\]
Подставим \(t\) обратно в параметрические уравнения прямой, чтобы найти точку пересечения \(Q\):
\[
Q\left(x_0 + a t, y_0 + b t, z_0 + c t\right).
\]

     Шаг 4: Симметричная точка

Точка \(P'\), симметричная точке \(P\) относительно прямой, находится как:
\[
P' = 2Q - P.
\]
Координаты \(P'\):
\[
P'\left(2x_Q - x_1, 2y_Q - y_1, 2z_Q - z_1\right).
\]

    2. Прямая задана как пересечение двух плоскостей

Пусть прямая задана как пересечение двух плоскостей:
\[
\begin{cases}
A_1x + B_1y + C_1z + D_1 = 0, \\
A_2x + B_2y + C_2z + D_2 = 0.
\end{cases}
\]

     Шаг 1: Направляющий вектор прямой

Направляющий вектор прямой \(\vec{s}\) можно найти как векторное произведение нормальных векторов плоскостей:
\[
\vec{s} = \begin{vmatrix}
\mathbf{i} & \mathbf{j} & \mathbf{k} \\
A_1 & B_1 & C_1 \\
A_2 & B_2 & C_2
\end{vmatrix}.
\]
Пусть \(\vec{s} = (a, b, c)\).

     Шаг 2: Уравнение плоскости, перпендикулярной прямой и проходящей через точку \(P\)

Уравнение плоскости:
\[
a(x - x_1) + b(y - y_1) + c(z - z_1) = 0.
\]

     Шаг 3: Точка пересечения плоскости и прямой

Решаем систему уравнений:
\[
\begin{cases}
A_1x + B_1y + C_1z + D_1 = 0, \\
A_2x + B_2y + C_2z + D_2 = 0, \\
a(x - x_1) + b(y - y_1) + c(z - z_1) = 0.
\end{cases}
\]
Находим точку пересечения \(Q\).

     Шаг 4: Симметричная точка

Точка \(P'\):
\[
P' = 2Q - P.
\]

    3. Прямая задана канонически

Пусть прямая задана канонически:
\[
\frac{x - x_0}{a} = \frac{y - y_0}{b} = \frac{z - z_0}{c}.
\]

     Шаг 1: Направляющий вектор прямой

Направляющий вектор прямой \(\vec{s} = (a, b, c)\).

     Шаг 2: Уравнение плоскости, перпендикулярной прямой и проходящей через точку \(P\)

Уравнение плоскости:
\[
a(x - x_1) + b(y - y_1) + c(z - z_1) = 0.
\]

     Шаг 3: Точка пересечения плоскости и прямой

Подставляем параметрические уравнения прямой в уравнение плоскости:
\[
a(x_0 + at - x_1) + b(y_0 + bt - y_1) + c(z_0 + ct - z_1) = 0.
\]
Находим \(t\) и точку \(Q\).

     Шаг 4: Симметричная точка

Точка \(P'\):
\[
P' = 2Q - P.
\]

Заключение

Таким образом, независимо от способа задания прямой, алгоритм нахождения симметричной точки относительно прямой остается одинаковым. Основные шаги включают нахождение точки пересечения плоскости, перпендикулярной прямой и проходящей через данную точку, и последующее нахождение симметричной точки относительно этой точки пересечения.
\section*{Задача}
Для вычисления объема тетраэдра с заданными вершинами \( A(x_1, y_1, z_1) \), \( B(x_2, y_2, z_2) \), \( C(x_3, y_3, z_3) \) и \( D(x_4, y_4, z_4) \) в трехмерном пространстве, можно использовать формулу объема тетраэдра через определитель матрицы. Объем \( V \) тетраэдра вычисляется по следующей формуле:

\[
V = \frac{1}{6} \left| \det \begin{pmatrix}
x_2 - x_1 & y_2 - y_1 & z_2 - z_1 \\
x_3 - x_1 & y_3 - y_1 & z_3 - z_1 \\
x_4 - x_1 & y_4 - y_1 & z_4 - z_1 \\
\end{pmatrix} \right|
\]

 Пошаговое решение: 

1.  Вычисление векторов: 
   - Вектор \( \vec{AB} = (x_2 - x_1, y_2 - y_1, z_2 - z_1) \)
   - Вектор \( \vec{AC} = (x_3 - x_1, y_3 - y_1, z_3 - z_1) \)
   - Вектор \( \vec{AD} = (x_4 - x_1, y_4 - y_1, z_4 - z_1) \)

2.  Составление матрицы: 
   \[
   \begin{pmatrix}
   x_2 - x_1 & y_2 - y_1 & z_2 - z_1 \\
   x_3 - x_1 & y_3 - y_1 & z_3 - z_1 \\
   x_4 - x_1 & y_4 - y_1 & z_4 - z_1 \\
   \end{pmatrix}
   \]

3.  Вычисление определителя матрицы: 
   \[
   \det \begin{pmatrix}
   a & b & c \\
   d & e & f \\
   g & h & i \\
   \end{pmatrix} = a(ei - fh) - b(di - fg) + c(dh - eg)
   \]
   Применительно к нашей матрице:
   \[
   \det = (x_2 - x_1)\left[(y_3 - y_1)(z_4 - z_1) - (z_3 - z_1)(y_4 - y_1)\right] \\
   - (y_2 - y_1)\left[(x_3 - x_1)(z_4 - z_1) - (z_3 - z_1)(x_4 - x_1)\right] \\
   + (z_2 - z_1)\left[(x_3 - x_1)(y_4 - y_1) - (y_3 - y_1)(x_4 - x_1)\right]
   \]

4.  Вычисление объема: 
   \[
   V = \frac{1}{6} \left| \det \right|
   \]

 Пример вычисления: 

Пусть вершины тетраэдра заданы координатами:
- \( A(0, 0, 0) \)
- \( B(1, 0, 0) \)
- \( C(0, 1, 0) \)
- \( D(0, 0, 1) \)

1.  Векторы: 
   - \( \vec{AB} = (1, 0, 0) \)
   - \( \vec{AC} = (0, 1, 0) \)
   - \( \vec{AD} = (0, 0, 1) \)

2.  Матрица: 
   \[
   \begin{pmatrix}
   1 & 0 & 0 \\
   0 & 1 & 0 \\
   0 & 0 & 1 \\
   \end{pmatrix}
   \]

3.  Определитель: 
   \[
   \det = 1 \cdot (1 \cdot 1 - 0 \cdot 0) - 0 \cdot (0 \cdot 1 - 0 \cdot 0) + 0 \cdot (0 \cdot 0 - 1 \cdot 0) = 1
   \]

4.  Объем: 
   \[
   V = \frac{1}{6} \left| 1 \right| = \frac{1}{6}
   \]
Таким образом, объем тетраэдра равен \( \frac{1}{6} \).
\section*{Задача}
Чтобы найти расстояние между двумя параллельными плоскостями, заданными уравнениями:

\[
P_1: \quad a x + b y + c z + d_1 = 0
\]
\[
P_2: \quad a x + b y + c z + d_2 = 0,
\]

где коэффициенты \( a \), \( b \), \( c \) одинаковы для обеих плоскостей (что гарантирует их параллельность), расстояние \( h \) между ними вычисляется по формуле:

\[
h = \frac{|d_2 - d_1|}{\sqrt{a^2 + b^2 + c^2}}.
\]

 Пошаговое решение: 

1.  Проверка параллельности плоскостей: 
   Убедимся, что коэффициенты при \( x \), \( y \), \( z \) в уравнениях плоскостей совпадают:
   \[
   \frac{a_1}{a_2} = \frac{b_1}{b_2} = \frac{c_1}{c_2}.
   \]
   Если это условие выполняется, плоскости параллельны.

2.  Применение формулы расстояния: 
   Если плоскости параллельны, расстояние между ними вычисляется по формуле:
   \[
   h = \frac{|d_2 - d_1|}{\sqrt{a^2 + b^2 + c^2}}.
   \]

 Пример вычисления: 

Пусть даны две параллельные плоскости:
\[
P_1: \quad 2x - 3y + 6z + 5 = 0,
\]
\[
P_2: \quad 2x - 3y + 6z - 10 = 0.
\]

1.  Проверка параллельности: 
   Коэффициенты при \( x \), \( y \), \( z \) совпадают:
   \[
   \frac{2}{2} = \frac{-3}{-3} = \frac{6}{6} = 1.
   \]
   Плоскости параллельны.

2.  Применение формулы расстояния: 
   Здесь \( a = 2 \), \( b = -3 \), \( c = 6 \), \( d_1 = 5 \), \( d_2 = -10 \). Подставляем в формулу:
   \[
   h = \frac{|d_2 - d_1|}{\sqrt{a^2 + b^2 + c^2}} = \frac{|-10 - 5|}{\sqrt{2^2 + (-3)^2 + 6^2}} = \frac{15}{\sqrt{4 + 9 + 36}} = \frac{15}{\sqrt{49}} = \frac{15}{7}.
   \]

Таким образом, расстояние между плоскостями равно \( \frac{15}{7} \).
\section*{Задача}
\textbf{АХТУНГ, Я ПРОЕБАЛ УСЛОВИЕ ЗАДАЧИ ПОЭТОМУ ПРИШЛОСЬ ВЫДУМЫВАТЬ}\\ 
Рассмотрим задачу на составление уравнения гиперболоида, используя канонический вид уравнения.

---

 Задача: 

Составьте уравнение однополостного гиперболоида, если известно, что оси симметрии служат осями ортонормированного репера, а гиперболоид проходит через точки \( A(3, 0, 0) \), \( B(0, 2, 0) \) и \( C(0, 0, 4) \).

---

 Решение: 

1.  Канонический вид уравнения однополостного гиперболоида: 

   \[
   \frac{x^2}{a^2} + \frac{y^2}{b^2} - \frac{z^2}{c^2} = 1,
   \]

   где \( a \), \( b \), \( c \) — полуоси гиперболоида.

2.  Подставляем координаты точек в уравнение: 

   - Для точки \( A(3, 0, 0) \):

     \[
     \frac{3^2}{a^2} + \frac{0^2}{b^2} - \frac{0^2}{c^2} = 1 \implies \frac{9}{a^2} = 1 \implies a^2 = 9 \implies a = 3.
     \]

   - Для точки \( B(0, 2, 0) \):

     \[
     \frac{0^2}{a^2} + \frac{2^2}{b^2} - \frac{0^2}{c^2} = 1 \implies \frac{4}{b^2} = 1 \implies b^2 = 4 \implies b = 2.
     \]

   - Для точки \( C(0, 0, 4) \):

     \[
     \frac{0^2}{a^2} + \frac{0^2}{b^2} - \frac{4^2}{c^2} = 1 \implies -\frac{16}{c^2} = 1 \implies c^2 = -16.
     \]

     Здесь возникает противоречие, так как \( c^2 \) не может быть отрицательным. Это означает, что точка \( C(0, 0, 4) \) не лежит на однополостном гиперболоиде с заданными параметрами.

3.  Исправление задачи: 

   Поскольку точка \( C(0, 0, 4) \) не подходит для однополостного гиперболоида, рассмотрим двуполостный гиперболоид, каноническое уравнение которого имеет вид:

   \[
   \frac{x^2}{a^2} + \frac{y^2}{b^2} - \frac{z^2}{c^2} = -1.
   \]

   Подставим координаты точки \( C(0, 0, 4) \):

   \[
   \frac{0^2}{a^2} + \frac{0^2}{b^2} - \frac{4^2}{c^2} = -1 \implies -\frac{16}{c^2} = -1 \implies c^2 = 16 \implies c = 4.
   \]

4.  Итоговое уравнение двуполостного гиперболоида: 

   Подставляем найденные значения \( a = 3 \), \( b = 2 \), \( c = 4 \):

   \[
   \frac{x^2}{9} + \frac{y^2}{4} - \frac{z^2}{16} = -1.
   \]

---

 Ответ: 

Уравнение двуполостного гиперболоида:

\[
\frac{x^2}{9} + \frac{y^2}{4} - \frac{z^2}{16} = -1.
\]
\section*{Задача}
Рассмотрим задачу на составление уравнения эллипсоида.

---

 Задача: 

Составьте уравнение эллипсоида, если известно, что оси симметрии служат осями ортонормированного репера, а эллипсоид проходит через точки \( A(2, 0, 0) \), \( B(0, 3, 0) \) и \( C(0, 0, 5) \).

---

 Решение: 

1.  Канонический вид уравнения эллипсоида: 

   \[
   \frac{x^2}{a^2} + \frac{y^2}{b^2} + \frac{z^2}{c^2} = 1,
   \]

   где \( a \), \( b \), \( c \) — полуоси эллипсоида.

2.  Подставляем координаты точек в уравнение: 

   - Для точки \( A(2, 0, 0) \):

     \[
     \frac{2^2}{a^2} + \frac{0^2}{b^2} + \frac{0^2}{c^2} = 1 \implies \frac{4}{a^2} = 1 \implies a^2 = 4 \implies a = 2.
     \]

   - Для точки \( B(0, 3, 0) \):

     \[
     \frac{0^2}{a^2} + \frac{3^2}{b^2} + \frac{0^2}{c^2} = 1 \implies \frac{9}{b^2} = 1 \implies b^2 = 9 \implies b = 3.
     \]

   - Для точки \( C(0, 0, 5) \):

     \[
     \frac{0^2}{a^2} + \frac{0^2}{b^2} + \frac{5^2}{c^2} = 1 \implies \frac{25}{c^2} = 1 \implies c^2 = 25 \implies c = 5.
     \]

3.  Итоговое уравнение эллипсоида: 

   Подставляем найденные значения \( a = 2 \), \( b = 3 \), \( c = 5 \):

   \[
   \frac{x^2}{4} + \frac{y^2}{9} + \frac{z^2}{25} = 1.
   \]

---

 Ответ: 

Уравнение эллипсоида:

\[
\frac{x^2}{4} + \frac{y^2}{9} + \frac{z^2}{25} = 1.
\]
    \section*{Задача}1: Вычислить площадь треугольника по трем вершинам

 Условие:   
Даны три вершины треугольника в трехмерном пространстве:  
\( A(1, 2, 3) \), \( B(4, 5, 6) \), \( C(7, 8, 9) \).  
Найти площадь треугольника \( ABC \).

 Решение:   
Площадь треугольника можно найти как половину модуля векторного произведения векторов \( \vec{AB} \) и \( \vec{AC} \).

1. Найдем векторы \( \vec{AB} \) и \( \vec{AC} \):
   \[
   \vec{AB} = B - A = (4-1, 5-2, 6-3) = (3, 3, 3)
   \]
   \[
   \vec{AC} = C - A = (7-1, 8-2, 9-3) = (6, 6, 6)
   \]

2. Найдем векторное произведение \( \vec{AB} \times \vec{AC} \):
   \[
   \vec{AB} \times \vec{AC} = 
   \begin{vmatrix}
   \mathbf{i} & \mathbf{j} & \mathbf{k} \\
   3 & 3 & 3 \\
   6 & 6 & 6 \\
   \end{vmatrix}
   = \mathbf{i}(3 \cdot 6 - 3 \cdot 6) - \mathbf{j}(3 \cdot 6 - 3 \cdot 6) + \mathbf{k}(3 \cdot 6 - 3 \cdot 6)
   = (0, 0, 0)
   \]

3. Модуль векторного произведения равен нулю, следовательно, площадь треугольника:
   \[
   S = \frac{1}{2} \| \vec{AB} \times \vec{AC} \| = \frac{1}{2} \cdot 0 = 0
   \]

 Ответ:   
Площадь треугольника \( ABC \) равна 0 (точки лежат на одной прямой).

---

    \section*{Задача}2: Найти точку на оси, равноудаленную от заданной точки и плоскости

 Условие:   
Даны точка \( M(1, 2, 3) \) и плоскость \( 2x - y + 3z = 4 \).  
Найти точку на оси \( Oz \), которая равноудалена от точки \( M \) и плоскости.

 Решение:   
Точка на оси \( Oz \) имеет координаты \( P(0, 0, z) \).

1. Расстояние от точки \( P(0, 0, z) \) до плоскости \( 2x - y + 3z = 4 \):
   \[
   d_{\text{плоскость}} = \frac{|2 \cdot 0 - 1 \cdot 0 + 3 \cdot z - 4|}{\sqrt{2^2 + (-1)^2 + 3^2}} = \frac{|3z - 4|}{\sqrt{14}}
   \]

2. Расстояние от точки \( P(0, 0, z) \) до точки \( M(1, 2, 3) \):
   \[
   d_{\text{точка}} = \sqrt{(1-0)^2 + (2-0)^2 + (3-z)^2} = \sqrt{1 + 4 + (3-z)^2}
   \]

3. Приравняем расстояния:
   \[
   \frac{|3z - 4|}{\sqrt{14}} = \sqrt{1 + 4 + (3-z)^2}
   \]

4. Возведем обе части в квадрат:
   \[
   \frac{(3z - 4)^2}{14} = 5 + (3 - z)^2
   \]

5. Решим уравнение:
   \[
   (3z - 4)^2 = 14(5 + (3 - z)^2)
   \]
   \[
   9z^2 - 24z + 16 = 70 + 14(9 - 6z + z^2)
   \]
   \[
   9z^2 - 24z + 16 = 70 + 126 - 84z + 14z^2
   \]
   \[
   9z^2 - 24z + 16 = 196 - 84z + 14z^2
   \]
   \[
   -5z^2 + 60z - 180 = 0
   \]
   \[
   z^2 - 12z + 36 = 0
   \]
   \[
   (z - 6)^2 = 0 \Rightarrow z = 6
   \]

 Ответ:   
Точка на оси \( Oz \) имеет координаты \( P(0, 0, 6) \).

---

    \section*{Задача}3: Составить уравнение конической поверхности

 Условие:   
Даны вершина конической поверхности \( V(1, 2, 3) \) и направляющая кривая \( x^2 + y^2 = 4 \) в плоскости \( z = 0 \).  
Составить уравнение конической поверхности.

 Решение:   
Уравнение конической поверхности можно записать как:
\[
\frac{(x - x_0)^2}{a^2} + \frac{(y - y_0)^2}{b^2} = \frac{(z - z_0)^2}{c^2}
\]
где \( (x_0, y_0, z_0) \) — вершина конуса, а \( a, b, c \) — параметры, определяющие форму конуса.

1. Вершина конуса \( V(1, 2, 3) \), поэтому уравнение примет вид:
   \[
   \frac{(x - 1)^2}{a^2} + \frac{(y - 2)^2}{b^2} = \frac{(z - 3)^2}{c^2}
   \]

2. Направляющая кривая \( x^2 + y^2 = 4 \) в плоскости \( z = 0 \) задает окружность радиуса 2.  
   Это означает, что конус симметричен относительно оси \( z \), и \( a = b = 2 \).

3. Подставим \( a = b = 2 \) в уравнение:
   \[
   \frac{(x - 1)^2}{4} + \frac{(y - 2)^2}{4} = \frac{(z - 3)^2}{c^2}
   \]

4. Упростим уравнение:
   \[
   (x - 1)^2 + (y - 2)^2 = \frac{4(z - 3)^2}{c^2}
   \]

 Ответ:   
Уравнение конической поверхности:
\[
(x - 1)^2 + (y - 2)^2 = \frac{4(z - 3)^2}{c^2}
\]

---

    \section*{Задача}4: Вычислить площадь параллелограмма, построенного на векторах

 Условие:   
Даны два вектора \( \vec{a} = 2\vec{m} + 3\vec{n} \) и \( \vec{b} = 4\vec{m} - \vec{n} \), где \( \vec{m} \) и \( \vec{n} \) — базисные векторы.  
Длины векторов \( \vec{m} \) и \( \vec{n} \) равны 1, а угол между ними \( \theta = 60^\circ \).  
Найти площадь параллелограмма, построенного на векторах \( \vec{a} \) и \( \vec{b} \).

 Решение:   
Площадь параллелограмма равна модулю векторного произведения векторов \( \vec{a} \) и \( \vec{b} \).

1. Выразим векторное произведение:
   \[
   \vec{a} \times \vec{b} = (2\vec{m} + 3\vec{n}) \times (4\vec{m} - \vec{n})
   \]

2. Раскроем скобки:
   \[
   \vec{a} \times \vec{b} = 2\vec{m} \times 4\vec{m} + 2\vec{m} \times (-\vec{n}) + 3\vec{n} \times 4\vec{m} + 3\vec{n} \times (-\vec{n})
   \]

3. Упростим:
   \[
   \vec{a} \times \vec{b} = 8(\vec{m} \times \vec{m}) - 2(\vec{m} \times \vec{n}) + 12(\vec{n} \times \vec{m}) - 3(\vec{n} \times \vec{n})
   \]

4. Учтем, что \( \vec{m} \times \vec{m} = 0 \) и \( \vec{n} \times \vec{n} = 0 \), а также \( \vec{n} \times \vec{m} = -\vec{m} \times \vec{n} \):
   \[
   \vec{a} \times \vec{b} = -2(\vec{m} \times \vec{n}) - 12(\vec{m} \times \vec{n}) = -14(\vec{m} \times \vec{n})
   \]

5. Модуль векторного произведения:
   \[
   \| \vec{a} \times \vec{b} \| = 14 \| \vec{m} \times \vec{n} \|
   \]

6. Найдем \( \| \vec{m} \times \vec{n} \| \):
   \[
   \| \vec{m} \times \vec{n} \| = \| \vec{m} \| \| \vec{n} \| \sin \theta = 1 \cdot 1 \cdot \sin 60^\circ = \frac{\sqrt{3}}{2}
   \]

7. Площадь параллелограмма:
   \[
   S = 14 \cdot \frac{\sqrt{3}}{2} = 7\sqrt{3}
   \]

 Ответ:   
Площадь параллелограмма равна \( 7\sqrt{3} \).

---

    \section*{Задача}5: Составить уравнение плоскости, равноудаленной от двух параллельных плоскостей

 Условие:   
Даны две параллельные плоскости:
\[
2x - 3y + 4z = 5 \quad \text{и} \quad 2x - 3y + 4z = 10.
\]
Составить уравнение плоскости, равноудаленной от этих двух плоскостей.

 Решение:   
Плоскость, равноудаленная от двух параллельных плоскостей, будет параллельна им и находиться посередине между ними.

1. Уравнения плоскостей имеют одинаковые коэффициенты при \( x, y, z \), что подтверждает их параллельность.

2. Найдем расстояние между плоскостями.  
   Расстояние между двумя параллельными плоскостями \( Ax + By + Cz + D_1 = 0 \) и \( Ax + By + Cz + D_2 = 0 \) вычисляется по формуле:
   \[
   d = \frac{|D_2 - D_1|}{\sqrt{A^2 + B^2 + C^2}}
   \]
   Для данных плоскостей:
   \[
   d = \frac{|10 - 5|}{\sqrt{2^2 + (-3)^2 + 4^2}} = \frac{5}{\sqrt{29}}
   \]

3. Плоскость, равноудаленная от данных плоскостей, будет находиться на расстоянии \( \frac{d}{2} \) от каждой из них.  
   Уравнение такой плоскости:
   \[
   2x - 3y + 4z = \frac{5 + 10}{2} = 7.5
   \]

 Ответ:   
Уравнение плоскости:
\[
2x - 3y + 4z = 7.5
\]

---

    \section*{Задача}6: Найти координаты вектора, перпендикулярного двум заданным векторам

 Условие:   
Даны два вектора \( \vec{a} = (1, 2, 3) \) и \( \vec{b} = (4, 5, 6) \).  
Найти вектор \( \vec{c} \), перпендикулярный \( \vec{a} \) и \( \vec{b} \), и образующий с осью \( Ox \) острый угол. Длина вектора \( \vec{c} \) равна 1.

 Решение:   
Вектор \( \vec{c} \) можно найти как векторное произведение \( \vec{a} \times \vec{b} \).

1. Найдем векторное произведение:
   \[
   \vec{c} = \vec{a} \times \vec{b} = 
   \begin{vmatrix}
   \mathbf{i} & \mathbf{j} & \mathbf{k} \\
   1 & 2 & 3 \\
   4 & 5 & 6 \\
   \end{vmatrix}
   = \mathbf{i}(2 \cdot 6 - 3 \cdot 5) - \mathbf{j}(1 \cdot 6 - 3 \cdot 4) + \mathbf{k}(1 \cdot 5 - 2 \cdot 4)
   \]
   \[
   \vec{c} = \mathbf{i}(12 - 15) - \mathbf{j}(6 - 12) + \mathbf{k}(5 - 8) = (-3, 6, -3)
   \]

2. Найдем длину вектора \( \vec{c} \):
   \[
   \| \vec{c} \| = \sqrt{(-3)^2 + 6^2 + (-3)^2} = \sqrt{9 + 36 + 9} = \sqrt{54} = 3\sqrt{6}
   \]

3. Нормируем вектор \( \vec{c} \):
   \[
   \vec{c}_{\text{норм}} = \frac{\vec{c}}{\| \vec{c} \|} = \left( \frac{-3}{3\sqrt{6}}, \frac{6}{3\sqrt{6}}, \frac{-3}{3\sqrt{6}} \right) = \left( \frac{-1}{\sqrt{6}}, \frac{2}{\sqrt{6}}, \frac{-1}{\sqrt{6}} \right)
   \]

4. Проверим, что вектор \( \vec{c}_{\text{норм}} \) образует с осью \( Ox \) острый угол.  
   Угол между вектором \( \vec{c}_{\text{норм}} \) и осью \( Ox \) определяется через скалярное произведение:
   \[
   \cos \theta = \vec{c}_{\text{норм}} \cdot \mathbf{i} = \frac{-1}{\sqrt{6}}
   \]
   Так как \( \cos \theta < 0 \), угол тупой.  
   Чтобы получить острый угол, умножим вектор на -1:
   \[
   \vec{c}_{\text{норм}} = \left( \frac{1}{\sqrt{6}}, \frac{-2}{\sqrt{6}}, \frac{1}{\sqrt{6}} \right)
   \]

 Ответ:   
Вектор \( \vec{c} \), перпендикулярный \( \vec{a} \) и \( \vec{b} \), и образующий с осью \( Ox \) острый угол:
\[
\vec{c} = \left( \frac{1}{\sqrt{6}}, \frac{-2}{\sqrt{6}}, \frac{1}{\sqrt{6}} \right)
\]

---

    \section*{Задача}7: Составить уравнение плоскости, перпендикулярной прямой через точку

 Условие:   
Дана прямая \( \frac{x - 1}{2} = \frac{y + 3}{-1} = \frac{z - 2}{4} \) и точка \( M(3, -1, 5) \).  
Составить уравнение плоскости, проходящей через точку \( M \) и перпендикулярной данной прямой.

 Решение:   
Плоскость, перпендикулярная прямой, имеет нормальный вектор, совпадающий с направляющим вектором прямой.

1. Направляющий вектор прямой:
   \[
   \vec{v} = (2, -1, 4)
   \]

2. Уравнение плоскости, проходящей через точку \( M(3, -1, 5) \) с нормальным вектором \( \vec{v} \):
   \[
   2(x - 3) - 1(y + 1) + 4(z - 5) = 0
   \]

3. Упростим уравнение:
   \[
   2x - 6 - y - 1 + 4z - 20 = 0
   \]
   \[
   2x - y + 4z - 27 = 0
   \]

 Ответ:   
Уравнение плоскости:
\[
2x - y + 4z - 27 = 0
\]

---

    \section*{Задача}8: Составить уравнение цилиндрической поверхности

 Условие:   
Дана направляющая кривая \( x^2 + y^2 = 9 \) в плоскости \( z = 0 \) и образующая, параллельная вектору \( \vec{v} = (1, 1, 1) \).  
Составить уравнение цилиндрической поверхности.

 Решение:   
Уравнение цилиндрической поверхности можно записать как:
\[
(x - z)^2 + (y - z)^2 = 9
\]

 Ответ:   
Уравнение цилиндрической поверхности:
\[
(x - z)^2 + (y - z)^2 = 9
\]

---

    \section*{Задача}9: Найти четвертую вершину параллелограмма

 Условие:   
Даны три вершины параллелограмма:  
\( A(1, 2, 3) \), \( B(4, 5, 6) \), \( C(7, 8, 9) \).  
Найти четвертую вершину \( D \).

 Решение:   
В параллелограмме диагонали делятся пополам.  
Следовательно, середина диагонали \( AC \) совпадает с серединой диагонали \( BD \).

1. Найдем середину диагонали \( AC \):
   \[
   M = \left( \frac{1 + 7}{2}, \frac{2 + 8}{2}, \frac{3 + 9}{2} \right) = (4, 5, 6)
   \]

2. Пусть \( D(x, y, z) \). Тогда середина диагонали \( BD \):
   \[
   M = \left( \frac{4 + x}{2}, \frac{5 + y}{2}, \frac{6 + z}{2} \right) = (4, 5, 6)
   \]

3. Решим уравнения:
   \[
   \frac{4 + x}{2} = 4 \Rightarrow 4 + x = 8 \Rightarrow x = 4
   \]
   \[
   \frac{5 + y}{2} = 5 \Rightarrow 5 + y = 10 \Rightarrow y = 5
   \]
   \[
   \frac{6 + z}{2} = 6 \Rightarrow 6 + z = 12 \Rightarrow z = 6
   \]

 Ответ:   
Четвертая вершина параллелограмма \( D(4, 5, 6) \).

---

    \section*{Задача}10: Составить уравнение плоскости через точку, параллельной другой плоскости

 Условие:   
Дана плоскость \( 2x - 3y + 4z = 5 \) и точка \( M(1, 2, 3) \).  
Составить уравнение плоскости, проходящей через точку \( M \) и параллельной данной плоскости.

 Решение:   
Плоскость, параллельная данной, имеет тот же нормальный вектор.

1. Нормальный вектор данной плоскости:
   \[
   \vec{n} = (2, -3, 4)
   \]

2. Уравнение плоскости, проходящей через точку \( M(1, 2, 3) \) с нормальным вектором \( \vec{n} \):
   \[
   2(x - 1) - 3(y - 2) + 4(z - 3) = 0
   \]

3. Упростим уравнение:
   \[
   2x - 2 - 3y + 6 + 4z - 12 = 0
   \]
   \[
   2x - 3y + 4z - 8 = 0
   \]

 Ответ:   
Уравнение плоскости:
\[
2x - 3y + 4z - 8 = 0
\]

---
\section*{Задача}
Рассмотрим задачу на составление уравнения конической поверхности, если известны вершина и направляющая.

---

 Задача: 

Составьте уравнение конической поверхности, если известно, что её вершина находится в точке \( V(1, 2, 3) \), а направляющая задана уравнением окружности в плоскости \( Oxy \):

\[
\begin{cases}
x^2 + y^2 = 4, \\
z = 0.
\end{cases}
\]

---

 Решение: 

1.  Общий вид уравнения конической поверхности: 

   Коническая поверхность образуется прямыми (образующими), проходящими через вершину \( V(x_0, y_0, z_0) \) и каждую точку направляющей. Уравнение конической поверхности можно записать в параметрическом виде или в неявном виде, исключив параметры.

2.  Направляющая: 

   Направляющая задана окружностью в плоскости \( Oxy \):

   \[
   \begin{cases}
   x^2 + y^2 = 4, \\
   z = 0.
   \end{cases}
   \]

   Это окружность радиуса 2 с центром в точке \( (0, 0, 0) \).

3.  Уравнение образующей: 

   Любая точка \( P(x, y, z) \) на конической поверхности лежит на прямой, соединяющей вершину \( V(1, 2, 3) \) и точку направляющей \( (x_1, y_1, 0) \). Параметрическое уравнение прямой:

   \[
   \begin{cases}
   x = x_0 + t(x_1 - x_0), \\
   y = y_0 + t(y_1 - y_0), \\
   z = z_0 + t(z_1 - z_0),
   \end{cases}
   \]

   где \( t \) — параметр, \( (x_1, y_1, 0) \) — точка на направляющей.

4.  Исключение параметра \( t \): 

   Выразим \( t \) из третьего уравнения:

   \[
   z = 3 + t(0 - 3) \implies z = 3 - 3t \implies t = \frac{3 - z}{3}.
   \]

   Подставим \( t \) в первые два уравнения:

   \[
   x = 1 + \left(\frac{3 - z}{3}\right)(x_1 - 1), \\
   y = 2 + \left(\frac{3 - z}{3}\right)(y_1 - 2).
   \]

5.  Условие принадлежности точки \( (x_1, y_1, 0) \) направляющей: 

   Точка \( (x_1, y_1, 0) \) лежит на окружности \( x_1^2 + y_1^2 = 4 \). Подставим \( x_1 \) и \( y_1 \) из выражений выше:

   \[
   \left(\frac{3(x - 1)}{3 - z} + 1\right)^2 + \left(\frac{3(y - 2)}{3 - z} + 2\right)^2 = 4.
   \]

6.  Упрощение уравнения: 

   После раскрытия скобок и упрощения получим уравнение конической поверхности:

   \[
   (x - 1)^2 + (y - 2)^2 = \frac{4}{9}(z - 3)^2.
   \]

---

 Ответ: 

Уравнение конической поверхности:

\[
(x - 1)^2 + (y - 2)^2 = \frac{4}{9}(z - 3)^2.
\]

---

 Примечание: 

Это уравнение описывает конус с вершиной в точке \( V(1, 2, 3) \) и направляющей в виде окружности \( x^2 + y^2 = 4 \) в плоскости \( Oxy \).


\section*{Задача}
Рассмотрим задачу на составление уравнения цилиндра.
---

 Задача: 

Составьте уравнение кругового цилиндра, если известно, что его ось симметрии совпадает с осью \( Oz \), а радиус цилиндра равен 4.

---

 Решение: 

1.  Канонический вид уравнения кругового цилиндра: 

   Если ось цилиндра совпадает с осью \( Oz \), то уравнение цилиндра имеет вид:

   \[
   x^2 + y^2 = r^2,
   \]

   где \( r \) — радиус цилиндра.

2.  Подставляем известное значение радиуса: 

   По условию задачи радиус \( r = 4 \), поэтому:

   \[
   x^2 + y^2 = 4^2 \implies x^2 + y^2 = 16.
   \]

3.  Итоговое уравнение цилиндра: 

   Уравнение кругового цилиндра с осью \( Oz \) и радиусом 4:

   \[
   x^2 + y^2 = 16.
   \]

---

 Ответ: 

Уравнение цилиндра:

\[
x^2 + y^2 = 16.
\]

---

 Примечание: 

Если требуется составить уравнение цилиндра другого типа (например, эллиптического или гиперболического), то канонический вид уравнения будет зависеть от формы направляющей линии. Например:

- Для  эллиптического цилиндра :

  \[
  \frac{x^2}{a^2} + \frac{y^2}{b^2} = 1.
  \]

- Для  гиперболического цилиндра :

  \[
  \frac{x^2}{a^2} - \frac{y^2}{b^2} = 1.
  \]

Если нужно, можно придумать задачу и для этих случаев.
% _____
% \section{Типы задач на 5}
% 1)Взаимное расположение k-плоскостей. 
% 2-3) Составить ур-ние поверхности 2-го рода 
% 4) Составление уравнений k-плоскостей в n-мерном пространстве 
% 5) Составление уравнение k-плоскостей, параллельное векторному пространству. 
%
%
%
\end{document}
