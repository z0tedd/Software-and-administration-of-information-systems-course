% Options for packages loaded elsewhere
\PassOptionsToPackage{unicode}{hyperref}
\PassOptionsToPackage{hyphens}{url}
%
\documentclass[
]{article}

\usepackage{titlesec}
\usepackage{microtype}
\usepackage{cmap}\usepackage{mathtext}
\usepackage[T2A]{fontenc}
\usepackage[utf8]{inputenc}\usepackage[english,russian]{babel}\usepackage{ fancyhdr}\usepackage{amsmath,amsfonts,amssymb,amsthm,mathtools} \usepackage{icomma}\usepackage{euscript}\usepackage{mathrsfs}\usepackage{tempora}
\usepackage{amsmath,amssymb}
\usepackage{iftex}
\ifPDFTeX
  \usepackage[T1]{fontenc}
  \usepackage[utf8]{inputenc}
  \usepackage{textcomp} % provide euro and other symbols
\else % if luatex or xetex
  \usepackage{unicode-math} % this also loads fontspec
  \defaultfontfeatures{Scale=MatchLowercase}
  \defaultfontfeatures[\rmfamily]{Ligatures=TeX,Scale=1}
\fi
\usepackage{lmodern}
\ifPDFTeX\else
  % xetex/luatex font selection
\fi
% Use upquote if available, for straight quotes in verbatim environments
\IfFileExists{upquote.sty}{\usepackage{upquote}}{}
\IfFileExists{microtype.sty}{% use microtype if available
  \usepackage[]{microtype}
  \UseMicrotypeSet[protrusion]{basicmath} % disable protrusion for tt fonts
}{}
\makeatletter
\@ifundefined{KOMAClassName}{% if non-KOMA class
  \IfFileExists{parskip.sty}{%
    \usepackage{parskip}
  }{% else
    \setlength{\parindent}{0pt}
    \setlength{\parskip}{6pt plus 2pt minus 1pt}}
}{% if KOMA class
  \KOMAoptions{parskip=half}}
\makeatother
\usepackage{xcolor}
\usepackage{color}
\usepackage{fancyvrb}
\newcommand{\VerbBar}{|}
\newcommand{\VERB}{\Verb[commandchars=\\\{\}]}
\DefineVerbatimEnvironment{Highlighting}{Verbatim}{commandchars=\\\{\}}
% Add ',fontsize=\small' for more characters per line
\newenvironment{Shaded}{}{}
\newcommand{\AlertTok}[1]{\textcolor[rgb]{1.00,0.00,0.00}{\textbf{#1}}}
\newcommand{\AnnotationTok}[1]{\textcolor[rgb]{0.38,0.63,0.69}{\textbf{\textit{#1}}}}
\newcommand{\AttributeTok}[1]{\textcolor[rgb]{0.49,0.56,0.16}{#1}}
\newcommand{\BaseNTok}[1]{\textcolor[rgb]{0.25,0.63,0.44}{#1}}
\newcommand{\BuiltInTok}[1]{\textcolor[rgb]{0.00,0.50,0.00}{#1}}
\newcommand{\CharTok}[1]{\textcolor[rgb]{0.25,0.44,0.63}{#1}}
\newcommand{\CommentTok}[1]{\textcolor[rgb]{0.38,0.63,0.69}{\textit{#1}}}
\newcommand{\CommentVarTok}[1]{\textcolor[rgb]{0.38,0.63,0.69}{\textbf{\textit{#1}}}}
\newcommand{\ConstantTok}[1]{\textcolor[rgb]{0.53,0.00,0.00}{#1}}
\newcommand{\ControlFlowTok}[1]{\textcolor[rgb]{0.00,0.44,0.13}{\textbf{#1}}}
\newcommand{\DataTypeTok}[1]{\textcolor[rgb]{0.56,0.13,0.00}{#1}}
\newcommand{\DecValTok}[1]{\textcolor[rgb]{0.25,0.63,0.44}{#1}}
\newcommand{\DocumentationTok}[1]{\textcolor[rgb]{0.73,0.13,0.13}{\textit{#1}}}
\newcommand{\ErrorTok}[1]{\textcolor[rgb]{1.00,0.00,0.00}{\textbf{#1}}}
\newcommand{\ExtensionTok}[1]{#1}
\newcommand{\FloatTok}[1]{\textcolor[rgb]{0.25,0.63,0.44}{#1}}
\newcommand{\FunctionTok}[1]{\textcolor[rgb]{0.02,0.16,0.49}{#1}}
\newcommand{\ImportTok}[1]{\textcolor[rgb]{0.00,0.50,0.00}{\textbf{#1}}}
\newcommand{\InformationTok}[1]{\textcolor[rgb]{0.38,0.63,0.69}{\textbf{\textit{#1}}}}
\newcommand{\KeywordTok}[1]{\textcolor[rgb]{0.00,0.44,0.13}{\textbf{#1}}}
\newcommand{\NormalTok}[1]{#1}
\newcommand{\OperatorTok}[1]{\textcolor[rgb]{0.40,0.40,0.40}{#1}}
\newcommand{\OtherTok}[1]{\textcolor[rgb]{0.00,0.44,0.13}{#1}}
\newcommand{\PreprocessorTok}[1]{\textcolor[rgb]{0.74,0.48,0.00}{#1}}
\newcommand{\RegionMarkerTok}[1]{#1}
\newcommand{\SpecialCharTok}[1]{\textcolor[rgb]{0.25,0.44,0.63}{#1}}
\newcommand{\SpecialStringTok}[1]{\textcolor[rgb]{0.73,0.40,0.53}{#1}}
\newcommand{\StringTok}[1]{\textcolor[rgb]{0.25,0.44,0.63}{#1}}
\newcommand{\VariableTok}[1]{\textcolor[rgb]{0.10,0.09,0.49}{#1}}
\newcommand{\VerbatimStringTok}[1]{\textcolor[rgb]{0.25,0.44,0.63}{#1}}
\newcommand{\WarningTok}[1]{\textcolor[rgb]{0.38,0.63,0.69}{\textbf{\textit{#1}}}}
\setlength{\emergencystretch}{3em} % prevent overfull lines
\providecommand{\tightlist}{%
  \setlength{\itemsep}{0pt}\setlength{\parskip}{0pt}}
\setcounter{secnumdepth}{-\maxdimen} % remove section numbering
\ifLuaTeX
  \usepackage{selnolig}  % disable illegal ligatures
\fi
\IfFileExists{bookmark.sty}{\usepackage{bookmark}}{\usepackage{hyperref}}
\IfFileExists{xurl.sty}{\usepackage{xurl}}{} % add URL line breaks if available
\urlstyle{same}
\hypersetup{
  hidelinks,
  pdfcreator={LaTeX via pandoc}}

\author{}
\date{}

\begin{document}

Конечно, вот ответы на все вопросы из списка:

\begin{enumerate}
\def\labelenumi{\arabic{enumi}.}
\item
  \textbf{История возникновения языка программирования C++. Версии
  стандарта языка. IDE для программирования на C++. Средства отладки
  приложений в IDE.}

  \textbf{Ответ:} Язык программирования C++ был разработан Бьярне
  Страуструпом в 1983 году как расширение языка C. Основная цель
  создания C++ заключалась в добавлении объектно-ориентированного
  программирования к C, сохранив при этом его эффективность. Первым
  стандартом языка стал C++98, утвержденный в 1998 году. Далее
  последовали C++03, C++11 (который внес значительные изменения и
  улучшения), C++14, C++17 и последний стандарт на момент 2023 года ---
  C++20. Популярные IDE для C++ включают Visual Studio, Code::Blocks,
  CLion и Eclipse CDT. Эти IDE предоставляют инструменты для написания,
  компиляции и отладки программ. Средства отладки в IDE обычно включают
  точки останова (breakpoints), просмотр переменных в реальном времени,
  трассировку выполнения программы и профилирование производительности.
\item
  \textbf{Переменные и константы, их объявление и определение. Правила
  задания идентификаторов. Базовые типы данных, спецификаторы типов.
  Суффиксы для литералов.}

  \textbf{Ответ:} Переменные в C++ объявляются следующим образом:
  \texttt{тип\ имя\_переменной;}, например, \texttt{int\ x;}. Константы
  объявляются с использованием ключевого слова \texttt{const}:
  \texttt{const\ int\ y\ =\ 10;}. Идентификаторы должны начинаться с
  буквы или подчеркивания и могут содержать буквы, цифры и
  подчеркивания. Например, \texttt{myVar}, \texttt{\_temp},
  \texttt{value1}. Базовые типы данных включают: \texttt{int},
  \texttt{float}, \texttt{double}, \texttt{char}, \texttt{bool}.
  Спецификаторы типов (модификаторы) могут быть: \texttt{signed},
  \texttt{unsigned}, \texttt{short}, \texttt{long}. Суффиксы для
  литералов используются для указания типа литерала: \texttt{u} или
  \texttt{U} для unsigned, \texttt{l} или \texttt{L} для long,
  \texttt{ll} или \texttt{LL} для long long, \texttt{f} или \texttt{F}
  для float.
\item
  \textbf{Функции преобразования типа. Явные и неявные, безопасные и
  небезопасные неявные преобразования. Функция static\_cast. Оператор
  приведения типа в стиле C.}

  \textbf{Ответ:} Преобразование типов в C++ бывает явным и неявным.
  Явное преобразование выполняется программистом с использованием
  операторов приведения, например,
  \texttt{static\_cast\textless{}int\textgreater{}(x)}. Неявное
  преобразование выполняется компилятором автоматически, когда это
  возможно. Безопасные преобразования не приводят к потере данных или
  изменению значения, например, преобразование \texttt{int} к
  \texttt{float}. Небезопасные преобразования могут приводить к потере
  данных или изменению значения, например, преобразование \texttt{float}
  к \texttt{int}. \texttt{static\_cast} используется для явного
  преобразования одного типа в другой, когда преобразование логически
  допустимо, например, \texttt{int} к \texttt{float}. Пример
  использования:
  \texttt{int\ x\ =\ static\_cast\textless{}int\textgreater{}(3.14);}.
  Оператор приведения в стиле C выглядит как \texttt{(type)value},
  например, \texttt{(int)3.14}.
\item
  \textbf{Математические функции и функции округления,
  тригонометрические функции. Принцип хранения действительных чисел в
  памяти компьютера. Математические функции языка C для параметров с
  типами int, float и double.}

  \textbf{Ответ:} В C++ доступны различные математические функции, такие
  как \texttt{sqrt} (квадратный корень), \texttt{pow} (возведение в
  степень), \texttt{abs} (модуль числа). Для округления используются
  функции \texttt{ceil} (округление вверх), \texttt{floor} (округление
  вниз), \texttt{round} (обычное округление). Тригонометрические функции
  включают \texttt{sin}, \texttt{cos}, \texttt{tan}. Действительные
  числа хранятся в формате с плавающей точкой, который делится на три
  части: знак, экспонента и мантисса. Математические функции в C++
  поддерживают параметры типов \texttt{int}, \texttt{float},
  \texttt{double}, например,
  \texttt{int\ x\ =\ abs(-5);\ float\ y\ =\ sqrt(25.0f);\ double\ z\ =\ pow(2.0,\ 3.0);}.
\item
  \textbf{Основные арифметические операции, оператор присваивания,
  цепочка присваиваний, модификации оператора присваивания, префиксные и
  постфиксные инкремент и декремент. Приоритет выполнения операций.
  Комментарии в C++.}

  \textbf{Ответ:} Основные арифметические операции включают: сложение
  (\texttt{+}), вычитание (\texttt{-}), умножение (\texttt{*}), деление
  (\texttt{/}), остаток от деления (\texttt{\%}). Оператор присваивания
  \texttt{=} используется для присвоения значения переменной, например,
  \texttt{int\ x\ =\ 5;}. Цепочка присваиваний позволяет присваивать
  одно и то же значение нескольким переменным:
  \texttt{int\ a,\ b,\ c;\ a\ =\ b\ =\ c\ =\ 10;}. Модификации оператора
  присваивания включают \texttt{+=}, \texttt{-=}, \texttt{*=},
  \texttt{/=}, \texttt{\%=}. Префиксные (\texttt{++x}, \texttt{-\/-x}) и
  постфиксные (\texttt{x++}, \texttt{x-\/-}) операторы инкремента и
  декремента увеличивают или уменьшают значение переменной на единицу.
  Приоритет выполнения операций определяется правилами языка: сначала
  выполняются операции в скобках, затем инкремент и декремент, умножение
  и деление, и наконец сложение и вычитание. Комментарии в C++ могут
  быть однострочными (\texttt{//}) и многострочными
  (\texttt{/*\ ...\ */}).
\item
  \textbf{Условные операторы и оператор множественного выбора,
  использование перечисляемого типа в switch. Операторы отношения,
  логические операции, пустой оператор, составной оператор, его влияние
  на область видимости переменных и функций.}

  \textbf{Ответ:} Условные операторы включают \texttt{if},
  \texttt{else\ if}, \texttt{else}. Пример:
  \texttt{if\ (x\ \textgreater{}\ 0)\ \{\ /*\ код\ */\ \}\ else\ \{\ /*\ код\ */\ \}}.
  Оператор множественного выбора \texttt{switch} используется для выбора
  одного из нескольких вариантов. Пример:

\begin{Shaded}
\begin{Highlighting}[]
\KeywordTok{enum}\NormalTok{ Color }\OperatorTok{\{}\NormalTok{ RED}\OperatorTok{,}\NormalTok{ GREEN}\OperatorTok{,}\NormalTok{ BLUE }\OperatorTok{\};}
\NormalTok{Color c }\OperatorTok{=}\NormalTok{ RED}\OperatorTok{;}
\ControlFlowTok{switch} \OperatorTok{(}\NormalTok{c}\OperatorTok{)} \OperatorTok{\{}
    \ControlFlowTok{case}\NormalTok{ RED}\OperatorTok{:} \CommentTok{/* код */} \ControlFlowTok{break}\OperatorTok{;}
    \ControlFlowTok{case}\NormalTok{ GREEN}\OperatorTok{:} \CommentTok{/* код */} \ControlFlowTok{break}\OperatorTok{;}
    \ControlFlowTok{case}\NormalTok{ BLUE}\OperatorTok{:} \CommentTok{/* код */} \ControlFlowTok{break}\OperatorTok{;}
\OperatorTok{\}}
\end{Highlighting}
\end{Shaded}

  Операторы отношения включают \texttt{==}, \texttt{!=},
  \texttt{\textgreater{}}, \texttt{\textless{}},
  \texttt{\textgreater{}=}, \texttt{\textless{}=}. Логические операции:
  \texttt{\&\&} (логическое И), \texttt{\textbar{}\textbar{}}
  (логическое ИЛИ), \texttt{!} (логическое НЕ). Пустой оператор
  \texttt{;} используется для завершения выражений. Составной оператор
  \texttt{\{\ ...\ \}} группирует несколько операторов в блок, влияя на
  область видимости переменных и функций.
\item
  \textbf{Вложенные конструкции if-else, использование операторов
  присваивание и «запятая» в условии оператора if. Побитовые логические
  операции и модификации оператора присваивания для работы с ними,
  тернарный оператор условия.}

  \textbf{Ответ:} Вложенные конструкции \texttt{if-else} позволяют
  создавать многоуровневую логику принятия решений. Пример:

\begin{Shaded}
\begin{Highlighting}[]
\ControlFlowTok{if} \OperatorTok{(}\NormalTok{x }\OperatorTok{\textgreater{}} \DecValTok{0}\OperatorTok{)} \OperatorTok{\{}
    \ControlFlowTok{if} \OperatorTok{(}\NormalTok{y }\OperatorTok{\textgreater{}} \DecValTok{0}\OperatorTok{)} \OperatorTok{\{}
        \CommentTok{// код}
    \OperatorTok{\}} \ControlFlowTok{else} \OperatorTok{\{}
        \CommentTok{// код}
    \OperatorTok{\}}
\OperatorTok{\}} \ControlFlowTok{else} \OperatorTok{\{}
    \CommentTok{// код}
\OperatorTok{\}}
\end{Highlighting}
\end{Shaded}

  Операторы присваивания и «запятая» в условии \texttt{if} используются
  для выполнения нескольких операций. Пример:

\begin{Shaded}
\begin{Highlighting}[]
\ControlFlowTok{if} \OperatorTok{((}\NormalTok{x }\OperatorTok{=}\NormalTok{ func}\OperatorTok{())} \OperatorTok{!=} \DecValTok{0}\OperatorTok{)} \OperatorTok{\{}
    \CommentTok{// код}
\OperatorTok{\}}
\end{Highlighting}
\end{Shaded}

  Побитовые логические операции включают \texttt{\&} (побитовое И),
  \texttt{\textbar{}} (побитовое ИЛИ), \texttt{\^{}} (побитовое
  исключающее ИЛИ), \texttt{\textasciitilde{}} (побитовое НЕ),
  \texttt{\textless{}\textless{}} (сдвиг влево),
  \texttt{\textgreater{}\textgreater{}} (сдвиг вправо). Модификации
  оператора присваивания для побитовых операций: \texttt{\&=},
  \texttt{\textbar{}=}, \texttt{\^{}=},
  \texttt{\textless{}\textless{}=},
  \texttt{\textgreater{}\textgreater{}=}. Тернарный оператор \texttt{?:}
  используется для краткой записи условий. Пример:
  \texttt{int\ y\ =\ (x\ \textgreater{}\ 0)\ ?\ 1\ :\ -1;}.
\item
  \textbf{Цикл со счетчиком, цикл с предусловием и цикл с постусловием.
  Операторы досрочного завершения тела цикла. Использование оператора
  «запятая» в заголовке цикла. Организация циклов при помощи меток и
  безусловных переходов.}

  \textbf{Ответ:} Цикл со счетчиком \texttt{for}:

  ```cpp for (int i = 0; i \textless{} 10; ++i) \{ // код \}
\end{enumerate}

\begin{verbatim}
  Цикл с предусловием `while`:
  ```cpp
  while (x < 10) {
      // код
  }
\end{verbatim}

Цикл с постусловием \texttt{do-while}:

\begin{Shaded}
\begin{Highlighting}[]
\ControlFlowTok{do} \OperatorTok{\{}
    \CommentTok{// код}
\OperatorTok{\}} \ControlFlowTok{while} \OperatorTok{(}\NormalTok{x }\OperatorTok{\textless{}} \DecValTok{10}\OperatorTok{);}
\end{Highlighting}
\end{Shaded}

Операторы досрочного завершения \texttt{break} и \texttt{continue}:

\begin{Shaded}
\begin{Highlighting}[]
\ControlFlowTok{for} \OperatorTok{(}\DataTypeTok{int}\NormalTok{ i }\OperatorTok{=} \DecValTok{0}\OperatorTok{;}\NormalTok{ i }\OperatorTok{\textless{}} \DecValTok{10}\OperatorTok{;} \OperatorTok{++}\NormalTok{i}\OperatorTok{)} \OperatorTok{\{}
    \ControlFlowTok{if} \OperatorTok{(}\NormalTok{i }\OperatorTok{==} \DecValTok{5}\OperatorTok{)} \ControlFlowTok{break}\OperatorTok{;} \CommentTok{// завершает цикл}
    \ControlFlowTok{if} \OperatorTok{(}\NormalTok{i }\OperatorTok{\%} \DecValTok{2} \OperatorTok{==} \DecValTok{0}\OperatorTok{)} \ControlFlowTok{continue}\OperatorTok{;} \CommentTok{// переходит к следующей итерации}
    \CommentTok{// код}
\OperatorTok{\}}
\end{Highlighting}
\end{Shaded}

Оператор «запятая» в заголовке цикла:

\begin{Shaded}
\begin{Highlighting}[]
\ControlFlowTok{for} \OperatorTok{(}\DataTypeTok{int}\NormalTok{ i }\OperatorTok{=} \DecValTok{0}\OperatorTok{,}\NormalTok{ j }\OperatorTok{=} \DecValTok{10}\OperatorTok{;}\NormalTok{ i }\OperatorTok{\textless{}}\NormalTok{ j}\OperatorTok{;} \OperatorTok{++}\NormalTok{i}\OperatorTok{,} \OperatorTok{{-}{-}}\NormalTok{j}\OperatorTok{)} \OperatorTok{\{}
    \CommentTok{// код}
\OperatorTok{\}}
\end{Highlighting}
\end{Shaded}

Организация циклов с использованием меток и безусловных переходов:

\begin{Shaded}
\begin{Highlighting}[]
\ControlFlowTok{for} \OperatorTok{(}\DataTypeTok{int}\NormalTok{ i }\OperatorTok{=} \DecValTok{0}\OperatorTok{;}\NormalTok{ i }\OperatorTok{\textless{}} \DecValTok{10}\OperatorTok{;} \OperatorTok{++}\NormalTok{i}\OperatorTok{)} \OperatorTok{\{}
\NormalTok{loop\_start}\OperatorTok{:}
    \ControlFlowTok{if} \OperatorTok{(}\NormalTok{i }\OperatorTok{==} \DecValTok{5}\OperatorTok{)} \ControlFlowTok{goto}\NormalTok{ loop\_end}\OperatorTok{;}
    \CommentTok{// код}
\OperatorTok{\}}
\NormalTok{loop\_end}\OperatorTok{:}
\CommentTok{// код после цикла}
\end{Highlighting}
\end{Shaded}

\begin{enumerate}
\def\labelenumi{\arabic{enumi}.}
\setcounter{enumi}{8}
\item
  \textbf{Вложенные циклы. Использование операторов прерывания во
  вложенных циклах. Организация вложенных циклов при помощи меток и
  безусловных переходов.}

  \textbf{Ответ:} Вложенные циклы позволяют выполнять многократные
  повторения внутри других циклов. Пример:

\begin{Shaded}
\begin{Highlighting}[]
\ControlFlowTok{for} \OperatorTok{(}\DataTypeTok{int}\NormalTok{ i }\OperatorTok{=} \DecValTok{0}\OperatorTok{;}\NormalTok{ i }\OperatorTok{\textless{}} \DecValTok{10}\OperatorTok{;} \OperatorTok{++}\NormalTok{i}\OperatorTok{)} \OperatorTok{\{}
    \ControlFlowTok{for} \OperatorTok{(}\DataTypeTok{int}\NormalTok{ j }\OperatorTok{=} \DecValTok{0}\OperatorTok{;}\NormalTok{ j }\OperatorTok{\textless{}} \DecValTok{10}\OperatorTok{;} \OperatorTok{++}\NormalTok{j}\OperatorTok{)} \OperatorTok{\{}
        \CommentTok{// код}
        \ControlFlowTok{if} \OperatorTok{(}\NormalTok{j }\OperatorTok{==} \DecValTok{5}\OperatorTok{)} \ControlFlowTok{break}\OperatorTok{;} \CommentTok{// прерывание внутреннего цикла}
    \OperatorTok{\}}
\OperatorTok{\}}
\end{Highlighting}
\end{Shaded}

  Прерывание во вложенных циклах можно осуществлять с помощью меток и
  операторов \texttt{goto}:

\begin{Shaded}
\begin{Highlighting}[]
\ControlFlowTok{for} \OperatorTok{(}\DataTypeTok{int}\NormalTok{ i }\OperatorTok{=} \DecValTok{0}\OperatorTok{;}\NormalTok{ i }\OperatorTok{\textless{}} \DecValTok{10}\OperatorTok{;} \OperatorTok{++}\NormalTok{i}\OperatorTok{)} \OperatorTok{\{}
    \ControlFlowTok{for} \OperatorTok{(}\DataTypeTok{int}\NormalTok{ j }\OperatorTok{=} \DecValTok{0}\OperatorTok{;}\NormalTok{ j }\OperatorTok{\textless{}} \DecValTok{10}\OperatorTok{;} \OperatorTok{++}\NormalTok{j}\OperatorTok{)} \OperatorTok{\{}
        \ControlFlowTok{if} \OperatorTok{(}\NormalTok{j }\OperatorTok{==} \DecValTok{5}\OperatorTok{)} \ControlFlowTok{goto}\NormalTok{ end\_loops}\OperatorTok{;} \CommentTok{// прерывание обоих циклов}
        \CommentTok{// код}
    \OperatorTok{\}}
\OperatorTok{\}}
\NormalTok{end\_loops}\OperatorTok{:}
\CommentTok{// код после циклов}
\end{Highlighting}
\end{Shaded}
\item
  \textbf{Алгоритмы нахождения делителей числа, определение простоты
  числа, вычисления факториала натурального числа, выделения цифр
  натурального числа.}

  \textbf{Ответ:} Нахождение делителей числа:

\begin{Shaded}
\begin{Highlighting}[]
\ControlFlowTok{for} \OperatorTok{(}\DataTypeTok{int}\NormalTok{ i }\OperatorTok{=} \DecValTok{1}\OperatorTok{;}\NormalTok{ i }\OperatorTok{\textless{}=}\NormalTok{ n}\OperatorTok{;} \OperatorTok{++}\NormalTok{i}\OperatorTok{)} \OperatorTok{\{}
    \ControlFlowTok{if} \OperatorTok{(}\NormalTok{n }\OperatorTok{\%}\NormalTok{ i }\OperatorTok{==} \DecValTok{0}\OperatorTok{)} \OperatorTok{\{}
        \CommentTok{// i {-} делитель n}
    \OperatorTok{\}}
\OperatorTok{\}}
\end{Highlighting}
\end{Shaded}

  Определение простоты числа:

\begin{Shaded}
\begin{Highlighting}[]
\DataTypeTok{bool}\NormalTok{ isPrime}\OperatorTok{(}\DataTypeTok{int}\NormalTok{ n}\OperatorTok{)} \OperatorTok{\{}
    \ControlFlowTok{if} \OperatorTok{(}\NormalTok{n }\OperatorTok{\textless{}=} \DecValTok{1}\OperatorTok{)} \ControlFlowTok{return} \KeywordTok{false}\OperatorTok{;}
    \ControlFlowTok{for} \OperatorTok{(}\DataTypeTok{int}\NormalTok{ i }\OperatorTok{=} \DecValTok{2}\OperatorTok{;}\NormalTok{ i }\OperatorTok{\textless{}=}\NormalTok{ sqrt}\OperatorTok{(}\NormalTok{n}\OperatorTok{);} \OperatorTok{++}\NormalTok{i}\OperatorTok{)} \OperatorTok{\{}
        \ControlFlowTok{if} \OperatorTok{(}\NormalTok{n }\OperatorTok{\%}\NormalTok{ i }\OperatorTok{==} \DecValTok{0}\OperatorTok{)} \ControlFlowTok{return} \KeywordTok{false}\OperatorTok{;}
    \OperatorTok{\}}
    \ControlFlowTok{return} \KeywordTok{true}\OperatorTok{;}
\OperatorTok{\}}
\end{Highlighting}
\end{Shaded}

  Вычисление факториала:

\begin{Shaded}
\begin{Highlighting}[]
\DataTypeTok{int}\NormalTok{ factorial}\OperatorTok{(}\DataTypeTok{int}\NormalTok{ n}\OperatorTok{)} \OperatorTok{\{}
    \DataTypeTok{int}\NormalTok{ result }\OperatorTok{=} \DecValTok{1}\OperatorTok{;}
    \ControlFlowTok{for} \OperatorTok{(}\DataTypeTok{int}\NormalTok{ i }\OperatorTok{=} \DecValTok{1}\OperatorTok{;}\NormalTok{ i }\OperatorTok{\textless{}=}\NormalTok{ n}\OperatorTok{;} \OperatorTok{++}\NormalTok{i}\OperatorTok{)} \OperatorTok{\{}
\NormalTok{        result }\OperatorTok{*=}\NormalTok{ i}\OperatorTok{;}
    \OperatorTok{\}}
    \ControlFlowTok{return}\NormalTok{ result}\OperatorTok{;}
\OperatorTok{\}}
\end{Highlighting}
\end{Shaded}

  Выделение цифр числа:

\begin{Shaded}
\begin{Highlighting}[]
\DataTypeTok{void}\NormalTok{ extractDigits}\OperatorTok{(}\DataTypeTok{int}\NormalTok{ n}\OperatorTok{)} \OperatorTok{\{}
    \ControlFlowTok{while} \OperatorTok{(}\NormalTok{n }\OperatorTok{\textgreater{}} \DecValTok{0}\OperatorTok{)} \OperatorTok{\{}
        \DataTypeTok{int}\NormalTok{ digit }\OperatorTok{=}\NormalTok{ n }\OperatorTok{\%} \DecValTok{10}\OperatorTok{;}
\NormalTok{        n }\OperatorTok{/=} \DecValTok{10}\OperatorTok{;}
        \CommentTok{// код для обработки digit}
    \OperatorTok{\}}
\OperatorTok{\}}
\end{Highlighting}
\end{Shaded}
\item
  \textbf{Статические одномерные массивы. Способы инициализации
  элементов массива. Обращение к содержимому ячейки массива. Оператор и
  функция sizeof. Ручной ввод-вывод массивов. Заполнение элементов
  массива случайными числами.}

  \textbf{Ответ:} Объявление и инициализация статического одномерного
  массива:

\begin{Shaded}
\begin{Highlighting}[]
\DataTypeTok{int}\NormalTok{ arr}\OperatorTok{[}\DecValTok{5}\OperatorTok{]} \OperatorTok{=} \OperatorTok{\{}\DecValTok{1}\OperatorTok{,} \DecValTok{2}\OperatorTok{,} \DecValTok{3}\OperatorTok{,} \DecValTok{4}\OperatorTok{,} \DecValTok{5}\OperatorTok{\};}
\end{Highlighting}
\end{Shaded}

  Обращение к элементам массива:

\begin{Shaded}
\begin{Highlighting}[]
\DataTypeTok{int}\NormalTok{ x }\OperatorTok{=}\NormalTok{ arr}\OperatorTok{[}\DecValTok{2}\OperatorTok{];} \CommentTok{// доступ к третьему элементу}
\NormalTok{arr}\OperatorTok{[}\DecValTok{0}\OperatorTok{]} \OperatorTok{=} \DecValTok{10}\OperatorTok{;} \CommentTok{// изменение первого элемента}
\end{Highlighting}
\end{Shaded}

  Оператор и функция \texttt{sizeof}:

\begin{Shaded}
\begin{Highlighting}[]
\DataTypeTok{int}\NormalTok{ size }\OperatorTok{=} \KeywordTok{sizeof}\OperatorTok{(}\NormalTok{arr}\OperatorTok{)} \OperatorTok{/} \KeywordTok{sizeof}\OperatorTok{(}\NormalTok{arr}\OperatorTok{[}\DecValTok{0}\OperatorTok{]);} \CommentTok{// количество элементов в массиве}
\end{Highlighting}
\end{Shaded}

  Ручной ввод массива:

\begin{Shaded}
\begin{Highlighting}[]
\ControlFlowTok{for} \OperatorTok{(}\DataTypeTok{int}\NormalTok{ i }\OperatorTok{=} \DecValTok{0}\OperatorTok{;}\NormalTok{ i }\OperatorTok{\textless{}}\NormalTok{ size}\OperatorTok{;} \OperatorTok{++}\NormalTok{i}\OperatorTok{)} \OperatorTok{\{}
\NormalTok{    cin }\OperatorTok{\textgreater{}\textgreater{}}\NormalTok{ arr}\OperatorTok{[}\NormalTok{i}\OperatorTok{];}
\OperatorTok{\}}
\end{Highlighting}
\end{Shaded}

  Вывод массива:

\begin{Shaded}
\begin{Highlighting}[]
\ControlFlowTok{for} \OperatorTok{(}\DataTypeTok{int}\NormalTok{ i }\OperatorTok{=} \DecValTok{0}\OperatorTok{;}\NormalTok{ i }\OperatorTok{\textless{}}\NormalTok{ size}\OperatorTok{;} \OperatorTok{++}\NormalTok{i}\OperatorTok{)} \OperatorTok{\{}
\NormalTok{    cout }\OperatorTok{\textless{}\textless{}}\NormalTok{ arr}\OperatorTok{[}\NormalTok{i}\OperatorTok{]} \OperatorTok{\textless{}\textless{}} \StringTok{" "}\OperatorTok{;}
\OperatorTok{\}}
\end{Highlighting}
\end{Shaded}

  Заполнение случайными числами:

\begin{Shaded}
\begin{Highlighting}[]
\PreprocessorTok{\#include }\ImportTok{\textless{}cstdlib\textgreater{}}
\PreprocessorTok{\#include }\ImportTok{\textless{}ctime\textgreater{}}
\NormalTok{srand}\OperatorTok{(}\NormalTok{time}\OperatorTok{(}\DecValTok{0}\OperatorTok{));} \CommentTok{// инициализация генератора случайных чисел}
\ControlFlowTok{for} \OperatorTok{(}\DataTypeTok{int}\NormalTok{ i }\OperatorTok{=} \DecValTok{0}\OperatorTok{;}\NormalTok{ i }\OperatorTok{\textless{}}\NormalTok{ size}\OperatorTok{;} \OperatorTok{++}\NormalTok{i}\OperatorTok{)} \OperatorTok{\{}
\NormalTok{    arr}\OperatorTok{[}\NormalTok{i}\OperatorTok{]} \OperatorTok{=}\NormalTok{ rand}\OperatorTok{()} \OperatorTok{\%} \DecValTok{100}\OperatorTok{;} \CommentTok{// случайное число от 0 до 99}
\OperatorTok{\}}
\end{Highlighting}
\end{Shaded}
\item
  \textbf{Статические многомерные массивы. Способы инициализации
  элементов массива. Обращение к содержимому ячейки массива. Оператор и
  функция sizeof. Ручной ввод-вывод массивов. Заполнение элементов
  массива случайными числами.}

  \textbf{Ответ:} Объявление и инициализация статического многомерного
  массива:

\begin{Shaded}
\begin{Highlighting}[]
\DataTypeTok{int}\NormalTok{ matrix}\OperatorTok{[}\DecValTok{2}\OperatorTok{][}\DecValTok{3}\OperatorTok{]} \OperatorTok{=} \OperatorTok{\{\{}\DecValTok{1}\OperatorTok{,} \DecValTok{2}\OperatorTok{,} \DecValTok{3}\OperatorTok{\},} \OperatorTok{\{}\DecValTok{4}\OperatorTok{,} \DecValTok{5}\OperatorTok{,} \DecValTok{6}\OperatorTok{\}\};}
\end{Highlighting}
\end{Shaded}

  Обращение к элементам массива:

\begin{Shaded}
\begin{Highlighting}[]
\DataTypeTok{int}\NormalTok{ x }\OperatorTok{=}\NormalTok{ matrix}\OperatorTok{[}\DecValTok{1}\OperatorTok{][}\DecValTok{2}\OperatorTok{];} \CommentTok{// доступ к элементу в 2{-}й строке и 3{-}м столбце}
\NormalTok{matrix}\OperatorTok{[}\DecValTok{0}\OperatorTok{][}\DecValTok{0}\OperatorTok{]} \OperatorTok{=} \DecValTok{10}\OperatorTok{;} \CommentTok{// изменение элемента в 1{-}й строке и 1{-}м столбце}
\end{Highlighting}
\end{Shaded}

  Оператор и функция \texttt{sizeof}:

\begin{Shaded}
\begin{Highlighting}[]
\DataTypeTok{int}\NormalTok{ rows }\OperatorTok{=} \KeywordTok{sizeof}\OperatorTok{(}\NormalTok{matrix}\OperatorTok{)} \OperatorTok{/} \KeywordTok{sizeof}\OperatorTok{(}\NormalTok{matrix}\OperatorTok{[}\DecValTok{0}\OperatorTok{]);}
\DataTypeTok{int}\NormalTok{ cols }\OperatorTok{=} \KeywordTok{sizeof}\OperatorTok{(}\NormalTok{matrix}\OperatorTok{[}\DecValTok{0}\OperatorTok{])} \OperatorTok{/} \KeywordTok{sizeof}\OperatorTok{(}\NormalTok{matrix}\OperatorTok{[}\DecValTok{0}\OperatorTok{][}\DecValTok{0}\OperatorTok{]);}
\end{Highlighting}
\end{Shaded}

  Ручной ввод массива:

\begin{Shaded}
\begin{Highlighting}[]
\ControlFlowTok{for} \OperatorTok{(}\DataTypeTok{int}\NormalTok{ i }\OperatorTok{=} \DecValTok{0}\OperatorTok{;}\NormalTok{ i }\OperatorTok{\textless{}}\NormalTok{ rows}\OperatorTok{;} \OperatorTok{++}\NormalTok{i}\OperatorTok{)} \OperatorTok{\{}
    \ControlFlowTok{for} \OperatorTok{(}\DataTypeTok{int}\NormalTok{ j }\OperatorTok{=} \DecValTok{0}\OperatorTok{;}\NormalTok{ j }\OperatorTok{\textless{}}\NormalTok{ cols}\OperatorTok{;} \OperatorTok{++}\NormalTok{j}\OperatorTok{)} \OperatorTok{\{}
\NormalTok{        cin }\OperatorTok{\textgreater{}\textgreater{}}\NormalTok{ matrix}\OperatorTok{[}\NormalTok{i}\OperatorTok{][}\NormalTok{j}\OperatorTok{];}
    \OperatorTok{\}}
\OperatorTok{\}}
\end{Highlighting}
\end{Shaded}

  Вывод массива:

\begin{Shaded}
\begin{Highlighting}[]
\ControlFlowTok{for} \OperatorTok{(}\DataTypeTok{int}\NormalTok{ i }\OperatorTok{=} \DecValTok{0}\OperatorTok{;}\NormalTok{ i }\OperatorTok{\textless{}}\NormalTok{ rows}\OperatorTok{;} \OperatorTok{++}\NormalTok{i}\OperatorTok{)} \OperatorTok{\{}
    \ControlFlowTok{for} \OperatorTok{(}\DataTypeTok{int}\NormalTok{ j }\OperatorTok{=} \DecValTok{0}\OperatorTok{;}\NormalTok{ j }\OperatorTok{\textless{}}\NormalTok{ cols}\OperatorTok{;} \OperatorTok{++}\NormalTok{j}\OperatorTok{)} \OperatorTok{\{}
\NormalTok{        cout }\OperatorTok{\textless{}\textless{}}\NormalTok{ matrix}\OperatorTok{[}\NormalTok{i}\OperatorTok{][}\NormalTok{j}\OperatorTok{]} \OperatorTok{\textless{}\textless{}} \StringTok{" "}\OperatorTok{;}
    \OperatorTok{\}}
\NormalTok{    cout }\OperatorTok{\textless{}\textless{}}\NormalTok{ endl}\OperatorTok{;}
\OperatorTok{\}}
\end{Highlighting}
\end{Shaded}

  Заполнение случайными числами:

\begin{Shaded}
\begin{Highlighting}[]
\PreprocessorTok{\#include }\ImportTok{\textless{}cstdlib\textgreater{}}
\PreprocessorTok{\#include }\ImportTok{\textless{}ctime\textgreater{}}
\NormalTok{srand}\OperatorTok{(}\NormalTok{time}\OperatorTok{(}\DecValTok{0}\OperatorTok{));} \CommentTok{// инициализация генератора случайных чисел}
\ControlFlowTok{for} \OperatorTok{(}\DataTypeTok{int}\NormalTok{ i }\OperatorTok{=} \DecValTok{0}\OperatorTok{;}\NormalTok{ i }\OperatorTok{\textless{}}\NormalTok{ rows}\OperatorTok{;} \OperatorTok{++}\NormalTok{i}\OperatorTok{)} \OperatorTok{\{}
    \ControlFlowTok{for} \OperatorTok{(}\DataTypeTok{int}\NormalTok{ j }\OperatorTok{=} \DecValTok{0}\OperatorTok{;}\NormalTok{ j }\OperatorTok{\textless{}}\NormalTok{ cols}\OperatorTok{;} \OperatorTok{++}\NormalTok{j}\OperatorTok{)} \OperatorTok{\{}
\NormalTok{        matrix}\OperatorTok{[}\NormalTok{i}\OperatorTok{][}\NormalTok{j}\OperatorTok{]} \OperatorTok{=}\NormalTok{ rand}\OperatorTok{()} \OperatorTok{\%} \DecValTok{100}\OperatorTok{;} \CommentTok{// случайное число от 0 до 99}
    \OperatorTok{\}}
\OperatorTok{\}}
\end{Highlighting}
\end{Shaded}
\item
  \textbf{Динамические одномерные массивы в языке C++. Понятие
  указателя, операции над указателями, инициализация указателя.
  Динамическое распределение памяти для одномерных массивов с
  использованием операторов new и delete.}

  \textbf{Ответ:} Объявление указателя и динамическое распределение
  памяти:

\begin{Shaded}
\begin{Highlighting}[]
\DataTypeTok{int}\OperatorTok{*}\NormalTok{ arr }\OperatorTok{=} \KeywordTok{new} \DataTypeTok{int}\OperatorTok{[}\DecValTok{10}\OperatorTok{];} \CommentTok{// выделение памяти для массива из 10 элементов}
\end{Highlighting}
\end{Shaded}

  Обращение к элементам массива через указатель:

\begin{Shaded}
\begin{Highlighting}[]
\NormalTok{arr}\OperatorTok{[}\DecValTok{0}\OperatorTok{]} \OperatorTok{=} \DecValTok{1}\OperatorTok{;}
\DataTypeTok{int}\NormalTok{ x }\OperatorTok{=}\NormalTok{ arr}\OperatorTok{[}\DecValTok{1}\OperatorTok{];}
\end{Highlighting}
\end{Shaded}

  Освобождение памяти:

\begin{Shaded}
\begin{Highlighting}[]
\KeywordTok{delete}\OperatorTok{[]}\NormalTok{ arr}\OperatorTok{;}
\end{Highlighting}
\end{Shaded}
\item
  \textbf{Динамические одномерные массивы в языке C. Понятие указателя,
  операции над указателями, инициализация указателя. Динамическое
  распределение памяти для одномерных массивов с использованием функций
  malloc, calloc, realloc, free.}

  \textbf{Ответ:} Объявление указателя и динамическое распределение
  памяти:

\begin{Shaded}
\begin{Highlighting}[]
\DataTypeTok{int}\OperatorTok{*}\NormalTok{ arr }\OperatorTok{=} \OperatorTok{(}\DataTypeTok{int}\OperatorTok{*)}\NormalTok{malloc}\OperatorTok{(}\DecValTok{10} \OperatorTok{*} \KeywordTok{sizeof}\OperatorTok{(}\DataTypeTok{int}\OperatorTok{));} \CommentTok{// выделение памяти для массива из 10 элементов}
\end{Highlighting}
\end{Shaded}

  Обращение к элементам массива через указатель:

\begin{Shaded}
\begin{Highlighting}[]
\NormalTok{arr}\OperatorTok{[}\DecValTok{0}\OperatorTok{]} \OperatorTok{=} \DecValTok{1}\OperatorTok{;}
\DataTypeTok{int}\NormalTok{ x }\OperatorTok{=}\NormalTok{ arr}\OperatorTok{[}\DecValTok{1}\OperatorTok{];}
\end{Highlighting}
\end{Shaded}

  Освобождение памяти:

\begin{Shaded}
\begin{Highlighting}[]
\NormalTok{free}\OperatorTok{(}\NormalTok{arr}\OperatorTok{);}
\end{Highlighting}
\end{Shaded}

  Использование \texttt{calloc} и \texttt{realloc}:

\begin{Shaded}
\begin{Highlighting}[]
\DataTypeTok{int}\OperatorTok{*}\NormalTok{ arr }\OperatorTok{=} \OperatorTok{(}\DataTypeTok{int}\OperatorTok{*)}\NormalTok{calloc}\OperatorTok{(}\DecValTok{10}\OperatorTok{,} \KeywordTok{sizeof}\OperatorTok{(}\DataTypeTok{int}\OperatorTok{));} \CommentTok{// выделение и инициализация памяти}
\NormalTok{arr }\OperatorTok{=} \OperatorTok{(}\DataTypeTok{int}\OperatorTok{*)}\NormalTok{realloc}\OperatorTok{(}\NormalTok{arr}\OperatorTok{,} \DecValTok{20} \OperatorTok{*} \KeywordTok{sizeof}\OperatorTok{(}\DataTypeTok{int}\OperatorTok{));} \CommentTok{// изменение размера массива}
\NormalTok{free}\OperatorTok{(}\NormalTok{arr}\OperatorTok{);}
\end{Highlighting}
\end{Shaded}
\item
  \textbf{Алгоритмы сортировки элементов массива: пузырьковый, вставкой,
  выбором.}

  \textbf{Ответ:} Сортировка пузырьком:

\begin{Shaded}
\begin{Highlighting}[]


\DataTypeTok{void}\NormalTok{ bubbleSort}\OperatorTok{(}\DataTypeTok{int}\NormalTok{ arr}\OperatorTok{[],} \DataTypeTok{int}\NormalTok{ size}\OperatorTok{)} \OperatorTok{\{}
    \ControlFlowTok{for} \OperatorTok{(}\DataTypeTok{int}\NormalTok{ i }\OperatorTok{=} \DecValTok{0}\OperatorTok{;}\NormalTok{ i }\OperatorTok{\textless{}}\NormalTok{ size }\OperatorTok{{-}} \DecValTok{1}\OperatorTok{;} \OperatorTok{++}\NormalTok{i}\OperatorTok{)} \OperatorTok{\{}
        \ControlFlowTok{for} \OperatorTok{(}\DataTypeTok{int}\NormalTok{ j }\OperatorTok{=} \DecValTok{0}\OperatorTok{;}\NormalTok{ j }\OperatorTok{\textless{}}\NormalTok{ size }\OperatorTok{{-}}\NormalTok{ i }\OperatorTok{{-}} \DecValTok{1}\OperatorTok{;} \OperatorTok{++}\NormalTok{j}\OperatorTok{)} \OperatorTok{\{}
            \ControlFlowTok{if} \OperatorTok{(}\NormalTok{arr}\OperatorTok{[}\NormalTok{j}\OperatorTok{]} \OperatorTok{\textgreater{}}\NormalTok{ arr}\OperatorTok{[}\NormalTok{j }\OperatorTok{+} \DecValTok{1}\OperatorTok{])} \OperatorTok{\{}
                \BuiltInTok{std::}\NormalTok{swap}\OperatorTok{(}\NormalTok{arr}\OperatorTok{[}\NormalTok{j}\OperatorTok{],}\NormalTok{ arr}\OperatorTok{[}\NormalTok{j }\OperatorTok{+} \DecValTok{1}\OperatorTok{]);}
            \OperatorTok{\}}
        \OperatorTok{\}}
    \OperatorTok{\}}
\OperatorTok{\}}
\end{Highlighting}
\end{Shaded}

  Сортировка вставкой:

\begin{Shaded}
\begin{Highlighting}[]
\DataTypeTok{void}\NormalTok{ insertionSort}\OperatorTok{(}\DataTypeTok{int}\NormalTok{ arr}\OperatorTok{[],} \DataTypeTok{int}\NormalTok{ size}\OperatorTok{)} \OperatorTok{\{}
    \ControlFlowTok{for} \OperatorTok{(}\DataTypeTok{int}\NormalTok{ i }\OperatorTok{=} \DecValTok{1}\OperatorTok{;}\NormalTok{ i }\OperatorTok{\textless{}}\NormalTok{ size}\OperatorTok{;} \OperatorTok{++}\NormalTok{i}\OperatorTok{)} \OperatorTok{\{}
        \DataTypeTok{int}\NormalTok{ key }\OperatorTok{=}\NormalTok{ arr}\OperatorTok{[}\NormalTok{i}\OperatorTok{];}
        \DataTypeTok{int}\NormalTok{ j }\OperatorTok{=}\NormalTok{ i }\OperatorTok{{-}} \DecValTok{1}\OperatorTok{;}
        \ControlFlowTok{while} \OperatorTok{(}\NormalTok{j }\OperatorTok{\textgreater{}=} \DecValTok{0} \OperatorTok{\&\&}\NormalTok{ arr}\OperatorTok{[}\NormalTok{j}\OperatorTok{]} \OperatorTok{\textgreater{}}\NormalTok{ key}\OperatorTok{)} \OperatorTok{\{}
\NormalTok{            arr}\OperatorTok{[}\NormalTok{j }\OperatorTok{+} \DecValTok{1}\OperatorTok{]} \OperatorTok{=}\NormalTok{ arr}\OperatorTok{[}\NormalTok{j}\OperatorTok{];}
            \OperatorTok{{-}{-}}\NormalTok{j}\OperatorTok{;}
        \OperatorTok{\}}
\NormalTok{        arr}\OperatorTok{[}\NormalTok{j }\OperatorTok{+} \DecValTok{1}\OperatorTok{]} \OperatorTok{=}\NormalTok{ key}\OperatorTok{;}
    \OperatorTok{\}}
\OperatorTok{\}}
\end{Highlighting}
\end{Shaded}

  Сортировка выбором:

\begin{Shaded}
\begin{Highlighting}[]
\DataTypeTok{void}\NormalTok{ selectionSort}\OperatorTok{(}\DataTypeTok{int}\NormalTok{ arr}\OperatorTok{[],} \DataTypeTok{int}\NormalTok{ size}\OperatorTok{)} \OperatorTok{\{}
    \ControlFlowTok{for} \OperatorTok{(}\DataTypeTok{int}\NormalTok{ i }\OperatorTok{=} \DecValTok{0}\OperatorTok{;}\NormalTok{ i }\OperatorTok{\textless{}}\NormalTok{ size }\OperatorTok{{-}} \DecValTok{1}\OperatorTok{;} \OperatorTok{++}\NormalTok{i}\OperatorTok{)} \OperatorTok{\{}
        \DataTypeTok{int}\NormalTok{ minIndex }\OperatorTok{=}\NormalTok{ i}\OperatorTok{;}
        \ControlFlowTok{for} \OperatorTok{(}\DataTypeTok{int}\NormalTok{ j }\OperatorTok{=}\NormalTok{ i }\OperatorTok{+} \DecValTok{1}\OperatorTok{;}\NormalTok{ j }\OperatorTok{\textless{}}\NormalTok{ size}\OperatorTok{;} \OperatorTok{++}\NormalTok{j}\OperatorTok{)} \OperatorTok{\{}
            \ControlFlowTok{if} \OperatorTok{(}\NormalTok{arr}\OperatorTok{[}\NormalTok{j}\OperatorTok{]} \OperatorTok{\textless{}}\NormalTok{ arr}\OperatorTok{[}\NormalTok{minIndex}\OperatorTok{])} \OperatorTok{\{}
\NormalTok{                minIndex }\OperatorTok{=}\NormalTok{ j}\OperatorTok{;}
            \OperatorTok{\}}
        \OperatorTok{\}}
        \BuiltInTok{std::}\NormalTok{swap}\OperatorTok{(}\NormalTok{arr}\OperatorTok{[}\NormalTok{i}\OperatorTok{],}\NormalTok{ arr}\OperatorTok{[}\NormalTok{minIndex}\OperatorTok{]);}
    \OperatorTok{\}}
\OperatorTok{\}}
\end{Highlighting}
\end{Shaded}
\end{enumerate}

\subsubsection{16. Алгоритмы поиска минимального (максимального),
нахождение суммы (произведения) элементов, сдвиги элементов одномерных
массивов}\label{ux430ux43bux433ux43eux440ux438ux442ux43cux44b-ux43fux43eux438ux441ux43aux430-ux43cux438ux43dux438ux43cux430ux43bux44cux43dux43eux433ux43e-ux43cux430ux43aux441ux438ux43cux430ux43bux44cux43dux43eux433ux43e-ux43dux430ux445ux43eux436ux434ux435ux43dux438ux435-ux441ux443ux43cux43cux44b-ux43fux440ux43eux438ux437ux432ux435ux434ux435ux43dux438ux44f-ux44dux43bux435ux43cux435ux43dux442ux43eux432-ux441ux434ux432ux438ux433ux438-ux44dux43bux435ux43cux435ux43dux442ux43eux432-ux43eux434ux43dux43eux43cux435ux440ux43dux44bux445-ux43cux430ux441ux441ux438ux432ux43eux432}

\paragraph{Алгоритм поиска минимального (максимального)
элемента}\label{ux430ux43bux433ux43eux440ux438ux442ux43c-ux43fux43eux438ux441ux43aux430-ux43cux438ux43dux438ux43cux430ux43bux44cux43dux43eux433ux43e-ux43cux430ux43aux441ux438ux43cux430ux43bux44cux43dux43eux433ux43e-ux44dux43bux435ux43cux435ux43dux442ux430}

Для нахождения минимального или максимального элемента в одномерном
массиве используется следующий алгоритм:

\begin{enumerate}
\def\labelenumi{\arabic{enumi}.}
\tightlist
\item
  Инициализировать переменную для хранения минимального (максимального)
  значения первым элементом массива.
\item
  Перебрать все элементы массива:

  \begin{itemize}
  \tightlist
  \item
    Если текущий элемент меньше (больше) сохраненного значения, обновить
    минимальное (максимальное) значение.
  \end{itemize}
\item
  После завершения цикла в переменной будет храниться минимальное
  (максимальное) значение массива.
\end{enumerate}

Пример на языке C++:

\begin{Shaded}
\begin{Highlighting}[]
\PreprocessorTok{\#include }\ImportTok{\textless{}iostream\textgreater{}}
\KeywordTok{using} \KeywordTok{namespace}\NormalTok{ std}\OperatorTok{;}

\DataTypeTok{int}\NormalTok{ main}\OperatorTok{()} \OperatorTok{\{}
    \DataTypeTok{int}\NormalTok{ arr}\OperatorTok{[]} \OperatorTok{=} \OperatorTok{\{}\DecValTok{4}\OperatorTok{,} \DecValTok{2}\OperatorTok{,} \DecValTok{8}\OperatorTok{,} \DecValTok{1}\OperatorTok{,} \DecValTok{5}\OperatorTok{\};}
    \DataTypeTok{int}\NormalTok{ n }\OperatorTok{=} \KeywordTok{sizeof}\OperatorTok{(}\NormalTok{arr}\OperatorTok{)} \OperatorTok{/} \KeywordTok{sizeof}\OperatorTok{(}\NormalTok{arr}\OperatorTok{[}\DecValTok{0}\OperatorTok{]);}

    \DataTypeTok{int}\NormalTok{ min }\OperatorTok{=}\NormalTok{ arr}\OperatorTok{[}\DecValTok{0}\OperatorTok{];}
    \DataTypeTok{int}\NormalTok{ max }\OperatorTok{=}\NormalTok{ arr}\OperatorTok{[}\DecValTok{0}\OperatorTok{];}

    \ControlFlowTok{for} \OperatorTok{(}\DataTypeTok{int}\NormalTok{ i }\OperatorTok{=} \DecValTok{1}\OperatorTok{;}\NormalTok{ i }\OperatorTok{\textless{}}\NormalTok{ n}\OperatorTok{;}\NormalTok{ i}\OperatorTok{++)} \OperatorTok{\{}
        \ControlFlowTok{if} \OperatorTok{(}\NormalTok{arr}\OperatorTok{[}\NormalTok{i}\OperatorTok{]} \OperatorTok{\textless{}}\NormalTok{ min}\OperatorTok{)}
\NormalTok{            min }\OperatorTok{=}\NormalTok{ arr}\OperatorTok{[}\NormalTok{i}\OperatorTok{];}
        \ControlFlowTok{if} \OperatorTok{(}\NormalTok{arr}\OperatorTok{[}\NormalTok{i}\OperatorTok{]} \OperatorTok{\textgreater{}}\NormalTok{ max}\OperatorTok{)}
\NormalTok{            max }\OperatorTok{=}\NormalTok{ arr}\OperatorTok{[}\NormalTok{i}\OperatorTok{];}
    \OperatorTok{\}}

\NormalTok{    cout }\OperatorTok{\textless{}\textless{}} \StringTok{"Minimum: "} \OperatorTok{\textless{}\textless{}}\NormalTok{ min }\OperatorTok{\textless{}\textless{}}\NormalTok{ endl}\OperatorTok{;}
\NormalTok{    cout }\OperatorTok{\textless{}\textless{}} \StringTok{"Maximum: "} \OperatorTok{\textless{}\textless{}}\NormalTok{ max }\OperatorTok{\textless{}\textless{}}\NormalTok{ endl}\OperatorTok{;}

    \ControlFlowTok{return} \DecValTok{0}\OperatorTok{;}
\OperatorTok{\}}
\end{Highlighting}
\end{Shaded}

\paragraph{Алгоритм нахождения суммы (произведения)
элементов}\label{ux430ux43bux433ux43eux440ux438ux442ux43c-ux43dux430ux445ux43eux436ux434ux435ux43dux438ux44f-ux441ux443ux43cux43cux44b-ux43fux440ux43eux438ux437ux432ux435ux434ux435ux43dux438ux44f-ux44dux43bux435ux43cux435ux43dux442ux43eux432}

Для нахождения суммы или произведения элементов одномерного массива:

\begin{enumerate}
\def\labelenumi{\arabic{enumi}.}
\tightlist
\item
  Инициализировать переменную для суммы нулем (для произведения
  единицей).
\item
  Перебрать все элементы массива:

  \begin{itemize}
  \tightlist
  \item
    Для суммы: прибавить текущий элемент к переменной суммы.
  \item
    Для произведения: умножить текущий элемент на переменную
    произведения.
  \end{itemize}
\item
  После завершения цикла в переменной будет храниться сумма
  (произведение) всех элементов массива.
\end{enumerate}

Пример на языке C++:

\begin{Shaded}
\begin{Highlighting}[]
\PreprocessorTok{\#include }\ImportTok{\textless{}iostream\textgreater{}}
\KeywordTok{using} \KeywordTok{namespace}\NormalTok{ std}\OperatorTok{;}

\DataTypeTok{int}\NormalTok{ main}\OperatorTok{()} \OperatorTok{\{}
    \DataTypeTok{int}\NormalTok{ arr}\OperatorTok{[]} \OperatorTok{=} \OperatorTok{\{}\DecValTok{4}\OperatorTok{,} \DecValTok{2}\OperatorTok{,} \DecValTok{8}\OperatorTok{,} \DecValTok{1}\OperatorTok{,} \DecValTok{5}\OperatorTok{\};}
    \DataTypeTok{int}\NormalTok{ n }\OperatorTok{=} \KeywordTok{sizeof}\OperatorTok{(}\NormalTok{arr}\OperatorTok{)} \OperatorTok{/} \KeywordTok{sizeof}\OperatorTok{(}\NormalTok{arr}\OperatorTok{[}\DecValTok{0}\OperatorTok{]);}

    \DataTypeTok{int}\NormalTok{ sum }\OperatorTok{=} \DecValTok{0}\OperatorTok{;}
    \DataTypeTok{int}\NormalTok{ product }\OperatorTok{=} \DecValTok{1}\OperatorTok{;}

    \ControlFlowTok{for} \OperatorTok{(}\DataTypeTok{int}\NormalTok{ i }\OperatorTok{=} \DecValTok{0}\OperatorTok{;}\NormalTok{ i }\OperatorTok{\textless{}}\NormalTok{ n}\OperatorTok{;}\NormalTok{ i}\OperatorTok{++)} \OperatorTok{\{}
\NormalTok{        sum }\OperatorTok{+=}\NormalTok{ arr}\OperatorTok{[}\NormalTok{i}\OperatorTok{];}
\NormalTok{        product }\OperatorTok{*=}\NormalTok{ arr}\OperatorTok{[}\NormalTok{i}\OperatorTok{];}
    \OperatorTok{\}}

\NormalTok{    cout }\OperatorTok{\textless{}\textless{}} \StringTok{"Sum: "} \OperatorTok{\textless{}\textless{}}\NormalTok{ sum }\OperatorTok{\textless{}\textless{}}\NormalTok{ endl}\OperatorTok{;}
\NormalTok{    cout }\OperatorTok{\textless{}\textless{}} \StringTok{"Product: "} \OperatorTok{\textless{}\textless{}}\NormalTok{ product }\OperatorTok{\textless{}\textless{}}\NormalTok{ endl}\OperatorTok{;}

    \ControlFlowTok{return} \DecValTok{0}\OperatorTok{;}
\OperatorTok{\}}
\end{Highlighting}
\end{Shaded}

\paragraph{Алгоритм сдвига элементов
массива}\label{ux430ux43bux433ux43eux440ux438ux442ux43c-ux441ux434ux432ux438ux433ux430-ux44dux43bux435ux43cux435ux43dux442ux43eux432-ux43cux430ux441ux441ux438ux432ux430}

Для сдвига элементов массива влево или вправо:

\begin{enumerate}
\def\labelenumi{\arabic{enumi}.}
\tightlist
\item
  Сдвиг влево:

  \begin{itemize}
  \tightlist
  \item
    Сохранить первый элемент массива.
  \item
    Переместить каждый элемент массива на одну позицию влево.
  \item
    Последний элемент заменить сохраненным первым элементом.
  \end{itemize}
\item
  Сдвиг вправо:

  \begin{itemize}
  \tightlist
  \item
    Сохранить последний элемент массива.
  \item
    Переместить каждый элемент массива на одну позицию вправо.
  \item
    Первый элемент заменить сохраненным последним элементом.
  \end{itemize}
\end{enumerate}

Пример на языке C++ (сдвиг влево):

\begin{Shaded}
\begin{Highlighting}[]
\PreprocessorTok{\#include }\ImportTok{\textless{}iostream\textgreater{}}
\KeywordTok{using} \KeywordTok{namespace}\NormalTok{ std}\OperatorTok{;}

\DataTypeTok{void}\NormalTok{ shiftLeft}\OperatorTok{(}\DataTypeTok{int}\NormalTok{ arr}\OperatorTok{[],} \DataTypeTok{int}\NormalTok{ n}\OperatorTok{)} \OperatorTok{\{}
    \DataTypeTok{int}\NormalTok{ temp }\OperatorTok{=}\NormalTok{ arr}\OperatorTok{[}\DecValTok{0}\OperatorTok{];}
    \ControlFlowTok{for} \OperatorTok{(}\DataTypeTok{int}\NormalTok{ i }\OperatorTok{=} \DecValTok{0}\OperatorTok{;}\NormalTok{ i }\OperatorTok{\textless{}}\NormalTok{ n }\OperatorTok{{-}} \DecValTok{1}\OperatorTok{;}\NormalTok{ i}\OperatorTok{++)} \OperatorTok{\{}
\NormalTok{        arr}\OperatorTok{[}\NormalTok{i}\OperatorTok{]} \OperatorTok{=}\NormalTok{ arr}\OperatorTok{[}\NormalTok{i }\OperatorTok{+} \DecValTok{1}\OperatorTok{];}
    \OperatorTok{\}}
\NormalTok{    arr}\OperatorTok{[}\NormalTok{n }\OperatorTok{{-}} \DecValTok{1}\OperatorTok{]} \OperatorTok{=}\NormalTok{ temp}\OperatorTok{;}
\OperatorTok{\}}

\DataTypeTok{int}\NormalTok{ main}\OperatorTok{()} \OperatorTok{\{}
    \DataTypeTok{int}\NormalTok{ arr}\OperatorTok{[]} \OperatorTok{=} \OperatorTok{\{}\DecValTok{4}\OperatorTok{,} \DecValTok{2}\OperatorTok{,} \DecValTok{8}\OperatorTok{,} \DecValTok{1}\OperatorTok{,} \DecValTok{5}\OperatorTok{\};}
    \DataTypeTok{int}\NormalTok{ n }\OperatorTok{=} \KeywordTok{sizeof}\OperatorTok{(}\NormalTok{arr}\OperatorTok{)} \OperatorTok{/} \KeywordTok{sizeof}\OperatorTok{(}\NormalTok{arr}\OperatorTok{[}\DecValTok{0}\OperatorTok{]);}

\NormalTok{    shiftLeft}\OperatorTok{(}\NormalTok{arr}\OperatorTok{,}\NormalTok{ n}\OperatorTok{);}

    \ControlFlowTok{for} \OperatorTok{(}\DataTypeTok{int}\NormalTok{ i }\OperatorTok{=} \DecValTok{0}\OperatorTok{;}\NormalTok{ i }\OperatorTok{\textless{}}\NormalTok{ n}\OperatorTok{;}\NormalTok{ i}\OperatorTok{++)} \OperatorTok{\{}
\NormalTok{        cout }\OperatorTok{\textless{}\textless{}}\NormalTok{ arr}\OperatorTok{[}\NormalTok{i}\OperatorTok{]} \OperatorTok{\textless{}\textless{}} \StringTok{" "}\OperatorTok{;}
    \OperatorTok{\}}

    \ControlFlowTok{return} \DecValTok{0}\OperatorTok{;}
\OperatorTok{\}}
\end{Highlighting}
\end{Shaded}

Пример на языке C++ (сдвиг вправо):

\begin{Shaded}
\begin{Highlighting}[]
\PreprocessorTok{\#include }\ImportTok{\textless{}iostream\textgreater{}}
\KeywordTok{using} \KeywordTok{namespace}\NormalTok{ std}\OperatorTok{;}

\DataTypeTok{void}\NormalTok{ shiftRight}\OperatorTok{(}\DataTypeTok{int}\NormalTok{ arr}\OperatorTok{[],} \DataTypeTok{int}\NormalTok{ n}\OperatorTok{)} \OperatorTok{\{}
    \DataTypeTok{int}\NormalTok{ temp }\OperatorTok{=}\NormalTok{ arr}\OperatorTok{[}\NormalTok{n }\OperatorTok{{-}} \DecValTok{1}\OperatorTok{];}
    \ControlFlowTok{for} \OperatorTok{(}\DataTypeTok{int}\NormalTok{ i }\OperatorTok{=}\NormalTok{ n }\OperatorTok{{-}} \DecValTok{1}\OperatorTok{;}\NormalTok{ i }\OperatorTok{\textgreater{}} \DecValTok{0}\OperatorTok{;}\NormalTok{ i}\OperatorTok{{-}{-})} \OperatorTok{\{}
\NormalTok{        arr}\OperatorTok{[}\NormalTok{i}\OperatorTok{]} \OperatorTok{=}\NormalTok{ arr}\OperatorTok{[}\NormalTok{i }\OperatorTok{{-}} \DecValTok{1}\OperatorTok{];}
    \OperatorTok{\}}
\NormalTok{    arr}\OperatorTok{[}\DecValTok{0}\OperatorTok{]} \OperatorTok{=}\NormalTok{ temp}\OperatorTok{;}
\OperatorTok{\}}

\DataTypeTok{int}\NormalTok{ main}\OperatorTok{()} \OperatorTok{\{}
    \DataTypeTok{int}\NormalTok{ arr}\OperatorTok{[]} \OperatorTok{=} \OperatorTok{\{}\DecValTok{4}\OperatorTok{,} \DecValTok{2}\OperatorTok{,} \DecValTok{8}\OperatorTok{,} \DecValTok{1}\OperatorTok{,} \DecValTok{5}\OperatorTok{\};}
    \DataTypeTok{int}\NormalTok{ n }\OperatorTok{=} \KeywordTok{sizeof}\OperatorTok{(}\NormalTok{arr}\OperatorTok{)} \OperatorTok{/} \KeywordTok{sizeof}\OperatorTok{(}\NormalTok{arr}\OperatorTok{[}\DecValTok{0}\OperatorTok{]);}

\NormalTok{    shiftRight}\OperatorTok{(}\NormalTok{arr}\OperatorTok{,}\NormalTok{ n}\OperatorTok{);}

    \ControlFlowTok{for} \OperatorTok{(}\DataTypeTok{int}\NormalTok{ i }\OperatorTok{=} \DecValTok{0}\OperatorTok{;}\NormalTok{ i }\OperatorTok{\textless{}}\NormalTok{ n}\OperatorTok{;}\NormalTok{ i}\OperatorTok{++)} \OperatorTok{\{}
\NormalTok{        cout }\OperatorTok{\textless{}\textless{}}\NormalTok{ arr}\OperatorTok{[}\NormalTok{i}\OperatorTok{]} \OperatorTok{\textless{}\textless{}} \StringTok{" "}\OperatorTok{;}
    \OperatorTok{\}}

    \ControlFlowTok{return} \DecValTok{0}\OperatorTok{;}
\OperatorTok{\}}
\end{Highlighting}
\end{Shaded}

Эти алгоритмы позволяют эффективно выполнять основные операции с
одномерными массивами.

\subsubsection{17. Особенности описания многомерных динамических
массивов в языке C++. Динамическое распределение памяти для многомерных
массивов с использованием операторов new и
delete}\label{ux43eux441ux43eux431ux435ux43dux43dux43eux441ux442ux438-ux43eux43fux438ux441ux430ux43dux438ux44f-ux43cux43dux43eux433ux43eux43cux435ux440ux43dux44bux445-ux434ux438ux43dux430ux43cux438ux447ux435ux441ux43aux438ux445-ux43cux430ux441ux441ux438ux432ux43eux432-ux432-ux44fux437ux44bux43aux435-c.-ux434ux438ux43dux430ux43cux438ux447ux435ux441ux43aux43eux435-ux440ux430ux441ux43fux440ux435ux434ux435ux43bux435ux43dux438ux435-ux43fux430ux43cux44fux442ux438-ux434ux43bux44f-ux43cux43dux43eux433ux43eux43cux435ux440ux43dux44bux445-ux43cux430ux441ux441ux438ux432ux43eux432-ux441-ux438ux441ux43fux43eux43bux44cux437ux43eux432ux430ux43dux438ux435ux43c-ux43eux43fux435ux440ux430ux442ux43eux440ux43eux432-new-ux438-delete}

\paragraph{Особенности описания многомерных динамических массивов в
C++}\label{ux43eux441ux43eux431ux435ux43dux43dux43eux441ux442ux438-ux43eux43fux438ux441ux430ux43dux438ux44f-ux43cux43dux43eux433ux43eux43cux435ux440ux43dux44bux445-ux434ux438ux43dux430ux43cux438ux447ux435ux441ux43aux438ux445-ux43cux430ux441ux441ux438ux432ux43eux432-ux432-c}

В языке C++ многомерные динамические массивы могут быть описаны и
выделены с использованием указателей и операторов \texttt{new} и
\texttt{delete}. Основной особенностью является необходимость создания
массива указателей для каждой измерения массива.

\paragraph{Динамическое распределение памяти для многомерных
массивов}\label{ux434ux438ux43dux430ux43cux438ux447ux435ux441ux43aux43eux435-ux440ux430ux441ux43fux440ux435ux434ux435ux43bux435ux43dux438ux435-ux43fux430ux43cux44fux442ux438-ux434ux43bux44f-ux43cux43dux43eux433ux43eux43cux435ux440ux43dux44bux445-ux43cux430ux441ux441ux438ux432ux43eux432}

Для создания двумерного динамического массива сначала создается массив
указателей на строки, а затем для каждой строки выделяется память под
элементы.

Пример создания и удаления двумерного динамического массива:

\begin{Shaded}
\begin{Highlighting}[]
\PreprocessorTok{\#include }\ImportTok{\textless{}iostream\textgreater{}}
\KeywordTok{using} \KeywordTok{namespace}\NormalTok{ std}\OperatorTok{;}

\DataTypeTok{int}\NormalTok{ main}\OperatorTok{()} \OperatorTok{\{}
    \DataTypeTok{int}\NormalTok{ rows }\OperatorTok{=} \DecValTok{3}\OperatorTok{;}
    \DataTypeTok{int}\NormalTok{ cols }\OperatorTok{=} \DecValTok{4}\OperatorTok{;}

    \CommentTok{// Создание массива указателей на строки}
    \DataTypeTok{int}\OperatorTok{**}\NormalTok{ array }\OperatorTok{=} \KeywordTok{new} \DataTypeTok{int}\OperatorTok{*[}\NormalTok{rows}\OperatorTok{];}

    \CommentTok{// Выделение памяти для каждого ряда}
    \ControlFlowTok{for} \OperatorTok{(}\DataTypeTok{int}\NormalTok{ i }\OperatorTok{=} \DecValTok{0}\OperatorTok{;}\NormalTok{ i }\OperatorTok{\textless{}}\NormalTok{ rows}\OperatorTok{;} \OperatorTok{++}\NormalTok{i}\OperatorTok{)} \OperatorTok{\{}
\NormalTok{        array}\OperatorTok{[}\NormalTok{i}\OperatorTok{]} \OperatorTok{=} \KeywordTok{new} \DataTypeTok{int}\OperatorTok{[}\NormalTok{cols}\OperatorTok{];}
    \OperatorTok{\}}

    \CommentTok{// Инициализация и вывод массива}
    \ControlFlowTok{for} \OperatorTok{(}\DataTypeTok{int}\NormalTok{ i }\OperatorTok{=} \DecValTok{0}\OperatorTok{;}\NormalTok{ i }\OperatorTok{\textless{}}\NormalTok{ rows}\OperatorTok{;} \OperatorTok{++}\NormalTok{i}\OperatorTok{)} \OperatorTok{\{}
        \ControlFlowTok{for} \OperatorTok{(}\DataTypeTok{int}\NormalTok{ j }\OperatorTok{=} \DecValTok{0}\OperatorTok{;}\NormalTok{ j }\OperatorTok{\textless{}}\NormalTok{ cols}\OperatorTok{;} \OperatorTok{++}\NormalTok{j}\OperatorTok{)} \OperatorTok{\{}
\NormalTok{            array}\OperatorTok{[}\NormalTok{i}\OperatorTok{][}\NormalTok{j}\OperatorTok{]} \OperatorTok{=}\NormalTok{ i }\OperatorTok{*}\NormalTok{ cols }\OperatorTok{+}\NormalTok{ j}\OperatorTok{;}
\NormalTok{            cout }\OperatorTok{\textless{}\textless{}}\NormalTok{ array}\OperatorTok{[}\NormalTok{i}\OperatorTok{][}\NormalTok{j}\OperatorTok{]} \OperatorTok{\textless{}\textless{}} \StringTok{" "}\OperatorTok{;}
        \OperatorTok{\}}
\NormalTok{        cout }\OperatorTok{\textless{}\textless{}}\NormalTok{ endl}\OperatorTok{;}
    \OperatorTok{\}}

    \CommentTok{// Удаление массива}
    \ControlFlowTok{for} \OperatorTok{(}\DataTypeTok{int}\NormalTok{ i }\OperatorTok{=} \DecValTok{0}\OperatorTok{;}\NormalTok{ i }\OperatorTok{\textless{}}\NormalTok{ rows}\OperatorTok{;} \OperatorTok{++}\NormalTok{i}\OperatorTok{)} \OperatorTok{\{}
        \KeywordTok{delete}\OperatorTok{[]}\NormalTok{ array}\OperatorTok{[}\NormalTok{i}\OperatorTok{];}
    \OperatorTok{\}}
    \KeywordTok{delete}\OperatorTok{[]}\NormalTok{ array}\OperatorTok{;}

    \ControlFlowTok{return} \DecValTok{0}\OperatorTok{;}
\OperatorTok{\}}
\end{Highlighting}
\end{Shaded}

В этом примере сначала выделяется память для массива указателей на
строки, затем выделяется память для каждого ряда. После использования
массива память освобождается в обратном порядке: сначала освобождаются
строки, затем массив указателей.

\subsubsection{18. Особенности описания многомерных динамических
массивов в языке C. Динамическое распределение памяти для многомерных
массивов с использованием функций malloc, calloc, realloc,
free}\label{ux43eux441ux43eux431ux435ux43dux43dux43eux441ux442ux438-ux43eux43fux438ux441ux430ux43dux438ux44f-ux43cux43dux43eux433ux43eux43cux435ux440ux43dux44bux445-ux434ux438ux43dux430ux43cux438ux447ux435ux441ux43aux438ux445-ux43cux430ux441ux441ux438ux432ux43eux432-ux432-ux44fux437ux44bux43aux435-c.-ux434ux438ux43dux430ux43cux438ux447ux435ux441ux43aux43eux435-ux440ux430ux441ux43fux440ux435ux434ux435ux43bux435ux43dux438ux435-ux43fux430ux43cux44fux442ux438-ux434ux43bux44f-ux43cux43dux43eux433ux43eux43cux435ux440ux43dux44bux445-ux43cux430ux441ux441ux438ux432ux43eux432-ux441-ux438ux441ux43fux43eux43bux44cux437ux43eux432ux430ux43dux438ux435ux43c-ux444ux443ux43dux43aux446ux438ux439-malloc-calloc-realloc-free}

\paragraph{Особенности описания многомерных динамических массивов в
C}\label{ux43eux441ux43eux431ux435ux43dux43dux43eux441ux442ux438-ux43eux43fux438ux441ux430ux43dux438ux44f-ux43cux43dux43eux433ux43eux43cux435ux440ux43dux44bux445-ux434ux438ux43dux430ux43cux438ux447ux435ux441ux43aux438ux445-ux43cux430ux441ux441ux438ux432ux43eux432-ux432-c-1}

В языке C многомерные динамические массивы описываются и создаются с
использованием функций выделения памяти \texttt{malloc},
\texttt{calloc}, \texttt{realloc} и освобождения памяти \texttt{free}.

\paragraph{Динамическое распределение памяти для многомерных
массивов}\label{ux434ux438ux43dux430ux43cux438ux447ux435ux441ux43aux43eux435-ux440ux430ux441ux43fux440ux435ux434ux435ux43bux435ux43dux438ux435-ux43fux430ux43cux44fux442ux438-ux434ux43bux44f-ux43cux43dux43eux433ux43eux43cux435ux440ux43dux44bux445-ux43cux430ux441ux441ux438ux432ux43eux432-1}

Процесс создания двумерного динамического массива аналогичен процессу в
C++, но с использованием функций выделения памяти из стандартной
библиотеки C.

Пример создания и удаления двумерного динамического массива:

\begin{Shaded}
\begin{Highlighting}[]
\PreprocessorTok{\#include }\ImportTok{\textless{}stdio.h\textgreater{}}
\PreprocessorTok{\#include }\ImportTok{\textless{}stdlib.h\textgreater{}}

\DataTypeTok{int}\NormalTok{ main}\OperatorTok{()} \OperatorTok{\{}
    \DataTypeTok{int}\NormalTok{ rows }\OperatorTok{=} \DecValTok{3}\OperatorTok{;}
    \DataTypeTok{int}\NormalTok{ cols }\OperatorTok{=} \DecValTok{4}\OperatorTok{;}

    \CommentTok{// Создание массива указателей на строки}
    \DataTypeTok{int}\OperatorTok{**}\NormalTok{ array }\OperatorTok{=} \OperatorTok{(}\DataTypeTok{int}\OperatorTok{**)}\NormalTok{malloc}\OperatorTok{(}\NormalTok{rows }\OperatorTok{*} \KeywordTok{sizeof}\OperatorTok{(}\DataTypeTok{int}\OperatorTok{*));}

    \CommentTok{// Выделение памяти для каждого ряда}
    \ControlFlowTok{for} \OperatorTok{(}\DataTypeTok{int}\NormalTok{ i }\OperatorTok{=} \DecValTok{0}\OperatorTok{;}\NormalTok{ i }\OperatorTok{\textless{}}\NormalTok{ rows}\OperatorTok{;} \OperatorTok{++}\NormalTok{i}\OperatorTok{)} \OperatorTok{\{}
\NormalTok{        array}\OperatorTok{[}\NormalTok{i}\OperatorTok{]} \OperatorTok{=} \OperatorTok{(}\DataTypeTok{int}\OperatorTok{*)}\NormalTok{malloc}\OperatorTok{(}\NormalTok{cols }\OperatorTok{*} \KeywordTok{sizeof}\OperatorTok{(}\DataTypeTok{int}\OperatorTok{));}
    \OperatorTok{\}}

    \CommentTok{// Инициализация и вывод массива}
    \ControlFlowTok{for} \OperatorTok{(}\DataTypeTok{int}\NormalTok{ i }\OperatorTok{=} \DecValTok{0}\OperatorTok{;}\NormalTok{ i }\OperatorTok{\textless{}}\NormalTok{ rows}\OperatorTok{;} \OperatorTok{++}\NormalTok{i}\OperatorTok{)} \OperatorTok{\{}
        \ControlFlowTok{for} \OperatorTok{(}\DataTypeTok{int}\NormalTok{ j }\OperatorTok{=} \DecValTok{0}\OperatorTok{;}\NormalTok{ j }\OperatorTok{\textless{}}\NormalTok{ cols}\OperatorTok{;} \OperatorTok{++}\NormalTok{j}\OperatorTok{)} \OperatorTok{\{}
\NormalTok{            array}\OperatorTok{[}\NormalTok{i}\OperatorTok{][}\NormalTok{j}\OperatorTok{]} \OperatorTok{=}\NormalTok{ i }\OperatorTok{*}\NormalTok{ cols }\OperatorTok{+}\NormalTok{ j}\OperatorTok{;}
\NormalTok{            printf}\OperatorTok{(}\StringTok{"}\SpecialCharTok{\%d}\StringTok{ "}\OperatorTok{,}\NormalTok{ array}\OperatorTok{[}\NormalTok{i}\OperatorTok{][}\NormalTok{j}\OperatorTok{]);}
        \OperatorTok{\}}
\NormalTok{        printf}\OperatorTok{(}\StringTok{"}\SpecialCharTok{\textbackslash{}n}\StringTok{"}\OperatorTok{);}
    \OperatorTok{\}}

    \CommentTok{// Удаление массива}
    \ControlFlowTok{for} \OperatorTok{(}\DataTypeTok{int}\NormalTok{ i }\OperatorTok{=} \DecValTok{0}\OperatorTok{;}\NormalTok{ i }\OperatorTok{\textless{}}\NormalTok{ rows}\OperatorTok{;} \OperatorTok{++}\NormalTok{i}\OperatorTok{)} \OperatorTok{\{}
\NormalTok{        free}\OperatorTok{(}\NormalTok{array}\OperatorTok{[}\NormalTok{i}\OperatorTok{]);}
    \OperatorTok{\}}
\NormalTok{    free}\OperatorTok{(}\NormalTok{array}\OperatorTok{);}

    \ControlFlowTok{return} \DecValTok{0}\OperatorTok{;}
\OperatorTok{\}}
\end{Highlighting}
\end{Shaded}

В этом примере память выделяется с помощью функции \texttt{malloc}.
После использования массива память освобождается с помощью функции
\texttt{free}.

\subsubsection{19. Динамические массивы. Обращение к i-й ячейке
одномерного массива с использованием указателя и смещения. Обращение к
ячейке двумерного массива на пересечении i-й строки и j-го столбца с
использованием указателя и
смещения}\label{ux434ux438ux43dux430ux43cux438ux447ux435ux441ux43aux438ux435-ux43cux430ux441ux441ux438ux432ux44b.-ux43eux431ux440ux430ux449ux435ux43dux438ux435-ux43a-i-ux439-ux44fux447ux435ux439ux43aux435-ux43eux434ux43dux43eux43cux435ux440ux43dux43eux433ux43e-ux43cux430ux441ux441ux438ux432ux430-ux441-ux438ux441ux43fux43eux43bux44cux437ux43eux432ux430ux43dux438ux435ux43c-ux443ux43aux430ux437ux430ux442ux435ux43bux44f-ux438-ux441ux43cux435ux449ux435ux43dux438ux44f.-ux43eux431ux440ux430ux449ux435ux43dux438ux435-ux43a-ux44fux447ux435ux439ux43aux435-ux434ux432ux443ux43cux435ux440ux43dux43eux433ux43e-ux43cux430ux441ux441ux438ux432ux430-ux43dux430-ux43fux435ux440ux435ux441ux435ux447ux435ux43dux438ux438-i-ux439-ux441ux442ux440ux43eux43aux438-ux438-j-ux433ux43e-ux441ux442ux43eux43bux431ux446ux430-ux441-ux438ux441ux43fux43eux43bux44cux437ux43eux432ux430ux43dux438ux435ux43c-ux443ux43aux430ux437ux430ux442ux435ux43bux44f-ux438-ux441ux43cux435ux449ux435ux43dux438ux44f}

\paragraph{Динамические
массивы}\label{ux434ux438ux43dux430ux43cux438ux447ux435ux441ux43aux438ux435-ux43cux430ux441ux441ux438ux432ux44b}

Динамические массивы позволяют выделять память во время выполнения
программы, что делает их более гибкими по сравнению со статическими
массивами.

\paragraph{Обращение к i-й ячейке одномерного
массива}\label{ux43eux431ux440ux430ux449ux435ux43dux438ux435-ux43a-i-ux439-ux44fux447ux435ux439ux43aux435-ux43eux434ux43dux43eux43cux435ux440ux43dux43eux433ux43e-ux43cux430ux441ux441ux438ux432ux430}

Обращение к элементам одномерного массива с использованием указателей и
смещения:

\begin{Shaded}
\begin{Highlighting}[]
\PreprocessorTok{\#include }\ImportTok{\textless{}iostream\textgreater{}}
\KeywordTok{using} \KeywordTok{namespace}\NormalTok{ std}\OperatorTok{;}

\DataTypeTok{int}\NormalTok{ main}\OperatorTok{()} \OperatorTok{\{}
    \DataTypeTok{int}\NormalTok{ n }\OperatorTok{=} \DecValTok{5}\OperatorTok{;}
    \DataTypeTok{int}\OperatorTok{*}\NormalTok{ array }\OperatorTok{=} \KeywordTok{new} \DataTypeTok{int}\OperatorTok{[}\NormalTok{n}\OperatorTok{];}

    \CommentTok{// Инициализация массива}
    \ControlFlowTok{for} \OperatorTok{(}\DataTypeTok{int}\NormalTok{ i }\OperatorTok{=} \DecValTok{0}\OperatorTok{;}\NormalTok{ i }\OperatorTok{\textless{}}\NormalTok{ n}\OperatorTok{;} \OperatorTok{++}\NormalTok{i}\OperatorTok{)} \OperatorTok{\{}
\NormalTok{        array}\OperatorTok{[}\NormalTok{i}\OperatorTok{]} \OperatorTok{=}\NormalTok{ i}\OperatorTok{;}
    \OperatorTok{\}}

    \CommentTok{// Обращение к элементам с использованием указателя и смещения}
    \ControlFlowTok{for} \OperatorTok{(}\DataTypeTok{int}\NormalTok{ i }\OperatorTok{=} \DecValTok{0}\OperatorTok{;}\NormalTok{ i }\OperatorTok{\textless{}}\NormalTok{ n}\OperatorTok{;} \OperatorTok{++}\NormalTok{i}\OperatorTok{)} \OperatorTok{\{}
\NormalTok{        cout }\OperatorTok{\textless{}\textless{}} \OperatorTok{*(}\NormalTok{array }\OperatorTok{+}\NormalTok{ i}\OperatorTok{)} \OperatorTok{\textless{}\textless{}} \StringTok{" "}\OperatorTok{;}
    \OperatorTok{\}}
\NormalTok{    cout }\OperatorTok{\textless{}\textless{}}\NormalTok{ endl}\OperatorTok{;}

    \KeywordTok{delete}\OperatorTok{[]}\NormalTok{ array}\OperatorTok{;}

    \ControlFlowTok{return} \DecValTok{0}\OperatorTok{;}
\OperatorTok{\}}
\end{Highlighting}
\end{Shaded}

\paragraph{Обращение к ячейке двумерного
массива}\label{ux43eux431ux440ux430ux449ux435ux43dux438ux435-ux43a-ux44fux447ux435ux439ux43aux435-ux434ux432ux443ux43cux435ux440ux43dux43eux433ux43e-ux43cux430ux441ux441ux438ux432ux430}

Обращение к элементам двумерного массива с использованием указателей и
смещения:

\begin{Shaded}
\begin{Highlighting}[]
\PreprocessorTok{\#include }\ImportTok{\textless{}iostream\textgreater{}}
\KeywordTok{using} \KeywordTok{namespace}\NormalTok{ std}\OperatorTok{;}

\DataTypeTok{int}\NormalTok{ main}\OperatorTok{()} \OperatorTok{\{}
    \DataTypeTok{int}\NormalTok{ rows }\OperatorTok{=} \DecValTok{3}\OperatorTok{;}
    \DataTypeTok{int}\NormalTok{ cols }\OperatorTok{=} \DecValTok{4}\OperatorTok{;}
    \DataTypeTok{int}\OperatorTok{**}\NormalTok{ array }\OperatorTok{=} \KeywordTok{new} \DataTypeTok{int}\OperatorTok{*[}\NormalTok{rows}\OperatorTok{];}

    \ControlFlowTok{for} \OperatorTok{(}\DataTypeTok{int}\NormalTok{ i }\OperatorTok{=} \DecValTok{0}\OperatorTok{;}\NormalTok{ i }\OperatorTok{\textless{}}\NormalTok{ rows}\OperatorTok{;} \OperatorTok{++}\NormalTok{i}\OperatorTok{)} \OperatorTok{\{}
\NormalTok{        array}\OperatorTok{[}\NormalTok{i}\OperatorTok{]} \OperatorTok{=} \KeywordTok{new} \DataTypeTok{int}\OperatorTok{[}\NormalTok{cols}\OperatorTok{];}
    \OperatorTok{\}}

    \CommentTok{// Инициализация массива}
    \ControlFlowTok{for} \OperatorTok{(}\DataTypeTok{int}\NormalTok{ i }\OperatorTok{=} \DecValTok{0}\OperatorTok{;}\NormalTok{ i }\OperatorTok{\textless{}}\NormalTok{ rows}\OperatorTok{;} \OperatorTok{++}\NormalTok{i}\OperatorTok{)} \OperatorTok{\{}
        \ControlFlowTok{for} \OperatorTok{(}\DataTypeTok{int}\NormalTok{ j }\OperatorTok{=} \DecValTok{0}\OperatorTok{;}\NormalTok{ j }\OperatorTok{\textless{}}\NormalTok{ cols}\OperatorTok{;} \OperatorTok{++}\NormalTok{j}\OperatorTok{)} \OperatorTok{\{}
\NormalTok{            array}\OperatorTok{[}\NormalTok{i}\OperatorTok{][}\NormalTok{j}\OperatorTok{]} \OperatorTok{=}\NormalTok{ i }\OperatorTok{*}\NormalTok{ cols }\OperatorTok{+}\NormalTok{ j}\OperatorTok{;}
        \OperatorTok{\}}
    \OperatorTok{\}}

    \CommentTok{// Обращение к элементам с использованием указателя и смещения}
    \ControlFlowTok{for} \OperatorTok{(}\DataTypeTok{int}\NormalTok{ i }\OperatorTok{=} \DecValTok{0}\OperatorTok{;}\NormalTok{ i }\OperatorTok{\textless{}}\NormalTok{ rows}\OperatorTok{;} \OperatorTok{++}\NormalTok{i}\OperatorTok{)} \OperatorTok{\{}
        \ControlFlowTok{for} \OperatorTok{(}\DataTypeTok{int}\NormalTok{ j }\OperatorTok{=} \DecValTok{0}\OperatorTok{;}\NormalTok{ j }\OperatorTok{\textless{}}\NormalTok{ cols}\OperatorTok{;} \OperatorTok{++}\NormalTok{j}\OperatorTok{)} \OperatorTok{\{}
\NormalTok{            cout }\OperatorTok{\textless{}\textless{}} \OperatorTok{*(*(}\NormalTok{array }\OperatorTok{+}\NormalTok{ i}\OperatorTok{)} \OperatorTok{+}\NormalTok{ j}\OperatorTok{)} \OperatorTok{\textless{}\textless{}} \StringTok{" "}\OperatorTok{;}
        \OperatorTok{\}}
\NormalTok{        cout }\OperatorTok{\textless{}\textless{}}\NormalTok{ endl}\OperatorTok{;}
    \OperatorTok{\}}

    \ControlFlowTok{for} \OperatorTok{(}\DataTypeTok{int}\NormalTok{ i }\OperatorTok{=} \DecValTok{0}\OperatorTok{;}\NormalTok{ i }\OperatorTok{\textless{}}\NormalTok{ rows}\OperatorTok{;} \OperatorTok{++}\NormalTok{i}\OperatorTok{)} \OperatorTok{\{}
        \KeywordTok{delete}\OperatorTok{[]}\NormalTok{ array}\OperatorTok{[}\NormalTok{i}\OperatorTok{];}
    \OperatorTok{\}}
    \KeywordTok{delete}\OperatorTok{[]}\NormalTok{ array}\OperatorTok{;}

    \ControlFlowTok{return} \DecValTok{0}\OperatorTok{;}
\OperatorTok{\}}
\end{Highlighting}
\end{Shaded}

Этот подход позволяет более гибко и эффективно работать с массивами в
C++ и C.

\subsubsection{20. Алгоритмы обработки диагоналей и треугольников
квадратной матрицы, сортировка элементов матрицы по строке (по
столбцу)}\label{ux430ux43bux433ux43eux440ux438ux442ux43cux44b-ux43eux431ux440ux430ux431ux43eux442ux43aux438-ux434ux438ux430ux433ux43eux43dux430ux43bux435ux439-ux438-ux442ux440ux435ux443ux433ux43eux43bux44cux43dux438ux43aux43eux432-ux43aux432ux430ux434ux440ux430ux442ux43dux43eux439-ux43cux430ux442ux440ux438ux446ux44b-ux441ux43eux440ux442ux438ux440ux43eux432ux43aux430-ux44dux43bux435ux43cux435ux43dux442ux43eux432-ux43cux430ux442ux440ux438ux446ux44b-ux43fux43e-ux441ux442ux440ux43eux43aux435-ux43fux43e-ux441ux442ux43eux43bux431ux446ux443}

\paragraph{Обработка диагоналей и треугольников квадратной
матрицы}\label{ux43eux431ux440ux430ux431ux43eux442ux43aux430-ux434ux438ux430ux433ux43eux43dux430ux43bux435ux439-ux438-ux442ux440ux435ux443ux433ux43eux43bux44cux43dux438ux43aux43eux432-ux43aux432ux430ux434ux440ux430ux442ux43dux43eux439-ux43cux430ux442ux440ux438ux446ux44b}

В квадратной матрице можно выделить две главные диагонали:

\begin{itemize}
\tightlist
\item
  Главная диагональ: элементы с индексами (i, i).
\item
  Побочная диагональ: элементы с индексами (i, n-1-i), где n --- размер
  матрицы.
\end{itemize}

\subparagraph{Пример обработки главной диагонали (нахождение
суммы)}\label{ux43fux440ux438ux43cux435ux440-ux43eux431ux440ux430ux431ux43eux442ux43aux438-ux433ux43bux430ux432ux43dux43eux439-ux434ux438ux430ux433ux43eux43dux430ux43bux438-ux43dux430ux445ux43eux436ux434ux435ux43dux438ux435-ux441ux443ux43cux43cux44b}

\begin{Shaded}
\begin{Highlighting}[]
\PreprocessorTok{\#include }\ImportTok{\textless{}iostream\textgreater{}}
\KeywordTok{using} \KeywordTok{namespace}\NormalTok{ std}\OperatorTok{;}

\DataTypeTok{int}\NormalTok{ main}\OperatorTok{()} \OperatorTok{\{}
    \AttributeTok{const} \DataTypeTok{int}\NormalTok{ n }\OperatorTok{=} \DecValTok{3}\OperatorTok{;}
    \DataTypeTok{int}\NormalTok{ matrix}\OperatorTok{[}\NormalTok{n}\OperatorTok{][}\NormalTok{n}\OperatorTok{]} \OperatorTok{=} \OperatorTok{\{\{}\DecValTok{1}\OperatorTok{,} \DecValTok{2}\OperatorTok{,} \DecValTok{3}\OperatorTok{\},}
                        \OperatorTok{\{}\DecValTok{4}\OperatorTok{,} \DecValTok{5}\OperatorTok{,} \DecValTok{6}\OperatorTok{\},}
                        \OperatorTok{\{}\DecValTok{7}\OperatorTok{,} \DecValTok{8}\OperatorTok{,} \DecValTok{9}\OperatorTok{\}\};}
    \DataTypeTok{int}\NormalTok{ sum }\OperatorTok{=} \DecValTok{0}\OperatorTok{;}

    \ControlFlowTok{for} \OperatorTok{(}\DataTypeTok{int}\NormalTok{ i }\OperatorTok{=} \DecValTok{0}\OperatorTok{;}\NormalTok{ i }\OperatorTok{\textless{}}\NormalTok{ n}\OperatorTok{;} \OperatorTok{++}\NormalTok{i}\OperatorTok{)} \OperatorTok{\{}
\NormalTok{        sum }\OperatorTok{+=}\NormalTok{ matrix}\OperatorTok{[}\NormalTok{i}\OperatorTok{][}\NormalTok{i}\OperatorTok{];}
    \OperatorTok{\}}

\NormalTok{    cout }\OperatorTok{\textless{}\textless{}} \StringTok{"Sum of main diagonal: "} \OperatorTok{\textless{}\textless{}}\NormalTok{ sum }\OperatorTok{\textless{}\textless{}}\NormalTok{ endl}\OperatorTok{;}

    \ControlFlowTok{return} \DecValTok{0}\OperatorTok{;}
\OperatorTok{\}}
\end{Highlighting}
\end{Shaded}

\subparagraph{Пример обработки побочной диагонали (нахождение
суммы)}\label{ux43fux440ux438ux43cux435ux440-ux43eux431ux440ux430ux431ux43eux442ux43aux438-ux43fux43eux431ux43eux447ux43dux43eux439-ux434ux438ux430ux433ux43eux43dux430ux43bux438-ux43dux430ux445ux43eux436ux434ux435ux43dux438ux435-ux441ux443ux43cux43cux44b}

\begin{Shaded}
\begin{Highlighting}[]
\PreprocessorTok{\#include }\ImportTok{\textless{}iostream\textgreater{}}
\KeywordTok{using} \KeywordTok{namespace}\NormalTok{ std}\OperatorTok{;}

\DataTypeTok{int}\NormalTok{ main}\OperatorTok{()} \OperatorTok{\{}
    \AttributeTok{const} \DataTypeTok{int}\NormalTok{ n }\OperatorTok{=} \DecValTok{3}\OperatorTok{;}
    \DataTypeTok{int}\NormalTok{ matrix}\OperatorTok{[}\NormalTok{n}\OperatorTok{][}\NormalTok{n}\OperatorTok{]} \OperatorTok{=} \OperatorTok{\{\{}\DecValTok{1}\OperatorTok{,} \DecValTok{2}\OperatorTok{,} \DecValTok{3}\OperatorTok{\},}
                        \OperatorTok{\{}\DecValTok{4}\OperatorTok{,} \DecValTok{5}\OperatorTok{,} \DecValTok{6}\OperatorTok{\},}
                        \OperatorTok{\{}\DecValTok{7}\OperatorTok{,} \DecValTok{8}\OperatorTok{,} \DecValTok{9}\OperatorTok{\}\};}
    \DataTypeTok{int}\NormalTok{ sum }\OperatorTok{=} \DecValTok{0}\OperatorTok{;}

    \ControlFlowTok{for} \OperatorTok{(}\DataTypeTok{int}\NormalTok{ i }\OperatorTok{=} \DecValTok{0}\OperatorTok{;}\NormalTok{ i }\OperatorTok{\textless{}}\NormalTok{ n}\OperatorTok{;} \OperatorTok{++}\NormalTok{i}\OperatorTok{)} \OperatorTok{\{}
\NormalTok{        sum }\OperatorTok{+=}\NormalTok{ matrix}\OperatorTok{[}\NormalTok{i}\OperatorTok{][}\NormalTok{n }\OperatorTok{{-}} \DecValTok{1} \OperatorTok{{-}}\NormalTok{ i}\OperatorTok{];}
    \OperatorTok{\}}

\NormalTok{    cout }\OperatorTok{\textless{}\textless{}} \StringTok{"Sum of secondary diagonal: "} \OperatorTok{\textless{}\textless{}}\NormalTok{ sum }\OperatorTok{\textless{}\textless{}}\NormalTok{ endl}\OperatorTok{;}

    \ControlFlowTok{return} \DecValTok{0}\OperatorTok{;}
\OperatorTok{\}}
\end{Highlighting}
\end{Shaded}

\subparagraph{Обработка треугольников
матрицы}\label{ux43eux431ux440ux430ux431ux43eux442ux43aux430-ux442ux440ux435ux443ux433ux43eux43bux44cux43dux438ux43aux43eux432-ux43cux430ux442ux440ux438ux446ux44b}

Треугольники в квадратной матрице можно разделить на верхний и нижний
треугольники относительно главной диагонали.

Пример обработки верхнего треугольника (нахождение суммы)

\begin{Shaded}
\begin{Highlighting}[]
\PreprocessorTok{\#include }\ImportTok{\textless{}iostream\textgreater{}}
\KeywordTok{using} \KeywordTok{namespace}\NormalTok{ std}\OperatorTok{;}

\DataTypeTok{int}\NormalTok{ main}\OperatorTok{()} \OperatorTok{\{}
    \AttributeTok{const} \DataTypeTok{int}\NormalTok{ n }\OperatorTok{=} \DecValTok{3}\OperatorTok{;}
    \DataTypeTok{int}\NormalTok{ matrix}\OperatorTok{[}\NormalTok{n}\OperatorTok{][}\NormalTok{n}\OperatorTok{]} \OperatorTok{=} \OperatorTok{\{\{}\DecValTok{1}\OperatorTok{,} \DecValTok{2}\OperatorTok{,} \DecValTok{3}\OperatorTok{\},}
                        \OperatorTok{\{}\DecValTok{4}\OperatorTok{,} \DecValTok{5}\OperatorTok{,} \DecValTok{6}\OperatorTok{\},}
                        \OperatorTok{\{}\DecValTok{7}\OperatorTok{,} \DecValTok{8}\OperatorTok{,} \DecValTok{9}\OperatorTok{\}\};}
    \DataTypeTok{int}\NormalTok{ sum }\OperatorTok{=} \DecValTok{0}\OperatorTok{;}

    \ControlFlowTok{for} \OperatorTok{(}\DataTypeTok{int}\NormalTok{ i }\OperatorTok{=} \DecValTok{0}\OperatorTok{;}\NormalTok{ i }\OperatorTok{\textless{}}\NormalTok{ n}\OperatorTok{;} \OperatorTok{++}\NormalTok{i}\OperatorTok{)} \OperatorTok{\{}
        \ControlFlowTok{for} \OperatorTok{(}\DataTypeTok{int}\NormalTok{ j }\OperatorTok{=}\NormalTok{ i }\OperatorTok{+} \DecValTok{1}\OperatorTok{;}\NormalTok{ j }\OperatorTok{\textless{}}\NormalTok{ n}\OperatorTok{;} \OperatorTok{++}\NormalTok{j}\OperatorTok{)} \OperatorTok{\{}
\NormalTok{            sum }\OperatorTok{+=}\NormalTok{ matrix}\OperatorTok{[}\NormalTok{i}\OperatorTok{][}\NormalTok{j}\OperatorTok{];}
        \OperatorTok{\}}
    \OperatorTok{\}}

\NormalTok{    cout }\OperatorTok{\textless{}\textless{}} \StringTok{"Sum of upper triangle: "} \OperatorTok{\textless{}\textless{}}\NormalTok{ sum }\OperatorTok{\textless{}\textless{}}\NormalTok{ endl}\OperatorTok{;}

    \ControlFlowTok{return} \DecValTok{0}\OperatorTok{;}
\OperatorTok{\}}
\end{Highlighting}
\end{Shaded}

Пример обработки нижнего треугольника (нахождение суммы)

\begin{Shaded}
\begin{Highlighting}[]
\PreprocessorTok{\#include }\ImportTok{\textless{}iostream\textgreater{}}
\KeywordTok{using} \KeywordTok{namespace}\NormalTok{ std}\OperatorTok{;}

\DataTypeTok{int}\NormalTok{ main}\OperatorTok{()} \OperatorTok{\{}
    \AttributeTok{const} \DataTypeTok{int}\NormalTok{ n }\OperatorTok{=} \DecValTok{3}\OperatorTok{;}
    \DataTypeTok{int}\NormalTok{ matrix}\OperatorTok{[}\NormalTok{n}\OperatorTok{][}\NormalTok{n}\OperatorTok{]} \OperatorTok{=} \OperatorTok{\{\{}\DecValTok{1}\OperatorTok{,} \DecValTok{2}\OperatorTok{,} \DecValTok{3}\OperatorTok{\},}
                        \OperatorTok{\{}\DecValTok{4}\OperatorTok{,} \DecValTok{5}\OperatorTok{,} \DecValTok{6}\OperatorTok{\},}
                        \OperatorTok{\{}\DecValTok{7}\OperatorTok{,} \DecValTok{8}\OperatorTok{,} \DecValTok{9}\OperatorTok{\}\};}
    \DataTypeTok{int}\NormalTok{ sum }\OperatorTok{=} \DecValTok{0}\OperatorTok{;}

    \ControlFlowTok{for} \OperatorTok{(}\DataTypeTok{int}\NormalTok{ i }\OperatorTok{=} \DecValTok{0}\OperatorTok{;}\NormalTok{ i }\OperatorTok{\textless{}}\NormalTok{ n}\OperatorTok{;} \OperatorTok{++}\NormalTok{i}\OperatorTok{)} \OperatorTok{\{}
        \ControlFlowTok{for} \OperatorTok{(}\DataTypeTok{int}\NormalTok{ j }\OperatorTok{=} \DecValTok{0}\OperatorTok{;}\NormalTok{ j }\OperatorTok{\textless{}}\NormalTok{ i}\OperatorTok{;} \OperatorTok{++}\NormalTok{j}\OperatorTok{)} \OperatorTok{\{}
\NormalTok{            sum }\OperatorTok{+=}\NormalTok{ matrix}\OperatorTok{[}\NormalTok{i}\OperatorTok{][}\NormalTok{j}\OperatorTok{];}
        \OperatorTok{\}}
    \OperatorTok{\}}

\NormalTok{    cout }\OperatorTok{\textless{}\textless{}} \StringTok{"Sum of lower triangle: "} \OperatorTok{\textless{}\textless{}}\NormalTok{ sum }\OperatorTok{\textless{}\textless{}}\NormalTok{ endl}\OperatorTok{;}

    \ControlFlowTok{return} \DecValTok{0}\OperatorTok{;}
\OperatorTok{\}}
\end{Highlighting}
\end{Shaded}

\paragraph{Сортировка элементов матрицы по строке (по
столбцу)}\label{ux441ux43eux440ux442ux438ux440ux43eux432ux43aux430-ux44dux43bux435ux43cux435ux43dux442ux43eux432-ux43cux430ux442ux440ux438ux446ux44b-ux43fux43e-ux441ux442ux440ux43eux43aux435-ux43fux43e-ux441ux442ux43eux43bux431ux446ux443}

Пример сортировки по строкам

\begin{Shaded}
\begin{Highlighting}[]
\PreprocessorTok{\#include }\ImportTok{\textless{}iostream\textgreater{}}
\PreprocessorTok{\#include }\ImportTok{\textless{}algorithm\textgreater{}}
\KeywordTok{using} \KeywordTok{namespace}\NormalTok{ std}\OperatorTok{;}

\DataTypeTok{int}\NormalTok{ main}\OperatorTok{()} \OperatorTok{\{}
    \AttributeTok{const} \DataTypeTok{int}\NormalTok{ n }\OperatorTok{=} \DecValTok{3}\OperatorTok{;}
    \DataTypeTok{int}\NormalTok{ matrix}\OperatorTok{[}\NormalTok{n}\OperatorTok{][}\NormalTok{n}\OperatorTok{]} \OperatorTok{=} \OperatorTok{\{\{}\DecValTok{3}\OperatorTok{,} \DecValTok{2}\OperatorTok{,} \DecValTok{1}\OperatorTok{\},}
                        \OperatorTok{\{}\DecValTok{9}\OperatorTok{,} \DecValTok{8}\OperatorTok{,} \DecValTok{7}\OperatorTok{\},}
                        \OperatorTok{\{}\DecValTok{6}\OperatorTok{,} \DecValTok{5}\OperatorTok{,} \DecValTok{4}\OperatorTok{\}\};}

    \ControlFlowTok{for} \OperatorTok{(}\DataTypeTok{int}\NormalTok{ i }\OperatorTok{=} \DecValTok{0}\OperatorTok{;}\NormalTok{ i }\OperatorTok{\textless{}}\NormalTok{ n}\OperatorTok{;} \OperatorTok{++}\NormalTok{i}\OperatorTok{)} \OperatorTok{\{}
\NormalTok{        sort}\OperatorTok{(}\NormalTok{matrix}\OperatorTok{[}\NormalTok{i}\OperatorTok{],}\NormalTok{ matrix}\OperatorTok{[}\NormalTok{i}\OperatorTok{]} \OperatorTok{+}\NormalTok{ n}\OperatorTok{);}
    \OperatorTok{\}}

    \CommentTok{// Вывод отсортированной матрицы}
    \ControlFlowTok{for} \OperatorTok{(}\DataTypeTok{int}\NormalTok{ i }\OperatorTok{=} \DecValTok{0}\OperatorTok{;}\NormalTok{ i }\OperatorTok{\textless{}}\NormalTok{ n}\OperatorTok{;} \OperatorTok{++}\NormalTok{i}\OperatorTok{)} \OperatorTok{\{}
        \ControlFlowTok{for} \OperatorTok{(}\DataTypeTok{int}\NormalTok{ j }\OperatorTok{=} \DecValTok{0}\OperatorTok{;}\NormalTok{ j }\OperatorTok{\textless{}}\NormalTok{ n}\OperatorTok{;} \OperatorTok{++}\NormalTok{j}\OperatorTok{)} \OperatorTok{\{}
\NormalTok{            cout }\OperatorTok{\textless{}\textless{}}\NormalTok{ matrix}\OperatorTok{[}\NormalTok{i}\OperatorTok{][}\NormalTok{j}\OperatorTok{]} \OperatorTok{\textless{}\textless{}} \StringTok{" "}\OperatorTok{;}
        \OperatorTok{\}}
\NormalTok{        cout }\OperatorTok{\textless{}\textless{}}\NormalTok{ endl}\OperatorTok{;}
    \OperatorTok{\}}

    \ControlFlowTok{return} \DecValTok{0}\OperatorTok{;}
\OperatorTok{\}}
\end{Highlighting}
\end{Shaded}

Пример сортировки по столбцам

\begin{Shaded}
\begin{Highlighting}[]
\PreprocessorTok{\#include }\ImportTok{\textless{}iostream\textgreater{}}
\PreprocessorTok{\#include }\ImportTok{\textless{}algorithm\textgreater{}}
\KeywordTok{using} \KeywordTok{namespace}\NormalTok{ std}\OperatorTok{;}

\DataTypeTok{int}\NormalTok{ main}\OperatorTok{()} \OperatorTok{\{}
    \AttributeTok{const} \DataTypeTok{int}\NormalTok{ n }\OperatorTok{=} \DecValTok{3}\OperatorTok{;}
    \DataTypeTok{int}\NormalTok{ matrix}\OperatorTok{[}\NormalTok{n}\OperatorTok{][}\NormalTok{n}\OperatorTok{]} \OperatorTok{=} \OperatorTok{\{\{}\DecValTok{3}\OperatorTok{,} \DecValTok{2}\OperatorTok{,} \DecValTok{1}\OperatorTok{\},}
                        \OperatorTok{\{}\DecValTok{9}\OperatorTok{,} \DecValTok{8}\OperatorTok{,} \DecValTok{7}\OperatorTok{\},}
                        \OperatorTok{\{}\DecValTok{6}\OperatorTok{,} \DecValTok{5}\OperatorTok{,} \DecValTok{4}\OperatorTok{\}\};}

    \ControlFlowTok{for} \OperatorTok{(}\DataTypeTok{int}\NormalTok{ j }\OperatorTok{=} \DecValTok{0}\OperatorTok{;}\NormalTok{ j }\OperatorTok{\textless{}}\NormalTok{ n}\OperatorTok{;} \OperatorTok{++}\NormalTok{j}\OperatorTok{)} \OperatorTok{\{}
        \DataTypeTok{int}\NormalTok{ temp}\OperatorTok{[}\NormalTok{n}\OperatorTok{];}
        \ControlFlowTok{for} \OperatorTok{(}\DataTypeTok{int}\NormalTok{ i }\OperatorTok{=} \DecValTok{0}\OperatorTok{;}\NormalTok{ i }\OperatorTok{\textless{}}\NormalTok{ n}\OperatorTok{;} \OperatorTok{++}\NormalTok{i}\OperatorTok{)} \OperatorTok{\{}
\NormalTok{            temp}\OperatorTok{[}\NormalTok{i}\OperatorTok{]} \OperatorTok{=}\NormalTok{ matrix}\OperatorTok{[}\NormalTok{i}\OperatorTok{][}\NormalTok{j}\OperatorTok{];}
        \OperatorTok{\}}
\NormalTok{        sort}\OperatorTok{(}\NormalTok{temp}\OperatorTok{,}\NormalTok{ temp }\OperatorTok{+}\NormalTok{ n}\OperatorTok{);}
        \ControlFlowTok{for} \OperatorTok{(}\DataTypeTok{int}\NormalTok{ i }\OperatorTok{=} \DecValTok{0}\OperatorTok{;}\NormalTok{ i }\OperatorTok{\textless{}}\NormalTok{ n}\OperatorTok{;} \OperatorTok{++}\NormalTok{i}\OperatorTok{)} \OperatorTok{\{}
\NormalTok{            matrix}\OperatorTok{[}\NormalTok{i}\OperatorTok{][}\NormalTok{j}\OperatorTok{]} \OperatorTok{=}\NormalTok{ temp}\OperatorTok{[}\NormalTok{i}\OperatorTok{];}
        \OperatorTok{\}}
    \OperatorTok{\}}

    \CommentTok{// Вывод отсортированной матрицы}
    \ControlFlowTok{for} \OperatorTok{(}\DataTypeTok{int}\NormalTok{ i }\OperatorTok{=} \DecValTok{0}\OperatorTok{;}\NormalTok{ i }\OperatorTok{\textless{}}\NormalTok{ n}\OperatorTok{;} \OperatorTok{++}\NormalTok{i}\OperatorTok{)} \OperatorTok{\{}
        \ControlFlowTok{for} \OperatorTok{(}\DataTypeTok{int}\NormalTok{ j }\OperatorTok{=} \DecValTok{0}\OperatorTok{;}\NormalTok{ j }\OperatorTok{\textless{}}\NormalTok{ n}\OperatorTok{;} \OperatorTok{++}\NormalTok{j}\OperatorTok{)} \OperatorTok{\{}
\NormalTok{            cout }\OperatorTok{\textless{}\textless{}}\NormalTok{ matrix}\OperatorTok{[}\NormalTok{i}\OperatorTok{][}\NormalTok{j}\OperatorTok{]} \OperatorTok{\textless{}\textless{}} \StringTok{" "}\OperatorTok{;}
        \OperatorTok{\}}
\NormalTok{        cout }\OperatorTok{\textless{}\textless{}}\NormalTok{ endl}\OperatorTok{;}
    \OperatorTok{\}}

    \ControlFlowTok{return} \DecValTok{0}\OperatorTok{;}
\OperatorTok{\}}
\end{Highlighting}
\end{Shaded}

\subsubsection{21. Порядок расположения адресов переменных программы.
Понятие указателя, допустимые операции над указателями. Непрямой доступ
к значению переменной. Указатель на указатель, указатель на void,
указатель и модификатор
const}\label{ux43fux43eux440ux44fux434ux43eux43a-ux440ux430ux441ux43fux43eux43bux43eux436ux435ux43dux438ux44f-ux430ux434ux440ux435ux441ux43eux432-ux43fux435ux440ux435ux43cux435ux43dux43dux44bux445-ux43fux440ux43eux433ux440ux430ux43cux43cux44b.-ux43fux43eux43dux44fux442ux438ux435-ux443ux43aux430ux437ux430ux442ux435ux43bux44f-ux434ux43eux43fux443ux441ux442ux438ux43cux44bux435-ux43eux43fux435ux440ux430ux446ux438ux438-ux43dux430ux434-ux443ux43aux430ux437ux430ux442ux435ux43bux44fux43cux438.-ux43dux435ux43fux440ux44fux43cux43eux439-ux434ux43eux441ux442ux443ux43f-ux43a-ux437ux43dux430ux447ux435ux43dux438ux44e-ux43fux435ux440ux435ux43cux435ux43dux43dux43eux439.-ux443ux43aux430ux437ux430ux442ux435ux43bux44c-ux43dux430-ux443ux43aux430ux437ux430ux442ux435ux43bux44c-ux443ux43aux430ux437ux430ux442ux435ux43bux44c-ux43dux430-void-ux443ux43aux430ux437ux430ux442ux435ux43bux44c-ux438-ux43cux43eux434ux438ux444ux438ux43aux430ux442ux43eux440-const}

\paragraph{Порядок расположения адресов переменных
программы}\label{ux43fux43eux440ux44fux434ux43eux43a-ux440ux430ux441ux43fux43eux43bux43eux436ux435ux43dux438ux44f-ux430ux434ux440ux435ux441ux43eux432-ux43fux435ux440ux435ux43cux435ux43dux43dux44bux445-ux43fux440ux43eux433ux440ux430ux43cux43cux44b}

В программах на C++ переменные располагаются в различных сегментах
памяти:

\begin{itemize}
\tightlist
\item
  \textbf{Стек (stack)}: для локальных переменных и параметров функций.
\item
  \textbf{Куча (heap)}: для динамически выделяемой памяти (например, с
  помощью \texttt{new} или \texttt{malloc}).
\item
  \textbf{Сегмент данных (data segment)}: для глобальных и статических
  переменных.
\end{itemize}

\paragraph{Понятие
указателя}\label{ux43fux43eux43dux44fux442ux438ux435-ux443ux43aux430ux437ux430ux442ux435ux43bux44f}

Указатель --- это переменная, которая хранит адрес другой переменной.
Указатели позволяют эффективно работать с памятью, передавать большие
объекты по ссылке, а также создавать динамические структуры данных.

\paragraph{Допустимые операции над
указателями}\label{ux434ux43eux43fux443ux441ux442ux438ux43cux44bux435-ux43eux43fux435ux440ux430ux446ux438ux438-ux43dux430ux434-ux443ux43aux430ux437ux430ux442ux435ux43bux44fux43cux438}

\begin{itemize}
\tightlist
\item
  Присваивание адреса переменной указателю: \texttt{int\ *p\ =\ \&x;}
\item
  Разыменование указателя (доступ к значению по адресу):
  \texttt{*p\ =\ 10;}
\item
  Арифметика указателей: \texttt{p\ +\ 1}, \texttt{p\ -\ 1} (перемещение
  по массиву).
\end{itemize}

\paragraph{Непрямой доступ к значению
переменной}\label{ux43dux435ux43fux440ux44fux43cux43eux439-ux434ux43eux441ux442ux443ux43f-ux43a-ux437ux43dux430ux447ux435ux43dux438ux44e-ux43fux435ux440ux435ux43cux435ux43dux43dux43eux439}

Непрямой доступ к значению переменной осуществляется через разыменование
указателя:

\begin{Shaded}
\begin{Highlighting}[]
\DataTypeTok{int}\NormalTok{ x }\OperatorTok{=} \DecValTok{10}\OperatorTok{;}
\DataTypeTok{int} \OperatorTok{*}\NormalTok{p }\OperatorTok{=} \OperatorTok{\&}\NormalTok{x}\OperatorTok{;}
\OperatorTok{*}\NormalTok{p }\OperatorTok{=} \DecValTok{20}\OperatorTok{;} \CommentTok{// изменяет значение x на 20}
\end{Highlighting}
\end{Shaded}

\paragraph{Указатель на
указатель}\label{ux443ux43aux430ux437ux430ux442ux435ux43bux44c-ux43dux430-ux443ux43aux430ux437ux430ux442ux435ux43bux44c}

Указатель на указатель хранит адрес другого указателя:

\begin{Shaded}
\begin{Highlighting}[]
\DataTypeTok{int}\NormalTok{ x }\OperatorTok{=} \DecValTok{10}\OperatorTok{;}
\DataTypeTok{int} \OperatorTok{*}\NormalTok{p }\OperatorTok{=} \OperatorTok{\&}\NormalTok{x}\OperatorTok{;}
\DataTypeTok{int} \OperatorTok{**}\NormalTok{pp }\OperatorTok{=} \OperatorTok{\&}\NormalTok{p}\OperatorTok{;}

\NormalTok{cout }\OperatorTok{\textless{}\textless{}} \OperatorTok{**}\NormalTok{pp }\OperatorTok{\textless{}\textless{}}\NormalTok{ endl}\OperatorTok{;} \CommentTok{// выводит 10}
\end{Highlighting}
\end{Shaded}

\paragraph{Указатель на
void}\label{ux443ux43aux430ux437ux430ux442ux435ux43bux44c-ux43dux430-void}

Указатель типа \texttt{void*} может хранить адрес любой переменной, но
не может быть разыменован напрямую:

\begin{Shaded}
\begin{Highlighting}[]
\DataTypeTok{int}\NormalTok{ x }\OperatorTok{=} \DecValTok{10}\OperatorTok{;}
\DataTypeTok{void} \OperatorTok{*}\NormalTok{p }\OperatorTok{=} \OperatorTok{\&}\NormalTok{x}\OperatorTok{;}
\CommentTok{// int y = *p; // ошибка}
\DataTypeTok{int}\NormalTok{ y }\OperatorTok{=} \OperatorTok{*(}\DataTypeTok{int}\OperatorTok{*)}\NormalTok{p}\OperatorTok{;} \CommentTok{// необходимо приведение типа}
\end{Highlighting}
\end{Shaded}

\paragraph{Указатель и модификатор
const}\label{ux443ux43aux430ux437ux430ux442ux435ux43bux44c-ux438-ux43cux43eux434ux438ux444ux438ux43aux430ux442ux43eux440-const}

Модификатор \texttt{const} может применяться к указателям различными
способами:

\begin{itemize}
\tightlist
\item
  \texttt{const\ int\ *p}: указатель на константное значение (значение
  нельзя изменять через указатель).
\item
  \texttt{int\ *const\ p}: константный указатель (указатель не может
  указывать на другой адрес).
\item
  \texttt{const\ int\ *const\ p}: константный указатель на константное
  значение.
\end{itemize}

\subsubsection{22. Функциональное назначение и синтаксис определения
функций пользователя. Формальные и фактические параметры. Оператор
return и вызов функции. Механизм передачи параметров в функцию по
значению}\label{ux444ux443ux43dux43aux446ux438ux43eux43dux430ux43bux44cux43dux43eux435-ux43dux430ux437ux43dux430ux447ux435ux43dux438ux435-ux438-ux441ux438ux43dux442ux430ux43aux441ux438ux441-ux43eux43fux440ux435ux434ux435ux43bux435ux43dux438ux44f-ux444ux443ux43dux43aux446ux438ux439-ux43fux43eux43bux44cux437ux43eux432ux430ux442ux435ux43bux44f.-ux444ux43eux440ux43cux430ux43bux44cux43dux44bux435-ux438-ux444ux430ux43aux442ux438ux447ux435ux441ux43aux438ux435-ux43fux430ux440ux430ux43cux435ux442ux440ux44b.-ux43eux43fux435ux440ux430ux442ux43eux440-return-ux438-ux432ux44bux437ux43eux432-ux444ux443ux43dux43aux446ux438ux438.-ux43cux435ux445ux430ux43dux438ux437ux43c-ux43fux435ux440ux435ux434ux430ux447ux438-ux43fux430ux440ux430ux43cux435ux442ux440ux43eux432-ux432-ux444ux443ux43dux43aux446ux438ux44e-ux43fux43e-ux437ux43dux430ux447ux435ux43dux438ux44e}

\paragraph{Функциональное назначение и синтаксис определения
функций}\label{ux444ux443ux43dux43aux446ux438ux43eux43dux430ux43bux44cux43dux43eux435-ux43dux430ux437ux43dux430ux447ux435ux43dux438ux435-ux438-ux441ux438ux43dux442ux430ux43aux441ux438ux441-ux43eux43fux440ux435ux434ux435ux43bux435ux43dux438ux44f-ux444ux443ux43dux43aux446ux438ux439}

Функции позволяют разбивать программу на логические части, которые можно
многократно использовать. Функция определяет блок кода, который
выполняет конкретную задачу.

\begin{Shaded}
\begin{Highlighting}[]
\PreprocessorTok{\#include }\ImportTok{\textless{}iostream\textgreater{}}
\KeywordTok{using} \KeywordTok{namespace}\NormalTok{ std}\OperatorTok{;}

\CommentTok{// Объявление функции}
\DataTypeTok{int}\NormalTok{ sum}\OperatorTok{(}\DataTypeTok{int}\NormalTok{ a}\OperatorTok{,} \DataTypeTok{int}\NormalTok{ b}\OperatorTok{);}

\CommentTok{// Определение функции}
\DataTypeTok{int}\NormalTok{ sum}\OperatorTok{(}\DataTypeTok{int}\NormalTok{ a}\OperatorTok{,} \DataTypeTok{int}\NormalTok{ b}\OperatorTok{)} \OperatorTok{\{}
    \ControlFlowTok{return}\NormalTok{ a }\OperatorTok{+}\NormalTok{ b}\OperatorTok{;}
\OperatorTok{\}}

\DataTypeTok{int}\NormalTok{ main}\OperatorTok{()} \OperatorTok{\{}
    \DataTypeTok{int}\NormalTok{ result }\OperatorTok{=}\NormalTok{ sum}\OperatorTok{(}\DecValTok{3}\OperatorTok{,} \DecValTok{4}\OperatorTok{);} \CommentTok{// Вызов функции}
\NormalTok{    cout }\OperatorTok{\textless{}\textless{}} \StringTok{"Sum: "} \OperatorTok{\textless{}\textless{}}\NormalTok{ result }\OperatorTok{\textless{}\textless{}}\NormalTok{ endl}\OperatorTok{;}
    \ControlFlowTok{return} \DecValTok{0}\OperatorTok{;}
\OperatorTok{\}}
\end{Highlighting}
\end{Shaded}

\paragraph{Формальные и фактические
параметры}\label{ux444ux43eux440ux43cux430ux43bux44cux43dux44bux435-ux438-ux444ux430ux43aux442ux438ux447ux435ux441ux43aux438ux435-ux43fux430ux440ux430ux43cux435ux442ux440ux44b}

\begin{itemize}
\tightlist
\item
  \textbf{Формальные параметры}: переменные, указанные в определении
  функции (\texttt{int\ a,\ int\ b}).
\item
  \textbf{Фактические параметры}: значения, передаваемые при вызове
  функции (\texttt{sum(3,\ 4)}).
\end{itemize}

\paragraph{Оператор
return}\label{ux43eux43fux435ux440ux430ux442ux43eux440-return}

Оператор \texttt{return} завершает выполнение функции и возвращает
значение:

\begin{Shaded}
\begin{Highlighting}[]
\DataTypeTok{int}\NormalTok{ sum}\OperatorTok{(}\DataTypeTok{int}\NormalTok{ a}\OperatorTok{,} \DataTypeTok{int}\NormalTok{ b}\OperatorTok{)} \OperatorTok{\{}
    \ControlFlowTok{return}\NormalTok{ a }\OperatorTok{+}\NormalTok{ b}\OperatorTok{;} \CommentTok{// Возвращает сумму a и b}
\OperatorTok{\}}
\end{Highlighting}
\end{Shaded}

\paragraph{Механизм}\label{ux43cux435ux445ux430ux43dux438ux437ux43c}

передачи параметров в функцию по значению

Передача параметров по значению означает, что в функцию передается копия
аргумента. Изменения в параметрах функции не влияют на оригинальные
переменные:

\begin{Shaded}
\begin{Highlighting}[]
\DataTypeTok{void}\NormalTok{ increment}\OperatorTok{(}\DataTypeTok{int}\NormalTok{ x}\OperatorTok{)} \OperatorTok{\{}
\NormalTok{    x }\OperatorTok{=}\NormalTok{ x }\OperatorTok{+} \DecValTok{1}\OperatorTok{;}
\OperatorTok{\}}

\DataTypeTok{int}\NormalTok{ main}\OperatorTok{()} \OperatorTok{\{}
    \DataTypeTok{int}\NormalTok{ a }\OperatorTok{=} \DecValTok{5}\OperatorTok{;}
\NormalTok{    increment}\OperatorTok{(}\NormalTok{a}\OperatorTok{);}
\NormalTok{    cout }\OperatorTok{\textless{}\textless{}} \StringTok{"a: "} \OperatorTok{\textless{}\textless{}}\NormalTok{ a }\OperatorTok{\textless{}\textless{}}\NormalTok{ endl}\OperatorTok{;} \CommentTok{// a останется 5}
    \ControlFlowTok{return} \DecValTok{0}\OperatorTok{;}
\OperatorTok{\}}
\end{Highlighting}
\end{Shaded}


\end{document}
