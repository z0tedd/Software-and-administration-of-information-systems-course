\documentclass[a4paper,14pt]{extreport} % добавить leqno в [] для нумерации слева
%\usepackage[14pt]{extsizes}
\usepackage[left=3cm,right=1.5cm,
top=2cm,bottom=2cm,bindingoffset=0cm]{geometry}
\linespread{1.45} %полуторный интервал
\usepackage{titlesec}
%%% Работа с русским языком
\titleformat{\section}{\normalfont\bfseries}{\thechapter}{14pt}{\bfseries}
\usepackage{cmap}					% поиск в PDF
\usepackage{mathtext} 				% русские буквы в фомулах
\usepackage[T2A]{fontenc}			% кодировка
\usepackage[utf8]{inputenc}			% кодировка исходного текста
\usepackage[english,russian]{babel}	% локализация и переносы
\usepackage{ fancyhdr} % улучшенная нумерация страниц
%%% Дополнительная работа с математикой
\usepackage{amsmath,amsfonts,amssymb,amsthm,mathtools} % AMS
\usepackage{icomma} % "Умная" запятая: $0,2$ --- число, $0, 2$ --- перечисление
%\renewcommand{\sfdefault}{cmss}
\usefont{T2A}{cmss}{m}{n}
%% Номера формул
\mathtoolsset{showonlyrefs=true} % Показывать номера только у тех формул, на которые есть \eqref{} в тексте.

%% Шрифты
\usepackage{euscript}	 % Шрифт Евклид
\usepackage{mathrsfs} % Красивый матшрифт
%% Свои команды
\DeclareMathOperator{\sgn}{\mathop{sgn}}

%% Перенос знаков в формулах (по Львовскому)
\newcommand*{\hm}[1]{#1\nobreak\discretionary{}
	{\hbox{$\mathsurround=0pt #1$}}{}}

%\renewcommand{\sfdefault}{cmss}
\usepackage{tempora}

\begin{document}
  
\begin{center}
    \textbf{Самостоятельная работа по геометрии и топологии} \\
    \textbf{Вариант 1}
\end{center}

\noindent


\paragraph {1 задача}
 Составьте уравнение цилиндрической поверхности, образующие которой параллельны вектору $\vec{a} (1,2,0)$, а направляющая задана системой:
\[
  \alpha \text{:}
\begin{cases}
    \frac{x^2}{25} + \frac{y^2}{4}+ \frac{z^2}{9}= 1 \\
    x = 0\\
\end{cases}
\]
\textbf{Алгоритм}
1) Пусть $M(x_0, y_0, z_0) \in \alpha$.  Если она принадлежит $\alpha$, 
то ее ур-ние удолетворяет системе
\[  
\begin{cases}
    \frac{x_0^2}{25} + \frac{y_0^2}{4}+ \frac{z_0^2}{9}= 1 \\
    x_0 = 0\\
\end{cases}
=> M(0, y_0, z_0)
\]
2) Так как это цилиндр, то образующая l || $\vec{a}$, $M \in l$. \\ 
Напишем уравнение образующей. 
\[
  \frac{x-0}{1} = \frac{y-y_0}{2} = \frac{z-z_0}{0}
\]
\[
  \begin{cases}
     x = \frac{y-y_0}{2}\\ 
     x = \frac{z-z_0}{0}
  \end{cases} => 
  \begin{cases}
    y_0 = y - 2x \\ 
    z_0 = z
  \end{cases} 
\]
3) Подставим уравнение данной прямой в уравнение направляющей, получим ответ. 
\[
  \frac{(y-2x)^2}{4} + \frac{z^2}{9} = 1
\]

Ответ: $\frac{(y-2x)^2}{4} + \frac{z^2}{9} = 1$

\noindent
\textbf{№2.} Напишите уравнения прямолинейных образующих однополосного гиперболоида: $100x^2 - 36y^2 + 225z^2 = 900$, проходящих через точку $A(3; 2; 0,8)$.

Решение:\\
1) Пусть $\vec{l}(a,b,c)$ - направляющий вектор, прямолинейной образующей $ l $, где $ A \in l $
\[
  l: \begin{cases}
     x = at + 3\\ 
     y = bt + 2\\ 
     z = ct + 0.8\\
   \end{cases} \text{, Подставляем в уравнение из условия}
\]
\[
  100(at+3)^2 - 36(bt+2)^2 + 225(ct+0.8) - 900 = 0
\]
Так как ур-ние должно выполнятся $ \forall t $, то тогда должно выполнятся:
\[
\begin{aligned}
  &t^2|   &100a^2 -36b^2 + 225c^2 = 0\\
  &t^1|   &600at - 144b + 360c = 0\\
  &t^0|  &900-144+144-900 = 0\\
\end{aligned} =>
  \begin{cases}
    100a^2 -36b^2 + 225c^2 = 0\\
    600at - 144b + 360c = 0\\
  \end{cases}
\]  
Для направ. вектора $\vec{l}(a,b,c)$ возьмем 2 варианта $\vec{l}(0,b,c)$, $\vec{l}(1,b,c)$ \\
$\vec{l}(0,b,c)$: 
\[ 
  \begin{cases}
     -36b^2 + 225c^2 = 0\\
     - 144b + 360c = 0\\
  \end{cases} => b = \frac{360}{144}c = \frac{5}{2}c 
\]  
\[
  -36 * \frac{25}{4}c^2 + 225c^2 = 0 => 0*c^2 = 0 => c \in \mathbb{R}
\]
При с = 0, b = 0, получается нулевой вектор, поэтому данный вектором будет $ \vec{l1}(0,\frac{5}{2}c, c) \space c \in \mathbb{R}/ {0}$ \\
\\
$\vec{l}(1,b,c)$: 
\[ 
  \begin{cases}
     100 -36b^2 + 225c^2 = 0\\
     600 - 144b + 360c = 0\\
   \end{cases} => b = \frac{29}{12}, \space c = -\frac{7}{10}\text{ (Подсчитано на калькуляторе)}
\]  
$\vec{l2}(1,\frac{29}{12},-\frac{7}{10})$ \\
2) Подставляем значения напр. векторов в l: \\
\[
  l1:
  \begin{cases} 
     x = 3\\ 
     y = \frac{5}{2}ct + 2\\ 
     z = ct + 0.8\\
 \end{cases} \text{ , где } \space c \in \mathbb{R}/ {0} \text{ , } l2:
\begin{cases} 
     x = t+ 3\\ 
     y = \frac{29}{12}t + 2\\ 
     z = \frac{-7}{10}t + 0.8\\
\end{cases}
\]
Что и является ответом.\\

\noindent
\textbf{№3.} Напишите уравнения прямолинейных образующих гиперболического параболоида:
\[
\frac{y^2}{4} - \frac{z^2}{9} = 10x,
\]
параллельных вектору $\vec{a}(2; 10; -15)$.

\noindent
\textbf{№4*. №9.905 (1).}

905. Даны две точки $F_1$, $F_2$, расстояние между которыми равно $2c > 0$ и число $a > 0$. Найти фигуры:
\[
\begin{aligned}
    &1. \quad \Phi_1 = \{M | \rho(M, F_1) + \rho(M, F_2) = 2a, \quad a > c\}; \\
    &2. \quad \Phi_2 = \{M | \rho(M, F_1) - \rho(M, F_2) = 2a, \quad a < c\}.
\end{aligned}
\]










\paragraph {1 задача}
\end{document}
