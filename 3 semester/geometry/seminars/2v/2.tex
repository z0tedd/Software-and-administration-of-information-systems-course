\documentclass[a4paper,14pt]{extreport} % добавить leqno в [] для нумерации слева
%\usepackage[14pt]{extsizes}
\usepackage[left=3cm,right=1.5cm,
top=2cm,bottom=2cm,bindingoffset=0cm]{geometry}
\linespread{1.45} %полуторный интервал
\usepackage{titlesec}
%%% Работа с русским языком
\titleformat{\section}{\normalfont\bfseries}{\thechapter}{14pt}{\bfseries}
\usepackage{cmap}					% поиск в PDF
\usepackage{mathtext} 				% русские буквы в фомулах
\usepackage[T2A]{fontenc}			% кодировка
\usepackage[utf8]{inputenc}			% кодировка исходного текста
\usepackage[english,russian]{babel}	% локализация и переносы
\usepackage{ fancyhdr} % улучшенная нумерация страниц
%%% Дополнительная работа с математикой
\usepackage{amsmath,amsfonts,amssymb,amsthm,mathtools} % AMS
\usepackage{icomma} % "Умная" запятая: $0,2$ --- число, $0, 2$ --- перечисление
%\renewcommand{\sfdefault}{cmss}
\usefont{T2A}{cmss}{m}{n}
%% Номера формул
\mathtoolsset{showonlyrefs=true} % Показывать номера только у тех формул, на которые есть \eqref{} в тексте.

%% Шрифты
\usepackage{euscript}	 % Шрифт Евклид
\usepackage{mathrsfs} % Красивый матшрифт
%% Свои команды
\DeclareMathOperator{\sgn}{\mathop{sgn}}

%% Перенос знаков в формулах (по Львовскому)
\newcommand*{\hm}[1]{#1\nobreak\discretionary{}
	{\hbox{$\mathsurround=0pt #1$}}{}}

%\renewcommand{\sfdefault}{cmss}
\usepackage{tempora}

\begin{document}

\begin{align*}
&\frac{x^2}{100} + \frac{y^2}{4} + \frac{z^2}{9} = 1 \quad \text{(С эллиптической цилиндром ось у)}\\
&\text{A}(1, 2, 0)
\end{align*}

\section*{Алгоритм:}

1.$ \text{Пусть} M(x_0, y_0, z_0) \in \ell. \text{ Если она принадлежит } z, \text{ то ее проекция на плоскость } xOy \text{ имеет вид} $
\begin{align*}
&\frac{x_0^2}{100} + \frac{y_0^2}{4} = 1 \\
&x_0 = 0 \implies M(0, y_0, z_0)
\end{align*}

2.$ \text{Т.к. она параллельна плоскости } xOy, \text{ то образует } \ell \parallel xOz, M \in \ell $
\begin{align*}
&\frac{x - x_0}{1} = \frac{y - y_0}{2} = \frac{z - z_0}{0} = \frac{2 - 2z_0}{0}
\end{align*}

Подставим $z_0 = \frac{2}{2} = 1$

\begin{align*}
y &= y_0 \\
x &= 2 - 2y
\end{align*}

$\text{Подставим в} \quad \boxed{\text{множество ответов}}$

\begin{align*}
100x^2 - 36y^2 + 225z^2 - 300 = 0 \quad \text{(Канонический обpаз параболоида)}
\end{align*}

$\text{Через } A(1, 2, 0), (0, 1, 0), (0, 0, 2) \text{ищем } E(z), \text{ и } \ell.$

$\text{Подставим } (z = 1):$
\begin{align*}
100x^2 - 36y^2 + 225z^2 - 300 &= 0 \\
100x^2 - 36(2y + 4z)^2 + 225(1 + z)^2 - 300 &= 0
\end{align*}

$\text{Ищем параллельное направление}:$

\begin{align*}
l_2 : 
\begin{cases}
y = -x + z \\
z = t \\
\end{cases}
\end{align*}

Подставим в параболу:

\begin{align*}
100t^2 + 4t^2 - 36t^2 + 225t^2 &= 0 \\
\text{Решим квадратное уравнение:} \\
100 + 36 + 225 &= 800 \\
1000 - 36c^2 + 225c^2 - 800 &= 0 \\
800 + 448 + 360 &= 0
\end{align*}

$\text{Решим уравнение:} $

\begin{align*}
l(z) : 
\begin{cases}
x = 0 \\
y = 0 \\
\end{cases}
\end{align*}

\begin{align*}
z(1, B, C):
\begin{cases}
100 - 36c^2 + 225c^2 = 0 \\
800 + 448 + 360 = 0
\end{cases}
\end{align*}

$\text{Конус } C_1, C_2 - \text{решение}$
\end{document}
