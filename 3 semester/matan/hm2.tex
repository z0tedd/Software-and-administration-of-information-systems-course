


\documentclass{article}

\usepackage[left=3cm,right=1.5cm,
top=2cm,bottom=2cm,bindingoffset=0cm]{geometry}
\linespread{1.45} %полуторный интервал
\usepackage{cmap}					% поиск в PDF
\usepackage{mathtext} 				% русские буквы в фомулах
\usepackage[T2A]{fontenc}			% кодировка
\usepackage[utf8]{inputenc}			% кодировка исходного текста
\usepackage[english,russian]{babel}	% локализация и переносы
\usepackage{ fancyhdr} % улучшенная нумерация страниц
\usepackage{amsmath}

\begin{document}

Задача просит найти градиент функции двух переменных:

\[
f(x_1, x_2) = \frac{1}{1 + e^{-x_1}} + \frac{1}{1 + e^{-x_2}}
\]

в точке \( (0, 0) \).

 Найдем частные производные.

Функция состоит из двух слагаемых, каждое из которых зависит только от одной переменной. Рассмотрим отдельно каждую из них.

1. Частная производная по \( x_1 \):

\[
\frac{\partial f}{\partial x_1} = \frac{d}{d x_1} \left( \frac{1}{1 + e^{-x_1}} \right) = \frac{e^{-x_1}}{(1 + e^{-x_1})^2}
\]

Для точки \( (x_1, x_2) = (0, 0) \):

\[
\frac{\partial f}{\partial x_1} \bigg|_{(0, 0)} = \frac{e^0}{(1 + e^0)^2} = \frac{1}{(1+1)^2} = \frac{1}{4}
\]

2. Частная производная по \( x_2 \):

\[
\frac{\partial f}{\partial x_2} = \frac{d}{d x_2} \left( \frac{1}{1 + e^{-x_2}} \right) = \frac{e^{-x_2}}{(1 + e^{-x_2})^2}
\]

Для точки \( (x_2 = 0) \):

\[
\frac{\partial f}{\partial x_2} \bigg|_{(0, 0)} = \frac{e^0}{(1 + e^0)^2} = \frac{1}{(1+1)^2} = \frac{1}{4}
\]

 Градиент функции:

Градиент — это вектор, составленный из частных производных по каждой переменной:

\[
\nabla f(0, 0) = \left( \frac{\partial f}{\partial x_1}, \frac{\partial f}{\partial x_2} \right) = \left( \frac{1}{4}, \frac{1}{4} \right)
\]

Ответ:$1/4 1/4$.
\end{document}
