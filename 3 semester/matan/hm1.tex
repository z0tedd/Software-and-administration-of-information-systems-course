
\documentclass{article}

\usepackage[left=3cm,right=1.5cm,
top=2cm,bottom=2cm,bindingoffset=0cm]{geometry}
\linespread{1.45} %полуторный интервал
\usepackage{cmap}					% поиск в PDF
\usepackage{mathtext} 				% русские буквы в фомулах
\usepackage[T2A]{fontenc}			% кодировка
\usepackage[utf8]{inputenc}			% кодировка исходного текста
\usepackage[english,russian]{babel}	% локализация и переносы
\usepackage{ fancyhdr} % улучшенная нумерация страниц
\usepackage{amsmath}

\begin{document}

\title{Примеры решения задач по дифференциальному исчислению}
\author{}
\date{}
\section*{1. Площадь фигуры относительно оси Ox}



Площадь между графиком функции \( y = f(x) \), осью Ox и прямыми \( x = a \) и \( x = b \) вычисляется по формуле:
\[
S = \int_{a}^{b} |f(x)| \, dx
\]

\textbf{Пример:} Найти площадь фигуры, ограниченной графиком функции \( y = x^2 \), осью Ox и прямыми \( x = -1 \) и \( x = 1 \).

\[
S = \int_{-1}^{1} x^2 \, dx = \left[ \frac{x^3}{3} \right]_{-1}^{1} = \frac{1^3}{3} - \frac{(-1)^3}{3} = \frac{1}{3} - \left( -\frac{1}{3} \right) = \frac{2}{3}
\]

Ответ: \( S = \frac{2}{3} \).

\section*{2. Площадь фигуры относительно оси Oy}
Площадь между графиком функции \( x = g(y) \), осью Oy и прямыми \( y = c \) и \( y = d \) вычисляется по формуле:
\[
S = \int_{c}^{d} |g(y)| \, dy
\]

\textbf{Пример:} Найти площадь фигуры, ограниченной графиком функции \( x = y^2 \), осью Oy и прямыми \( y = -1 \) и \( y = 1 \).

\[
S = \int_{-1}^{1} y^2 \, dy = \left[ \frac{y^3}{3} \right]_{-1}^{1} = \frac{1^3}{3} - \frac{(-1)^3}{3} = \frac{1}{3} - \left( -\frac{1}{3} \right) = \frac{2}{3}
\]

Ответ: \( S = \frac{2}{3} \).

\section*{3. Площадь фигуры относительно кривой}
Площадь между двумя функциями \( y = f(x) \) и \( y = g(x) \) на интервале от \( x = a \) до \( x = b \) вычисляется по формуле:
\[
S = \int_{a}^{b} |f(x) - g(x)| \, dx
\]

\textbf{Пример:} Найти площадь между графиками функций \( y = x^2 \) и \( y = 2x \) на интервале от \( x = 0 \) до \( x = 2 \).

\[
S = \int_{0}^{2} |x^2 - 2x| \, dx
\]
Разбиваем на два участка, где функции пересекаются (при \( x = 0 \) и \( x = 2 \)):
\[
S = \int_{0}^{2} (2x - x^2) \, dx = \left[ x^2 - \frac{x^3}{3} \right]_{0}^{2} = (4 - \frac{8}{3}) - (0 - 0) = 4 - \frac{8}{3} = \frac{12}{3} - \frac{8}{3} = \frac{4}{3}
\]

Ответ: \( S = \frac{4}{3} \).

\section*{4. Объём фигуры, полученной при вращении относительно оси Ox}
Объём тела вращения при вращении графика функции \( y = f(x) \) вокруг оси Ox на интервале от \( x = a \) до \( x = b \) вычисляется по формуле:
\[
V = \pi \int_{a}^{b} [f(x)]^2 \, dx
\]

\textbf{Пример:} Найти объём тела, полученного при вращении графика функции \( y = \sqrt{x} \) вокруг оси Ox на интервале от \( x = 0 \) до \( x = 1 \).

\[
V = \pi \int_{0}^{1} (\sqrt{x})^2 \, dx = \pi \int_{0}^{1} x \, dx = \pi \left[ \frac{x^2}{2} \right]_{0}^{1} = \pi \cdot \frac{1}{2} = \frac{\pi}{2}
\]

Ответ: \( V = \frac{\pi}{2} \).

\section*{5. Объём фигуры, полученной при вращении относительно оси Oy}
Объём тела вращения при вращении графика функции \( x = g(y) \) вокруг оси Oy на интервале от \( y = c \) до \( y = d \) вычисляется по формуле:
\[
V = \pi \int_{c}^{d} [g(y)]^2 \, dy
\]

\textbf{Пример:} Найти объём тела, полученного при вращении графика функции \( x = y^2 \) вокруг оси Oy на интервале от \( y = 0 \) до \( y = 1 \).

\[
V = \pi \int_{0}^{1} (y^2)^2 \, dy = \pi \int_{0}^{1} y^4 \, dy = \pi \left[ \frac{y^5}{5} \right]_{0}^{1} = \pi \cdot \frac{1}{5} = \frac{\pi}{5}
\]

Ответ: \( V = \frac{\pi}{5} \).

\end{document}
